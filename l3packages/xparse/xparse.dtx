% \iffalse meta-comment
%
%% File: xparse.dtx
%
% Copyright (C) 1999 Frank Mittelbach, Chris Rowley, David Carlisle
%           (C) 2004-2008 Frank Mittelbach, The LaTeX Project
%           (C) 2009-2021 The LaTeX Project
%
% It may be distributed and/or modified under the conditions of the
% LaTeX Project Public License (LPPL), either version 1.3c of this
% license or (at your option) any later version.  The latest version
% of this license is in the file
%
%    https://www.latex-project.org/lppl.txt
%
% This file is part of the "l3packages bundle" (The Work in LPPL)
% and all files in that bundle must be distributed together.
%
% -----------------------------------------------------------------------
%
% The development version of the bundle can be found at
%
%    https://github.com/latex3/latex3
%
% for those people who are interested.
%
%<*driver|package>
% The version of expl3 required is tested as early as possible, as
% some really old versions do not define \ProvidesExplPackage.
\RequirePackage{expl3}[2018-04-12]
%<package>\@ifpackagelater{expl3}{2018-04-12}
%<package>  {}
%<package>  {%
%<package>    \PackageError{xparse}{Support package l3kernel too old}
%<package>      {%
%<package>        Please install an up to date version of l3kernel\MessageBreak
%<package>        using your TeX package manager or from CTAN.\MessageBreak
%<package>        \MessageBreak
%<package>        Loading xparse will abort!%
%<package>      }%
%<package>    \endinput
%<package>  }
%</driver|package>
%<*driver>
\documentclass[full]{l3doc}
\usepackage{amstext}
\begin{document}
  \DocInput{\jobname.dtx}
\end{document}
%</driver>
% \fi
%
% \providecommand\acro[1]{\textsc{\MakeLowercase{#1}}}
% \newenvironment{arg-description}{%
%   \begin{itemize}\def\makelabel##1{\hss\llap{\bfseries##1}}}{\end{itemize}}
%
% \title{^^A
%   The \textsf{xparse} package\\ Document command parser^^A
% }
%
% \author{^^A
%  The \LaTeX{} Project\thanks
%    {^^A
%      E-mail:
%        \href{mailto:latex-team@latex-project.org}
%          {latex-team@latex-project.org}^^A
%    }^^A
% }
%
% \date{Released 2021-11-12}
%
% \maketitle
%
% \begin{documentation}
%
% The \pkg{xparse} package provides a high-level interface for
% producing document-level commands. In that way, it is intended as
% a replacement for the \LaTeXe{} \cs{newcommand} macro. However,
% \pkg{xparse} works so that the interface to a function (optional
% arguments, stars and mandatory arguments, for example) is separate
% from the internal implementation. \pkg{xparse} provides a normalised
% input for the internal form of a function, independent of the
% document-level argument arrangement.
%
% At present, the functions in \pkg{xparse} which are regarded as
% \enquote{stable} are:
% \begin{itemize}
%   \item \cs{NewDocumentCommand}\\
%     \cs{RenewDocumentCommand}\\
%     \cs{ProvideDocumentCommand}\\
%     \cs{DeclareDocumentCommand}
%   \item \cs{NewDocumentEnvironment}\\
%     \cs{RenewDocumentEnvironment}\\
%     \cs{ProvideDocumentEnvironment}\\
%     \cs{DeclareDocumentEnvironment}
%   \item \cs{NewExpandableDocumentCommand}\\
%     \cs{RenewExpandableDocumentCommand}\\
%     \cs{ProvideExpandableDocumentCommand}\\
%     \cs{DeclareExpandableDocumentCommand}
%   \item \cs{IfNoValue(TF)}
%   \item \cs{IfValue(TF)}
%   \item \cs{IfBoolean(TF)}
% \end{itemize}
% with the other functions currently regarded as \enquote{experimental}. Please
% try all of the commands provided here, but be aware that the
% experimental ones may change or disappear.
%
% \section{Specifying arguments}
%
% Before introducing the functions used to create document commands,
% the method for specifying arguments with \pkg{xparse} will be
% illustrated. In order to allow each argument to be defined
% independently, \pkg{xparse} does not simply need to know the
% number of arguments for a function, but also the nature of each
% one. This is done by constructing an \emph{argument specification},
% which defines the number of arguments, the type of each argument
% and any additional information needed for \pkg{xparse} to read the
% user input and properly pass it through to internal functions.
%
% The basic form of the argument specifier is a list of letters, where
% each letter defines a type of argument. As will be described below,
% some of the types need additional information, such as default values.
% The argument types can be divided into two, those which define
% arguments that are mandatory (potentially raising an error if not
% found) and those which define optional arguments. The mandatory types
% are:
% \begin{itemize}[font=\ttfamily]
%   \item[m] A standard mandatory argument, which can either be a single
%     token alone or multiple tokens surrounded by curly braces |{}|.
%     Regardless of the input, the argument will be passed to the
%     internal code without the outer braces. This is the \pkg{xparse}
%     type specifier for a normal \TeX{} argument.
%   \item[r] Given as \texttt{r}\meta{token1}\meta{token2}, this denotes a
%     \enquote{required} delimited argument, where the delimiters are
%     \meta{token1} and \meta{token2}. If the opening delimiter
%     \meta{token1} is missing, the default marker |-NoValue-| will be
%     inserted after a suitable error.
%   \item[R] Given as \texttt{R}\meta{token1}\meta{token2}\marg{default},
%     this is a \enquote{required} delimited argument as for~\texttt{r},
%     but it has a user-definable recovery \meta{default} instead of
%     |-NoValue-|.
%   \item[v] Reads an argument \enquote{verbatim}, between the following
%     character and its next occurrence, in a way similar to the argument
%     of the \LaTeXe{} command \cs{verb}. Thus a \texttt{v}-type argument
%     is read between two identical characters, which cannot be any of |%|, |\|,
%     |#|, |{|, |}| or \verb*| |.
%     The verbatim argument can also be enclosed between braces, |{| and |}|.
%     A command with a verbatim
%     argument will produce an error when it appears within an argument of
%     another function.
%   \item[b] Only suitable in the argument specification of an
%     environment, it denotes the body of the environment, between
%     |\begin|\marg{environment} and |\end|\marg{environment}.  See
%     Section~\ref{sec:body} for details.
% \end{itemize}
% The types which define optional arguments are:
% \begin{itemize}[font=\ttfamily]
%   \item[o] A standard \LaTeX{} optional argument, surrounded with square
%     brackets, which will supply
%     the special |-NoValue-| marker if not given (as described later).
%   \item[d] Given as \texttt{d}\meta{token1}\meta{token2}, an optional
%     argument which is delimited by \meta{token1} and \meta{token2}.
%     As with \texttt{o}, if no
%     value is given the special marker |-NoValue-| is returned.
%   \item[O] Given as \texttt{O}\marg{default}, is like \texttt{o}, but
%     returns \meta{default} if no value is given.
%   \item[D] Given as \texttt{D}\meta{token1}\meta{token2}\marg{default},
%     it is as for \texttt{d}, but returns \meta{default} if no value is given.
%     Internally, the \texttt{o}, \texttt{d} and \texttt{O} types are
%     short-cuts to an appropriated-constructed \texttt{D} type argument.
%   \item[s] An optional star, which will result in a value
%     \cs{BooleanTrue} if a star is present and \cs{BooleanFalse}
%     otherwise (as described later).
%   \item[t] An optional \meta{token}, which will result in a value
%     \cs{BooleanTrue} if \meta{token} is present and \cs{BooleanFalse}
%     otherwise. Given as \texttt{t}\meta{token}.
%   \item[e] Given as \texttt{e}\marg{tokens}, a set of optional
%     \emph{embellishments}, each of which requires a \emph{value}.
%     If an embellishment is not present, |-NoValue-| is returned.  Each
%     embellishment gives one argument, ordered as for the list of
%     \meta{tokens} in the argument specification.  All \meta{tokens}
%     must be distinct.  \emph{This is an experimental type}.
%   \item[E] As for \texttt{e} but returns one or more \meta{defaults}
%     if values are not given: \texttt{E}\marg{tokens}\marg{defaults}. See
%     Section~\ref{sec:embellishment} for more details.
% \end{itemize}
%
% Using these specifiers, it is possible to create complex input syntax
% very easily. For example, given the argument definition
% `|s o o m O{default}|', the input `|*[Foo]{Bar}|' would be parsed as:
% \begin{itemize}[nolistsep]
%   \item |#1| = |\BooleanTrue|
%   \item |#2| = |Foo|
%   \item |#3| = |-NoValue-|
%   \item |#4| = |Bar|
%   \item |#5| = |default|
% \end{itemize}
% whereas `|[One][Two]{}[Three]|' would be parsed as:
% \begin{itemize}[nolistsep]
%   \item |#1| = |\BooleanFalse|
%   \item |#2| = |One|
%   \item |#3| = |Two|
%   \item |#4| = ||
%   \item |#5| = |Three|
% \end{itemize}
%
% Delimited argument types (\texttt{d}, \texttt{o} and \texttt{r}) are
% defined such that they require matched pairs of delimiters when collecting
% an argument. For example
% \begin{verbatim}
%   \NewDocumentCommand{\foo}{o}{#1}
%   \foo[[content]] % #1 = "[content]"
%   \foo[[]         % Error: missing closing "]"
% \end{verbatim}
% Also note that |{| and |}| cannot be used as delimiters as they are used
% by \TeX{} as grouping tokens. Implicit begin- or end-group tokens (\emph{e.g.},
% |\bgroup| and |\egroup|) are not allowed for delimited argument tipes.
% Arguments to be grabbed inside these tokens
% must be created as either \texttt{m}- or \texttt{g}-type arguments.
%
% Within delimited arguments, non-balanced or otherwise awkward tokens may
% be included by protecting the entire argument with a brace pair
% \begin{verbatim}
%   \NewDocumentCommand{\foobar}{o}{#1}
%   \foobar[{[}]         % Allowed as the "[" is 'hidden'
% \end{verbatim}
% These braces will be stripped only if they surround the \emph{entire} content
% of the optional argument
% \begin{verbatim}
%   \NewDocumentCommand{\foobaz}{o}{#1}
%   \foobaz[{abc}]         % => "abc"
%   \foobaz[ {abc}]         % => " {abc}"
% \end{verbatim}
%
% Two more characters have a special meaning when creating an argument
% specifier. First, \texttt{+} is used to make an argument long (to
% accept paragraph tokens). In contrast to \LaTeXe's \cs{newcommand},
% this applies on an argument-by-argument basis. So modifying the
% example to `|s o o +m O{default}|' means that the mandatory argument
% is now \cs{long}, whereas the optional arguments are not.
%
% Secondly, the character \texttt{>} is used to declare so-called
% \enquote{argument processors}, which can be used to modify the contents of an
% argument before it is passed to the macro definition. The use of
% argument processors is a somewhat advanced topic, (or at least a less
% commonly used feature) and is covered in Section~\ref{sec:processors}.
%
% When an optional argument is followed by a mandatory argument with the
% same delimiter, \pkg{xparse} issues a warning because the optional
% argument could not be omitted by the user, thus becoming in effect
% mandatory.  This can apply to \texttt{o}, \texttt{d}, \texttt{O},
% \texttt{D}, \texttt{s}, \texttt{t}, \texttt{e}, and \texttt{E} type
% arguments followed by \texttt{r} or \texttt{R}-type required
% arguments, but also to \texttt{g} or \texttt{G} type arguments
% followed by \texttt{m} type arguments.
%
% As \pkg{xparse} is also used to describe interfaces that have appeared
% in the wider \LaTeXe{} eco-system, it also defines additional argument
% types, described in Section~\ref{sec:backwards}: the mandatory types
% \texttt{l} and \texttt{u} and the optional brace group types
% \texttt{g} and \texttt{G}.  Their use is not recommended because it is
% simpler for a user if all packages use a similar syntax.  For the same
% reason, delimited arguments \texttt{r}, \texttt{R}, \texttt{d} and
% \texttt{D} should normally use delimiters that are naturally paired,
% such as |[| and |]| or |(| and |)|, or that are identical, such as |"|
% and~|"|.  A very common syntax is to have one optional argument
% \texttt{o} treated as a key--value list (using for instance
% \pkg{l3keys}) followed by some mandatory arguments~\texttt{m} (or
% \texttt{+m}).
%
% \subsection{Spacing and optional arguments}
%
% \TeX{} will find the first argument after a function name irrespective
% of any intervening spaces. This is true for both mandatory and
% optional arguments. So |\foo[arg]| and \verb*|\foo   [arg]| are
% equivalent. Spaces are also ignored when collecting arguments up
% to the last mandatory argument to be collected (as it must exist).
% So after
% \begin{verbatim}
%   \NewDocumentCommand \foo { m o m } { ... }
% \end{verbatim}
% the user input |\foo{arg1}[arg2]{arg3}| and
% \verb*|\foo{arg1}  [arg2]   {arg3}| will both be parsed in the same
% way.
% 
% The behavior of optional arguments \emph{after} any mandatory arguments is
% selectable. The standard settings will allow spaces here, and thus
% with
% \begin{verbatim}
%   \NewDocumentCommand \foobar { m o } { ... }
% \end{verbatim}
% both |\foobar{arg1}[arg2]| and \verb*|\foobar{arg1} [arg2]| will find an
% optional argument. This can be changed by giving the modified |!| in
% the argument specification:
% \begin{verbatim}
%   \NewDocumentCommand \foobar { m !o } { ... }
% \end{verbatim}
% where \verb*|\foobar{arg1} [arg2]| will not find an optional argument.
%
% There is one subtlety here due to the difference in handling by \TeX{}
% of \enquote{control symbols}, where the command name is made up of a single
% character, such as \enquote{\cmd{\\}}. Spaces are not ignored by \TeX{}
% here, and thus it is possible to require an optional argument directly
% follow such a command. The most common example is the use of \cmd{\\} in
% \pkg{amsmath} environments. In \pkg{xparse} terms it has signature
% \begin{verbatim}
%   \DeclareDocumentCommand \\ { !s !o } { ... }
% \end{verbatim}
%
% \subsection{Required delimited arguments}
%
% The contrast between a delimited (\texttt{D}-type) and \enquote{required
% delimited} (\texttt{R}-type) argument is that an error will be raised if
% the latter is missing. Thus for example
% \begin{verbatim}
%   \NewDocumentCommand {\foobaz} {r()m} {}
%   \foobaz{oops}
% \end{verbatim}
% will lead to an error message being issued. The marker |-NoValue-|
% (\texttt{r}-type) or user-specified default (for \texttt{R}-type) will be
% inserted to allow error recovery.
%
% \subsection{Verbatim arguments}
%
% Arguments of type~\texttt{v} are read in verbatim mode, which will
% result in the grabbed argument consisting of tokens of category codes
% $12$~(\enquote{other}) and $13$~(\enquote{active}), except spaces,
% which are given category code $10$~(\enquote{space}). The argument is
% delimited in a similar manner to the \LaTeXe{} \cs{verb} function, or
% by (correctly nested) pairs of braces.
%
% Functions containing verbatim arguments cannot appear in the arguments
% of other functions. The \texttt{v}~argument specifier includes code to check
% this, and will raise an error if the grabbed argument has already been
% tokenized by \TeX{} in an irreversible way.
%
% By default, an argument of type~\texttt{v} must be at most one line.
% Prefixing with \texttt{+} allows line breaks within the argument.
%
% Users should note that support for verbatim arguments is somewhat
% experimental. Feedback is therefore very welcome on the \texttt{LaTeX-L}
% mailing list.
%
% \subsection{Default values of arguments}
% \label{sec:defaultvaluesofarguments}
%
% Uppercase argument types (\texttt{O}, \texttt{D}, \ldots{}) allow to
% specify a default value to be used when the argument is missing; their
% lower-case counterparts use the special marker |-NoValue-|.  The
% default value can be expressed in terms of the value of any other
% arguments by using |#1|, |#2|, and so on.
% \begin{verbatim}
%   \NewDocumentCommand {\conjugate} { m O{#1ed} O{#2} } {(#1,#2,#3)}
%   \conjugate {walk}            % => (walk,walked,walked)
%   \conjugate {find} [found]    % => (find,found,found)
%   \conjugate {do} [did] [done] % => (do,did,done)
% \end{verbatim}
% The default values may refer to arguments that appear later in the
% argument specification.  For instance a command could accept two
% optional arguments, equal by default:
% \begin{verbatim}
%   \NewDocumentCommand {\margins} { O{#3} m O{#1} m } {(#1,#2,#3,#4)}
%   \margins {a} {b}              % => {(-NoValue-,a,-NoValue-,b)}
%   \margins [1cm] {a} {b}        % => {(1cm,a,1cm,b)}
%   \margins {a} [1cm] {b}        % => {(1cm,a,1cm,b)}
%   \margins [1cm] {a} [2cm] {b}  % => {(1cm,a,2cm,b)}
% \end{verbatim}
%
% Users should note that support for default arguments referring to
% other arguments is somewhat experimental. Feedback is therefore very
% welcome on the \texttt{LaTeX-L} mailing list.
%
% \subsection{Default values for \enquote{embellishments}}
% \label{sec:embellishment}
%
% The \texttt{E}-type argument allows one default value per test token.
% This is achieved by giving a list of defaults for each entry in the
% list, for example:
% \begin{verbatim}
%   E{^_}{{UP}{DOWN}}
% \end{verbatim}
% If the list of default values is \emph{shorter} than the list of test tokens,
% the special |-NoValue-| marker will be returned (as for the \texttt{e}-type
% argument). Thus for example
% \begin{verbatim}
%   E{^_}{{UP}}
% \end{verbatim}
% has default \texttt{UP} for the |^| test character, but will return the
% |-NoValue-| marker as a default for |_|. This allows mixing of explicit
% defaults with testing for missing values.
%
% \subsection{Body of an environment}
% \label{sec:body}
%
% While environments |\begin|\marg{environment} \dots{}
%   |\end|\marg{environment} are typically used in cases where the code
% implementing the \meta{environment} does not need to access the
% contents of the environment (its \enquote{body}), it is sometimes
% useful to have the body as a standard argument.
%
% This is achieved in \pkg{xparse} by ending the argument specification
% with~\texttt{b}. The approach taken in \pkg{xparse} is
% different from the earlier packages \pkg{environ} or \pkg{newenviron}:
% the body of the environment is provided to the code part as a usual
% argument |#1|, |#2| etc.\@, rather than stored in a macro such as
% \cs[no-index]{BODY}.
%
% For instance
% \begin{verbatim}
%   \NewDocumentEnvironment { twice }
%     { O{\ttfamily} +b }
%     {#2#1#2} {}
%   \begin{twice}[\itshape]
%     Hello world!
%   \end{twice}
% \end{verbatim}
% typesets \enquote{Hello world!{\itshape Hello world!}}.
%
% The prefix |+| is used to allow multiple paragraphs in the
% environment's body.  Argument processors can also be applied to
% \texttt{b}~arguments.
%
% By default, spaces are trimmed at both ends of the body: in the
% example there would otherwise be spaces coming from the ends the lines
% after |[\itshape]| and |world!|.  Putting the prefix |!| before
% \texttt{b} suppresses space-trimming.
%
% When \texttt{b} is used in the argument specification,
% the last argument of \cs{NewDocumentEnvironment}, which consists of
% an \meta{end code} to insert at |\end|\marg{environment}, is
% redundant since one can simply put that code at the end of the
% \meta{start code}.  Nevertheless this (empty) \meta{end code} must be
% provided.
%
% Environments that use this feature can be nested.
%
% Users should note that this feature is somewhat experimental. Feedback
% is therefore very welcome on the \texttt{LaTeX-L} mailing list.
%
% \subsection{Starred environments}
%
% Many packages define environments with and without \texttt{*} in their
% name, for instance \texttt{tabular} and \texttt{tabular*}.  At
% present, \pkg{xparse} does not provide specific tools to define these:
% one should simply define the two environment separately, for instance
% \begin{verbatim}
% \NewDocumentEnvironment { tabular } { o +m } {...} {...}
% \NewDocumentEnvironment { tabular* } { m o +m } {...} {...}
% \end{verbatim}
% Of course the implementation of these two environments, denoted
% \enquote{\texttt{...}} in this example, can rely on the same internal
% commands.
%
% Note that this situation is different from the \texttt{s} argument
% type: if the signature of an environment starts with~\texttt{s} then
% the star is searched for after the argument of \cs{begin}.  For
% instance, the following typesets \texttt{star}.
% \begin{verbatim}
% \NewDocumentEnvironment { envstar } { s }
%   {\IfBooleanTF {#1} {star} {no star}} {}
% \begin{envstar}*
% \end{envstar}
% \end{verbatim}
%
% \subsection{Backwards Compatibility}
% \label{sec:backwards}
%
% One role of \pkg{xparse} is to describe existing \LaTeX{} interfaces,
% including some that are rather unusual in \LaTeX{} (as opposed to
% formats such as plain \TeX{}) such as delimited arguments.  As such,
% the package defines some argument specifiers that should largely be
% avoided nowadays as using them in packages leads to inconsistent user
% interfaces.  The simplest syntax is often best, with argument
% specifications such as |mmmm| or |ommmm|, namely an optional argument
% followed by some standard mandatory ones.  The optional argument can
% be made to support key--value syntax using tools from \pkg{l3keys}.
%
% The argument types that are not recommended any longer are:
% \begin{itemize}[font=\ttfamily]
%   \item[l] A mandatory argument which reads everything up to the first
%     begin-group token: in standard \LaTeX{} this is a left brace.
%   \item[u] Reads a mandatory argument \enquote{until} \meta{tokens} are encountered,
%     where the desired \meta{tokens} are given as an argument to the
%     specifier: \texttt{u}\marg{tokens}.
%   \item[g] An optional argument given inside a pair of \TeX{} group
%     tokens (in standard \LaTeX{}, |{| \ldots |}|), which returns
%     |-NoValue-| if not present.
%   \item[G] As for \texttt{g} but returns \meta{default} if no value
%     is given: \texttt{G}\marg{default}.
% \end{itemize}
%
% \subsection{Details about argument delimiters}
%
% In normal (non-expandable) commands, the delimited types look for the
% initial delimiter by peeking ahead (using \pkg{expl3}'s |\peek_...|
% functions) looking for the delimiter token.  The token has to have the
% same meaning and \enquote{shape} of the token defined as delimiter.
% There are three possible cases of delimiters: character tokens, control
% sequence tokens, and active character tokens.  For all practical purposes
% of this description, active character tokens will behave exactly as
% control sequence tokens.
%
% \subsubsection{Character tokens}
%
% A character token is characterised by its character code, and its meaning
% is the category code~(|\catcode|).  When a command is defined, the meaning
% of the character token is fixed into the definition of the command and
% cannot change.  A command will correctly see an argument delimiter if
% the open delimiter has the same character and category codes as at the
% time of the definition.  For example in:
% \begin{verbatim}
%   \NewDocumentCommand { \foobar } { D<>{default} } {(#1)}
%   \foobar <hello> \par
%   \char_set_catcode_letter:N <
%   \foobar <hello>
% \end{verbatim}
% the output would be:
% \begin{verbatim}
%   (hello)
%   (default)<hello>
% \end{verbatim}
% as the open-delimter |<| changed in meaning between the two calls to
% |\foobar|, so the second one doesn't see the |<| as a valid delimiter.
% Commands assume that if a valid open-delimiter was found, a matching
% close-delimiter will also be there.  If it is not (either by being
% omitted or by changing in meaning), a low-level \TeX{} error is raised
% and the command call is aborted.
%
% \subsubsection{Control sequence tokens}
%
% A control sequence (or control character) token is characterised by is
% its name, and its meaning is its definition.
% A token cannot have two different meanings at the same time.
% When a control sequence is defined as delimiter in a command,
% it will be detected as delimiter whenever the control sequence name
% is found in the document regardless of its current definition.
% For example in:
% \begin{verbatim}
%   \cs_set:Npn \x { abc }
%   \NewDocumentCommand { \foobar } { D\x\y{default} } {(#1)}
%   \foobar \x hello\y \par
%   \cs_set:Npn \x { def }
%   \foobar \x hello\y
% \end{verbatim}
% the output would be:
% \begin{verbatim}
%   (hello)
%   (hello)
% \end{verbatim}
% with both calls to the command seeing the delimiter |\x|.
%
% \section{Declaring commands and environments}
%
% With the concept of an argument specifier defined, it is now
% possible to describe the methods available for creating both
% functions and environments using \pkg{xparse}.
%
% The interface-building commands are the preferred method for
% creating document-level functions in \LaTeX3. All of the functions
% generated in this way are naturally robust (using the \eTeX{}
% \cs{protected} mechanism).
%
% \begin{function}
%   {
%     \NewDocumentCommand     ,
%     \RenewDocumentCommand   ,
%     \ProvideDocumentCommand ,
%     \DeclareDocumentCommand
%   }
%   \begin{syntax}
%     \cs{NewDocumentCommand} \meta{function} \Arg{arg spec} \Arg{code}
%   \end{syntax}
%   This family of commands are used to create a document-level
%   \meta{function}. The argument specification for the function is
%   given by \meta{arg spec}, and the function expands to the
%   \meta{code} with |#1|, |#2|, etc.\ replaced by the arguments found
%   by \pkg{xparse}.
% \end{function}
%
%   As an example:
%   \begin{verbatim}
%     \NewDocumentCommand \chapter { s o m }
%       {
%         \IfBooleanTF {#1}
%           { \typesetstarchapter {#3} }
%           { \typesetnormalchapter {#2} {#3} }
%       }
%   \end{verbatim}
%   would be a way to define a \cs{chapter} command which would
%   essentially behave like the current \LaTeXe{} command (except that it
%   would accept an optional argument even when a \texttt{*} was parsed).
%   The \cs{typesetnormalchapter} could test its first argument for being
%   |-NoValue-| to see if an optional argument was present.
%
%   The difference between the \cs{New\ldots} \cs{Renew\ldots},
%   \cs{Provide\ldots} and \cs{Declare\ldots} versions is the behaviour
%   if \meta{function} is already defined.
%   \begin{itemize}
%    \item \cs{NewDocumentCommand} will issue an error if \meta{function}
%      has already been defined.
%    \item \cs{RenewDocumentCommand} will issue an error if \meta{function}
%      has not previously been defined.
%    \item \cs{ProvideDocumentCommand} creates a new definition for
%      \meta{function} only if one has not already been given.
%     \item \cs{DeclareDocumentCommand} will always create the new
%       definition, irrespective of any existing \meta{function} with the
%       same name.  This should be used sparingly.
%   \end{itemize}
%
%   \begin{texnote}
%      Unlike \LaTeXe{}'s \cs{newcommand} and relatives, the
%      \cs{NewDocumentCommand} family of functions do not prevent creation of
%      functions with names starting \cs{end\ldots}.
%   \end{texnote}
%
% \begin{function}
%   {
%     \NewDocumentEnvironment     ,
%     \RenewDocumentEnvironment   ,
%     \ProvideDocumentEnvironment ,
%     \DeclareDocumentEnvironment
%   }
%   \begin{syntax}
%     \cs{NewDocumentEnvironment} \Arg{environment} \Arg{arg spec}
%     ~~\Arg{start code} \Arg{end code}
%   \end{syntax}
%   These commands work in the same way as \cs{NewDocumentCommand},
%   etc.\@, but create environments (\cs{begin}\Arg{environment} \ldots{}
%   \cs{end}\Arg{environment}). Both the \meta{start code} and
%   \meta{end code}
%   may access the arguments as defined by \meta{arg spec}.
%   The arguments will be given following \cs{begin}\Arg{environment}.
% \end{function}
%
% \section{Other \pkg{xparse} commands}
%
% \subsection{Testing special values}
%
% Optional arguments created using \pkg{xparse} make use of dedicated
% variables to return information about the nature of the argument
% received.
%
% \begin{function}[EXP]{\IfNoValueT, \IfNoValueF, \IfNoValueTF}
%   \begin{syntax}
%     \cs{IfNoValueTF} \Arg{argument} \Arg{true code} \Arg{false code}
%     \cs{IfNoValueT} \Arg{argument} \Arg{true code}
%     \cs{IfNoValueF} \Arg{argument} \Arg{false code}
%   \end{syntax}
%   The \cs{IfNoValue(TF)} tests are used to check if \meta{argument} (|#1|,
%   |#2|, \emph{etc.}) is the special |-NoValue-| marker For example
%   \begin{verbatim}
%     \NewDocumentCommand \foo { o m }
%       {
%         \IfNoValueTF {#1}
%           { \DoSomethingJustWithMandatoryArgument {#2} }
%           {  \DoSomethingWithBothArguments {#1} {#2}   }
%       }
%   \end{verbatim}
%   will use a different internal function if the optional argument
%   is given than if it is not present.
%
%   Note that three tests are available, depending on which outcome
%   branches are required: \cs{IfNoValueTF}, \cs{IfNoValueT} and
%   \cs{IfNoValueF}.
%
%   As the \cs{IfNoValue(TF)} tests are expandable, it is possible to
%   test these values later, for example at the point of typesetting or
%   in an expansion context.
%
%   It is important to note that |-NoValue-| is constructed such that it
%   will \emph{not} match the simple text input |-NoValue-|, \emph{i.e.}
%   that
%   \begin{verbatim}
%     \IfNoValueTF{-NoValue-}
%   \end{verbatim}
%   will be logically \texttt{false}.
%
%   When two optional arguments follow each other (a syntax we typically
%   discourage), it can make sense to allow users of the command to
%   specify only the second argument by providing an empty first
%   argument.  Rather than testing separately for emptyness and for
%   |-NoValue-| it is then best to use the argument type~|O| with an
%   empty default value, and simply test for emptyness using the
%   \pkg{expl3} conditional \cs{tl_if_blank:nTF} or its \pkg{etoolbox}
%   analogue \tn{ifblank}.
% \end{function}
%
% \begin{function}[EXP]{\IfValueT, \IfValueF, \IfValueTF}
%   \begin{syntax}
%     \cs{IfValueTF} \Arg{argument} \Arg{true code} \Arg{false code}
%    \end{syntax}
%   The reverse form of the \cs{IfNoValue(TF)} tests are also available
%   as \cs{IfValue(TF)}. The context will determine which logical
%   form makes the most sense for a given code scenario.
% \end{function}
%
% \begin{variable}{\BooleanFalse, \BooleanTrue}
%   The \texttt{true} and \texttt{false} flags set when searching for
%   an optional character (using \texttt{s} or \texttt{t\meta{char}}) have
%   names which are accessible outside of code blocks.
% \end{variable}
%
% \begin{function}[EXP]{\IfBooleanT, \IfBooleanF, \IfBooleanTF}
%   \begin{syntax}
%     \cs{IfBooleanTF} \Arg{argument} \Arg{true code} \Arg{false code}
%   \end{syntax}
%   Used to test if \meta{argument} (|#1|, |#2|, \emph{etc.}) is
%   \cs{BooleanTrue} or \cs{BooleanFalse}. For example
%   \begin{verbatim}
%     \NewDocumentCommand \foo { s m }
%       {
%         \IfBooleanTF {#1}
%           { \DoSomethingWithStar {#2} }
%           { \DoSomethingWithoutStar {#2} }
%       }
%   \end{verbatim}
%   checks for a star as the first argument, then chooses the action to
%   take based on this information.
% \end{function}
%
% \subsection{Argument processors}
% \label{sec:processors}
%
% \pkg{xparse} introduces the idea of an argument processor, which is
% applied to an argument \emph{after} it has been grabbed by the
% underlying system but before it is passed to \meta{code}. An argument
% processor can therefore be used to regularise input at an early stage,
% allowing the internal functions to be completely independent of input
% form. Processors are applied to user input and to default values for
% optional arguments, but \emph{not} to the special |-NoValue-| marker.
%
% Each argument processor is specified by the syntax
% \texttt{>}\marg{processor} in the argument specification. Processors
% are applied from right to left, so that
% \begin{verbatim}
%   >{\ProcessorB} >{\ProcessorA} m
% \end{verbatim}
% would apply \cs{ProcessorA}
% followed by \cs{ProcessorB} to the tokens grabbed by the \texttt{m}
% argument.
%
% It might sometimes be useful to use the value of another argument as
% one of the arguments of a processor.  For example, using the
% \cs{SplitList} processor defined below,
% \begin{verbatim}
%   \NewDocumentCommand \foo { O{,} >{\SplitList{#1}} m } { \foobar{#2} }
%   \foo{a,b;c,d}
% \end{verbatim}
% results in |\foobar| receiving the argument |{a}{b;c}{d}| because
% \cs{SplitList} receives as its two arguments the optional one (whose
% value here is the default, a comma) and the mandatory one.  To
% summarize, first the arguments are searched for in the input, then any
% default argument is determined as explained in
% Section~\ref{sec:defaultvaluesofarguments}, then these default
% arguments are passed to any processor.  When referring to arguments
% (through |#1|, |#2| and so on) in a processor, the arguments used are
% always those before applying any processor.
%
% \begin{variable}{\ProcessedArgument}
%   \pkg{xparse} defines a very small set of processor functions. In the
%   main, it is anticipated that code writers will want to create their
%   own processors. These need to accept one argument, which is the
%   tokens as grabbed (or as returned by a previous processor function).
%   Processor functions should return the processed argument as the
%   variable \cs{ProcessedArgument}.
% \end{variable}
%
% \begin{function}{\ReverseBoolean}
%   \begin{syntax}
%     \cs{ReverseBoolean}
%   \end{syntax}
%   This processor reverses the logic of \cs{BooleanTrue} and
%   \cs{BooleanFalse}, so that the example from earlier would become
%   \begin{verbatim}
%     \NewDocumentCommand \foo { > { \ReverseBoolean } s m }
%       {
%         \IfBooleanTF #1
%           { \DoSomethingWithoutStar {#2} }
%           { \DoSomethingWithStar {#2} }
%       }
%   \end{verbatim}
% \end{function}
%
% \begin{function}[updated = 2012-02-12]{\SplitArgument}
%   \begin{syntax}
%     \cs{SplitArgument} \Arg{number} \Arg{token(s)}
%   \end{syntax}
%   This processor splits the argument given at each occurrence of the
%   \meta{tokens} up to a maximum of \meta{number} tokens (thus
%   dividing the input into $\text{\meta{number}} + 1$ parts).
%   An error is given if too many \meta{tokens} are present in the
%   input. The processed input is placed inside
%   $\text{\meta{number}} + 1$ sets of braces for further use.
%   If there are fewer than \Arg{number} of \Arg{tokens} in the argument
%   then |-NoValue-| markers are added at the end of the processed
%   argument.
%   \begin{verbatim}
%     \NewDocumentCommand \foo
%       { > { \SplitArgument { 2 } { ; } } m }
%       { \InternalFunctionOfThreeArguments #1 }
%   \end{verbatim}
%   If only a single character \meta{token} is used for the split, any
%   category code $13$ (active) character matching the \meta{token} will
%   be replaced before the split takes place.
%   Spaces are trimmed at each end of each item parsed.
% \end{function}
%
% \begin{function}{\SplitList}
%   \begin{syntax}
%     \cs{SplitList} \Arg{token(s)}
%   \end{syntax}
%   This processor splits the argument given at each occurrence of the
%   \meta{token(s)} where the number of items is not fixed. Each item is
%   then wrapped in braces within |#1|. The result is that the
%   processed argument can be further processed using a mapping function.
%   \begin{verbatim}
%     \NewDocumentCommand \foo
%       { > { \SplitList { ; } } m }
%       { \MappingFunction #1 }
%   \end{verbatim}
%   If only a single character \meta{token} is used for the split, any
%   category code $13$ (active) character matching the \meta{token} will
%   be replaced before the split takes place.
%   Spaces are trimmed at each end of each item parsed.
% \end{function}
%
% \begin{function}[EXP]{\ProcessList}
%   \begin{syntax}
%     \cs{ProcessList} \Arg{list} \Arg{function}
%   \end{syntax}
%   To support \cs{SplitList}, the function \cs{ProcessList} is available
%   to apply a \meta{function} to every entry in a \meta{list}. The
%   \meta{function} should absorb one argument: the list entry. For example
%   \begin{verbatim}
%     \NewDocumentCommand \foo
%       { > { \SplitList { ; } } m }
%       { \ProcessList {#1} { \SomeDocumentFunction } }
%   \end{verbatim}
%
%   \textbf{This function is experimental.}
% \end{function}
%
% \begin{function}{\TrimSpaces}
%   \begin{syntax}
%     \cs{TrimSpaces}
%   \end{syntax}
%   Removes any leading and trailing spaces (tokens with character code~$32$
%   and category code~$10$) for the ends of the argument. Thus for example
%   declaring a function
%   \begin{verbatim}
%     \NewDocumentCommand \foo
%       { > { \TrimSpaces } m }
%       { \showtokens {#1} }
%   \end{verbatim}
%   and using it in a document as
%   \begin{verbatim}
%     \foo{ hello world }
%   \end{verbatim}
%   will show \texttt{hello world} at the terminal, with the space at each
%   end removed. \cs{TrimSpaces} will remove multiple spaces from the ends of
%   the input in cases where these have been included such that the standard
%   \TeX{} conversion of multiple spaces to a single space does not apply.
%
%   \textbf{This function is experimental.}
% \end{function}
%
% \subsection{Fully-expandable document commands}
%
% There are \emph{very rare} occasion when it may be useful to create
% functions using a fully-expandable argument grabber. To support this,
% \pkg{xparse} can create expandable functions as well as the usual
% robust ones. This imposes a number of restrictions on the nature of
% the arguments accepted by a function, and the code it implements.
% This facility should only be used when \emph{absolutely necessary};
% if you do not understand when this might be, \emph{do not use these
% functions}!
%
% \begin{function}
%   {
%     \NewExpandableDocumentCommand     ,
%     \RenewExpandableDocumentCommand   ,
%     \ProvideExpandableDocumentCommand ,
%     \DeclareExpandableDocumentCommand
%   }
%   \begin{syntax}
%     \cs{NewExpandableDocumentCommand}
%     ~~~~\meta{function} \Arg{arg spec} \Arg{code}
%   \end{syntax}
%   This family of commands is used to create a document-level \meta{function},
%   which will grab its arguments in a fully-expandable manner. The
%   argument specification for the function is given by \meta{arg spec},
%   and the function will execute \meta{code}. In  general, \meta{code} will
%   also be fully expandable, although it is possible that this will
%   not be the case (for example, a function for use in a table might
%   expand so that \cs{omit} is the first non-expandable non-space token).
%
%   Parsing arguments expandably imposes a number of restrictions on
%   both the type of arguments that can be read and the error checking
%   available:
%   \begin{itemize}
%     \item The last argument (if any are present) must be one of the
%       mandatory types \texttt{m}, \texttt{r}, \texttt{R}, \texttt{l}
%       or~\texttt{u}.
%     \item All short arguments appear before long arguments.
%     \item The mandatory argument types \texttt{l} and \texttt{u} may
%       not be used after optional arguments.
%     \item The optional argument types \texttt{g}
%       and \texttt{G} are not available.
%     \item The \enquote{verbatim} argument type \texttt{v} is not available.
%     \item Argument processors (using \texttt{>}) are not available.
%     \item It is not possible to differentiate between, for example
%       |\foo[| and |\foo{[}|: in both cases the \texttt{[} will be
%       interpreted as the start of an optional argument. As a
%       result, checking for optional arguments is less robust than
%       in the standard version.
%   \end{itemize}
%   \pkg{xparse} will issue an error if an argument specifier is given
%   which does not conform to the first six requirements. The last
%   item is an issue when the function is used, and so is beyond the
%   scope of \pkg{xparse} itself.
% \end{function}
%
% \subsection{Access to the argument specification}
%
% The argument specifications for document commands and environments are
% available for examination and use.
%
% \begin{function}{\GetDocumentCommandArgSpec, \GetDocumentEnvironmentArgSpec}
%   \begin{syntax}
%     \cs{GetDocumentCommandArgSpec} \meta{function}
%     \cs{GetDocumentEnvironmentArgSpec} \Arg{environment}
%   \end{syntax}
%   These functions transfer the current argument specification for the
%   requested \meta{function} or \meta{environment} into the token list
%   variable \cs{ArgumentSpecification}. If the \meta{function} or
%   \meta{environment} has no known argument specification then an error
%   is issued. The assignment to \cs{ArgumentSpecification} is local to
%   the current \TeX{} group.
% \end{function}
%
% \begin{function}
%   {\ShowDocumentCommandArgSpec,  \ShowDocumentEnvironmentArgSpec}
%   \begin{syntax}
%     \cs{ShowDocumentCommandArgSpec} \meta{function}
%     \cs{ShowDocumentEnvironmentArgSpec} \Arg{environment}
%   \end{syntax}
%   These functions show the current argument specification for the
%   requested \meta{function} or \meta{environment} at the terminal. If
%   the \meta{function} or \meta{environment} has no known argument
%   specification then an error is issued.
% \end{function}
%
% \section{Load-time options}
%
% \DescribeOption{log-declarations}
% The package recognises the load-time option \texttt{log-declarations},
% which is a key--value option taking the value \texttt{true} and
% \texttt{false}. By default, the option is set to \texttt{false}, meaning
% that no command or environment declared is logged. By loading
% \pkg{xparse} using
% \begin{verbatim}
%   \usepackage[log-declarations=true]{xparse}
% \end{verbatim}
% each new, declared or renewed command or environment is logged.
%
% \end{documentation}
%
% \begin{implementation}
%
% \section{\pkg{xparse} implementation}
%
%    \begin{macrocode}
%<*2ekernel|package>
%    \end{macrocode}
%
%    \begin{macrocode}
%<@@=cmd>
%    \end{macrocode}
%
% This package file, along with the frozen versions in
% \pkg{xparse-2018-04-12}, \pkg{xparse-2020-10-01}, and
% |xparse-generic.tex|, is intended to work across different \LaTeX{}
% releases, respected the minimum \pkg{expl3} version required by
% \pkg{xparse}.
%
% We can't use the \pkg{latexrelease} mechanism with \cs{DeclareRelease}
% here because cases of rolling forwards or backwards differ, due to the
% prefix change from |xparse| to |cmd|, so we do our own checks to
% ensure the right version is loaded.
%
% All this loading assumes that if \pkg{latexrelease} should be loaded,
% it is before \pkg{xparse}.  The other way around was not tested at all
% and is likely to break.
%
% In releases prior to 2020-10-01, \cs{NewDocumentCommand} is undefined,
% so \pkg{xparse} was not loaded, thus we can load the full version from
% \pkg{xparse-2018-04-12}.  Otherwise we will load only the deprecated
% argument types from \pkg{xparse-2020-10-01}.  If we're in |2ekernel|
% mode, it means rolling forward so load only the code code from
% |xparse-generic.tex|.
%
% In |2ekernel| mode we anticipate the definition of two macros to parse
% dates so that we can compare \cs{fmtdate} properly.
%    \begin{macrocode}
%<*2ekernel>
\def\@parse@version#1/#2/#3#4#5\@nil{%
  \@parse@version@dash#1-#2-#3#4\@nil}
\def\@parse@version@dash#1-#2-#3#4#5\@nil{%
  \if\relax#2\relax\else#1\fi#2#3#4 }
%</2ekernel>
\ExplSyntaxOn
\cs_set_protected:Npn \@@_tmp:w #1
  {
    \DeclareOption* { \PassOptionsToPackage { \CurrentOption } {#1} }
    \ProcessOptions \relax
    \RequirePackage {#1}
  }
\cs_if_free:NTF \NewDocumentCommand
  {
    \ExplSyntaxOff
    \ifnum\expandafter
        \@parse@version\fmtversion//00\@nil <
        \@parse@version 2020-10-01//00\@nil
      \@@_tmp:w { xparse-2018-04-12 }
    \else
%<2ekernel>      \@@@@input xparse-generic.tex ~
%<package>      \@@_tmp:w { xparse-2020-10-01 }
    \fi
    \file_input_stop:
  }
%    \begin{macrocode}
%
% In case \cs{NewDocumentCommand} is already defined, we're either in
% \LaTeX{} 2020-10-01 or later.  In the former case, the code loaded in
% the kernel has the |__xparse| prefix, so we'll load
% \pkg{xparse-2020-10-01}, otherwise we'll continue in |xparse.sty|,
% which contains the final remains of \pkg{xparse} with the |__cmd|
% prefix.  To check that we simply look at an internal command with the
% |__cmd| prefix.
%    \end{macrocode}
  {
    \ExplSyntaxOff
    \cs_if_exist:NF \@@_start:nNNnnn
      {
        \@@_tmp:w { xparse-2020-10-01 }
        \file_input_stop:
      }
  }
\ExplSyntaxOff
%    \end{macrocode}
%
%    \begin{macrocode}
%</2ekernel|package>
%    \end{macrocode}
%
% In older releases, the prefix was |xparse|, but since the 2021 Spring
% release of \LaTeXe, the core code is included in the kernel, and this
% file only holds the deprecated argument specifiers |G|, |l|, and |u|.
% To match the prefix in the \LaTeXe{} kernel, so that the deprecated
% types work if |xparse.sty| is loaded, the prefix has changed to |cmd|.
%
%    \begin{macrocode}
%<*package>
%    \end{macrocode}
%
%    \begin{macrocode}
\ProvidesExplPackage{xparse}{2021-11-12}{}
  {L3 Experimental document command parser}
%    \end{macrocode}
%
% \subsection{Package options}
%
% \begin{variable}{\l_@@_options_clist}
% \begin{variable}{\l_@@_log_bool}
%   Key--value option to log information: done by hand to keep dependencies
%   down.
%    \begin{macrocode}
\clist_new:N \l_@@_options_clist
\DeclareOption* { \clist_put_right:NV \l_@@_options_clist \CurrentOption }
\ProcessOptions \relax
\cs_set_protected:Npn \@@_tmp:w #1
  {
    \keys_define:nn {#1}
      {
        log-declarations .bool_set:N = \l_@@_log_bool ,
        log-declarations .initial:n  = false
      }
    \keys_set:nV {#1} \l_@@_options_clist
    \bool_if:NF \l_@@_log_bool
      { \msg_redirect_module:nnn {#1} { info } { none } }
    \cs_new_protected:Npn \@@_unknown_argument_type_error:n ##1
      {
        \msg_error:nnxx {#1} { unknown-argument-type }
          { \@@_environment_or_command: } { \tl_to_str:n {##1} }
      }
  }
\msg_if_exist:nnTF { cmd } { define-command }
  { \@@_tmp:w { cmd } }
  { \@@_tmp:w { ltcmd } }
%    \end{macrocode}
% \end{variable}
% \end{variable}
%
% \subsection{Normalizing the argument specifications}
%
% \begin{macro}{\@@_normalize_arg_spec_loop:n}
%   Loop through the argument specification, calling an auxiliary
%   specific to each argument type.  If any argument is unknown stop the
%   definition.
%    \begin{macrocode}
\cs_gset_protected:Npn \@@_normalize_arg_spec_loop:n #1
  {
    \quark_if_recursion_tail_stop:n {#1}
    \int_incr:N \l_@@_current_arg_int
    \cs_if_exist_use:cF { @@_normalize_type_ \tl_to_str:n {#1} :w }
      {
        \@@_unknown_argument_type_error:n {#1}
        \@@_bad_def:wn
      }
  }
%    \end{macrocode}
% \end{macro}
%
% \begin{macro}{\@@_normalize_type_g:w}
%   These argument types are aliases of more general ones, for example
%   with the default argument |-NoValue-|.
%    \begin{macrocode}
\cs_new_protected:Npx \@@_normalize_type_g:w
  { \exp_not:N \@@_normalize_type_G:w { \exp_not:V \c_novalue_tl } }
%    \end{macrocode}
% \end{macro}
%
% \begin{macro}{\@@_normalize_type_G:w}
%   Optional argument types.  Check that all required data is present
%   (and consists of single characters if applicable) and check for
%   forbidden types for expandable commands.  Then in each case
%   store the data in \cs{l_@@_arg_spec_tl}, and
%   for later checks store in \cs{l_@@_last_delimiters_tl} the tokens
%   whose presence determines whether there is an optional argument (for
%   braces store |{}|, seen later as an empty delimiter).
%    \begin{macrocode}
\cs_new_protected:Npn \@@_normalize_type_G:w #1
  {
    \quark_if_recursion_tail_stop_do:nn {#1} { \@@_bad_arg_spec:wn }
    \@@_normalize_check_gv:N G
    \@@_add_arg_spec:n { G {#1} }
    \tl_put_right:Nn \l_@@_last_delimiters_tl { { } }
    \@@_normalize_arg_spec_loop:n
  }
%    \end{macrocode}
% \end{macro}
%
% \begin{macro}{\@@_normalize_type_l:w,\@@_normalize_type_u:w}
%   Mandatory arguments.  First check the required data is present,
%   consists of single characters where applicable, and that the argument
%   type is allowed for expandable commands if applicable.  Then save data in
%   \cs{l_@@_arg_spec_tl}, count the mandatory argument, and empty the
%   list of last delimiters.
%    \begin{macrocode}
\cs_new_protected:Npn \@@_normalize_type_l:w
  {
    \@@_normalize_check_lu:N l
    \@@_add_arg_spec_mandatory:n { l }
    \@@_normalize_arg_spec_loop:n
  }
\cs_new_protected:Npn \@@_normalize_type_u:w #1
  {
    \quark_if_recursion_tail_stop_do:nn {#1} { \@@_bad_arg_spec:wn }
    \@@_normalize_check_lu:N u
    \@@_add_arg_spec_mandatory:n { u {#1} }
    \@@_normalize_arg_spec_loop:n
  }
%    \end{macrocode}
% \end{macro}
%
% \subsection{Setting up a standard signature}
%
% \begin{macro}{\@@_add_type_G:w}
%   For the \texttt{G} type, the grabber and the default are added to the
%   signature.
%    \begin{macrocode}
\cs_new_protected:Npn \@@_add_type_G:w #1
  {
    \@@_flush_m_args:
    \@@_add_default:n {#1}
    \@@_add_grabber:N G
    \@@_prepare_signature:N
  }
%    \end{macrocode}
% \end{macro}
%
% \begin{macro}{\@@_add_type_l:w}
%   Finding \texttt{l} arguments is very simple: there is nothing to do
%   other than add the grabber.
%    \begin{macrocode}
\cs_new_protected:Npn \@@_add_type_l:w
  {
    \@@_flush_m_args:
    \@@_add_default:
    \@@_add_grabber:N l
    \@@_prepare_signature:N
  }
%    \end{macrocode}
% \end{macro}
%
% \begin{macro}{\@@_add_type_u:w}
%   At the set up stage, the \texttt{u} type argument is identical to the
%   \texttt{G} type except for the name of the grabber function.
%    \begin{macrocode}
\cs_new_protected:Npn \@@_add_type_u:w #1
  {
    \@@_flush_m_args:
    \@@_add_default:
    \@@_add_grabber:N u
    \tl_put_right:Nn \l_@@_signature_tl { {#1} }
    \@@_prepare_signature:N
  }
%    \end{macrocode}
% \end{macro}
%
% \subsection{Setting up expandable types}
%
% \begin{macro}{\@@_add_expandable_type_l:w}
%   Reuse type \texttt{u}, thanks to the fact that \TeX{} macros whose
%   parameter text ends with |#| in fact end up being delimited by an
%   open brace.
%    \begin{macrocode}
\cs_new_protected:Npn \@@_add_expandable_type_l:w
  { \@@_add_expandable_type_u:w ## }
%    \end{macrocode}
% \end{macro}
%
% \begin{macro}{\@@_add_expandable_type_u:w}
%   Define an auxiliary that will be used directly in the signature.  It
%   grabs one argument delimited by |#1| and places it before \cs{q_@@}.
%    \begin{macrocode}
\cs_new_protected:Npn \@@_add_expandable_type_u:w #1
  {
    \@@_add_default:
    \bool_if:NTF \l_@@_long_bool
      { \cs_set:cpn }
      { \cs_set_nopar:cpn }
      { \l_@@_expandable_aux_name_tl } ##1 \q_@@ ##2 ##3 ##4 #1
      { ##1 {##4} \q_@@ ##2 ##3 }
    \@@_add_expandable_grabber:nn { u }
      { \exp_not:c  { \l_@@_expandable_aux_name_tl } }
    \@@_prepare_signature:N
  }
%    \end{macrocode}
% \end{macro}
%
% \subsection{Copying a command and its internal structure}
%
% The apparatus for copying commands is almost entirely in |ltcmd.dtx|,
% preloaded in the \LaTeXe{} kernel.  The missing parts, regarding
% copying the deprecated argument types |G|, |l| and |u| boil down to
% copying the expandable grabber for |u|.
%
% \begin{macro}{\@@_copy_grabber_u:w}
%   An expandable |u|-type uses a dedicated grabber just like a
%   |D|-type, except that both its delimiter tokens are omitted, so to
%   copy that we just copy a |D|-type and leave the last two arguments
%   empty:
%    \begin{macrocode}
\cs_new_protected:Npn \@@_copy_grabber_u:w #1 #2 #3
  { \@@_copy_grabber_D:w {#1} {#2} {#3} { } { } }
%    \end{macrocode}
% \end{macro}
%
% \subsubsection{Showing the definition of a command}
%
% \begin{macro}{\c_@@_show_type_u_tl,\c_@@_show_type_G_tl}
%   Same as for copying, only small bits are missing here.  Namely, two
%   token lists that tell the \pkg{ltcmd} mechanism how to deal with the
%   argument types defined here.  Both |G| and |u| are classified
%   as~|3|: commands that take a default value.  That's not really true
%   for~|u|, but it's a close enough approximation to get the output we
%   want.
%    \begin{macrocode}
\tl_const:Nn \c_@@_show_type_u_tl { 3 }
\tl_const:Nn \c_@@_show_type_G_tl { 3 }
%    \end{macrocode}
% \end{macro}
%
% \subsection{Grabbing arguments}
%
% \begin{macro}{\@@_grab_G:w}
% \begin{macro}{\@@_grab_G_long:w}
% \begin{macro}{\@@_grab_G_obey_spaces:w}
% \begin{macro}{\@@_grab_G_long_obey_spaces:w}
% \begin{macro}{\@@_grab_G_aux:nNN}
%   Optional groups are checked by meaning, so that the same code will
%   work with, for example, \ConTeXt{}-like input.
%    \begin{macrocode}
\cs_new_protected:Npn \@@_grab_G:w #1 \@@_run_code:
  {
    \@@_grab_G_aux:nNN {#1} \cs_set_protected_nopar:Npn
      \@@_peek_nonspace:NTF
  }
\cs_new_protected:Npn \@@_grab_G_long:w #1 \@@_run_code:
  {
    \@@_grab_G_aux:nNN {#1} \cs_set_protected:Npn
      \@@_peek_nonspace:NTF
  }
\cs_new_protected:Npn \@@_grab_G_obey_spaces:w #1 \@@_run_code:
  {
    \@@_grab_G_aux:nNN {#1} \cs_set_protected_nopar:Npn
      \peek_meaning:NTF
  }
\cs_new_protected:Npn \@@_grab_G_long_obey_spaces:w #1 \@@_run_code:
  {
    \@@_grab_G_aux:nNN {#1} \cs_set_protected:Npn
      \peek_meaning:NTF
  }
\cs_new_protected:Npn \@@_grab_G_aux:nNN #1#2#3
  {
    \tl_set:Nn \l_@@_signature_tl {#1}
    \exp_after:wN #2 \l_@@_fn_tl ##1
      { \@@_add_arg:n {##1} }
    #3 \c_group_begin_token
      { \l_@@_fn_tl }
      { \@@_add_arg:o \c_novalue_tl }
  }
%    \end{macrocode}
% \end{macro}
% \end{macro}
% \end{macro}
% \end{macro}
% \end{macro}
%
% \begin{macro}{\@@_grab_l:w}
% \begin{macro}{\@@_grab_l_long:w}
% \begin{macro}{\@@_grab_l_aux:nN}
%   Argument grabbers for mandatory \TeX{} arguments are pretty simple.
%    \begin{macrocode}
\cs_new_protected:Npn \@@_grab_l:w #1 \@@_run_code:
  { \@@_grab_l_aux:nN {#1} \cs_set_protected_nopar:Npn }
\cs_new_protected:Npn \@@_grab_l_long:w #1 \@@_run_code:
  { \@@_grab_l_aux:nN {#1} \cs_set_protected:Npn }
\cs_new_protected:Npn \@@_grab_l_aux:nN #1#2
  {
    \tl_set:Nn \l_@@_signature_tl {#1}
    \exp_after:wN #2 \l_@@_fn_tl ##1##
      { \@@_add_arg:n {##1} }
    \l_@@_fn_tl
  }
%    \end{macrocode}
% \end{macro}
% \end{macro}
% \end{macro}
%
% \begin{macro}{\@@_grab_u:w}
% \begin{macro}{\@@_grab_u_long:w}
% \begin{macro}{\@@_grab_u_aux:nnN}
%   Grabbing up to a list of tokens is quite easy: define the grabber,
%   and then collect.
%    \begin{macrocode}
\cs_new_protected:Npn \@@_grab_u:w #1#2 \@@_run_code:
  { \@@_grab_u_aux:nnN {#1} {#2} \cs_set_protected_nopar:Npn }
\cs_new_protected:Npn \@@_grab_u_long:w #1#2 \@@_run_code:
  { \@@_grab_u_aux:nnN {#1} {#2} \cs_set_protected:Npn }
\cs_new_protected:Npn \@@_grab_u_aux:nnN #1#2#3
  {
    \tl_set:Nn \l_@@_signature_tl {#2}
    \exp_after:wN #3 \l_@@_fn_tl ##1 #1
      { \@@_add_arg:n {##1} }
    \l_@@_fn_tl
  }
%    \end{macrocode}
% \end{macro}
% \end{macro}
% \end{macro}
%
% \begin{macro}[EXP]{\@@_expandable_grab_u:w}
%   It turns out there is nothing to do: this is followed by an
%   auxiliary named after the function, that does everything.
%    \begin{macrocode}
\cs_new_eq:NN \@@_expandable_grab_u:w \prg_do_nothing:
%    \end{macrocode}
% \end{macro}
%
%    \begin{macrocode}
%</package>
%    \end{macrocode}
%
% \end{implementation}
%
% \PrintIndex
