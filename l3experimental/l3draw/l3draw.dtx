% \iffalse meta-comment
%
%% File: l3draw.dtx
%
% Copyright(C) 2018,2019 The LaTeX3 Project
%
% It may be distributed and/or modified under the conditions of the
% LaTeX Project Public License (LPPL), either version 1.3c of this
% license or (at your option) any later version.  The latest version
% of this license is in the file
%
%    http://www.latex-project.org/lppl.txt
%
% This file is part of the "l3experimental bundle" (The Work in LPPL)
% and all files in that bundle must be distributed together.
%
% -----------------------------------------------------------------------
%
% The development version of the bundle can be found at
%
%    https://github.com/latex3/latex3
%
% for those people who are interested.
%
%<*driver|package>
\RequirePackage{expl3}
%</driver|package>
%<*driver>
\documentclass[full]{l3doc}
\usepackage{l3draw}
% For creating code demonstrations
% This needs access to some code-level interfaces in listings
\usepackage{listings}
\makeatletter
\lst@RequireAspects{writefile}
\newsavebox\demo@box
\lstnewenvironment{demo}[1][code and example]
  {%
    \global\let\lst@intname\@empty
    \edef\demo@end{%
      \expandafter\noexpand\csname demo@@#1@end\endcsname
    }%
    \@nameuse{demo@@#1}%
  }
  {\demo@end}
\newcommand\demo@new[3]{%
  \@namedef{demo@@#1}{#2}%
  \@namedef{demo@@#1@end}{#3}%
}
\newcommand*\demo@common{%
  \setkeys{lst}
    {%
       basicstyle   = \small\ttfamily,
       basewidth    = 0.51em,
       gobble       = 5,
       language     = [LaTeX]{TeX},
    }%
}
\newcommand*\demo@input{%
  \ExplSyntaxOn
  \catcode`\^^M = 10\relax
  \catcode`\% = 14\relax
  \input{\jobname.tmp}%
  \ExplSyntaxOff
}
\demo@new{code and example}{%
  \setbox\demo@box=\hbox\bgroup
    \lst@BeginAlsoWriteFile{\jobname.tmp}%
    \demo@common
}{%
    \lst@EndWriteFile
  \egroup
  \begin{center}
    \ifdim\wd\demo@box > 0.75\linewidth
      \begin{minipage}{\linewidth}
        \usebox\demo@box
      \end{minipage}%
      \par
      \begin{minipage}{\linewidth}
        \demo@input
      \end{minipage}
    \else
      \begin{minipage}{0.25\linewidth}
        \demo@input
      \end{minipage}%
      \hfil
      \begin{minipage}{0.75\linewidth}
        \usebox\demo@box
      \end{minipage}%
    \fi
  \end{center}
}
\makeatother
\begin{document}
  \DocInput{\jobname.dtx}
\end{document}
%</driver>
% \fi
%
% \title{^^A
%   The \pkg{l3draw} package\\ Core drawing support^^A
% }
%
% \author{^^A
%  The \LaTeX3 Project\thanks
%    {^^A
%      E-mail:
%        \href{mailto:latex-team@latex-project.org}
%          {latex-team@latex-project.org}^^A
%    }^^A
% }
%
% \date{Released 2019-10-11}
%
% \maketitle
%
% \begin{documentation}
%
% \section{\pkg{l3draw} documentation}
%
% The \pkg{l3draw} package provides a set of tools for creating (vector)
% drawings in \pkg{expl3}. It is heavily inspired by the \pkg{pgf} layer of
% the Ti\textit{k}Z system, with many of the interfaces having the same form.
% However, the code provided here is build entirely on core \pkg{expl3} ideas
% and uses the \LaTeX3 FPU for numerical support.
%
% Numerical expressions in \pkg{l3draw} are handled as floating point
% expressions, unless otherwise noted. This means that they may contain or
% omit explicit units. Where units are omitted, they will automatically be
% taken as given in (\TeX{}) points.
%
% The code here is \emph{highly} experimental.
%
% \subsection{Drawings}
%
% \begin{function}{\draw_begin:, \draw_end:}
%   \begin{syntax}
%     \cs{draw_begin:}
%     ...
%     \cs{draw_end:}
%   \end{syntax}
%   Each drawing should be created within a \cs{draw_begin:}/\cs{draw_end:}
%   function pair. The \texttt{begin} function sets up a number of key
%   data structures for the rest of the functions here: unless otherwise
%   specified, use of |\draw_...| functions outside of this
%   \enquote{environment} is \emph{not supported}.
%
%   The drawing created within the environment will be inserted into
%   the typesetting stream by the \cs{draw_end:} function, which will
%   switch out of vertical mode if required.
%   \begin{demo}
%     \dim_new:N \l_mypos_dim
%     \draw_begin:
%       \draw_path_moveto:n { 0cm , \l_mypos_dim }
%       \draw_path_lineto:n { 1cm , \l_mypos_dim }
%       \dim_set:Nn \l_mypos_dim { 1cm }
%       \draw_path_lineto:n { 1cm , \l_mypos_dim }
%       \draw_path_close:
%       \draw_path_use_clear:n { stroke }
%     \draw_end:
%   \end{demo}
%
%   Within a drawing, the only commands that should appear are those directly
%   aimed at drawing (from \pkg{l3draw}) and those which produce \emph{no}
%   typeset output. For example, it is possible to include loops inside a
%   drawing using |\int_step_function:nnnn| or similar. On the other hand,
%   text should not be included directly in drawings, but should rather be
%   inserted using the appropriate \pkg{l3draw} command.
%
%   The drawing environment sets the following standard behaviors
%   \begin{itemize}
%      \item Non-zero rule for fill overlaps
%      \item Butt caps for lines
%      \item Mitering for line joins with a miter factor of $10$
%      \item Solid line strokes
%   \end{itemize}
% \end{function}
%
% Within a drawing, there are different ways of referring to a position.
% The co-ordinates of a point are given relative to the current
% \emph{drawing axes}. These can be manipulated and tracked at the code
% level. Underlying this is the \meta{canvas}, which is at the \pkg{draw}
% level essentially fixed and in line with the paper. Initially, the two
% sets of axes are coincident. (It is possible to manipulate the canvas
% axes at the driver level: this is then \enquote{transparent} to the
% \pkg{draw} level, and so should be used only when strictly required.)
%
% \begin{function}{\draw_suspend_begin:, \draw_suspend_end:}
%   \begin{syntax}
%     \cs{draw_suspend_begin:}
%     ...
%     \cs{draw_suspend_end:}
%   \end{syntax}
%   Suspends all of the drawing mechanisms to allow \enquote{normal}
%   material to be created. Typically, this environment will be applied
%   inside a box which may contain nested pictures.
%   \begin{demo}
%     \draw_begin:
%       \draw_path_moveto:n { 0cm , 0cm }
%       \hbox_set:Nn \l_tmpa_box
%         {
%           \draw_suspend_begin:
%             This~is~normal~text.
%             \draw_begin: % A subpicture
%               \draw_path_moveto:n { 1cm , 0cm }
%               \draw_path_lineto:n { 1cm , 1cm }
%               \draw_path_use_clear:n { stroke }
%             \draw_end:
%             More~text.
%           \draw_suspend_end:
%         }
%       \draw_box_use:N \l_tmpa_box
%       \draw_path_lineto:n { 0cm , 1cm }
%       \draw_path_use_clear:n { stroke }
%     \draw_end:
%   \end{demo}
% \end{function}
%
% \subsection{Graphics state}
%
% Within the drawing environment, a number of functions control how drawings
% will appear. Note that these all apply \emph{globally}, though some are
% rest at the start of each drawing (\cs{draw_begin:}).
%
% \begin{variable}{\g_draw_linewidth_default_dim}
%   The default value of the linewidth for stokes, set at the start
%   of every drawing (\cs{draw_begin:}).
% \end{variable}
%
% \begin{function}{\draw_linewidth:n}
%   \begin{syntax}
%     \cs{draw_linewidth:n} \Arg{width}
%   \end{syntax}
%   Sets the width to be used for stroking to the \meta{width} (an 
%   \meta{fp expr}).
% \end{function}
%
% \begin{function}{\draw_dash_pattern:nn}
%   \begin{syntax}
%     \cs{draw_dash_pattern:nn} \Arg{pattern} \Arg{phase}
%   \end{syntax}
%   Specifies a dash pattern. The \meta{pattern} itself is a comma-separated
%   list of entries which represent the \enquote{on} and \enquote{off}
%   parts of the line. These are all \meta{fp expr} and repeat as required.
%   Thus the \meta{pattern} may be of arbitrary length. The \meta{phase}
%   specifies where during the first \enquote{on} line the pattern should
%   start.
%   \begin{demo}
%     \draw_begin:
%       \draw_dash_pattern:nn
%         { 0.5cm , 0.5cm , 0.1cm , 0.2cm }
%         { 0cm }
%       \draw_path_moveto:n { 0cm , 0cm }
%       \draw_path_lineto:n { 2cm , 0cm }
%       \draw_path_use_clear:n { stroke }
%       \draw_dash_pattern:nn
%         { 0.5cm , 0.5cm , 0.1cm , 0.2cm }
%         { 0.1cm }
%       \draw_path_moveto:n { 0cm , 1mm }
%       \draw_path_lineto:n { 2cm , 1mm }
%       \draw_path_use_clear:n { stroke }
%       \draw_dash_pattern:nn
%         { 0.5cm , 0.5cm , 0.1cm , 0.2cm }
%         { 0.2cm }
%       \draw_path_moveto:n { 0cm , 2mm }
%       \draw_path_lineto:n { 2cm , 2mm }
%       \draw_path_use_clear:n { stroke }
%     \draw_end:
%   \end{demo}
%   Setting an empty pattern will produce a solid line.
%   
%   Note the \meta{pattern} interface here is different from that in \pkg{pgf}:
%   the list is comma-separated not given in brace groups.
% \end{function}
%
% \begin{function}{\draw_nonzero_rule:, \draw_evenodd_rule:}
%   \begin{syntax}
%     \cs{draw_nonzero_rule:}
%   \end{syntax}
%   Active either the non-zero winding number or the even-odd rule,
%   respectively, for determining what is inside a fill or clip area.
%   For technical reasons, these command are not influenced by scoping
%   and apply on an ongoing basis.
% \end{function}
%
% \begin{function}
%   {
%     \draw_cap_butt:      ,
%     \draw_cap_rectangle: ,
%     \draw_cap_round:
%   }
%   \begin{syntax}
%     \cs{draw_cap_butt:}
%   \end{syntax}
%   Sets the style of terminal stroke position to one of butt, rectangle or
%   round.
% \end{function}
%
% \begin{function}
%   {
%     \draw_join_bevel: ,
%     \draw_join_miter: ,
%     \draw_join_round:
%   }
%   \begin{syntax}
%     \cs{draw_cap_butt:}
%   \end{syntax}
%   Sets the style of stroke joins to one of bevel, miter or round.
% \end{function}
%
% \begin{function}{\draw_miterlimit:n}
%   \begin{syntax}
%     \cs{draw_miterlimit:n} \Arg{factor}
%   \end{syntax}
%   Sets the miter \meta{factor} of lines joined as a miter, as described in the
%   PDF and PostScript manuals. The \meta{factor} is an \meta{fp expr}.
% \end{function}
%
% \subsection{Scoping drawing elements}
%
% Scoping drawing elements is necessary to allowing nesting of subparts.
% These features have specific use requirements: the preconditions must be
% met. In particular, each type of drawing scope also constitutes a
% group (\cs{group_begin:}/\cs{group_end:} pair): as such, they must be
% nested correctly.
%
% \begin{function}{\draw_scope_begin:, \draw_scope_end:}
%   \begin{syntax}
%     \cs{draw_scope_begin:}
%     ...
%     \cs{draw_scope_end:}
%   \end{syntax}
%   Creates a scope for localisation of state settings within a drawing.
%   A scope forms a \TeX{} group but will also localise global state
%   variables (such as \cs{g_draw_linewidth_default_dim}), and driver-level
%   concepts such as the termination of lines.
%   \begin{demo}
%     \draw_begin:
%       \draw_scope_begin:
%         \group_begin:
%           \draw_linewidth:n { 2pt }
%           \draw_path_rectangle:nn { 0 , 0 } { 2ex , 2ex }
%           \draw_path_use:n  { stroke }
%         \group_end:
%         \draw_path_rectangle:nn { 3ex , 0ex } { 2ex , 2ex }
%         \draw_path_use:n  { stroke }
%       \draw_scope_end:
%       \draw_path_rectangle:nn { 6ex , 0ex } { 2ex , 2ex }
%       \draw_path_use_clear:n  { stroke }
%     \draw_end:
%   \end{demo}
%   Global graphical concepts restricted by scope are
%   \begin{itemize}
%     \item Line width
%     \item Stroke and fill color
%     \item Dash pattern
%     \item Line joining and capping, including the miter limit
%     \item Clipping paths
%     \item Canvas (driver) transformations
%   \end{itemize}
% \end{function}
%
% \subsection{Points}
%
% Functions supporting the calculation of points (co-ordinates) are expandable
% and may be used outside of the drawing environment. When used in this
% way, they all yield a co-ordinate tuple, for example
% \begin{verbatim}
%   \tl_set:Nx \l_tmpa_tl { \draw_point:nn { 1 } { 2 } }
%   \tl_show:N \l_tmpa_tl
% \end{verbatim}
% gives
% \begin{verbatim}
%   > \l_tmpa_tl=1pt,2pt.
%   <recently read> }
% \end{verbatim}
%
% This output form is then suitable as \emph{input} for subsequent point
% calculations, \emph{i.e.}~where a \meta{point} is required it may be
% given as a tuple. This \emph{may} include units and surrounding
% parentheses, for example
% \begin{verbatim}
%   1,2
%   (1,2)
%   1cm,3pt
%   (1pt,2cm)
%   2 * sind(30), 2^4in
% \end{verbatim}
% are all valid input forms. Notice that each part of the tuple may itself
% be a float point expression.
%
% Point co-ordinates are relative to the canvas axes, but can be transformed
% by \cs{draw_point_transform:n}. These manipulation is applied by many
% higher-level functions, for example path construction, and allows parts of
% a drawing to be rotated, scaled or skewed. This occurs before writing any
% data to the driver, and so such manipulations are tracked by the drawing
% mechanisms. See \cs{@@_backend_cm:nnnn} for backend-level
% manipulation of the canvas axes themselves.
%
% Notice that in contrast to \pkg{pgf} it is possible to give the positions
% of points \emph{directly}.
%
% \subsubsection{Basic point functions}
%
% \begin{function}[EXP]{\draw_point_polar:nn, \draw_point_polar:nnn}
%   \begin{syntax}
%     \cs{draw_point_polar:nn} \Arg{radius} \Arg{angle}
%     \cs{draw_point_polar:nnn} \Arg{radius-a} \Arg{radius-b} \Arg{angle}
%   \end{syntax}
%   Gives the co-ordinates of the point at \meta{angle} (an \meta{fp expr} in
%   \emph{degrees}) and \meta{radius}. The three-argument version accepts
%   two radii of different lengths.
%   
%   Note the interface here is somewhat different from that in \pkg{pgf}:
%   the one- and two-radii versions in \pkg{l3draw} use separate functions,
%   whilst in \pkg{pgf} they use the same function and a keyword.
% \end{function}
%
% \begin{function}[EXP]{\draw_point_unit_vector:n}
%   \begin{syntax}
%     \cs{draw_point_unit_vector:n} \Arg{point}
%   \end{syntax}
%   Expands to the co-ordinates of a unit vector in the direction of the
%   \meta{point} from the origin. If the \meta{point} is at the origin,
%   a vertical unit vector is returned
% \end{function}
%
% \begin{function}[EXP]{\draw_point_transform:n}
%   \begin{syntax}
%     \cs{draw_point_transform:n} \Arg{point}
%   \end{syntax}
%   Evaluates the position of the \meta{point} subject to the current
%   transformation matrix. This operation is applied automatically by
%   most higher-level functions (\emph{e.g.}~path manipulations).
% \end{function}
%
% \subsubsection{Points on a vector basis}
%
% As well as giving explicit values, it is possible to describe points
% in terms of underlying direction vectors. The latter are initially
% co-incident with the standard Cartesian axes, but may be altered by
% the user.
%
% \begin{function}{\draw_xvec:n, \draw_yvec:n, \draw_zvec:n}
%   \begin{syntax}
%     \cs{draw_xvec:n} \Arg{point}
%   \end{syntax}
%   Defines the appropriate base vector to point toward the \meta{point}
%   on the canvas. The standard settings for the $x$- and $y$-vectors are
%   $1\,\mathrm{cm}$ along the relevant canvas axis, whilst for the
%   $z$-vector an appropriate direction is taken.
% \end{function}
%
% \begin{function}[EXP]{\draw_point_vec:nn, \draw_point_vec:nnn}
%   \begin{syntax}
%     \cs{draw_point_vec:nn} \Arg{xscale} \Arg{yscale}
%     \cs{draw_point_vec:nnn} \Arg{xscale} \Arg{yscale} \Arg{zscale}
%   \end{syntax}
%   Expands to the co-ordinate of the point at \meta{xscale} times the
%   $x$-vector and \meta{yscale} times the $y$-vector. The three-argument
%   version extends this to include the $z$-vector.
% \end{function}
%
% \begin{function}[EXP]{\draw_point_vec_polar:nn, \draw_point_vec_polar:nnn}
%   \begin{syntax}
%     \cs{draw_point_vec_polar:nn} \Arg{radius} \Arg{angle}
%     \cs{draw_point_vec_polar:nnn} \Arg{radius-a} \Arg{radius-b} \Arg{angle}
%   \end{syntax}
%   Gives the co-ordinates of the point at \meta{angle} (an \meta{fp expr} in
%   \emph{degrees}) and \meta{radius}, relative to the prevailing
%   $x$- and $y$-vectors. The three-argument version accepts two radii of
%   different lengths.
%   
%   Note the interface here is somewhat different from that in \pkg{pgf}:
%   the one- and two-radii versions in \pkg{l3draw} use separate functions,
%   whilst in \pkg{pgf} they use the same function and a keyword.
% \end{function}
%
% \subsubsection{Intersections}
%
% \begin{function}[EXP]{\draw_point_intersect_lines:nnnn}
%   \begin{syntax}
%     \cs{draw_point_intersect_lines:nnnn} \Arg{point1} \Arg{point2} \Arg{point3} \Arg{point4}
%   \end{syntax}
%   Evaluates the point at the intersection of one line, joining
%   \meta{point1} and \meta{point2}, and a second line joining \meta{point3}
%   and \meta{point4}. If the lines do not intersect, or are coincident, and
%   error will occur.
% \end{function}
%
% \begin{function}[EXP]{\draw_point_intersect_circles:nnnnn}
%   \begin{syntax}
%     \cs{draw_point_intersect_circles:nnnnn}
%       \Arg{center1} \Arg{radius1} \Arg{center2} \Arg{radius2} \Arg{root}
%   \end{syntax}
%   Evaluates the point at the intersection of one circle with
%   \meta{center1} and \meta{radius1}, and a second circle with \meta{center2}
%   and \meta{radius2}. If the circles do not intersect, or are coincident, and
%   error will occur.
%
%   Note the interface here has a different argument ordering from that in
%   \pkg{pgf}, which has the two centers then the two radii.
% \end{function}
%
% \subsubsection{Interpolations}
%
% \begin{function}[EXP]{\draw_point_interpolate_line:nnn}
%   \begin{syntax}
%     \cs{draw_point_interpolate_line:nnn} \Arg{part} \Arg{point1} \Arg{point2}
%   \end{syntax}
%   Expands to the point which is \meta{part} way along the line joining
%   \meta{point1} and \meta{point2}. The \meta{part} may be an interpolation or
%   an extrapolation, and is a floating point value expressing a percentage
%   along the line, \emph{e.g.}~a value of \texttt{0.5} would be half-way
%   between the two points.
% \end{function}
%
% \begin{function}[EXP]{\draw_point_interpolate_distance:nnn}
%   \begin{syntax}
%     \cs{draw_point_interpolate_distance:nnn} \Arg{distance} \Arg{point expr1} \Arg{point expr2}
%   \end{syntax}
%   Expands to the point which is \meta{distance} way along the line joining
%   \meta{point1} and \meta{point2}. The \meta{distance} may be an interpolation
%   or an extrapolation.
% \end{function}
%
% \begin{function}[EXP]{\draw_point_interpolate_curve:nnnnnn}
%   \begin{syntax}
%     \cs{draw_point_interpolate_curve:nnnnnn} \Arg{part}
%       \Arg{start} \Arg{control1} \Arg{control2} \Arg{end}
%   \end{syntax}
%   Expands to the point which is \meta{part} way along the curve between
%   \meta{start} and \meta{end} and defined by \meta{control1} and
%   \meta{control2}. The \meta{part} may be an interpolation or
%   an extrapolation, and is a floating point value expressing a percentage
%   along the curve, \emph{e.g.}~a value of \texttt{0.5} would be half-way
%   along the curve.
% \end{function}
%
% \subsection{Paths}
%
% Paths are constructed by combining one or more operations before applying
% one or more actions. Thus until a path is \enquote{used}, it may be
% manipulated or indeed discarded entirely. Only one path is active at
% any one time, and the path is \emph{not} affected by \TeX{} grouping.
%
% \begin{function}{\draw_path_corner_arc:nn}
%   \begin{syntax}
%     \cs{draw_path_corner_arc:n} \Arg{length1} \Arg{length2}
%   \end{syntax}
%   Sets the degree of rounding applied to corners in a path: the two
%   \meta{length} values are the distances from the corner at which the curving
%   should start. The first \meta{length} applies to the part of the path
%   \enquote{leading in} to the corner (\emph{i.e.}~from the previous path
%   operation), and the second to that \enquote{leading out}. If both
%   values are \texttt{0pt} then corners will not be rounded. The values
%   apply within the scope of the current \TeX{} group.
%   \begin{demo}
%     \draw_begin:
%       \draw_path_corner_arc:nn { 5mm } { 5mm }
%       \draw_path_rectangle_corners:nn
%         { 0cm , 0cm } { 3cm , 2cm }
%       \draw_path_use_clear:n { stroke }
%     \draw_end:
%   \end{demo}
%   \begin{demo}
%     \draw_begin:
%       \draw_path_corner_arc:nn { 10mm } { 5mm }
%       \draw_path_moveto:n { 0cm , 0cm }
%       \draw_path_lineto:n { 0cm , 2cm }
%       \draw_path_lineto:n { 3cm , 2cm }
%       \draw_path_curveto:nnn
%         { 3cm , 0cm } { 2cm , 0cm } { 1cm , 0cm }
%       \draw_path_use_clear:n { stroke }
%     \draw_end:
%   \end{demo}
%   The corners created are quarter-circles for exact right-angles and
%   close approximations otherwise. Closing a path will result in rounding
%   correctly.
%   \begin{demo}
%     \draw_begin:
%       \draw_path_corner_arc:nn { 4pt } { 4pt }
%       \draw_path_moveto:n
%         { \draw_point_polar:nn { 1cm } { 0 } }
%       \int_step_inline:nnnn { 72 } { 72 } { 359 }
%         {
%           \draw_path_lineto:n
%             { \draw_point_polar:nn { 1cm } { #1 } }
%         }
%        \draw_path_close:
%       \draw_path_use_clear:n { stroke }
%     \draw_end:
%   \end{demo}
% \end{function}
%
% \begin{function}{\draw_path_close:}
%   \begin{syntax}
%     \cs{draw_path_close:}
%   \end{syntax}
%   Closes the current part of the path by appending a straight line from
%   the current point to the starting point of the path. In general, any
%   path to be \emph{filled} should include a close instructions.
% \end{function}
%
% \begin{function}{\draw_path_use:n, \draw_path_use_clear:n}
%   \begin{syntax}
%     \cs{draw_path_use:n} \Arg{action(s)}
%   \end{syntax}
%   Inserts the current path, carrying out one ore more possible \meta{actions}
%   (a comma list):
%   \begin{itemize}
%     \item \texttt{stroke} Draws a line along the current path
%     \item \texttt{draw} A synonym for \texttt{stroke}
%     \item \texttt{fill} Fills the interior of the path with the current
%       file color
%     \item \texttt{clip} Clips any content outside of the path
%   \end{itemize}
%   Actions are applied in the order given irrespective of the input order.
%   Note that whilst it is possible to use a path without clearing it, the
%   normal use case would be to clear the path (this resets data structures
%   at the macro level).
% \end{function}
%
% \subsubsection{Path operations on drawing axes}
%
% The standard path functions are all influenced by the active transformation
% matrix, \emph{i.e.}~the work relative to the drawing axes rather than
% the canvas.
%
% \begin{function}{\draw_path_moveto:n}
%   \begin{syntax}
%     \cs{draw_path_moveto:n} \Arg{point}
%   \end{syntax}
%   Moves the reference point of the path to the \meta{point}, but will
%   not join this to any previous point.
% \end{function}
%
% \begin{function}{\draw_path_lineto:n}
%   \begin{syntax}
%     \cs{draw_path_lineto:n} \Arg{point}
%   \end{syntax}
%   Joins the current path to the \meta{point} with a straight line.
%   In general, for reliable treatment by viewers, a \cs{draw_path_moveto:n}
%   operation should precede the first use of a \cs{draw_path_lineto:n}
%   on a path.
%   \begin{demo}
%     \draw_begin:
%       \draw_path_moveto:n { 0cm , 0cm }
%       \draw_path_lineto:n { 1cm , 1cm }
%       \draw_path_lineto:n { 2cm , 1cm }
%       \draw_path_lineto:n { 3cm , 0.5cm }
%       \draw_path_lineto:n { 3cm , 0cm }
%       \draw_color_fill:n {  yellow!80!black }
%       \draw_path_use_clear:n { fill , stroke }
%     \draw_end:
%   \end{demo}
%   \begin{demo}
%     \draw_begin:
%       \draw_path_moveto:n { 0cm , 0cm }
%       \draw_path_lineto:n { 1cm , 1cm }
%       \draw_path_lineto:n { 2cm , 1cm }
%       \draw_path_moveto:n { 2cm , 1cm }  % Begins a new part
%       \draw_path_lineto:n { 3cm , 0.5cm }
%       \draw_path_lineto:n { 3cm , 0cm }
%       \draw_color_fill:n {  yellow!80!black }
%       \draw_path_use_clear:n { fill , stroke }
%     \draw_end:
%   \end{demo}
% \end{function}
%
% \begin{function}{\draw_path_curveto:nnn}
%   \begin{syntax}
%     \cs{draw_path_curveto:nnn} \Arg{control1} \Arg{control2} \Arg{end}
%   \end{syntax}
%   Joins the current path to the \meta{end} with a curved line defined by
%   cubic Bézier points \meta{control1} and \meta{control2}. The bounding box
%   of the path (and image) will fully-contain the curve and control points,
%   \emph{i.e.}~it may be bigger than strictly necessary to contain the curve
%   \emph{alone}.
%   \begin{demo}
%     \draw_begin:
%       \draw_path_moveto:n { 0cm , 0cm }
%       \draw_path_curveto:nnn
%         { 1cm , 1cm } % First control
%         { 2cm , 1cm } % Second control
%         { 3cm , 0cm } % End
%       \draw_color_fill:n {  yellow!80!black }
%       \draw_path_use_clear:n { fill , stroke }
%     \draw_end:
%   \end{demo}
% \end{function}
%
% \begin{function}{\draw_path_curveto:nn}
%   \begin{syntax}
%     \cs{draw_path_curveto:nn} \Arg{control} \Arg{end}
%   \end{syntax}
%   Joins the current path to the \meta{end} with a curved line defined by
%   quadratic Bézier point \meta{control}. The bounding box
%   of the path (and image) will fully-contain the curve and computed (cubic)
%   control points, \emph{i.e.}~it may be bigger than strictly necessary to
%   contain the curve \emph{alone}.
%   \begin{demo}
%     \draw_begin:
%       \draw_path_moveto:n { 0cm , 0cm }
%       \draw_path_curveto:nn
%         { 1cm , 1cm }
%         { 2cm , 0cm }
%       \draw_color_fill:n {  yellow!80!black }
%       \draw_path_use_clear:n { fill , stroke }
%     \draw_end:
%   \end{demo}
% \end{function}
%
% \begin{function}{\draw_path_arc:nnn, \draw_path_arc:nnnn}
%   \begin{syntax}
%     \cs{draw_path_arc:nnn} \Arg{angle1} \Arg{angle2} \Arg{radius}
%     \cs{draw_path_arc:nnnn} \Arg{angle1} \Arg{angle2} \Arg{radius-a} \Arg{radius-b}
%   \end{syntax}
%   Joins the current path with an arc between \meta{angle1} and \meta{angle2}
%   and of \meta{radius}. The four-argument version accepts two radii of
%   different lengths.
%   \begin{demo}
%     \draw_begin:
%       \draw_path_moveto:n { 0cm , 0cm }
%       \draw_path_lineto:n { 0cm , 1cm }
%       \draw_path_arc:nnn { 180 } { 90 } { 0.5cm }
%       \draw_path_lineto:n { 3cm , 1.5cm }
%       \draw_path_arc:nnn { 90 } { -45 } { 0.5cm }
%       \draw_path_use_clear:n { fill }
%     \draw_end:
%   \end{demo}
%
%   Note the interface here has a different argument ordering from that in
%   \pkg{pgf}, which has the two centers then the two radii.
%   \begin{demo}
%     \draw_begin:
%       \draw_path_moveto:n { 0cm , 0cm }
%       \draw_path_arc:nnnn { 180 } { 45 } { 2cm } { 1cm }
%       \draw_path_use_clear:n { stroke }
%     \draw_end:
%   \end{demo}
% \end{function}
%
% \begin{function}{\draw_path_arc_axes:nnnn}
%   \begin{syntax}
%     \cs{draw_path_arc_axes:nnn} \Arg{angle1} \Arg{angle2} \Arg{vector1} \Arg{vector2}
%   \end{syntax}
%   Appends the portion of an ellipse from \meta{angle1} to \meta{angle2} of an
%   ellipse with axes along \meta{vector1} and \meta{vector2} to the current
%   path.
%   \begin{demo}
%     \draw_begin:
%       \draw_path_moveto:n { 0cm , 0cm }
%       \draw_path_lineto:n { 2cm , 5mm }
%       \draw_path_moveto:n { 0cm , 0cm }
%       \draw_path_lineto:n { 0cm , 1cm }
%       \draw_path_moveto:n { 2cm , 5mm }
%       \draw_path_arc_axes:nnnn { 0 } { 90 }
%         { 2cm , 5mm } { 0cm , 1cm }
%       \draw_path_use_clear:n { stroke }
%     \draw_end:
%   \end{demo}
% \end{function}
%
% \begin{function}{\draw_path_ellipse:nnn}
%   \begin{syntax}
%     \cs{draw_path_ellipse:nnn} \Arg{center} \Arg{vector1} \Arg{vector2}
%   \end{syntax}
%   Appends an ellipse at \meta{center} with axes along \meta{vector1} and
%   \meta{vector2} to the current path. A new part is started if the path
%   is already non-empty. Notice that the underlying drawing is constructed
%   from arcs with appropriate moves: the interfaces is a more efficient
%   convenience.
%   \begin{demo}
%     \draw_begin:
%       \draw_path_ellipse:nnn
%         { 1cm , 0cm }
%         { 1.5cm , 0cm }
%         { 0cm , 1cm }
%       \draw_path_use_clear:n { stroke }
%       \draw_color:n { red }
%       \draw_path_ellipse:nnn
%         { 1cm , 0cm }
%         { 1cm , 1cm }
%         { -0.5cm , 0.5cm }
%       \draw_path_use_clear:n { stroke }
%     \draw_end:
%   \end{demo}
%
%   Note that any transformation is applied to the completed ellipse rather than
%   to the axes.
% \end{function}
%
% \begin{function}{\draw_path_circle:nn}
%   \begin{syntax}
%     \cs{draw_path_circle:nn} \Arg{center} \Arg{radius}
%   \end{syntax}
%   Appends a circle of \meta{radius} at \meta{center} to the current path. 
%   This is a shortcut for \cs{draw_path_ellipse:nnn}.
% \end{function}
%
% \begin{function}{\draw_path_rectangle:nn, \draw_path_rectangle_corners:nn}
%   \begin{syntax}
%     \cs{draw_path_rectangle:nn} \Arg{lower-left} \Arg{displacement}
%     \cs{draw_path_rectangle_corners:nn} \Arg{lower-left} \Arg{top-right}
%   \end{syntax}
%   Appends a rectangle starting at \meta{lower-left} to the current path,
%   with the size of the rectangle determined either by a \meta{displacement}
%   or the position of the \meta{top-right}.
%   \begin{demo}
%     \draw_begin:
%       \draw_path_rectangle:nn
%         { 1cm , 0cm }
%         { 1.5cm , 1cm }
%       \draw_path_rectangle:nn
%         { 1.5cm , 0.25cm }
%         { 1.5cm , 1cm }
%       \draw_path_rectangle:nn
%         { 2cm , 0.5cm }
%         { 1.5cm , 1cm }
%       \draw_path_use_clear:n { draw }
%     \draw_end:
%   \end{demo}
%   \begin{demo}
%     \draw_begin:
%       \draw_path_rectangle_corners:nn
%         { 1cm , 0cm }
%         { 1.5cm , 1cm }
%       \draw_path_use_clear:n { draw }
%     \draw_end:
%   \end{demo}
% \end{function}
%
% \begin{function}{\draw_path_grid:nnnn}
%   \begin{syntax}
%     \cs{draw_path_grid:nnnn} \Arg{xstep} \Arg{ystep} \Arg{lower-left} \Arg{upper-right}
%   \end{syntax}
%   Constructs a grid of \meta{xstep} and \meta{ystep} inside the rectangle
%   defined by the \meta{lower-left} and the \meta{upper-right}, and appends
%   this to the current path. The grid will be aligned such that grid lines
%   pass through the origin, which may result in \enquote{protruding} ends
%   if the start/end positions do not fully align.
%   \begin{demo}
%     \draw_begin:
%       \draw_linewidth:n { 0.8pt }
%       \draw_path_grid:nnnn
%         { 1cm } { 1cm }
%         { -3mm , -3mm }
%         { 33mm , 23mm }
%       \draw_path_use_clear:n { stroke }
%       \draw_linewidth:n { 0.4pt }
%       \draw_path_grid:nnnn
%         { 1mm } { 1mm }
%         { -1.5mm , -1.5mm }
%         { 31.5mm , 21.5mm }
%       \draw_path_use_clear:n { stroke }
%     \draw_end:
%   \end{demo}
%
%   Any transformation is applied to the finished grid.
%   \begin{demo}
%     \draw_begin:
%       \draw_transform_rotate:n { 10 }
%       \draw_path_grid:nnnn
%         { 1mm } { 2mm }
%         { 0mm , 0mm }
%         { 30mm , 30mm }
%       \draw_path_use_clear:n { stroke }
%     \draw_end:
%   \end{demo}
% \end{function}
%
% \subsubsection{Path scope}
%
% \begin{function}{\draw_path_scope_begin:, \draw_path_scope_end:}
%   \begin{syntax}
%     \cs{draw_path_scope_begin:}
%     ...
%     \cs{draw_path_scope_end:}
%   \end{syntax}
%   Suspends (and saves) the current (partial) path, initialising a new
%   path within the scope. Path operations are written to output only when
%   used, so the scoped path is stored at the \pkg{expl3} level, not
%   in the output.
% \end{function}
%
% \subsubsection{Path operations on canvas axes}
%
% For \emph{specialist} work, a small number of functions are provided
% which work relative to the canvas axes, \emph{i.e.}~these functions
% ignore the transformation matrix.
%
% \begin{function}{\draw_path_canvas_moveto:n}
%   \begin{syntax}
%     \cs{draw_path_canvas_moveto:n} \Arg{canvas point}
%   \end{syntax}
%   Moves the reference point of the path to the \meta{canvas point}, but will
%   not join this to any previous point.
% \end{function}
%
% \begin{function}{\draw_path_canvas_lineto:n}
%   \begin{syntax}
%     \cs{draw_path_canvas_lineto:n} \Arg{canvas point}
%   \end{syntax}
%   Joins the current path to the \meta{canvas point} with a straight line.
% \end{function}
%
% \begin{function}{\draw_path_canvas_curveto:nnn}
%   \begin{syntax}
%     \cs{draw_path_canvas_curveto:nnn} \Arg{control1} \Arg{control2} \Arg{end}
%   \end{syntax}
%   Joins the current path to the \meta{end} with a curved line defined by
%   cubic Bézier points \meta{control1} and \meta{control2}. These positions
%   are given as canvas points.
% \end{function}
%
% \subsection{Bounding box}
%
% \begin{variable}{\l_draw_bb_update_bool}
%   All functions automatically update the bounding box of the image, unless
%   specified otherwise. This behavior is selectable using the
%   \cs{l_draw_bb_update_bool} boolean.
% \end{variable}
%
% \subsection{Boxes and coffins}
%
% \begin{function}{\draw_box_use:N}
%   \begin{syntax}
%     \cs{draw_box_use:N} \meta{box}
%   \end{syntax}
%   Inserts the \meta{box} into a drawing, taking account of the current
%   transformation matrix and shift, and adjusting the drawing bounding
%   box to contain the (apparent) size of the box if this is active
%   (see \cs{l_draw_bb_update_bool}).
%   \begin{demo}
%     \draw_begin:
%       \draw_path_moveto:n { 0cm , 0cm }
%       \draw_path_lineto:n { 0cm , 1cm }
%       \draw_path_use_clear:n { stroke }
%       \hbox_set:Nn \l_tmpa_box
%         { This~is~text. }
%       \draw_box_use:N \l_tmpa_box
%     \draw_end:
%   \end{demo}
%   \begin{demo}
%     \draw_begin:
%       \draw_transform_matrix_absolute:nnnn { 2 } { 0 } { 1 } { 2 }
%       \draw_path_moveto:n { 0cm , 0cm }
%       \draw_path_lineto:n { 0cm , 1cm }
%       \draw_path_use_clear:n { stroke }
%       \hbox_set:Nn \l_tmpa_box
%         { This~is~text. }
%       \draw_box_use:N \l_tmpa_box
%     \draw_end:
%   \end{demo}
% \end{function}
%
% \begin{function}{\draw_coffin_use:Nnn}
%   \begin{syntax}
%     \cs{draw_coffin_use:N} \meta{coffin} \Arg{hpole} \Arg{vpole}
%   \end{syntax}
%   Inserts the \meta{coffin} into a drawing, taking account of the current
%   transformation matrix and shift, and adjusting the drawing bounding
%   box to contain the (apparent) size of the box if this is active
%   (see \cs{l_draw_bb_update_bool}). The alignment point of the coffin to
%   the origin is specified by the intersection of the \meta{hpole} and the
%   \meta{vpole}.
%   \begin{demo}
%     \draw_begin:
%       \draw_path_moveto:n { 0cm , 0cm }
%       \draw_path_lineto:n { 0cm , 1cm }
%       \draw_path_use_clear:n { stroke }
%       \hcoffin_set:Nn \l_tmpa_coffin
%         { This~is~text. }
%       \draw_coffin_use:Nnn \l_tmpa_coffin { hc } { vc }
%     \draw_end:
%   \end{demo}
% \end{function}
%
% \subsection{Color}
%
% \begin{function}{\draw_color:n, \draw_color_fill:n, \draw_color_stroke:n}
%   \begin{syntax}
%     \cs{draw_color:n} \Arg{color expression}
%   \end{syntax}
%   Evaluates the \meta{color expression} as described for \pkg{l3color}.
% \end{function}
%
% \subsection{Transformations}
%
% Points are normally used unchanged relative to the canvas axes. This can
% be modified by applying a transformation matrix. The canvas axes themselves
% may be adjusted using \cs{driver_draw_cm:nnnn}: note that this
% is transparent to the drawing code so is not tracked.
%
% \begin{function}
%   {\draw_transform_matrix:nnnn, \draw_transform_matrix_absolute:nnnn}
%   \begin{syntax}
%     \cs{draw_transform_matrix:nnnn}
%       \Arg{a} \Arg{b} \Arg{c} \Arg{d}
%   \end{syntax}
%   Applies the transformation matrix $[ \meta{a} \meta{b} \meta{c} \meta{d}]$.
%   The basic applies the transformation in addition to those active; the
%   |absolute| version overwrites any active transformation.
%   This assignment is local.
% \end{function}
%
% \begin{function}{\draw_transform_shift:n, \draw_transform_shift_absolute:n}
%   \begin{syntax}
%     \cs{draw_transform_shift:n} \Arg{vector}
%   \end{syntax}
%   Applies the transformation \meta{vector} to points.
%   The basic applies the vector in addition to those active; the
%   |absolute| version overwrites any active vector.
%   This assignment is local.
% \end{function}
%
% \begin{function}{\draw_transform_triangle:nnn}
%   \begin{syntax}
%     \cs{draw_transform_triangle:nnn}
%       \Arg{origin} \Arg{point1} \Arg{point2}
%   \end{syntax}
%   Applies a transformation such that the co-ordinates $(0, 0)$, $(1, 0)$
%   and $(0, 1)$ are given by the \meta{origin}, \meta{point1} and
%   \meta{point2}, respectively.
%   This assignment is local.
% \end{function}
%
% \begin{function}{\draw_transform_rotate:n}
%   \begin{syntax}
%     \cs{draw_transform_rotate:n} \Arg{angle}
%   \end{syntax}
%   Applies a rotation by the \meta{angle}, measured anti-clockwise in degrees.
%   This rotation is \emph{additional} to any prevailing transformation.
%   This assignment is local.
% \end{function}
%
% \begin{function}
%   {
%     \draw_transform_scale:n,
%     \draw_transform_xscale:n,
%     \draw_transform_yscale:n
%   }
%   \begin{syntax}
%     \cs{draw_transform_scale:n} \Arg{scale}
%   \end{syntax}
%   Applies the \meta{scale} in either $x$ or $y$ (or both).
%   This scale is \emph{added} to any prevailing transformation.
%   This assignment is local.
% \end{function}
%
% \begin{function}
%   {
%     \draw_transform_xshift:n,
%     \draw_transform_yshift:n
%   }
%   \begin{syntax}
%     \cs{draw_transform_xshift:n} \Arg{xshift}
%   \end{syntax}
%   Applies an \meta{xshift} or \meta{yshift}, as appropriate. This shift is
%   \emph{added} to any prevailing one.
%   This assignment is local.
% \end{function}
%
% \begin{function}
%   {
%     \draw_transform_xslant:n,
%     \draw_transform_yslant:n
%   }
%   \begin{syntax}
%     \cs{draw_transform_xslant:n} \Arg{slant}
%   \end{syntax}
%   Applies the \meta{slant} (a factor) in either $x$ or $y$.
%   This slant is \emph{added} to any prevailing transformation.
%   This assignment is local.
% \end{function}
%
% \begin{function}
%   {\draw_transform_matrix_invert:, \draw_transform_shift_invert:}
%   \begin{syntax}
%     \cs{draw_transform_matrix_invert:}
%   \end{syntax}
%   Inverts the current transformation matrix or shift vector, as
%   appropriate.
%   This assignment is local.
% \end{function}
%
% \begin{function}
%   {\draw_transform_matrix_reset:, \draw_transform_shift_reset:}
%   \begin{syntax}
%     \cs{draw_transform_matrix_reset:}
%   \end{syntax}
%   Resets the current transformation matrix or shift vector, as
%   appropriate.
%   This assignment is local.
% \end{function}
%
% \subsection{Layers}
%
% Drawing layers may be used to alter the way in which elements are stacked
% on top of one another. In particular, they are useful when the nature of
% an element depends on another, but where it needs to be behind it's
% \enquote{parent}. A classic example is a filled background: this needs to
% know the size of the material it is behind.
%
% All drawings feature a layer called |main|. This layer should always be
% present, and is the one which \enquote{non-layered} content is added to.
%
% \begin{function}{\draw_layer_new:n}
%   \begin{syntax}
%     \cs{draw_layer_new:n} \Arg{layer}
%   \end{syntax}
%   Creates a new \meta{layer} which can then be used in drawing. The layer
%   |main| is pre-defined.
% \end{function}
%
% \begin{function}{\draw_layer_begin:n, \draw_layer_end:}
%   \begin{syntax}
%     \cs{draw_layer_begin:n} \Arg{layer}
%     ...
%     \cs{draw_layer_end:}
%   \end{syntax}
%   Begin and end collection of material for the \meta{layer}, which should
%   previously have been created (and which cannot be |main|). Material
%   collected for the layer is available globally within the current drawing.
% \end{function}
%
% \begin{variable}{\l_draw_layers_clist}
%   The list of active layers: may be given anywhere before \cs{draw_end:}.
%   The |main| layer should always be included.
% \end{variable}
%
% \end{documentation}
%
% \begin{implementation}
%
% \section{\pkg{l3draw} implementation}
%
%    \begin{macrocode}
%<*initex|package>
%    \end{macrocode}
%
%    \begin{macrocode}
%<@@=draw>
%    \end{macrocode}
%
%    \begin{macrocode}
%<*package>
\ProvidesExplPackage{l3draw}{2019-10-11}{}
  {L3 Experimental core drawing support}
%</package>
%    \end{macrocode}
%
%    \begin{macrocode}
\RequirePackage { l3color }
%    \end{macrocode}
%
% Everything else is in the sub-files!
%
%    \begin{macrocode}
%</initex|package>
%    \end{macrocode}
%
% \end{implementation}
%
% \PrintIndex
