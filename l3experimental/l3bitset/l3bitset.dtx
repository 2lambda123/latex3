% \iffalse meta-comment
%
%% File: l3bitset.dtx
%
% Copyright (C) 2020,2021 The LaTeX Project
%
% It may be distributed and/or modified under the conditions of the
% LaTeX Project Public License (LPPL), either version 1.3c of this
% license or (at your option) any later version.  The latest version
% of this license is in the file
%
%    http://www.latex-project.org/lppl.txt
%
% This file is part of the "l3bitset bundle" (The Work in LPPL)
% and all files in that bundle must be distributed together.
%
% -----------------------------------------------------------------------
%
% The development version of the bundle can be found at
%
%    https://github.com/latex3/latex3
%
% for those people who are interested.
%
%<*driver>
\documentclass[full]{l3doc}
\begin{document}
  \DocInput{\jobname.dtx}
\end{document}
%</driver>
% \fi
% \title{^^A
%   The \pkg{l3bitset} package  \\ Experimental bitsets ^^A
% }
%
% \author{^^A
%  The \LaTeX{} Project\thanks
%    {^^A
%      E-mail:
%        \href{mailto:latex-team@latex-project.org}
%          {latex-team@latex-project.org}^^A
%    }^^A
% }
%
% \date{Released 2021-01-26}
%
% \maketitle
% \begin{documentation}
%
% This package defines and implements the data type \texttt{bitset}, a vector of
% bits. The size of the vector may grow dynamically.
% Individual bits can be set and unset by index and by name.
% The index is like all other indices in \pkg{expl3} modules \emph{1-based}.
% A \texttt{bitset} can be output as binary number or---as needed e.g. in a
% PDF dictionary---as decimal (arabic) number.
% Currently only a small subset of the functions provided by the \pkg{bitset}
% package are implemented here, mainly the functions needed to use bitsets in
% PDF dictionaries.
%
% The bitset is stored as a string (but one shouldn't rely on the internal
% representation) and so the vector size is theoretically
% unlimited, only restricted by \TeX-memory. But the functions to set and clear
% bits uses integer functions for the index so bitsets can't be longer
% than $2^{31} - 1$.
% The export function
% \cs{bitset_to_arabic:N} can use functions from the \texttt{int} module only if
% the largest index used for this bitset is smaller then $32$, for longer
% bitsets \texttt{fp} is used and this is slower.
%
% \section{Creating bitsets}
% \begin{function}[added = 2020-12-13,updated=2020-12-29]
%   {\bitset_new:N,\bitset_new:c,\bitset_new:Nn, \bitset_new:cn}
%   \begin{syntax}
%     \cs{bitset_new:N}  \meta{bitset var} \\
%     \cs{bitset_new:Nn} \meta{bitset var}
%      \{
%         \meta{name1} |=| \meta{index1} |,|
%         \meta{name2} |=| \meta{index2} |,| \ldots{}
%      \}
%   \end{syntax}
% Creates a new \meta{bitset var} or raises an error if the name is already taken.
% The declaration is global. The \meta{bitset var} is initially $0$.
%
% Bitsets are implemented as string variables consisting of
% \texttt{1}'s and \texttt{0}'s.
% The rightmost number is the index position $1$, so
% the string variable can be viewed directly as the binary number.
% But one shouldn't rely on the internal representation, but use the
% dedicated \cs{bitset_to_bin:N} instead to get the binary number.
%
% The name--index pairs given in the second
% argument of \cs{bitset_new:Nn} declares names for some indices,
% which can be used to set and unset bits.
%
% \meta{index\ldots} should be a positive number.
% It can be an \meta{integer expression} which evaluates to a positive number.
% The expression is evaluated when the index is used, not a declaration time.
% The names \meta{name\ldots}
% should be unique. Using a number as name, e.g.~|10=1|, is allowed,
% but the index position |10| can then only be reached if some other
% name for it exists, e.g. |ten=10|.
% It is not necessary to give every index
% a name, and an index can have more than one name. The named index
% can be extended or changed with the next function.
%
% \end{function}
% \begin{function}[added = 2020-12-19]
%   {\bitset_addto_named_index:Nn}
%   \begin{syntax}
%     \cs{bitset_addto_named_index:Nn} \meta{bitset var}
%      \{
%         \meta{name1} |=| \meta{index1} |,|
%         \meta{name2} |=| \meta{index2} |,| \ldots{}
%      \}
%   \end{syntax}
% This extends or changes the name--index pairs for \meta{bitset var}
% globally as described for \cs{bitset_new:Nn}.
%
% For example after these settings
%
%  \begin{verbatim}
%  \bitset_new:Nn \l_pdfannot_F_bitset
%    {
%      Invisible      = 1,
%      Hidden         = 2,
%      Print          = 3,
%      NoZoom         = 4,
%      NoRotate       = 5,
%      NoView         = 6,
%      ReadOnly       = 7,
%      Locked         = 8,
%      ToggleNoView   = 9,
%      LockedContents = 10
%    }
%  \bitset_addto_named_index:Nn \l_pdfannot_F_bitset
%    {
%      print = 3
%    }
%  \end{verbatim}
%  it is possible to set bit 3 by using any of this alternatives:
%  \begin{verbatim}
%  \bitset_set_true:Nn \l_pdfannot_F_bitset {Print}
%  \bitset_set_true:Nn \l_pdfannot_F_bitset {print}
%  \bitset_set_true:Nn \l_pdfannot_F_bitset {3}
%  \end{verbatim}

% \end{function}
%

%
% \begin{function}[EXP, pTF,added = 2020-12-14]
%   { \bitset_if_exist:N, \bitset_if_exist:c  }
%   \begin{syntax}
%     \cs{bitset_if_exist_p:N} \meta{bitset var}
%     \cs{bitset_if_exist:NTF} \meta{bitset var} \Arg{true code} \Arg{false code}%
%   \end{syntax}
%   Tests whether the \meta{bitset var} exist.
% \end{function}

%
% \section{Setting and unsetting bits}
%
% \begin{function}[added = 2020-12-13, updated=2020-12-29]
%   { \bitset_set_true:Nn, \bitset_set_true:cn, \bitset_gset_true:Nn, \bitset_gset_true:cn  }
%   \begin{syntax}
%     \cs{bitset_set_true:Nn}   \meta{bitset var}  \Arg{name/index}\\
%     \cs{bitset_gset_true:Nn}  \meta{bitset var}  \Arg{name/index}
%   \end{syntax}
% This sets the bit of the index position represented by \Arg{name/index} to $1$.
% \Arg{name/index} should be either a declared name or a positive,
% unsigned explicit integer (not an integer expression).
% If it is both a name and a number, the name will take precedence.
% Index position are 1-based.
% If needed the length of the bit vector is enlarged.
% \end{function}
%
% \begin{function}[added = 2020-12-13,updated=2020-12-29]
%   { \bitset_set_false:Nn, \bitset_set_false:cn, \bitset_gset_false:Nn, \bitset_set_false:cn }
%   \begin{syntax}
%     \cs{bitset_set_false:Nn}   \meta{bitset var}  \Arg{name/index}\\
%     \cs{bitset_gset_false:Nn}  \meta{bitset var}  \Arg{name/index}
%   \end{syntax}
% This unsets the bit of the index position represented by \Arg{name/index} (sets
% it to $0$).
% \Arg{name/index} should be either a declared name or a positive,
% unsigned explicit integer (not an integer expression).
% If it is both a name and a number, the name will take precedence.
% The index is $1$-based. If the index position is larger
% than the current length of the bit vector
% nothing happens. If the leading (left most) bit is unset,
% zeros are not trimmed but stay in the bit vector and a still shown
% by \cs{bitset_show:N}.
% \end{function}
%
% \begin{function}[added = 2020-12-22]
%   {\bitset_clear:N,\bitset_clear:c,\bitset_gclear:N,\bitset_gclear:c}
%   \begin{syntax}
%     \cs{bitset_clear:N}  \meta{bitset var} \\
%     \cs{bitset_gclear:N}  \meta{bitset var}
%   \end{syntax}
% This resets the bitset to the initial state. The declared names are not changed.
% \end{function}
%
% \section{Using bitsets}
%
% \begin{function}[EXP,added = 2020-12-14, updated=2020-12-29]
%   { \bitset_item:Nn, \bitset_item:cn }
%   \begin{syntax}
%     \cs{bitset_item:Nn}   \meta{bitset var}  \Arg{name/index}
%   \end{syntax}
% \cs{bitset_item:Nn} outputs \texttt{1} if the bit with
% the index number represented by \Arg{name/index} is set and \texttt{0} otherwise.
%  \Arg{name/index} is a declared name or a positive, unsigned explicit number.
% \end{function}
%
% \begin{function}[EXP,added = 2020-12-13]
%   {\bitset_to_bin:N, \bitset_to_bin:c}
%   \begin{syntax}
%     \cs{bitset_to_bin:N} \meta{bitset var}
%   \end{syntax}
% This leaves the current value of the bitset expressed as
% a binary (string) number in the input stream.
% If no bit has been set yet, the output is zero.
% \end{function}
% \begin{function}[EXP,added = 2020-12-13]
%   {\bitset_to_arabic:N, \bitset_to_arabic:c}
%   \begin{syntax}
%     \cs{bitset_to_arabic:N} \meta{bitset var}
%   \end{syntax}
% This leaves the current value of the bitset expressed as
% a decimal number in the input stream. If no bit has been set yet,
% the output is zero. The function uses \cs{int_from_bin:n} if the largest
% index that have been set or unset is smaller then $32$, and a slower implementation
% based on \cs{fp_eval:n} otherwise.
% \end{function}
%
%
% \begin{function}[added = 2020-12-13]
%   {\bitset_show:N, \bitset_show:c}
%   \begin{syntax}
%     \cs{bitset_show:N} \meta{bitset var}
%   \end{syntax}
% Displays the binary and decimal value of the \meta{bitset var} on the terminal,
% \end{function}
%
% \begin{function}[added = 2020-12-13]
%   {\bitset_log:N, \bitset_log:c}
%   \begin{syntax}
%     \cs{bitset_log:N} \meta{bitset var}
%   \end{syntax}
% Writes the value of the \meta{bitset var} in the log file.
% \end{function}
%
% \end{documentation}
%
% \begin{implementation}
% \section{\pkg{l3bitset} implementation}
% \TestFiles{m3bitset001,m3bitset002}
%    \begin{macrocode}
%<*package>
%    \end{macrocode}
%
%    \begin{macrocode}
%<@@=bitset>
%    \end{macrocode}
%    \begin{macrocode}
\ProvidesExplPackage{l3bitset}{2020-12-22}{}
  {L3 Experimental bitset support}
%    \end{macrocode}
% A bitset is a string variable.
%  \begin{macro}
%    {
%      \bitset_new:N,  \bitset_new:c
%    }
%    \begin{macrocode}
\cs_new_protected:Npn \bitset_new:N #1
  {
    \__kernel_chk_if_free_cs:N #1
    \cs_gset_eq:NN #1 \c_zero_str
    \prop_new:c { g__bitset_\cs_to_str:N #1 _name_prop }
  }

\cs_new_protected:Npn \bitset_new:Nn #1 #2
  {
    \__kernel_chk_if_free_cs:N #1
    \cs_gset_eq:NN #1 \c_zero_str
    \prop_new:c { g__bitset_\cs_to_str:N #1 _name_prop }
    \prop_gset_from_keyval:cn
       { g__bitset_\cs_to_str:N #1 _name_prop }
       { #2 }
  }
\cs_generate_variant:Nn \bitset_new:N {c}
%    \end{macrocode}
% \end{macro}
% \begin{variable}{\l_@@_tmpa_prop}
% A scratch prop to be able to extend the names properties.
%    \begin{macrocode}
\prop_new:N \l_@@_tmpa_prop
%    \end{macrocode}
% \end{variable}
% \begin{macro}
%  {
%    \bitset_addto_named_index:Nn
%  }
%    \begin{macrocode}
\cs_new_protected:Npn  \bitset_addto_named_index:Nn #1 #2
  {
     \prop_set_from_keyval:Nn \l_@@_tmpa_prop
       { #2 }
     \prop_map_inline:Nn \l_@@_tmpa_prop
       {
         \prop_gput:cnn
           { g_@@_\cs_to_str:N #1 _name_prop }
           { ##1 }
           { ##2 }
       }
  }
%    \end{macrocode}
% \end{macro}
% \begin{macro}[pTF]
%   {
%     \bitset_if_exist:N, \bitset_if_exist:c
%   }
%
% Existence tests.
%    \begin{macrocode}
\prg_new_eq_conditional:NNn
  \bitset_if_exist:N \str_if_exist:N { p , T , F , TF }
\prg_new_eq_conditional:NNn
  \bitset_if_exist:c \str_if_exist:c
  { p , T , F , TF }
%    \end{macrocode}
% \end{macro}
% \begin{macro}
%   {
%     \@@_set_true:Nn, \@@_gset_true:Nn
%   }
% The internal command uses only numbers (integer expressions) for the
% position.
% A bit is set by either extending the string or by splitting it and
% then inserting an 1. It is not checked if the value was already 1.
%    \begin{macrocode}
% #1 name, #2 index (integer expression, 1-based)
\cs_new_protected:Npn \@@_set_true:Nn #1 #2
  {
    \int_compare:nNnT {#2 } > { 0 }
      {
        \int_compare:nNnTF {\str_count:N #1 } < { #2  }
          {
            %extend the str
            \exp_args:NNe
            \str_put_left:Nn #1 { \prg_replicate:nn {  #2 - \str_count:N #1 -1 } {0} }
            \str_put_left:Nn #1 { 1 }
          }
          {
            %replace value
            \str_set:Nx #1
              {
                \str_range:Nnn #1 {1}{-1 - (#2)}
                1
                \str_range:Nnn #1 {1 -(#2)}{-1}
             }
          }
      }
  }

%#1 name, #2 index (integer expression,1-based)
\cs_new_protected:Npn \@@_gset_true:Nn #1 #2
  {
    \int_compare:nNnT {#2 } > { 0 }
      {
        \int_compare:nNnTF {\str_count:N #1 } < { #2  }
          {
          %  %extend the str
            \exp_args:NNe
            \str_gput_left:Nn #1 { \prg_replicate:nn {  #2 - \str_count:N #1 -1 } {0} }
            \str_gput_left:Nn #1 { 1 }
          }
          {
            % replace the value
            \str_gset:Nx #1
              {
                \str_range:Nnn #1 {1}{-1 -(#2)}
                1
                \str_range:Nnn #1 {1 -(#2)}{-1}
              }
          }
      }
  }
%    \end{macrocode}
% \end{macro}
% \begin{macro}
%   {
%     \bitset_set_true:Nn, \bitset_set_true:cn,
%     \bitset_gset_true:Nn, \bitset_gset_true:cn,
%   }
% The user commands must first translate the argument to an index number.
%    \begin{macrocode}
% #1 name, #2 index (name or number)
\cs_new_protected:Npn \bitset_set_true:Nn #1 #2
  {
    \prop_if_in:cnTF { g_@@_\cs_to_str:N #1 _name_prop } {#2}
      {
        \@@_set_true:Nn #1
          {
            \prop_item:cn{ g_@@_\cs_to_str:N #1 _name_prop }{#2}
          }
      }
      {
        \regex_match:nnTF { ^[\d]+$ } {#2}
          {
            %is number
            \@@_set_true:Nn #1 { #2}
            \prop_gput:cnn { g_@@_\cs_to_str:N #1 _name_prop }{#2} {#2}
          }
          {
            \__kernel_msg_warning:nnxx { bitset } { bitset-unknown-name }
              { \token_to_str:N #1 }
              { \tl_to_str:n { #2} }
          }
      }
  }

%#1 name, #2 index (name or number)
\cs_new_protected:Npn \bitset_gset_true:Nn #1 #2
  {
    \prop_if_in:cnTF { g_@@_\cs_to_str:N #1 _name_prop } {#2}
      {
        \@@_gset_true:Nn #1
          {
            \prop_item:cn{ g_@@_\cs_to_str:N #1 _name_prop }{#2}
          }
      }
      {
        \regex_match:nnTF { ^[\d]+$ } {#2}
          {
            %is number
            \@@_gset_true:Nn #1 { #2}
            \prop_gput:cnn { g_@@_\cs_to_str:N #1 _name_prop }{#2} {#2}
          }
          {
            \__kernel_msg_warning:nnxx { bitset } { bitset-unknown-name }
              { \token_to_str:N #1 }
              { \tl_to_str:n { #2} }
          }
      }
  }
\cs_generate_variant:Nn \bitset_set_true:Nn  {cn}
\cs_generate_variant:Nn \bitset_gset_true:Nn {cn}
%    \end{macrocode}
% \end{macro}

% \begin{macro}
%   {
%     \@@_set_false:Nn, \@@_gset_false:Nn
%   }
% The internal command uses only numbers (integer expressions) for the
% position.
% Unsetting a bit has only to do something if the string is longer than then index.
%    \begin{macrocode}
\cs_new_protected:Npn \@@_set_false:Nn #1 #2  %#1 name, #2 index (1-based)
 {
   \int_compare:nNnT {#2 } > { 0 }
     {
       \int_compare:nNnT {\str_count:N #1 } > { #2 -1 }
         {
           % need to replace the str
           \str_set:Nx #1
             {
               \str_range:Nnn #1 {1}{-1 - (#2)}
                0
               \str_range:Nnn #1 { 1 - (#2)}{-1}
             }
         }
     }
 }

\cs_new_protected:Npn \@@_gset_false:Nn #1 #2  %#1 name, #2 index (1-based)
 {
   \int_compare:nNnT {#2 } > { 0 }
     {
       \int_compare:nNnT {\str_count:N #1 } > { #2 -1 }
         {
           % need to replace the str
           \str_gset:Nx #1
             {
               \str_range:Nnn #1 {1}{-1 -(#2)}
                0
               \str_range:Nnn #1 {1 - (#2)}{-1}
             }
         }
     }
 }
%    \end{macrocode}
%  \end{macro}

% \begin{macro}
%   {
%     \bitset_set_false:Nn,  \bitset_set_false:cn,
%     \bitset_gset_false:Nn, \bitset_gset_false:cn
%   }
%    \begin{macrocode}
\cs_new_protected:Npn \bitset_set_false:Nn #1 #2  %#1 name, #2 index (1-based)
 {
    \prop_if_in:cnTF { g_@@_\cs_to_str:N #1 _name_prop } {#2}
      {
        \@@_set_false:Nn #1
          {
            \prop_item:cn{ g_@@_\cs_to_str:N #1 _name_prop }{#2}
          }
      }
      {
        \regex_match:nnTF { ^[\d]+$ } {#2}
          {
            %is number
            \@@_set_false:Nn #1 { #2}
            \prop_gput:cnn { g_@@_\cs_to_str:N #1 _name_prop }{#2} {#2}
          }
          {
            \__kernel_msg_warning:nnxx { bitset } { bitset-unknown-name }
              { \token_to_str:N #1 }
              { \tl_to_str:n { #2} }
          }
      }
 }

\cs_new_protected:Npn \bitset_gset_false:Nn #1 #2  %#1 name, #2 index (1-based)
 {
   \prop_if_in:cnTF { g_@@_\cs_to_str:N #1 _name_prop } {#2}
     {
       \@@_gset_false:Nn #1
         {
           \prop_item:cn{ g_@@_\cs_to_str:N #1 _name_prop }{#2}
         }
     }
     {
       \regex_match:nnTF { ^[\d]+$ } {#2}
         {
           %is number
           \@@_gset_false:Nn #1 { #2}
           \prop_gput:cnn { g_@@_\cs_to_str:N #1 _name_prop }{#2} {#2}
         }
         {
           \__kernel_msg_warning:nnxx { bitset } { bitset-unknown-name }
             { \token_to_str:N #1 }
             { \tl_to_str:n { #2} }
         }
     }
 }
\cs_generate_variant:Nn \bitset_set_false:Nn {cn}
\cs_generate_variant:Nn \bitset_gset_false:Nn {cn}
%    \end{macrocode}
%  \end{macro}
% \begin{macro}
%   {
%     \bitset_clear:N,  \bitset_clear:c,
%     \bitset_gclear:N, \bitset_gclear:c
%   }
%    \begin{macrocode}
\cs_new_protected:Npn \bitset_clear:N #1
  {
    \str_set_eq:NN #1 \c_zero_str
  }
\cs_new_protected:Npn \bitset_gclear:N #1
  {
    \str_gset_eq:NN #1 \c_zero_str
  }
\cs_generate_variant:Nn \bitset_clear:N {c}
\cs_generate_variant:Nn \bitset_gclear:N {c}
%    \end{macrocode}
% \end{macro}
% \begin{macro}
%   {
%     \bitset_to_arabic:N, \bitset_to_arabic:c,
%     \bitset_to_bin:N,    \bitset_to_bin:c,
%   }
%   The naming of the commands follow the names in the \texttt{int} module.
%   \cs{bitset_to_arabic:N} uses \cs{int_from_bin:n} if the string is shorter
%   then 32 and  the slower \cs{fp_eval} for larger bitsets.
%
%    \begin{macrocode}
\cs_new:Npn \bitset_to_arabic:N #1
  {
    \int_compare:nNnTF { \str_count:N #1 } < {32}
      { \exp_args:No \int_from_bin:n {#1} }
      {
        \exp_after:wN \@@_to_int:nN \exp_after:wN 0
        #1 \q_recursion_tail \q_recursion_stop
      }
  }

\cs_new:Npn \@@_to_int:nN #1#2
  {
    \quark_if_recursion_tail_stop_do:Nn #2 {#1}
    \exp_args:Nf \@@_to_int:nN { \fp_eval:n { #1 * 2 + #2 } }
  }

\cs_new:Npn \bitset_to_bin:N #1
  {
    #1
  }

\cs_generate_variant:Nn \bitset_to_arabic:N  {c}
\cs_generate_variant:Nn \bitset_to_bin:N {c}
%    \end{macrocode}
% \end{macro}
% \begin{macro}
%   {
%     \bitset_item:Nn, \bitset_item:cn
%   }
% All bits that have been set at anytime have an entry in the prop,
% so we can take everything else as 0.
%    \begin{macrocode}
\cs_new:Npn \bitset_item:Nn #1 #2
  {
    \prop_if_in:cnTF { g_@@_\cs_to_str:N #1 _name_prop } {#2}
      {
        \int_eval:n
          {
            \str_item:Nn #1
              { 0 - ( \prop_item:cn { g_@@_\cs_to_str:N #1 _name_prop }{#2} ) }
            +0
          }
     }
     {
       0
     }
  }
\cs_generate_variant:Nn \bitset_item:Nn {cn}
%    \end{macrocode}
% \end{macro}

% \begin{macro}
%   {
%     \bitset_show:N, \bitset_show:c,
%     \bitset_log:N,  \bitset_log:c
%   }
%    \begin{macrocode}
\cs_new_protected:Npn   \bitset_show:N { \@@_show:NN \msg_show:nnxxxx }
\cs_generate_variant:Nn \bitset_show:N { c }
\cs_new_protected:Npn   \bitset_log:N  { \@@_show:NN \msg_log:nnxxxx }
\cs_generate_variant:Nn \bitset_log:N  { c }
\cs_new_protected:Npn   \bitset_show_named_index:N { \@@_show_named_index:NN \msg_show:nnxxxx }
\cs_generate_variant:Nn \bitset_show_named_index:N { c }
\cs_new_protected:Npn \__bitset_show:NN #1#2
  {
    \__kernel_chk_defined:NT #2
      {
        #1 { LaTeX/bitset } { show-bitset }
           { \token_to_str:N #2 }
           { \bitset_to_bin:N #2  }
           { \bitset_to_arabic:N #2  }
           { }
      }
  }

\cs_new_protected:Npn \__bitset_show_named_index:NN #1#2
  {
    \__kernel_chk_defined:NT #2
      {
        #1 { LaTeX/bitset } { show-names }
           { \token_to_str:N #2 }
           { \prop_map_function:cN { g_@@_\cs_to_str:N #2 _name_prop }  \msg_show_item:nn  }
           { }
           { }
      }
  }
%    \end{macrocode}
% \end{macro}

% \subsection{Messages}
%    \begin{macrocode}
 \__kernel_msg_new:nnn { bitset } { show-bitset }
  {
    The~bitset~#1~has~the~representation: \\
    >~binary:~#2  \\
    >~arabic:~#3 .
  }
\__kernel_msg_new:nnn { bitset } { show-names }
  {
    The~bitset~#1~
    \tl_if_empty:nTF {#2}
      { knows~no~names~yet \\>~ . }
      { knows~the~name/index~pairs~(without~outer~braces): #2 . }
  }
\__kernel_msg_new:nnn { bitset } { bitset-unknown-name }
  { The~name~'#2'~is~unknown~for~bitset~\tl_to_str:n {#1} }

%    \end{macrocode}
%    \begin{macrocode}
%</package>
%    \end{macrocode}
% \end{macro}%
% \end{implementation}
%
% \PrintIndex
