% \iffalse meta-comment
%
%% File: l3bootstrap.dtx
%
% Copyright (C) 2011-2021 The LaTeX Project
%
% It may be distributed and/or modified under the conditions of the
% LaTeX Project Public License (LPPL), either version 1.3c of this
% license or (at your option) any later version.  The latest version
% of this license is in the file
%
%    https://www.latex-project.org/lppl.txt
%
% This file is part of the "l3kernel bundle" (The Work in LPPL)
% and all files in that bundle must be distributed together.
%
% -----------------------------------------------------------------------
%
% The development version of the bundle can be found at
%
%    https://github.com/latex3/latex3
%
% for those people who are interested.
%
%<*driver>
\documentclass[full,kernel]{l3doc}
\begin{document}
  \DocInput{\jobname.dtx}
\end{document}
%</driver>
% \fi
%
% \title{^^A
%   The \pkg{l3bootstrap} package\\ Bootstrap code^^A
% }
%
% \author{^^A
%  The \LaTeX{} Project\thanks
%    {^^A
%      E-mail:
%        \href{mailto:latex-team@latex-project.org}
%          {latex-team@latex-project.org}^^A
%    }^^A
% }
%
% \date{Released 2021-05-27}
%
% \maketitle
%
% \begin{documentation}
%
% \section{Using the \LaTeX3 modules}
%
% The modules documented in \file{source3} are designed to be used on top of
% \LaTeXe{} and are loaded all as one with the usual |\usepackage{expl3}| or
% |\RequirePackage{expl3}| instructions.
%
% As the modules use a coding syntax different from standard
% \LaTeXe{} it provides a few functions for setting it up.
%
% \begin{function}[updated = 2011-08-13]{\ExplSyntaxOn, \ExplSyntaxOff}
%   \begin{syntax}
%     \cs{ExplSyntaxOn} \meta{code} \cs{ExplSyntaxOff}
%   \end{syntax}
%   The \cs{ExplSyntaxOn} function switches to a category code
%   regime in which spaces and new lines are ignored, and in which the colon (|:|)
%   and underscore (|_|) are treated as \enquote{letters}, thus allowing
%   access to the names of code functions and variables. Within this
%   environment, |~| is used to input a space. The \cs{ExplSyntaxOff}
%   reverts to the document category code regime.
%   \begin{texnote}
%     Spaces introduced by~|~| behave much in the same way as normal
%     space characters in the standard category code regime: they are
%     ignored after a control word or at the start of a line, and
%     multiple consecutive~|~| are equivalent to a single one.  However,
%     |~|~is \emph{not} ignored at the end of a line.
%   \end{texnote}
% \end{function}
%
% \begin{function}[updated = 2017-03-19]
%   {\ProvidesExplPackage, \ProvidesExplClass, \ProvidesExplFile}
%   \begin{syntax}
%     |\RequirePackage{expl3}| \\
%     \cs{ProvidesExplPackage} \Arg{package} \Arg{date} \Arg{version} \Arg{description}
%   \end{syntax}
%   These functions act broadly in the same way as the corresponding
%   \LaTeXe{} kernel
%   functions \tn{ProvidesPackage}, \tn{ProvidesClass} and
%   \tn{ProvidesFile}. However, they also implicitly switch
%   \cs{ExplSyntaxOn} for the remainder of the code with the file. At the
%   end of the file, \cs{ExplSyntaxOff} will be called to reverse this.
%   (This is the same concept as \LaTeXe{} provides in turning on
%   \tn{makeatletter} within package and class code.) The \meta{date} should
%   be given in the format \meta{year}/\meta{month}/\meta{day} or in the ISO
%   date format \meta{year}-\meta{month}-\meta{day}. If the
%   \meta{version} is given then it will be prefixed with \texttt{v} in
%   the package identifier line.
% \end{function}
%
% \begin{function}[updated = 2012-06-04]{\GetIdInfo}
%   \begin{syntax}
%     |\RequirePackage{l3bootstrap}|
%     \cs{GetIdInfo} |$Id:| \meta{SVN info field} |$| \Arg{description}
%   \end{syntax}
%   Extracts all information from a SVN field. Spaces are not
%   ignored in these fields. The information pieces are stored in
%   separate control sequences with \cs{ExplFileName} for the part of the
%   file name leading up to the period, \cs{ExplFileDate} for date,
%   \cs{ExplFileVersion} for version and \cs{ExplFileDescription} for the
%   description.
% \end{function}
%
% To summarize: Every single package using this syntax should identify
% itself using one of the above methods. Special care is taken so that
% every package or class file loaded with \tn{RequirePackage} or similar
% are loaded with usual \LaTeXe{} category codes and the \LaTeX3 category code
% scheme is reloaded when needed afterwards. See implementation for
% details. If you use the \cs{GetIdInfo} command you can use the
% information when loading a package with
% \begin{verbatim}
%   \ProvidesExplPackage{\ExplFileName}
%     {\ExplFileDate}{\ExplFileVersion}{\ExplFileDescription}
% \end{verbatim}
%
% \end{documentation}
%
% \begin{implementation}
%
% \section{\pkg{l3bootstrap} implementation}
%
%    \begin{macrocode}
%<*package>
%<@@=kernel>
%    \end{macrocode}
%
% \subsection{\LuaTeX{}-specific code}
%
% Depending on the versions available, the \LaTeX{} format may not have
% the raw |\Umath| primitive names available. We fix that globally:
% it should cause no issues. Older \LuaTeX{} versions do not have
% a pre-built table of the primitive names here so sort one out
% ourselves. These end up globally-defined but at that is true with
% a newer format anyway and as they all start |\U| this should be
% reasonably safe.
%    \begin{macrocode}
\begingroup
  \expandafter\ifx\csname directlua\endcsname\relax
  \else
    \directlua{%
      local i
      local t = { }
      for _,i in pairs(tex.extraprimitives("luatex")) do
        if string.match(i,"^U") then
          if not string.match(i,"^Uchar$") then %$
            table.insert(t,i)
          end
        end
      end
      tex.enableprimitives("", t)
    }%
  \fi
\endgroup
%    \end{macrocode}
%
% \subsection{The \tn{pdfstrcmp} primitive in \XeTeX{}}
%
% Only \pdfTeX{} has a primitive called \tn{pdfstrcmp}. The \XeTeX{}
% version is just \tn{strcmp}, so there is some shuffling to do. As
% this is still a real primitive, using the \pdfTeX{} name is \enquote{safe}.
%    \begin{macrocode}
\begingroup\expandafter\expandafter\expandafter\endgroup
  \expandafter\ifx\csname pdfstrcmp\endcsname\relax
  \let\pdfstrcmp\strcmp
\fi
%    \end{macrocode}
%
% \subsection{Loading support \Lua{} code}
%
% When \LuaTeX{} is used there are various pieces of \Lua{} code which need to
% be loaded. The code itself is defined in \pkg{l3luatex} and is extracted into
% a separate file. Thus here the task is to load the \Lua{} code both now and
% (if required) at the start of each job.
%    \begin{macrocode}
\begingroup\expandafter\expandafter\expandafter\endgroup
\expandafter\ifx\csname directlua\endcsname\relax
\else
  \ifnum\luatexversion<110 %
  \else
%    \end{macrocode}
%   For \LuaTeX{} we make sure the basic support is loaded:
%   this is only necessary in plain.
%
%   Additionally we just ensure that \TeX{} has seen the csnames \cs{prg_return_true:}
%   and \cs{prg_return_false:} before the Lua code builds these tokens.
%    \begin{macrocode}
    \begingroup\expandafter\expandafter\expandafter\endgroup
    \expandafter\ifx\csname newcatcodetable\endcsname\relax
      %%
%% This is file `ltluatex.tex',
%% generated with the docstrip utility.
%%
%% The original source files were:
%%
%% ltluatex.dtx  (with options: `tex,plain')
%% 
%% This is a generated file.
%% 
%% The source is maintained by the LaTeX Project team and bug
%% reports for it can be opened at https://latex-project.org/bugs.html
%% (but please observe conditions on bug reports sent to that address!)
%% 
%% 
%% Copyright (C) 1993-2021
%% The LaTeX Project and any individual authors listed elsewhere
%% in this file.
%% 
%% This file was generated from file(s) of the LaTeX base system.
%% --------------------------------------------------------------
%% 
%% It may be distributed and/or modified under the
%% conditions of the LaTeX Project Public License, either version 1.3c
%% of this license or (at your option) any later version.
%% The latest version of this license is in
%%    https://www.latex-project.org/lppl.txt
%% and version 1.3c or later is part of all distributions of LaTeX
%% version 2008 or later.
%% 
%% This file has the LPPL maintenance status "maintained".
%% 
%% This file may only be distributed together with a copy of the LaTeX
%% base system. You may however distribute the LaTeX base system without
%% such generated files.
%% 
%% The list of all files belonging to the LaTeX base distribution is
%% given in the file `manifest.txt'. See also `legal.txt' for additional
%% information.
%% 
%% The list of derived (unpacked) files belonging to the distribution
%% and covered by LPPL is defined by the unpacking scripts (with
%% extension .ins) which are part of the distribution.
\ifx\newluafunction\undefined\else\expandafter\endinput\fi
\ifx
  \ProvidesFile\undefined\begingroup\def\ProvidesFile
  #1#2[#3]{\endgroup\immediate\write-1{File: #1 #3}}
\fi
\ProvidesFile{ltluatex.tex}%
[2021/01/15 v1.1s
  LuaTeX support for plain TeX (core)
]
\edef\etatcatcode{\the\catcode`\@}
\catcode`\@=11
\ifnum\luatexversion<60 %
  \wlog{***************************************************}
  \wlog{* LuaTeX version too old for ltluatex support *}
  \wlog{***************************************************}
  \expandafter\endinput
\fi
\long\def\@gobble#1{}
\long\def\@firstofone#1{#1}
\directlua{tex.enableprimitives("",tex.extraprimitives("luatex"))}
\ifx\e@alloc\@undefined
  \ifx\documentclass\@undefined
    \ifx\loccount\@undefined
      \input{etex.src}%
    \fi
    \catcode`\@=11 %
    \outer\expandafter\def\csname newfam\endcsname
                          {\alloc@8\fam\chardef\et@xmaxfam}
  \else
    \RequirePackage{etex}
    \expandafter\def\csname newfam\endcsname
                    {\alloc@8\fam\chardef\et@xmaxfam}
    \expandafter\let\expandafter\new@mathgroup\csname newfam\endcsname
  \fi
\edef \et@xmaxregs {\ifx\directlua\@undefined 32768\else 65536\fi}
\edef \et@xmaxfam {\ifx\Umathcode\@undefined\sixt@@n\else\@cclvi\fi}
\count 270=\et@xmaxregs % locally allocates \count registers
\count 271=\et@xmaxregs % ditto for \dimen registers
\count 272=\et@xmaxregs % ditto for \skip registers
\count 273=\et@xmaxregs % ditto for \muskip registers
\count 274=\et@xmaxregs % ditto for \box registers
\count 275=\et@xmaxregs % ditto for \toks registers
\count 276=\et@xmaxregs % ditto for \marks classes
\expandafter\let\csname newcount\expandafter\expandafter\endcsname
                \csname globcount\endcsname
\expandafter\let\csname newdimen\expandafter\expandafter\endcsname
                \csname globdimen\endcsname
\expandafter\let\csname newskip\expandafter\expandafter\endcsname
                \csname globskip\endcsname
\expandafter\let\csname newbox\expandafter\expandafter\endcsname
                \csname globbox\endcsname
\chardef\e@alloc@top=65535
\let\e@alloc@chardef\chardef
\def\e@alloc#1#2#3#4#5#6{%
  \global\advance#3\@ne
  \e@ch@ck{#3}{#4}{#5}#1%
  \allocationnumber#3\relax
  \global#2#6\allocationnumber
  \wlog{\string#6=\string#1\the\allocationnumber}}%
\gdef\e@ch@ck#1#2#3#4{%
  \ifnum#1<#2\else
    \ifnum#1=#2\relax
      #1\@cclvi
      \ifx\count#4\advance#1 10 \fi
    \fi
    \ifnum#1<#3\relax
    \else
      \errmessage{No room for a new \string#4}%
    \fi
  \fi}%
\expandafter\csname newcount\endcsname\e@alloc@attribute@count
\expandafter\csname newcount\endcsname\e@alloc@ccodetable@count
\expandafter\csname newcount\endcsname\e@alloc@luafunction@count
\expandafter\csname newcount\endcsname\e@alloc@whatsit@count
\expandafter\csname newcount\endcsname\e@alloc@bytecode@count
\expandafter\csname newcount\endcsname\e@alloc@luachunk@count
\fi
\ifx\e@alloc@attribute@count\@undefined
  \countdef\e@alloc@attribute@count=258
  \e@alloc@attribute@count=\z@
\fi
\def\newattribute#1{%
  \e@alloc\attribute\attributedef
    \e@alloc@attribute@count\m@ne\e@alloc@top#1%
}
\def\setattribute#1#2{#1=\numexpr#2\relax}
\def\unsetattribute#1{#1=-"7FFFFFFF\relax}
\ifx\e@alloc@ccodetable@count\@undefined
  \countdef\e@alloc@ccodetable@count=259
  \e@alloc@ccodetable@count=\z@
\fi
\def\newcatcodetable#1{%
  \e@alloc\catcodetable\chardef
    \e@alloc@ccodetable@count\m@ne{"8000}#1%
  \initcatcodetable\allocationnumber
}
\newcatcodetable\catcodetable@initex
\newcatcodetable\catcodetable@string
\begingroup
  \def\setrangecatcode#1#2#3{%
    \ifnum#1>#2 %
      \expandafter\@gobble
    \else
      \expandafter\@firstofone
    \fi
      {%
        \catcode#1=#3 %
        \expandafter\setrangecatcode\expandafter
          {\number\numexpr#1 + 1\relax}{#2}{#3}
      }%
  }
  \@firstofone{%
    \catcodetable\catcodetable@initex
      \catcode0=12 %
      \catcode13=12 %
      \catcode37=12 %
      \setrangecatcode{65}{90}{12}%
      \setrangecatcode{97}{122}{12}%
      \catcode92=12 %
      \catcode127=12 %
      \savecatcodetable\catcodetable@string
    \endgroup
  }%
\newcatcodetable\catcodetable@latex
\newcatcodetable\catcodetable@atletter
\begingroup
  \def\parseunicodedataI#1;#2;#3;#4\relax{%
    \parseunicodedataII#1;#3;#2 First>\relax
  }%
  \def\parseunicodedataII#1;#2;#3 First>#4\relax{%
    \ifx\relax#4\relax
      \expandafter\parseunicodedataIII
    \else
      \expandafter\parseunicodedataIV
    \fi
      {#1}#2\relax%
  }%
  \def\parseunicodedataIII#1#2#3\relax{%
    \ifnum 0%
      \if L#21\fi
      \if M#21\fi
      >0 %
      \catcode"#1=11 %
    \fi
  }%
  \def\parseunicodedataIV#1#2#3\relax{%
    \read\unicoderead to \unicodedataline
    \if L#2%
      \count0="#1 %
      \expandafter\parseunicodedataV\unicodedataline\relax
    \fi
  }%
  \def\parseunicodedataV#1;#2\relax{%
    \loop
      \unless\ifnum\count0>"#1 %
        \catcode\count0=11 %
        \advance\count0 by 1 %
    \repeat
  }%
  \def\storedpar{\par}%
  \chardef\unicoderead=\numexpr\count16 + 1\relax
  \openin\unicoderead=UnicodeData.txt %
  \loop\unless\ifeof\unicoderead %
    \read\unicoderead to \unicodedataline
    \unless\ifx\unicodedataline\storedpar
      \expandafter\parseunicodedataI\unicodedataline\relax
    \fi
  \repeat
  \closein\unicoderead
  \@firstofone{%
    \catcode64=12 %
    \savecatcodetable\catcodetable@latex
    \catcode64=11 %
    \savecatcodetable\catcodetable@atletter
   }
\endgroup
\ifx\e@alloc@luafunction@count\@undefined
  \countdef\e@alloc@luafunction@count=260
  \e@alloc@luafunction@count=\z@
\fi
\def\newluafunction{%
  \e@alloc\luafunction\e@alloc@chardef
    \e@alloc@luafunction@count\m@ne\e@alloc@top
}
\ifx\e@alloc@whatsit@count\@undefined
  \countdef\e@alloc@whatsit@count=261
  \e@alloc@whatsit@count=\z@
\fi
\def\newwhatsit#1{%
  \e@alloc\whatsit\e@alloc@chardef
    \e@alloc@whatsit@count\m@ne\e@alloc@top#1%
}
\ifx\e@alloc@bytecode@count\@undefined
  \countdef\e@alloc@bytecode@count=262
  \e@alloc@bytecode@count=\z@
\fi
\def\newluabytecode#1{%
  \e@alloc\luabytecode\e@alloc@chardef
    \e@alloc@bytecode@count\m@ne\e@alloc@top#1%
}

\ifx\e@alloc@luachunk@count\@undefined
  \countdef\e@alloc@luachunk@count=263
  \e@alloc@luachunk@count=\z@
\fi
\def\newluachunkname#1{%
  \e@alloc\luachunk\e@alloc@chardef
    \e@alloc@luachunk@count\m@ne\e@alloc@top#1%
    {\escapechar\m@ne
    \directlua{lua.name[\the\allocationnumber]="\string#1"}}%
}
\def\now@and@everyjob#1{%
  \everyjob\expandafter{\the\everyjob
    #1%
  }%
  #1%
}
  \begingroup
    \attributedef\attributezero=0 %
    \chardef     \charzero     =0 %
    \countdef    \CountZero    =0 %
    \dimendef    \dimenzero    =0 %
    \mathchardef \mathcharzero =0 %
    \muskipdef   \muskipzero   =0 %
    \skipdef     \skipzero     =0 %
    \toksdef     \tokszero     =0 %
    \directlua{require("ltluatex")}
  \endgroup
\catcode`\@=\etatcatcode\relax
\endinput
%%
%% End of file `ltluatex.tex'.
%
    \fi
    \begingroup\edef\ignored{%
      \expandafter\noexpand\csname prg_return_true:\endcsname
      \expandafter\noexpand\csname prg_return_false:\endcsname
    }\endgroup
    \directlua{require("expl3")}%
%    \end{macrocode}
%   As the user might be making a custom format, no assumption is made about
%   matching package mode with only loading the \Lua{} code once. Instead, a
%   query to \Lua{} reveals what mode is in operation.
%    \begin{macrocode}
    \ifnum 0%
      \directlua{
        if status.ini_version then
          tex.write("1")
        end
      }>0 %
      \everyjob\expandafter{%
        \the\expandafter\everyjob
        \csname\detokenize{lua_now:n}\endcsname{require("expl3")}%
      }%
    \fi
  \fi
\fi
%    \end{macrocode}
%
% \subsection{Engine requirements}
%
% The code currently requires \eTeX{} and functionality equivalent to
% \tn{pdfstrcmp}, and also driver and Unicode character support. This is
% available in a reasonably-wide range of engines.
%
% For \LuaTeX, we require at least Lua 5.3 and the |token.set_lua| function.
% This is available at least since \LuaTeX{} 1.10.
%    \begin{macrocode}
\begingroup
  \def\next{\endgroup}%
  \def\ShortText{Required primitives not found}%
  \def\LongText%
    {%
      LaTeX3 requires the e-TeX primitives and additional functionality as
      described in the README file.
      \LineBreak
      These are available in the engines\LineBreak
      - pdfTeX v1.40\LineBreak
      - XeTeX v0.99992\LineBreak
      - LuaTeX v1.10\LineBreak
      - e-(u)pTeX mid-2012\LineBreak
      or later.\LineBreak
      \LineBreak
    }%
  \ifnum0%
    \expandafter\ifx\csname pdfstrcmp\endcsname\relax
    \else
      \expandafter\ifx\csname pdftexversion\endcsname\relax
        \expandafter\ifx\csname Ucharcat\endcsname\relax
          \expandafter\ifx\csname kanjiskip\endcsname\relax
          \else
            1%
          \fi
        \else
          1%
        \fi
      \else
        \ifnum\pdftexversion<140 \else 1\fi
      \fi
    \fi
    \expandafter\ifx\csname directlua\endcsname\relax
    \else
      \ifnum\luatexversion<110 \else 1\fi
    \fi
    =0 %
      \newlinechar`\^^J %
      \def\LineBreak{\noexpand\MessageBreak}%
      \expandafter\ifx\csname PackageError\endcsname\relax
        \def\LineBreak{^^J}%
        \def\PackageError#1#2#3%
          {%
            \errhelp{#3}%
            \errmessage{#1 Error: #2}%
          }%
      \fi
      \edef\next
        {%
          \noexpand\PackageError{expl3}{\ShortText}
            {\LongText Loading of expl3 will abort!}%
          \endgroup
          \noexpand\endinput
        }%
  \fi
\next
%    \end{macrocode}
%
% \subsection{Extending allocators}
%
% The ability to extend \TeX{}'s allocation routine to allow for \eTeX{} has
% been around since 1997 in the \pkg{etex} package.
% Loading this support is delayed until here as we are now sure that the
% \eTeX{} extensions and \tn{pdfstrcmp} or equivalent are available. Thus
% there is no danger of an \enquote{uncontrolled} error if the engine
% requirements are not met.
%
% For \LaTeXe{} we need to make sure that the extended pool is being used:
% \pkg{expl3} uses a lot of registers. For formats from 2015 onward there is
% nothing to do as this is automatic. For older formats, the \pkg{etex}
% package needs to be loaded to do the job. In that case, some inserts are
% reserved also as these have to be from the standard pool. Note that
% \tn{reserveinserts} is \tn{outer} and so is accessed here by csname. In
% earlier versions, loading \pkg{etex} was done directly and so
% \tn{reserveinserts} appeared in the code: this then required a \tn{relax}
% after \tn{RequirePackage} to prevent an error with \enquote{unsafe}
% definitions as seen for example with \pkg{capoptions}. The optional loading
% here is done using a group and \tn{ifx} test as we are not quite in the
% position to have a single name for \tn{pdfstrcmp} just yet.
%    \begin{macrocode}
\begingroup
  \def\@tempa{LaTeX2e}%
  \def\next{}%
  \ifx\fmtname\@tempa
    \expandafter\ifx\csname extrafloats\endcsname\relax
      \def\next
        {%
          \RequirePackage{etex}%
          \csname reserveinserts\endcsname{32}%
        }%
    \fi
  \fi
\expandafter\endgroup
\next
%    \end{macrocode}
%
% \subsection{The \LaTeX3 code environment}
%
% The code environment is now set up.
%
% \begin{macro}{\ExplSyntaxOff}
%   Before changing any category codes, in package mode we need to save
%   the situation before loading. Note the set up here means that once applied
%   \cs{ExplSyntaxOff} becomes a \enquote{do nothing} command until
%   \cs{ExplSyntaxOn} is used.
%    \begin{macrocode}
\protected\edef\ExplSyntaxOff
  {%
    \protected\def\noexpand\ExplSyntaxOff{}%
    \catcode   9 = \the\catcode   9\relax
    \catcode  32 = \the\catcode  32\relax
    \catcode  34 = \the\catcode  34\relax
    \catcode  38 = \the\catcode  38\relax
    \catcode  58 = \the\catcode  58\relax
    \catcode  94 = \the\catcode  94\relax
    \catcode  95 = \the\catcode  95\relax
    \catcode 124 = \the\catcode 124\relax
    \catcode 126 = \the\catcode 126\relax
    \endlinechar = \the\endlinechar\relax
    \chardef\csname\detokenize{l_@@_expl_bool}\endcsname = 0\relax
  }%
%    \end{macrocode}
% \end{macro}
%
% The code environment is now set up.
%    \begin{macrocode}
\catcode 9   = 9\relax
\catcode 32  = 9\relax
\catcode 34  = 12\relax
\catcode 38 =  4\relax
\catcode 58  = 11\relax
\catcode 94  = 7\relax
\catcode 95  = 11\relax
\catcode 124 = 12\relax
\catcode 126 = 10\relax
\endlinechar = 32\relax
%    \end{macrocode}
%
% \begin{variable}{\l_@@_expl_bool}
%   The status for code syntax: this is on at present.
%    \begin{macrocode}
\chardef\l_@@_expl_bool = 1\relax
%    \end{macrocode}
%\end{variable}
%
% \begin{macro}{\ExplSyntaxOn}
%  The idea here is that multiple \cs{ExplSyntaxOn} calls are not
%  going to mess up category codes, and that multiple calls to
%  \cs{ExplSyntaxOff} are also not wasting time. Applying
%  \cs{ExplSyntaxOn} alters the definition of \cs{ExplSyntaxOff}
%  and so in package mode this function should not be used until after
%  the end of the loading process!
%    \begin{macrocode}
\protected \def \ExplSyntaxOn
  {
    \bool_if:NF \l_@@_expl_bool
      {
        \cs_set_protected:Npx \ExplSyntaxOff
          {
            \char_set_catcode:nn { 9 }   { \char_value_catcode:n { 9 } }
            \char_set_catcode:nn { 32 }  { \char_value_catcode:n { 32 } }
            \char_set_catcode:nn { 34 }  { \char_value_catcode:n { 34 } }
            \char_set_catcode:nn { 38 }  { \char_value_catcode:n { 38 } }
            \char_set_catcode:nn { 58 }  { \char_value_catcode:n { 58 } }
            \char_set_catcode:nn { 94 }  { \char_value_catcode:n { 94 } }
            \char_set_catcode:nn { 95 }  { \char_value_catcode:n { 95 } }
            \char_set_catcode:nn { 124 } { \char_value_catcode:n { 124 } }
            \char_set_catcode:nn { 126 } { \char_value_catcode:n { 126 } }
            \tex_endlinechar:D =
              \tex_the:D \tex_endlinechar:D \scan_stop:
            \bool_set_false:N \l_@@_expl_bool
            \cs_set_protected:Npn \ExplSyntaxOff { }
          }
      }
    \char_set_catcode_ignore:n           { 9 }   % tab
    \char_set_catcode_ignore:n           { 32 }  % space
    \char_set_catcode_other:n            { 34 }  % double quote
    \char_set_catcode_alignment:n        { 38 }  % ampersand
    \char_set_catcode_letter:n           { 58 }  % colon
    \char_set_catcode_math_superscript:n { 94 }  % circumflex
    \char_set_catcode_letter:n           { 95 }  % underscore
    \char_set_catcode_other:n            { 124 } % pipe
    \char_set_catcode_space:n            { 126 } % tilde
    \tex_endlinechar:D = 32 \scan_stop:
    \bool_set_true:N \l_@@_expl_bool
  }
%    \end{macrocode}
% \end{macro}
%
%    \begin{macrocode}
%</package>
%    \end{macrocode}
%
% \end{implementation}
%
% \PrintIndex
