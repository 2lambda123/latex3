% \iffalse meta-comment
%
%% File: l3msg.dtx
%
% Copyright (C) 2009-2019 The LaTeX3 Project
%
% It may be distributed and/or modified under the conditions of the
% LaTeX Project Public License (LPPL), either version 1.3c of this
% license or (at your option) any later version.  The latest version
% of this license is in the file
%
%    https://www.latex-project.org/lppl.txt
%
% This file is part of the "l3kernel bundle" (The Work in LPPL)
% and all files in that bundle must be distributed together.
%
% -----------------------------------------------------------------------
%
% The development version of the bundle can be found at
%
%    https://github.com/latex3/latex3
%
% for those people who are interested.
%
%<*driver>
\documentclass[full,kernel]{l3doc}
\begin{document}
  \DocInput{\jobname.dtx}
\end{document}
%</driver>
% \fi
%
% \title{^^A
%   The \pkg{l3msg} package\\ Messages^^A
% }
%
% \author{^^A
%  The \LaTeX3 Project\thanks
%    {^^A
%      E-mail:
%        \href{mailto:latex-team@latex-project.org}
%          {latex-team@latex-project.org}^^A
%    }^^A
% }
%
% \date{Released 2019-05-28}
%
% \maketitle
%
% \begin{documentation}
%
% Messages need to be passed to the user by modules, either when errors
% occur or to indicate how the code is proceeding. The \pkg{l3msg}
% module provides a consistent method for doing this (as opposed to
% writing directly to the terminal or log).
%
% The system used by \pkg{l3msg} to create messages divides the process
% into two distinct parts. Named messages are created in the first part
% of the process; at this stage, no decision is made about the type of
% output that the message will produce. The second part of the process
% is actually producing a message. At this stage a choice of message
% \emph{class} has to be made, for example \texttt{error},
% \texttt{warning} or \texttt{info}.
%
% By separating out the creation and use of messages, several benefits
% are available. First, the messages can be altered later without
% needing details of where they are used in the code. This makes it
% possible to alter the language used, the detail level and so on.
% Secondly, the output which results from a given message can be
% altered. This can be done on a message class, module or message name
% basis. In this way, message behaviour can be altered and messages can
% be entirely suppressed.
%
% \section{Creating new messages}
%
% All messages have to be created before they can be used.
% The text of messages is automatically wrapped to the length
% available in the console. As a result, formatting is only needed
% where it helps to show meaning. In particular, |\\| may be
% used to force a new line and \verb*|\ | forces an explicit space.
% Additionally, |\{|, |\#|, |\}|, |\%| and |\~| can be used to produce
% the corresponding character.
%
% Messages may be subdivided \emph{by one level} using the~|/|
% character.  This is used within the message filtering system to allow
% for example the \LaTeX{} kernel messages to belong to the module
% \texttt{LaTeX} while still being filterable at a more granular level.
% Thus for example
% \begin{verbatim}
%   \msg_new:nnnn { mymodule } { submodule / message } ...
% \end{verbatim}
% will allow to filter out specifically messages from the \texttt{submodule}.
%
% \begin{function}[updated = 2011-08-16]{\msg_new:nnnn, \msg_new:nnn}
%   \begin{syntax}
%     \cs{msg_new:nnnn} \Arg{module} \Arg{message} \Arg{text} \Arg{more text}
%   \end{syntax}
%   Creates a \meta{message} for a given \meta{module}.
%   The message is defined to first give \meta{text} and then
%   \meta{more text} if the user requests it. If no \meta{more text} is
%   available then a standard text is given instead. Within \meta{text}
%   and \meta{more text} four parameters (|#1| to |#4|) can be used:
%   these will be supplied at the time the message is
%   used. An error is raised if the \meta{message} already exists.
% \end{function}
%
% \begin{function}{\msg_set:nnnn, \msg_set:nnn, \msg_gset:nnnn, \msg_gset:nnn}
%   \begin{syntax}
%     \cs{msg_set:nnnn} \Arg{module} \Arg{message} \Arg{text} \Arg{more text}
%   \end{syntax}
%   Sets up the text for a \meta{message} for a given \meta{module}.
%   The message is defined to first give \meta{text} and then
%   \meta{more text} if the user requests it. If no \meta{more text} is
%   available then a standard text is given instead. Within \meta{text}
%   and \meta{more text} four parameters (|#1| to |#4|) can be used:
%   these will be supplied at the time the message is used.
% \end{function}
%
% \begin{function}[EXP, pTF, added = 2012-03-03]{\msg_if_exist:nn}
%   \begin{syntax}
%     \cs{msg_if_exist_p:nn} \Arg{module} \Arg{message}
%     \cs{msg_if_exist:nnTF} \Arg{module} \Arg{message} \Arg{true code} \Arg{false code}
%   \end{syntax}
%   Tests whether the \meta{message} for the \meta{module} is currently
%   defined.
% \end{function}
%
% \section{Contextual information for messages}
%
% \begin{function}[rEXP]{\msg_line_context:}
%   \begin{syntax}
%     \cs{msg_line_context:}
%   \end{syntax}
%   Prints the current line number when a message is given, and
%   thus suitable for giving context to messages. The number itself
%   is proceeded by the text |on line|.
% \end{function}
%
% \begin{function}[EXP]{\msg_line_number:}
%   \begin{syntax}
%     \cs{msg_line_number:}
%   \end{syntax}
%   Prints the current line number when a message is given.
% \end{function}
%
% \begin{function}[EXP]{\msg_fatal_text:n}
%   \begin{syntax}
%     \cs{msg_fatal_text:n} \Arg{module}
%   \end{syntax}
%   Produces the standard text
%   \begin{quote}
%     \ttfamily
%     Fatal Package \meta{module} Error
%   \end{quote}
%   This function can be redefined to alter the language in which the
%   message is given, using |#1| as the name of the \meta{module} to
%   be included.
% \end{function}
%
% \begin{function}[EXP]{\msg_critical_text:n}
%   \begin{syntax}
%     \cs{msg_critical_text:n} \Arg{module}
%   \end{syntax}
%   Produces the standard text
%   \begin{quote}
%     \ttfamily
%     Critical Package \meta{module} Error
%   \end{quote}
%   This function can be redefined to alter the language in which the
%   message is given, using |#1| as the name of the \meta{module} to
%   be included.
% \end{function}
%
% \begin{function}[EXP]{\msg_error_text:n}
%   \begin{syntax}
%     \cs{msg_error_text:n} \Arg{module}
%   \end{syntax}
%   Produces the standard text
%   \begin{quote}
%     \ttfamily
%     Package \meta{module} Error
%   \end{quote}
%   This function can be redefined to alter the language in which the
%   message is given, using |#1| as the name of the \meta{module} to
%   be included.
% \end{function}
%
% \begin{function}[EXP]{\msg_warning_text:n}
%   \begin{syntax}
%     \cs{msg_warning_text:n} \Arg{module}
%   \end{syntax}
%   Produces the standard text
%   \begin{quote}
%     \ttfamily
%     Package \meta{module} Warning
%   \end{quote}
%   This function can be redefined to alter the language in which the
%   message is given, using |#1| as the name of the \meta{module} to
%   be included.  The \meta{type} of \meta{module} may be adjusted:
%   |Package| is the standard outcome: see \cs{msg_module_type:n}.
% \end{function}
%
% \begin{function}[EXP]{\msg_info_text:n}
%   \begin{syntax}
%     \cs{msg_info_text:n} \Arg{module}
%   \end{syntax}
%   Produces the standard text:
%   \begin{quote}
%     \ttfamily
%     Package \meta{module} Info
%   \end{quote}
%   This function can be redefined to alter the language in which the
%   message is given, using |#1| as the name of the \meta{module} to
%   be included. The \meta{type} of \meta{module} may be adjusted:
%   |Package| is the standard outcome: see \cs{msg_module_type:n}.
% \end{function}
%
% \begin{function}[EXP, added = 2018-10-10]{\msg_module_name:n}
%   \begin{syntax}
%     \cs{msg_module_name:n} \Arg{module}
%   \end{syntax}
%   Expands to the public name of the \meta{module} as defined by
%   \cs{g_msg_module_name_prop} (or otherwise leaves the \meta{module}
%   unchanged).
% \end{function}
%
% \begin{function}[EXP, added = 2018-10-10]{\msg_module_type:n}
%   \begin{syntax}
%     \cs{msg_module_type:n} \Arg{module}
%   \end{syntax}
%   Expands to the description which applies to the \meta{module},
%   for example a |Package| or |Class|. The information here is defined
%   in \cs{g_msg_module_type_prop}, and will default to |Package| if an
%   entry is not present.
% \end{function}
%
% \begin{function}[EXP,updated = 2018-09-30]{\msg_see_documentation_text:n}
%   \begin{syntax}
%     \cs{msg_see_documentation_text:n} \Arg{module}
%   \end{syntax}
%   Produces the standard text
%   \begin{quote}
%     \ttfamily
%
%     See the \meta{module} documentation for further information.
%   \end{quote}
%   This function can be redefined to alter the language in which the
%   message is given, using |#1| as the name of the \meta{module} to
%   be included. The name of the \meta{module} may be altered by use of
%   \cs{g_msg_module_documentation_prop}
% \end{function}
%
% \begin{variable}[added = 2018-10-10]{\g_msg_module_name_prop}
%   Provides a mapping between the module name used for messages, and that
%   for documentation. For example, \LaTeX{}3 core messages are stored in
%   the reserved |LaTeX| tree, but are printed as |LaTeX3|.
% \end{variable}
%
% \begin{variable}[added = 2018-10-10]{\g_msg_module_type_prop}
%   Provides a mapping between the module name used for messages, and that
%   type of module. For example, for \LaTeX{}3 core messages, an empty entry
%   is set here meaning that they are not described using the standard
%   |Package| text.
% \end{variable}
%
% \section{Issuing messages}
%
% Messages behave differently depending on the message class. In all cases,
% the message may be issued supplying~$0$ to~$4$ arguments.  If the number of
% arguments supplied here does not match the number in the definition of the
% message, extra arguments are ignored, or empty arguments added (of
% course the sense of the message may be impaired). The four arguments are
% converted to strings before being added to the message text: the
% \texttt{x}-type variants should be used to expand material.
%
% \begin{function}[updated = 2012-08-11]
%   {
%     \msg_fatal:nnnnnn ,
%     \msg_fatal:nnnnn  ,
%     \msg_fatal:nnnn   ,
%     \msg_fatal:nnn    ,
%     \msg_fatal:nn     ,
%     \msg_fatal:nnxxxx ,
%     \msg_fatal:nnxxx  ,
%     \msg_fatal:nnxx   ,
%     \msg_fatal:nnx
%   }
%   \begin{syntax}
%     \cs{msg_fatal:nnnnnn} \Arg{module} \Arg{message} \Arg{arg one} \Arg{arg two} \Arg{arg three} \Arg{arg four}
%   \end{syntax}
%   Issues \meta{module} error \meta{message}, passing \meta{arg one} to
%   \meta{arg four} to the text-creating functions. After issuing a
%   fatal error the \TeX{} run halts. No PDF file will be produced in
%   this case (DVI mode runs may produce a truncated DVI file).
% \end{function}
%
% \begin{function}[updated = 2012-08-11]
%   {
%     \msg_critical:nnnnnn ,
%     \msg_critical:nnnnn  ,
%     \msg_critical:nnnn   ,
%     \msg_critical:nnn    ,
%     \msg_critical:nn     ,
%     \msg_critical:nnxxxx ,
%     \msg_critical:nnxxx  ,
%     \msg_critical:nnxx   ,
%     \msg_critical:nnx
%   }
%   \begin{syntax}
%     \cs{msg_critical:nnnnnn} \Arg{module} \Arg{message} \Arg{arg one} \Arg{arg two} \Arg{arg three} \Arg{arg four}
%   \end{syntax}
%   Issues \meta{module} error \meta{message}, passing \meta{arg one} to
%   \meta{arg four} to the text-creating functions.  After issuing a
%   critical error, \TeX{} stops reading the current input file.
%   This may halt the \TeX{} run (if the current file is the main file)
%   or may abort reading a sub-file.
%   \begin{texnote}
%     The \TeX{} \tn{endinput} primitive is used to exit the file.  In
%     particular, the rest of the current line remains in the input
%     stream.
%   \end{texnote}
% \end{function}
%
% \begin{function}[updated = 2012-08-11]
%   {
%     \msg_error:nnnnnn ,
%     \msg_error:nnnnn  ,
%     \msg_error:nnnn   ,
%     \msg_error:nnn    ,
%     \msg_error:nn     ,
%     \msg_error:nnxxxx ,
%     \msg_error:nnxxx  ,
%     \msg_error:nnxx   ,
%     \msg_error:nnx
%   }
%   \begin{syntax}
%     \cs{msg_error:nnnnnn} \Arg{module} \Arg{message} \Arg{arg one} \Arg{arg two} \Arg{arg three} \Arg{arg four}
%   \end{syntax}
%   Issues \meta{module} error \meta{message}, passing \meta{arg one} to
%   \meta{arg four} to the text-creating functions.  The error
%   interrupts processing and issues the text at the terminal.  After user
%   input, the run continues.
% \end{function}
%
% \begin{function}[updated = 2012-08-11]
%   {
%     \msg_warning:nnnnnn ,
%     \msg_warning:nnnnn  ,
%     \msg_warning:nnnn   ,
%     \msg_warning:nnn    ,
%     \msg_warning:nn     ,
%     \msg_warning:nnxxxx ,
%     \msg_warning:nnxxx  ,
%     \msg_warning:nnxx   ,
%     \msg_warning:nnx
%   }
%   \begin{syntax}
%     \cs{msg_warning:nnxxxx} \Arg{module} \Arg{message} \Arg{arg one} \Arg{arg two} \Arg{arg three} \Arg{arg four}
%   \end{syntax}
%   Issues \meta{module} warning \meta{message}, passing \meta{arg one}
%   to \meta{arg four} to the text-creating functions. The warning text
%   is added to the log file and the terminal, but the \TeX{} run
%   is not interrupted.
% \end{function}
%
% \begin{function}[updated = 2012-08-11]
%   {
%     \msg_info:nnnnnn ,
%     \msg_info:nnnnn  ,
%     \msg_info:nnnn   ,
%     \msg_info:nnn    ,
%     \msg_info:nn     ,
%     \msg_info:nnxxxx ,
%     \msg_info:nnxxx  ,
%     \msg_info:nnxx   ,
%     \msg_info:nnx
%   }
%   \begin{syntax}
%     \cs{msg_info:nnnnnn} \Arg{module} \Arg{message} \Arg{arg one} \Arg{arg two} \Arg{arg three} \Arg{arg four}
%   \end{syntax}
%   Issues \meta{module} information \meta{message}, passing
%   \meta{arg one} to \meta{arg four} to the text-creating functions.
%   The information text is added to the log file.
% \end{function}
%
% \begin{function}[updated = 2012-08-11]
%   {
%     \msg_log:nnnnnn ,
%     \msg_log:nnnnn  ,
%     \msg_log:nnnn   ,
%     \msg_log:nnn    ,
%     \msg_log:nn     ,
%     \msg_log:nnxxxx ,
%     \msg_log:nnxxx  ,
%     \msg_log:nnxx   ,
%     \msg_log:nnx
%   }
%   \begin{syntax}
%     \cs{msg_log:nnnnnn} \Arg{module} \Arg{message} \Arg{arg one} \Arg{arg two} \Arg{arg three} \Arg{arg four}
%   \end{syntax}
%   Issues \meta{module} information \meta{message}, passing
%   \meta{arg one} to \meta{arg four} to the text-creating functions.
%   The information text is added to the log file: the output
%   is briefer than \cs{msg_info:nnnnnn}.
% \end{function}
%
% \begin{function}[updated = 2012-08-11]
%   {
%     \msg_none:nnnnnn ,
%     \msg_none:nnnnn  ,
%     \msg_none:nnnn   ,
%     \msg_none:nnn    ,
%     \msg_none:nn     ,
%     \msg_none:nnxxxx ,
%     \msg_none:nnxxx  ,
%     \msg_none:nnxx   ,
%     \msg_none:nnx
%   }
%   \begin{syntax}
%     \cs{msg_none:nnnnnn} \Arg{module} \Arg{message} \Arg{arg one} \Arg{arg two} \Arg{arg three} \Arg{arg four}
%   \end{syntax}
%   Does nothing: used as a message class to prevent any output at
%   all (see the discussion of message redirection).
% \end{function}
%
% \section{Redirecting messages}
%
% Each message has a \enquote{name}, which can be used to alter the behaviour
% of the message when it is given. Thus we might have
% \begin{verbatim}
%   \msg_new:nnnn { module } { my-message } { Some~text } { Some~more~text }
% \end{verbatim}
% to define a message, with
% \begin{verbatim}
%   \msg_error:nn { module } { my-message }
% \end{verbatim}
% when it is used. With no filtering, this raises an error. However, we
% could alter the behaviour with
% \begin{verbatim}
%   \msg_redirect_class:nn { error } { warning }
% \end{verbatim}
% to turn all errors into warnings, or with
% \begin{verbatim}
%   \msg_redirect_module:nnn { module } { error } { warning }
% \end{verbatim}
% to alter only messages from that module, or even
% \begin{verbatim}
%   \msg_redirect_name:nnn { module } { my-message } { warning }
% \end{verbatim}
% to target just one message. Redirection applies first to individual messages,
% then to messages from one module and finally to messages of one class. Thus
% it is possible to select out an individual message for special treatment
% even if the entire class is already redirected.
%
% Multiple redirections are possible.  Redirections can be cancelled by
% providing an empty argument for the target class.  Redirection to a
% missing class raises an error immediately.  Infinite loops are
% prevented by eliminating the redirection starting from the target of
% the redirection that caused the loop to appear.  Namely, if
% redirections are requested as $A \to B$, $B \to C$ and $C \to A$ in
% this order, then the $A \to B$ redirection is cancelled.
%
% \begin{function}[updated = 2012-04-27]{\msg_redirect_class:nn}
%   \begin{syntax}
%     \cs{msg_redirect_class:nn} \Arg{class one} \Arg{class two}
%   \end{syntax}
%   Changes the behaviour of messages of \meta{class one} so that they
%   are processed using the code for those of \meta{class two}.
% \end{function}
%
% \begin{function}[updated = 2012-04-27]{\msg_redirect_module:nnn}
%   \begin{syntax}
%     \cs{msg_redirect_module:nnn} \Arg{module} \Arg{class one} \Arg{class two}
%   \end{syntax}
%   Redirects message of \meta{class one} for \meta{module} to act as
%   though they were from \meta{class two}. Messages of \meta{class one}
%   from sources other than \meta{module} are not affected by this
%   redirection. This function can be used to make some messages
%   \enquote{silent} by default. For example, all of the
%   \texttt{warning} messages of \meta{module} could be turned off with:
%   \begin{verbatim}
%     \msg_redirect_module:nnn { module } { warning } { none }
%   \end{verbatim}
% \end{function}
%
% \begin{function}[updated = 2012-04-27]{\msg_redirect_name:nnn}
%   \begin{syntax}
%     \cs{msg_redirect_name:nnn} \Arg{module} \Arg{message} \Arg{class}
%   \end{syntax}
%   Redirects a specific \meta{message} from a specific \meta{module} to
%   act as a member of \meta{class} of messages.  No further redirection
%   is performed.  This function can be used to make a selected message
%   \enquote{silent} without changing global parameters:
%   \begin{verbatim}
%     \msg_redirect_name:nnn { module } { annoying-message } { none }
%   \end{verbatim}
% \end{function}
%
% \end{documentation}
%
% \begin{implementation}
%
% \section{\pkg{l3msg} implementation}
%
%    \begin{macrocode}
%<*initex|package>
%    \end{macrocode}
%
%    \begin{macrocode}
%<@@=msg>
%    \end{macrocode}
%
% \begin{variable}{\l_@@_internal_tl}
%   A general scratch for the module.
%    \begin{macrocode}
\tl_new:N \l_@@_internal_tl
%    \end{macrocode}
% \end{variable}
%
% \begin{variable}{\l_@@_name_str, \l_@@_text_str}
%   Used to save module info when creating messages.
%    \begin{macrocode}
\str_new:N \l_@@_name_str
\str_new:N \l_@@_text_str
%    \end{macrocode}
% \end{variable}
%
% \subsection{Creating messages}
%
% Messages are created and used separately, so there two parts to
% the code here. First, a mechanism for creating message text.
% This is pretty simple, as there is not actually a lot to do.
%
% \begin{variable}{\c_@@_text_prefix_tl, \c_@@_more_text_prefix_tl}
%   Locations for the text of messages.
%    \begin{macrocode}
\tl_const:Nn \c_@@_text_prefix_tl      { msg~text~>~ }
\tl_const:Nn \c_@@_more_text_prefix_tl { msg~extra~text~>~ }
%    \end{macrocode}
% \end{variable}
%
% \begin{macro}[EXP,pTF]{\msg_if_exist:nn}
%   Test whether the control sequence containing the message text exists
%   or not.
%    \begin{macrocode}
\prg_new_conditional:Npnn \msg_if_exist:nn #1#2 { p , T , F , TF }
  {
    \cs_if_exist:cTF { \c_@@_text_prefix_tl #1 / #2 }
      { \prg_return_true: } { \prg_return_false: }
  }
%    \end{macrocode}
% \end{macro}
%
% \begin{macro}{\@@_chk_if_free:nn}
%   This auxiliary is similar to \cs{__kernel_chk_if_free_cs:N}, and is used when
%   defining messages with \cs{msg_new:nnnn}.
%    \begin{macrocode}
\__kernel_patch:nnNNpn { }
  {
    \__kernel_debug_log:x
      { Defining~message~ #1 / #2 ~\msg_line_context: }
  }
\cs_new_protected:Npn \@@_chk_free:nn #1#2
  {
    \msg_if_exist:nnT {#1} {#2}
      {
        \__kernel_msg_error:nnxx { kernel } { message-already-defined }
          {#1} {#2}
      }
  }
%    \end{macrocode}
% \end{macro}
%
% \begin{macro}{\msg_new:nnnn, \msg_new:nnn}
% \begin{macro}{\msg_gset:nnnn, \msg_gset:nnn}
% \begin{macro}{\msg_set:nnnn, \msg_set:nnn}
%   Setting a message simply means saving the appropriate text
%   into two functions. A sanity check first.
%    \begin{macrocode}
\cs_new_protected:Npn \msg_new:nnnn #1#2
  {
    \@@_chk_free:nn {#1} {#2}
    \msg_gset:nnnn {#1} {#2}
  }
\cs_new_protected:Npn \msg_new:nnn #1#2#3
  { \msg_new:nnnn {#1} {#2} {#3} { } }
\cs_new_protected:Npn \msg_set:nnnn #1#2#3#4
  {
    \cs_set:cpn { \c_@@_text_prefix_tl #1 / #2 }
      ##1##2##3##4 {#3}
    \cs_set:cpn { \c_@@_more_text_prefix_tl #1 / #2 }
      ##1##2##3##4 {#4}
  }
\cs_new_protected:Npn \msg_set:nnn #1#2#3
  { \msg_set:nnnn {#1} {#2} {#3} { } }
\cs_new_protected:Npn \msg_gset:nnnn #1#2#3#4
  {
    \cs_gset:cpn { \c_@@_text_prefix_tl #1 / #2 }
      ##1##2##3##4 {#3}
    \cs_gset:cpn { \c_@@_more_text_prefix_tl #1 / #2 }
      ##1##2##3##4 {#4}
  }
\cs_new_protected:Npn \msg_gset:nnn #1#2#3
  { \msg_gset:nnnn {#1} {#2} {#3} { } }
%    \end{macrocode}
% \end{macro}
% \end{macro}
% \end{macro}
%
% \subsection{Messages: support functions and text}
%
% \begin{variable}
%   {
%     \c_@@_coding_error_text_tl ,
%     \c_@@_continue_text_tl     ,
%     \c_@@_critical_text_tl     ,
%     \c_@@_fatal_text_tl        ,
%     \c_@@_help_text_tl         ,
%     \c_@@_no_info_text_tl      ,
%     \c_@@_on_line_text_tl      ,
%     \c_@@_return_text_tl       ,
%     \c_@@_trouble_text_tl
%   }
% Simple pieces of text for messages.
%    \begin{macrocode}
\tl_const:Nn \c_@@_coding_error_text_tl
  {
    This~is~a~coding~error.
    \\ \\
  }
\tl_const:Nn \c_@@_continue_text_tl
  { Type~<return>~to~continue }
\tl_const:Nn \c_@@_critical_text_tl
  { Reading~the~current~file~'\g_file_curr_name_str'~will~stop. }
\tl_const:Nn \c_@@_fatal_text_tl
  { This~is~a~fatal~error:~LaTeX~will~abort. }
\tl_const:Nn \c_@@_help_text_tl
  { For~immediate~help~type~H~<return> }
\tl_const:Nn \c_@@_no_info_text_tl
  {
    LaTeX~does~not~know~anything~more~about~this~error,~sorry.
    \c_@@_return_text_tl
  }
\tl_const:Nn \c_@@_on_line_text_tl { on~line }
\tl_const:Nn \c_@@_return_text_tl
  {
    \\ \\
    Try~typing~<return>~to~proceed.
    \\
    If~that~doesn't~work,~type~X~<return>~to~quit.
  }
\tl_const:Nn \c_@@_trouble_text_tl
  {
    \\ \\
    More~errors~will~almost~certainly~follow: \\
    the~LaTeX~run~should~be~aborted.
  }
%    \end{macrocode}
% \end{variable}
%
% \begin{macro}{\msg_line_number:, \msg_line_context:}
%   For writing the line number nicely. \cs{msg_line_context:} was set up
%   earlier, so this is not \texttt{new}.
%    \begin{macrocode}
\cs_new:Npn \msg_line_number: { \int_use:N \tex_inputlineno:D }
\cs_gset:Npn \msg_line_context:
  {
    \c_@@_on_line_text_tl
    \c_space_tl
    \msg_line_number:
  }
%    \end{macrocode}
% \end{macro}
%
% \subsection{Showing messages: low level mechanism}
%
% \begin{macro}{\@@_interrupt:Nnnn}
% \begin{macro}{\@@_no_more_text:nnnn}
%   The low-level interruption macro is rather opaque, unfortunately.
%   Depending on the availability of more information there is a choice
%   of how to set up the further help.  We feed the extra help text and
%   the message itself to a wrapping auxiliary, in this order because we
%   must first setup \TeX{}'s \tn{errhelp} register before issuing an
%   \tn{errmessage}. To deal with the various cases of critical or fatal
%   errors with and without help text, there is a bit of argument-passing
%   to do.
%    \begin{macrocode}
\cs_new_protected:Npn \@@_interrupt:NnnnN #1#2#3#4#5
  {
    \str_set:Nx \l_@@_text_str { #1 {#2} }
    \str_set:Nx \l_@@_name_str { \msg_module_name:n {#2} }
    \cs_if_eq:cNTF
      { \c_@@_more_text_prefix_tl #2 / #3 }
      \@@_no_more_text:nnnn
      {
        \@@_interrupt_wrap:nnn
          { \use:c { \c_@@_text_prefix_tl #2 / #3 } #4 }
          { \c_@@_continue_text_tl }
          {
             \c_@@_no_info_text_tl
             \tl_if_empty:NF #5
               { \\ \\ #5 }
          }
      }
      {
        \@@_interrupt_wrap:nnn
          { \use:c { \c_@@_text_prefix_tl #2 / #3 } #4 }
          { \c_@@_help_text_tl }
          {
             \use:c { \c_@@_more_text_prefix_tl #2 / #3 } #4
             \tl_if_empty:NF #5
               { \\ \\ #5 }
          }
      }
  }
\cs_new:Npn \@@_no_more_text:nnnn #1#2#3#4 { }
%    \end{macrocode}
% \end{macro}
% \end{macro}
%
% \begin{macro}{\@@_interrupt_wrap:nnn}
% \begin{macro}{\@@_interrupt_text:n, \@@_interrupt_more_text:n}
%   First setup \TeX{}'s \tn{errhelp} register with the extra help |#1|,
%   then build a nice-looking error message with |#2|.  Everything is
%   done using \texttt{x}-type expansion as the new line markers are
%   different for the two type of text and need to be correctly set up.
%   The auxiliary \cs{@@_interrupt_more_text:n} receives its argument
%   as a line-wrapped string, which is thus unaffected by expansion.
%   We ave to split the main text into two parts as only the \enquote{message}
%   itself is wrapped with a leader: the generic help is wrapped at full
%   width. We also have to allow for the two characters used by \tn{errmessage}
%   itself.
%    \begin{macrocode}
\cs_new_protected:Npn \@@_interrupt_wrap:nnn #1#2#3
  {
    \iow_wrap:nnnN { \\ #3 } { } { } \@@_interrupt_more_text:n
    \group_begin:
      \int_sub:Nn \l_iow_line_count_int { 2 }
      \iow_wrap:nxnN { \l_@@_text_str : ~ #1 }
        {
          ( \l_@@_name_str )
          \prg_replicate:nn
            {
                \str_count:N \l_@@_text_str
              - \str_count:N \l_@@_name_str
              + 2
            }
            { ~ }
        }
        { } \@@_interrupt_text:n
    \iow_wrap:nnnN { \l_@@_internal_tl \\ \\ #2 } { } { }
      \@@_interrupt:n
  }
\cs_new_protected:Npn \@@_interrupt_text:n #1
  {
    \group_end:
    \tl_set:Nn \l_@@_internal_tl {#1}
  }
\cs_new_protected:Npn \@@_interrupt_more_text:n #1
  { \exp_args:Nx \tex_errhelp:D { #1 \iow_newline: } }
%    \end{macrocode}
% \end{macro}
% \end{macro}
%
% \begin{macro}{\@@_interrupt:n}
%   The business end of the process starts by producing some visual
%   separation of the message from the main part of the log.  The error
%   message needs to be printed with everything made
%   \enquote{invisible}: \TeX{}'s own information involves the macro in
%   which \tn{errmessage} is called, and the end of the argument of the
%   \tn{errmessage}, including the closing brace.  We use an active |!|
%   to call the \tn{errmessage} primitive, and end its argument with
%   \cs{use_none:n} \Arg{spaces} which fills the output with spaces.  Two
%   trailing closing braces are turned into spaces to hide them as well.
%   The group in which we alter the definition of the active |!| is
%   closed before producing the message: this ensures that tokens
%   inserted by typing |I| in the command-line are inserted after
%   the message is entirely cleaned up.
%
%   The \cs{__kernel_iow_with:Nnn} auxiliary, defined in \pkg{l3file}, expects
%   an \meta{integer variable}, an integer \meta{value}, and some
%   \meta{code}.  It runs the \meta{code} after ensuring that the
%   \meta{integer variable} takes the given \meta{value}, then restores
%   the former value of the \meta{integer variable} if needed.  We use
%   it to ensure that the \tn{newlinechar} is $10$, as needed for
%   \cs{iow_newline:} to work, and that \tn{errorcontextlines} is $-1$,
%   to avoid showing irrelevant context.  Note that restoring the former
%   value of these integers requires inserting tokens after the
%   \tn{errmessage}, which go in the way of tokens which could be
%   inserted by the user.  This is unavoidable.
%    \begin{macrocode}
\group_begin:
  \char_set_lccode:nn { 38 } { 32 } % &
  \char_set_lccode:nn { 46 } { 32 } % .
  \char_set_lccode:nn { 123 } { 32 } % {
  \char_set_lccode:nn { 125 } { 32 } % }
  \char_set_catcode_active:N \&
\tex_lowercase:D
  {
    \group_end:
    \cs_new_protected:Npn \@@_interrupt:n #1
      {
        \iow_term:n { }
        \__kernel_iow_with:Nnn \tex_newlinechar:D { `\^^J }
          {
            \__kernel_iow_with:Nnn \tex_errorcontextlines:D { -1 }
              {
                \group_begin:
                  \cs_set_protected:Npn &
                    {
                      \tex_errmessage:D
                        {
                          #1
                          \use_none:n
                            { ............................................ }
                        }
                    }
                  \exp_after:wN
                \group_end:
                &
              }
          }
      }
  }
%    \end{macrocode}
% \end{macro}
%
% \subsection{Displaying messages}
%
% \LaTeX{} is handling error messages and so the \TeX{} ones are disabled.
% This is already done by the \LaTeXe{} kernel, so to avoid messing up any
% deliberate change by a user this is only set in format mode.
%    \begin{macrocode}
%<*initex>
\int_gset:Nn \tex_errorcontextlines:D { -1 }
%</initex>
%    \end{macrocode}
%
% \begin{macro}[EXP]
%   {
%     \msg_fatal_text:n    ,
%     \msg_critical_text:n ,
%     \msg_error_text:n    ,
%     \msg_warning_text:n  ,
%     \msg_info_text:n
%   }
% \begin{macro}[EXP]{\@@_text:nn}
% \begin{macro}[EXP]{\@@_text:n}
%   A function for issuing messages: both the text and order could
%   in principle vary. The module name may be empty for kernel messages,
%   hence the slightly contorted code path for a space.
%    \begin{macrocode}
\cs_new:Npn \msg_fatal_text:n #1
  {
    Fatal ~
    \msg_error_text:n {#1}
  }
\cs_new:Npn \msg_critical_text:n #1
  {
    Critical ~
    \msg_error_text:n {#1}
  }
\cs_new:Npn \msg_error_text:n #1
  { \@@_text:nn {#1} { Error } }
\cs_new:Npn \msg_warning_text:n #1
  { \@@_text:nn {#1} { Warning } }
\cs_new:Npn \msg_info_text:n #1
  { \@@_text:nn {#1} { Info } }
\cs_new:Npn \@@_text:nn #1#2
  {
    \exp_args:Nf \@@_text:n { \msg_module_type:n {#1} }
    \msg_module_name:n {#1} ~
    #2
  }
\cs_new:Npn \@@_text:n #1
  {
    \tl_if_blank:nF {#1}
      { #1 ~ }
  }
%    \end{macrocode}
% \end{macro}
% \end{macro}
% \end{macro}
%
% \begin{variable}{\g_msg_module_name_prop, \g_msg_module_type_prop}
%   For storing public module information: the kernel data is set up
%   in advance.
%    \begin{macrocode}
\prop_new:N \g_msg_module_name_prop
\prop_gput:Nnn \g_msg_module_name_prop { LaTeX } { LaTeX3 }
\prop_new:N \g_msg_module_type_prop
\prop_gput:Nnn \g_msg_module_type_prop { LaTeX } { }
%    \end{macrocode}
% \end{variable}
%
% \begin{macro}[EXP]{\msg_module_type:n}
%   Contextual footer information, with the potential to give modules an
%   alternative name.
%    \begin{macrocode}
\cs_new:Npn \msg_module_type:n #1
  {
    \prop_if_in:NnTF \g_msg_module_type_prop {#1}
      { \prop_item:Nn \g_msg_module_type_prop {#1} }
%<*initex>
      { Module }
%</initex>
%<*package>
      { Package }
%</package>
  }
%    \end{macrocode}
% \end{macro}
%
% \begin{macro}[EXP]{\msg_module_name:n, \msg_see_documentation_text:n}
%   Contextual footer information, with the potential to give modules an
%   alternative name.
%    \begin{macrocode}
\cs_new:Npn \msg_module_name:n #1
  {
    \prop_if_in:NnTF \g_msg_module_name_prop {#1}
      { \prop_item:Nn \g_msg_module_name_prop {#1} }
      {#1}
  }
\cs_new:Npn \msg_see_documentation_text:n #1
  {
    See~the~ \msg_module_name:n {#1} ~
    documentation~for~further~information.
  }
%    \end{macrocode}
% \end{macro}
%
% \begin{macro}{\@@_class_new:nn}
%    \begin{macrocode}
\group_begin:
  \cs_set_protected:Npn \@@_class_new:nn #1#2
    {
      \prop_new:c { l_@@_redirect_ #1 _prop }
      \cs_new_protected:cpn { @@_ #1 _code:nnnnnn }
          ##1##2##3##4##5##6 {#2}
      \cs_new_protected:cpn { msg_ #1 :nnnnnn } ##1##2##3##4##5##6
        {
          \use:x
            {
              \exp_not:n { \@@_use:nnnnnnn {#1} {##1} {##2} }
                { \tl_to_str:n {##3} } { \tl_to_str:n {##4} }
                { \tl_to_str:n {##5} } { \tl_to_str:n {##6} }
            }
        }
      \cs_new_protected:cpx { msg_ #1 :nnnnn } ##1##2##3##4##5
        { \exp_not:c { msg_ #1 :nnnnnn } {##1} {##2} {##3} {##4} {##5} { } }
      \cs_new_protected:cpx { msg_ #1 :nnnn } ##1##2##3##4
        { \exp_not:c { msg_ #1 :nnnnnn } {##1} {##2} {##3} {##4} { } { } }
      \cs_new_protected:cpx { msg_ #1 :nnn } ##1##2##3
        { \exp_not:c { msg_ #1 :nnnnnn } {##1} {##2} {##3} { } { } { } }
      \cs_new_protected:cpx { msg_ #1 :nn } ##1##2
        { \exp_not:c { msg_ #1 :nnnnnn } {##1} {##2} { } { } { } { } }
      \cs_new_protected:cpx { msg_ #1 :nnxxxx } ##1##2##3##4##5##6
        {
          \use:x
            {
              \exp_not:N \exp_not:n
                { \exp_not:c { msg_ #1 :nnnnnn } {##1} {##2} }
                {##3} {##4} {##5} {##6}
            }
        }
      \cs_new_protected:cpx { msg_ #1 :nnxxx } ##1##2##3##4##5
        { \exp_not:c { msg_ #1 :nnxxxx } {##1} {##2} {##3} {##4} {##5} { } }
      \cs_new_protected:cpx { msg_ #1 :nnxx } ##1##2##3##4
        { \exp_not:c { msg_ #1 :nnxxxx } {##1} {##2} {##3} {##4} { } { } }
      \cs_new_protected:cpx { msg_ #1 :nnx } ##1##2##3
        { \exp_not:c { msg_ #1 :nnxxxx } {##1} {##2} {##3} { } { } { } }
    }
%    \end{macrocode}
% \end{macro}
%
% \begin{macro}
%   {
%     \msg_fatal:nnnnnn ,
%     \msg_fatal:nnnnn  ,
%     \msg_fatal:nnnn   ,
%     \msg_fatal:nnn    ,
%     \msg_fatal:nn     ,
%     \msg_fatal:nnxxxx ,
%     \msg_fatal:nnxxx  ,
%     \msg_fatal:nnxx   ,
%     \msg_fatal:nnx
%   }
%  \begin{macro}{\@@_fatal_exit:}
%   For fatal errors, after the error message \TeX{} bails out. We force
%   a bail out rather than using \tn{end} as this means it does not
%   matter if we are in a context where normally the run cannot end.
%    \begin{macrocode}
  \@@_class_new:nn { fatal }
    {
      \@@_interrupt:NnnnN
        \msg_fatal_text:n {#1} {#2}
        { {#3} {#4} {#5} {#6} }
        \c_@@_fatal_text_tl
      \@@_fatal_exit:
    }
  \cs_new_protected:Npn \@@_fatal_exit:
    {
      \tex_batchmode:D
      \tex_read:D -1 to \l_@@_internal_tl
    }
%    \end{macrocode}
% \end{macro}
% \end{macro}
%
% \begin{macro}
%   {
%     \msg_critical:nnnnnn ,
%     \msg_critical:nnnnn  ,
%     \msg_critical:nnnn   ,
%     \msg_critical:nnn    ,
%     \msg_critical:nn     ,
%     \msg_critical:nnxxxx ,
%     \msg_critical:nnxxx  ,
%     \msg_critical:nnxx   ,
%     \msg_critical:nnx
%   }
%   Not quite so bad: just end the current file.
%    \begin{macrocode}
  \@@_class_new:nn { critical }
    {
      \@@_interrupt:NnnnN
        \msg_critical_text:n {#1} {#2}
        { {#3} {#4} {#5} {#6} }
        \c_@@_critical_text_tl
      \tex_endinput:D
    }
%    \end{macrocode}
% \end{macro}
%
% \begin{macro}
%   {
%     \msg_error:nnnnnn ,
%     \msg_error:nnnnn  ,
%     \msg_error:nnnn   ,
%     \msg_error:nnn    ,
%     \msg_error:nn     ,
%     \msg_error:nnxxxx ,
%     \msg_error:nnxxx  ,
%     \msg_error:nnxx   ,
%     \msg_error:nnx
%   }
%   For an error, the interrupt routine is called.  We check if there is
%   a \enquote{more text} by comparing that control sequence with a
%   permanently empty text.
%    \begin{macrocode}
  \@@_class_new:nn { error }
    {
      \@@_interrupt:NnnnN
        \msg_error_text:n {#1} {#2}
        { {#3} {#4} {#5} {#6} }
        \c_empty_tl
    }
%    \end{macrocode}
% \end{macro}
%
% \begin{macro}
%   {
%     \msg_warning:nnnnnn ,
%     \msg_warning:nnnnn  ,
%     \msg_warning:nnnn   ,
%     \msg_warning:nnn    ,
%     \msg_warning:nn     ,
%     \msg_warning:nnxxxx ,
%     \msg_warning:nnxxx  ,
%     \msg_warning:nnxx   ,
%     \msg_warning:nnx
%   }
%   Warnings are printed to the terminal.
%    \begin{macrocode}
  \@@_class_new:nn { warning }
    {
      \str_set:Nx \l_@@_text_str { \msg_warning_text:n {#1} }
      \str_set:Nx \l_@@_name_str { \msg_module_name:n {#1} }
      \iow_term:n { }
      \iow_wrap:nxnN
        {
          \l_@@_text_str : ~
          \use:c { \c_@@_text_prefix_tl #1 / #2 } {#3} {#4} {#5} {#6}
        }
        {
          ( \l_@@_name_str )
          \prg_replicate:nn
            {
                \str_count:N \l_@@_text_str
              - \str_count:N \l_@@_name_str
            }
            { ~ }
        }
        { } \iow_term:n
      \iow_term:n { }
    }
%    \end{macrocode}
% \end{macro}
%
% \begin{macro}
%   {
%     \msg_info:nnnnnn ,
%     \msg_info:nnnnn  ,
%     \msg_info:nnnn   ,
%     \msg_info:nnn    ,
%     \msg_info:nn     ,
%     \msg_info:nnxxxx ,
%     \msg_info:nnxxx  ,
%     \msg_info:nnxx   ,
%     \msg_info:nnx
%   }
%   Information only goes into the log.
%    \begin{macrocode}
  \@@_class_new:nn { info }
    {
      \str_set:Nx \l_@@_text_str { \msg_info_text:n {#1} }
      \str_set:Nx \l_@@_name_str { \msg_module_name:n {#1} }
      \iow_log:n { }
      \iow_wrap:nxnN
        {
          \l_@@_text_str : ~
          \use:c { \c_@@_text_prefix_tl #1 / #2 } {#3} {#4} {#5} {#6}
        }
        {
          ( \l_@@_name_str )
          \prg_replicate:nn
             {
                 \str_count:N \l_@@_text_str
               - \str_count:N \l_@@_name_str
             }
            { ~ }
         }
         { } \iow_log:n
      \iow_log:n { }
    }
%    \end{macrocode}
% \end{macro}
%
% \begin{macro}
%   {
%     \msg_log:nnnnnn ,
%     \msg_log:nnnnn  ,
%     \msg_log:nnnn   ,
%     \msg_log:nnn    ,
%     \msg_log:nn     ,
%     \msg_log:nnxxxx ,
%     \msg_log:nnxxx  ,
%     \msg_log:nnxx   ,
%     \msg_log:nnx
%   }
%   \enquote{Log} data is very similar to information, but with no extras
%   added.
%    \begin{macrocode}
  \@@_class_new:nn { log }
    {
      \iow_wrap:nnnN
        { \use:c { \c_@@_text_prefix_tl #1 / #2 } {#3} {#4} {#5} {#6} }
        { } { } \iow_log:n
    }
%    \end{macrocode}
% \end{macro}
%
% \begin{macro}
%   {
%     \msg_none:nnnnnn ,
%     \msg_none:nnnnn  ,
%     \msg_none:nnnn   ,
%     \msg_none:nnn    ,
%     \msg_none:nn     ,
%     \msg_none:nnxxxx ,
%     \msg_none:nnxxx  ,
%     \msg_none:nnxx   ,
%     \msg_none:nnx
%   }
%   The \texttt{none} message type is needed so that input can be gobbled.
%    \begin{macrocode}
  \@@_class_new:nn { none } { }
%    \end{macrocode}
% \end{macro}
%
% \begin{macro}
%   {
%     \msg_show:nnnnnn ,
%     \msg_show:nnnnn  ,
%     \msg_show:nnnn   ,
%     \msg_show:nnn    ,
%     \msg_show:nn     ,
%     \msg_show:nnxxxx ,
%     \msg_show:nnxxx  ,
%     \msg_show:nnxx   ,
%     \msg_show:nnx
%   }
% \begin{macro}{\@@_show:n, \@@_show:w, \@@_show_dot:w, \@@_show:nn}
%   The \texttt{show} message type is used for \cs{seq_show:N} and
%   similar complicated data structures.  Wrap the given text with a
%   trailing dot (important later) then pass it to \cs{@@_show:n}.  If
%   there is |\\>~| (or if the whole thing starts with |>~|) we split
%   there, print the first part and show the second part using
%   \tn{showtokens} (the \cs{exp_after:wN} ensure a nice display).  Note
%   that that primitive adds a leading |>~| and trailing dot.  That is
%   why we included a trailing dot before wrapping and removed it
%   afterwards.  If there is no |\\>~| do the same but with an empty
%   second part which adds a spurious but inevitable |>~.|
%    \begin{macrocode}
  \@@_class_new:nn { show }
    {
      \iow_wrap:nnnN
        { \use:c { \c_@@_text_prefix_tl #1 / #2 } {#3} {#4} {#5} {#6} }
        { } { } \@@_show:n
    }
  \cs_new_protected:Npn \@@_show:n #1
    {
      \tl_if_in:nnTF { ^^J #1 } { ^^J > ~ }
        {
          \tl_if_in:nnTF { #1 \q_mark } { . \q_mark }
            { \@@_show_dot:w } { \@@_show:w }
          ^^J #1 \q_stop
        }
        { \@@_show:nn { ? #1 } { } }
    }
  \cs_new:Npn \@@_show_dot:w #1 ^^J > ~ #2 . \q_stop
    { \@@_show:nn {#1} {#2} }
  \cs_new:Npn \@@_show:w #1 ^^J > ~ #2 \q_stop
    { \@@_show:nn {#1} {#2} }
  \cs_new_protected:Npn \@@_show:nn #1#2
    {
      \tl_if_empty:nF {#1}
        { \exp_args:No \iow_term:n { \use_none:n #1 } }
      \tl_set:Nn \l_@@_internal_tl {#2}
      \__kernel_iow_with:Nnn \tex_newlinechar:D { 10 }
        {
          \__kernel_iow_with:Nnn \tex_errorcontextlines:D { -1 }
            {
              \tex_showtokens:D \exp_after:wN \exp_after:wN \exp_after:wN
                { \exp_after:wN \l_@@_internal_tl }
            }
        }
    }
%    \end{macrocode}
% \end{macro}
% \end{macro}
%
% End the group to eliminate \cs{@@_class_new:nn}.
%    \begin{macrocode}
\group_end:
%    \end{macrocode}
%
% \begin{macro}{\@@_class_chk_exist:nT}
%   Checking that a message class exists.  We build this from
%   \cs{cs_if_free:cTF} rather than \cs{cs_if_exist:cTF} because that
%   avoids reading the second argument earlier than necessary.
%    \begin{macrocode}
\cs_new:Npn \@@_class_chk_exist:nT #1
  {
    \cs_if_free:cTF { @@_ #1 _code:nnnnnn }
      { \__kernel_msg_error:nnx { kernel } { message-class-unknown } {#1} }
  }
%    \end{macrocode}
% \end{macro}
%
% \begin{variable}{\l_@@_class_tl, \l_@@_current_class_tl}
%   Support variables needed for the redirection system.
%    \begin{macrocode}
\tl_new:N \l_@@_class_tl
\tl_new:N \l_@@_current_class_tl
%    \end{macrocode}
% \end{variable}
%
% \begin{variable}{\l_@@_redirect_prop}
%   For redirection of individually-named messages
%    \begin{macrocode}
\prop_new:N \l_@@_redirect_prop
%    \end{macrocode}
% \end{variable}
%
% \begin{variable}{\l_@@_hierarchy_seq}
%   During redirection, split the message name into a sequence:
%   |{/module/submodule}|, |{/module}|, and |{}|.
%    \begin{macrocode}
\seq_new:N \l_@@_hierarchy_seq
%    \end{macrocode}
% \end{variable}
%
% \begin{variable}{\l_@@_class_loop_seq}
%   Classes encountered when following redirections to check for loops.
%    \begin{macrocode}
\seq_new:N \l_@@_class_loop_seq
%    \end{macrocode}
% \end{variable}
%
% \begin{macro}{\@@_use:nnnnnnn}
% \begin{macro}
%   {
%     \@@_use_redirect_name:n  , \@@_use_hierarchy:nwwN ,
%     \@@_use_redirect_module:n, \@@_use_code:
%   }
%   Actually using a message is a multi-step process.  First, some
%   safety checks on the message and class requested.  The code and
%   arguments are then stored to avoid passing them around.  The
%   assignment to \cs{@@_use_code:} is similar to \cs{tl_set:Nn}.
%   The message is eventually produced with whatever \cs{l_@@_class_tl}
%   is when \cs{@@_use_code:} is called.
%   Here is also a good place to suppress tracing output if the
%   \pkg{trace} package is loaded since all (non-expandable) messages go
%   through this auxiliary.
%    \begin{macrocode}
\cs_new_protected:Npn \@@_use:nnnnnnn #1#2#3#4#5#6#7
  {
%<package>    \use:c { conditionally@traceoff }
    \msg_if_exist:nnTF {#2} {#3}
      {
        \@@_class_chk_exist:nT {#1}
          {
            \tl_set:Nn \l_@@_current_class_tl {#1}
            \cs_set_protected:Npx \@@_use_code:
              {
                \exp_not:n
                  {
                    \use:c { @@_ \l_@@_class_tl _code:nnnnnn }
                      {#2} {#3} {#4} {#5} {#6} {#7}
                  }
              }
            \@@_use_redirect_name:n { #2 / #3 }
          }
      }
      { \__kernel_msg_error:nnxx { kernel } { message-unknown } {#2} {#3} }
%<package>    \use:c { conditionally@traceon }
  }
\cs_new_protected:Npn \@@_use_code: { }
%    \end{macrocode}
%   The first check is for a individual message redirection. If this
%   applies then no further redirection is attempted.  Otherwise, split
%   the message name into \meta{module}, \meta{submodule} and \meta{message}
%   (with an
%   arbitrary number of slashes), and store |{/module/submodule}|,
%   |{/module}| and |{}| into \cs{l_@@_hierarchy_seq}.  We then
%   map through this sequence, applying the most specific redirection.
%    \begin{macrocode}
\cs_new_protected:Npn \@@_use_redirect_name:n #1
  {
    \prop_get:NnNTF \l_@@_redirect_prop { / #1 } \l_@@_class_tl
      { \@@_use_code: }
      {
        \seq_clear:N \l_@@_hierarchy_seq
        \@@_use_hierarchy:nwwN { }
          #1 \q_mark \@@_use_hierarchy:nwwN
          /  \q_mark \use_none_delimit_by_q_stop:w
          \q_stop
        \@@_use_redirect_module:n { }
      }
  }
\cs_new_protected:Npn \@@_use_hierarchy:nwwN #1#2 / #3 \q_mark #4
  {
    \seq_put_left:Nn \l_@@_hierarchy_seq {#1}
    #4 { #1 / #2 } #3 \q_mark #4
  }
%    \end{macrocode}
%   At this point, the items of \cs{l_@@_hierarchy_seq} are the
%   various levels at which we should look for a redirection.
%   Redirections which are less specific than the argument of
%   \cs{@@_use_redirect_module:n} are not attempted.  This argument is
%   empty for a class redirection, \texttt{/module} for a module
%   redirection, \emph{etc.}  Loop through the sequence to find the most
%   specific redirection, with module |##1|.  The loop is interrupted
%   after testing for a redirection for |##1| equal to the argument |#1|
%   (least specific redirection allowed).  When a redirection is found,
%   break the mapping, then if the redirection targets the same class,
%   output the code with that class, and otherwise set the target as the
%   new current class, and search for further redirections.  Those
%   redirections should be at least as specific as |##1|.
%    \begin{macrocode}
\cs_new_protected:Npn \@@_use_redirect_module:n #1
  {
    \seq_map_inline:Nn \l_@@_hierarchy_seq
      {
        \prop_get:cnNTF { l_@@_redirect_ \l_@@_current_class_tl _prop }
          {##1} \l_@@_class_tl
          {
            \seq_map_break:n
              {
                \tl_if_eq:NNTF \l_@@_current_class_tl \l_@@_class_tl
                  { \@@_use_code: }
                  {
                    \tl_set_eq:NN \l_@@_current_class_tl \l_@@_class_tl
                    \@@_use_redirect_module:n {##1}
                  }
              }
          }
          {
            \str_if_eq:nnT {##1} {#1}
              {
                \tl_set_eq:NN \l_@@_class_tl \l_@@_current_class_tl
                \seq_map_break:n { \@@_use_code: }
              }
          }
      }
  }
%    \end{macrocode}
% \end{macro}
% \end{macro}
%
% \begin{macro}{\msg_redirect_name:nnn}
%   Named message always use the given class even if that class is
%   redirected further.  An empty target class cancels any existing
%   redirection for that message.
%    \begin{macrocode}
\cs_new_protected:Npn \msg_redirect_name:nnn #1#2#3
  {
    \tl_if_empty:nTF {#3}
      { \prop_remove:Nn \l_@@_redirect_prop { / #1 / #2 } }
      {
        \@@_class_chk_exist:nT {#3}
          { \prop_put:Nnn \l_@@_redirect_prop { / #1 / #2 } {#3} }
      }
  }
%    \end{macrocode}
% \end{macro}
%
% \begin{macro}{\msg_redirect_class:nn, \msg_redirect_module:nnn}
% \begin{macro}{\@@_redirect:nnn, \@@_redirect_loop_chk:nnn}
% \begin{macro}{\@@_redirect_loop_list:n}
%   If the target class is empty, eliminate the corresponding
%   redirection.  Otherwise, add the redirection.  We must then check
%   for a loop: as an initialization, we start by storing the initial
%   class in \cs{l_@@_current_class_tl}.
%    \begin{macrocode}
\cs_new_protected:Npn \msg_redirect_class:nn
  { \@@_redirect:nnn { } }
\cs_new_protected:Npn \msg_redirect_module:nnn #1
  { \@@_redirect:nnn { / #1 } }
\cs_new_protected:Npn \@@_redirect:nnn #1#2#3
  {
    \@@_class_chk_exist:nT {#2}
      {
        \tl_if_empty:nTF {#3}
          { \prop_remove:cn { l_@@_redirect_ #2 _prop } {#1} }
          {
            \@@_class_chk_exist:nT {#3}
              {
                \prop_put:cnn { l_@@_redirect_ #2 _prop } {#1} {#3}
                \tl_set:Nn \l_@@_current_class_tl {#2}
                \seq_clear:N \l_@@_class_loop_seq
                \@@_redirect_loop_chk:nnn {#2} {#3} {#1}
              }
          }
      }
  }
%    \end{macrocode}
%   Since multiple redirections can only happen with increasing
%   specificity, a loop requires that all steps are of the same
%   specificity.  The new redirection can thus only create a loop with
%   other redirections for the exact same module, |#1|, and not
%   submodules.  After some initialization above, follow redirections
%   with \cs{l_@@_class_tl}, and keep track in
%   \cs{l_@@_class_loop_seq} of the various classes encountered.  A
%   redirection from a class to itself, or the absence of redirection
%   both mean that there is no loop.  A redirection to the initial class
%   marks a loop.  To break it, we must decide which redirection to
%   cancel.  The user most likely wants the newly added redirection to
%   hold with no further redirection.  We thus remove the redirection
%   starting from |#2|, target of the new redirection.  Note that no
%   message is emitted by any of the underlying functions: otherwise we
%   may get an infinite loop because of a message from the message
%   system itself.
%    \begin{macrocode}
\cs_new_protected:Npn \@@_redirect_loop_chk:nnn #1#2#3
  {
    \seq_put_right:Nn \l_@@_class_loop_seq {#1}
    \prop_get:cnNT { l_@@_redirect_ #1 _prop } {#3} \l_@@_class_tl
      {
        \str_if_eq:VnF \l_@@_class_tl {#1}
          {
            \tl_if_eq:NNTF \l_@@_class_tl \l_@@_current_class_tl
              {
                \prop_put:cnn { l_@@_redirect_ #2 _prop } {#3} {#2}
                \__kernel_msg_warning:nnxxxx
                  { kernel } { message-redirect-loop }
                  { \seq_item:Nn \l_@@_class_loop_seq { 1 } }
                  { \seq_item:Nn \l_@@_class_loop_seq { 2 } }
                  {#3}
                  {
                    \seq_map_function:NN \l_@@_class_loop_seq
                      \@@_redirect_loop_list:n
                    { \seq_item:Nn \l_@@_class_loop_seq { 1 } }
                  }
              }
              { \@@_redirect_loop_chk:onn \l_@@_class_tl {#2} {#3} }
          }
      }
  }
\cs_generate_variant:Nn \@@_redirect_loop_chk:nnn { o }
\cs_new:Npn \@@_redirect_loop_list:n #1 { {#1} ~ => ~ }
%    \end{macrocode}
% \end{macro}
% \end{macro}
% \end{macro}
%
% \subsection{Kernel-specific functions}
%
% \begin{macro}{\__kernel_msg_new:nnnn, \__kernel_msg_new:nnn}
% \begin{macro}{\__kernel_msg_set:nnnn, \__kernel_msg_set:nnn}
%   The kernel needs some messages of its own. These are created using
%   pre-built functions. Two functions are provided: one more general
%   and one which only has the short text part.
%    \begin{macrocode}
\cs_new_protected:Npn \__kernel_msg_new:nnnn #1#2
  { \msg_new:nnnn { LaTeX } { #1 / #2 } }
\cs_new_protected:Npn \__kernel_msg_new:nnn #1#2
  { \msg_new:nnn { LaTeX } { #1 / #2 } }
\cs_new_protected:Npn \__kernel_msg_set:nnnn #1#2
  { \msg_set:nnnn { LaTeX } { #1 / #2 } }
\cs_new_protected:Npn \__kernel_msg_set:nnn #1#2
  { \msg_set:nnn { LaTeX } { #1 / #2 } }
%    \end{macrocode}
% \end{macro}
% \end{macro}
%
% \begin{macro}{\@@_kernel_class_new:nN}
% \begin{macro}{\@@_kernel_class_new_aux:nN}
%   All the functions for kernel messages come in variants ranging
%   from~$0$ to~$4$ arguments.  Those with less than~$4$ arguments are
%   defined in terms of the $4$-argument variant, in a way very similar
%   to \cs{@@_class_new:nn}.  This auxiliary is destroyed at the end of
%   the group.
%    \begin{macrocode}
\group_begin:
  \cs_set_protected:Npn \@@_kernel_class_new:nN #1
    { \@@_kernel_class_new_aux:nN { __kernel_msg_ #1 } }
  \cs_set_protected:Npn \@@_kernel_class_new_aux:nN #1#2
    {
      \cs_new_protected:cpn { #1 :nnnnnn } ##1##2##3##4##5##6
        {
          \use:x
            {
              \exp_not:n { #2 { LaTeX } { ##1 / ##2 } }
                { \tl_to_str:n {##3} } { \tl_to_str:n {##4} }
                { \tl_to_str:n {##5} } { \tl_to_str:n {##6} }
            }
        }
      \cs_new_protected:cpx { #1 :nnnnn } ##1##2##3##4##5
        { \exp_not:c { #1 :nnnnnn } {##1} {##2} {##3} {##4} {##5} { } }
      \cs_new_protected:cpx { #1 :nnnn } ##1##2##3##4
        { \exp_not:c { #1 :nnnnnn } {##1} {##2} {##3} {##4} { } { } }
      \cs_new_protected:cpx { #1 :nnn } ##1##2##3
        { \exp_not:c { #1 :nnnnnn } {##1} {##2} {##3} { } { } { } }
      \cs_new_protected:cpx { #1 :nn } ##1##2
        { \exp_not:c { #1 :nnnnnn } {##1} {##2} { } { } { } { } }
      \cs_new_protected:cpx { #1 :nnxxxx } ##1##2##3##4##5##6
        {
          \use:x
            {
              \exp_not:N \exp_not:n
                { \exp_not:c { #1 :nnnnnn } {##1} {##2} }
                {##3} {##4} {##5} {##6}
            }
        }
      \cs_new_protected:cpx { #1 :nnxxx } ##1##2##3##4##5
        { \exp_not:c { #1 :nnxxxx } {##1} {##2} {##3} {##4} {##5} { } }
      \cs_new_protected:cpx { #1 :nnxx } ##1##2##3##4
        { \exp_not:c { #1 :nnxxxx } {##1} {##2} {##3} {##4} { } { } }
      \cs_new_protected:cpx { #1 :nnx } ##1##2##3
        { \exp_not:c { #1 :nnxxxx } {##1} {##2} {##3} { } { } { } }
    }
%    \end{macrocode}
% \end{macro}
% \end{macro}
%
% \begin{macro}
%   {
%     \__kernel_msg_fatal:nnnnnn ,
%     \__kernel_msg_fatal:nnnnn  ,
%     \__kernel_msg_fatal:nnnn   ,
%     \__kernel_msg_fatal:nnn    ,
%     \__kernel_msg_fatal:nn     ,
%     \__kernel_msg_fatal:nnxxxx ,
%     \__kernel_msg_fatal:nnxxx  ,
%     \__kernel_msg_fatal:nnxx   ,
%     \__kernel_msg_fatal:nnx    ,
%     \__kernel_msg_error:nnnnnn ,
%     \__kernel_msg_error:nnnnn  ,
%     \__kernel_msg_error:nnnn   ,
%     \__kernel_msg_error:nnn    ,
%     \__kernel_msg_error:nn     ,
%     \__kernel_msg_error:nnxxxx ,
%     \__kernel_msg_error:nnxxx  ,
%     \__kernel_msg_error:nnxx   ,
%     \__kernel_msg_error:nnx
%   }
%   Neither fatal kernel errors nor kernel errors can be redirected.  We
%   directly use the code for (non-kernel) fatal errors and errors,
%   adding the \enquote{\LaTeX{}} module name.  Three functions are
%   already defined by \pkg{l3basics}; we need to undefine them to avoid
%   errors.
%    \begin{macrocode}
  \@@_kernel_class_new:nN { fatal } \@@_fatal_code:nnnnnn
  \cs_undefine:N \__kernel_msg_error:nnxx
  \cs_undefine:N \__kernel_msg_error:nnx
  \cs_undefine:N \__kernel_msg_error:nn
  \@@_kernel_class_new:nN { error } \@@_error_code:nnnnnn
%    \end{macrocode}
% \end{macro}
%
% \begin{macro}
%   {
%     \__kernel_msg_warning:nnnnnn ,
%     \__kernel_msg_warning:nnnnn  ,
%     \__kernel_msg_warning:nnnn   ,
%     \__kernel_msg_warning:nnn    ,
%     \__kernel_msg_warning:nn     ,
%     \__kernel_msg_warning:nnxxxx ,
%     \__kernel_msg_warning:nnxxx  ,
%     \__kernel_msg_warning:nnxx   ,
%     \__kernel_msg_warning:nnx    ,
%     \__kernel_msg_info:nnnnnn    ,
%     \__kernel_msg_info:nnnnn     ,
%     \__kernel_msg_info:nnnn      ,
%     \__kernel_msg_info:nnn       ,
%     \__kernel_msg_info:nn        ,
%     \__kernel_msg_info:nnxxxx    ,
%     \__kernel_msg_info:nnxxx     ,
%     \__kernel_msg_info:nnxx      ,
%     \__kernel_msg_info:nnx
%   }
%   Kernel messages which can be redirected simply use the machinery for
%   normal messages, with the module name \enquote{\LaTeX{}}.
%    \begin{macrocode}
  \@@_kernel_class_new:nN { warning } \msg_warning:nnxxxx
  \@@_kernel_class_new:nN { info } \msg_info:nnxxxx
%    \end{macrocode}
% \end{macro}
%
% End the group to eliminate \cs{@@_kernel_class_new:nN}.
%    \begin{macrocode}
\group_end:
%    \end{macrocode}
%
% Error messages needed to actually implement the message system
% itself.
%    \begin{macrocode}
\__kernel_msg_new:nnnn { kernel } { message-already-defined }
  { Message~'#2'~for~module~'#1'~already~defined. }
  {
    \c_@@_coding_error_text_tl
    LaTeX~was~asked~to~define~a~new~message~called~'#2'\\
    by~the~module~'#1':~this~message~already~exists.
    \c_@@_return_text_tl
  }
\__kernel_msg_new:nnnn { kernel } { message-unknown }
  { Unknown~message~'#2'~for~module~'#1'. }
  {
    \c_@@_coding_error_text_tl
    LaTeX~was~asked~to~display~a~message~called~'#2'\\
    by~the~module~'#1':~this~message~does~not~exist.
    \c_@@_return_text_tl
  }
\__kernel_msg_new:nnnn { kernel } { message-class-unknown }
  { Unknown~message~class~'#1'. }
  {
    LaTeX~has~been~asked~to~redirect~messages~to~a~class~'#1':\\
    this~was~never~defined.
    \c_@@_return_text_tl
  }
\__kernel_msg_new:nnnn { kernel } { message-redirect-loop }
  {
    Message~redirection~loop~caused~by~ {#1} ~=>~ {#2}
    \tl_if_empty:nF {#3} { ~for~module~' \use_none:n #3 ' } .
  }
  {
    Adding~the~message~redirection~ {#1} ~=>~ {#2}
    \tl_if_empty:nF {#3} { ~for~the~module~' \use_none:n #3 ' } ~
    created~an~infinite~loop\\\\
    \iow_indent:n { #4 \\\\ }
  }
%    \end{macrocode}
%
% Messages for earlier kernel modules plus a few for \pkg{l3keys} which
% cover coding errors.
%    \begin{macrocode}
\__kernel_msg_new:nnnn { kernel } { bad-number-of-arguments }
  { Function~'#1'~cannot~be~defined~with~#2~arguments. }
  {
    \c_@@_coding_error_text_tl
    LaTeX~has~been~asked~to~define~a~function~'#1'~with~
    #2~arguments.~
    TeX~allows~between~0~and~9~arguments~for~a~single~function.
  }
\__kernel_msg_new:nnn { kernel } { char-active }
  { Cannot~generate~active~chars. }
\__kernel_msg_new:nnn { kernel } { char-invalid-catcode }
  { Invalid~catcode~for~char~generation. }
\__kernel_msg_new:nnn { kernel } { char-null-space }
  { Cannot~generate~null~char~as~a~space. }
\__kernel_msg_new:nnn { kernel } { char-out-of-range }
  { Charcode~requested~out~of~engine~range. }
\__kernel_msg_new:nnn { kernel } { char-space }
  { Cannot~generate~space~chars. }
\__kernel_msg_new:nnnn { kernel } { command-already-defined }
  { Control~sequence~#1~already~defined. }
  {
    \c_@@_coding_error_text_tl
    LaTeX~has~been~asked~to~create~a~new~control~sequence~'#1'~
    but~this~name~has~already~been~used~elsewhere. \\ \\
    The~current~meaning~is:\\
    \ \ #2
  }
\__kernel_msg_new:nnnn { kernel } { command-not-defined }
  { Control~sequence~#1~undefined. }
  {
    \c_@@_coding_error_text_tl
    LaTeX~has~been~asked~to~use~a~control~sequence~'#1':\\
    this~has~not~been~defined~yet.
  }
\__kernel_msg_new:nnnn { kernel } { empty-search-pattern }
  { Empty~search~pattern. }
  {
    \c_@@_coding_error_text_tl
    LaTeX~has~been~asked~to~replace~an~empty~pattern~by~'#1':~that~
    would~lead~to~an~infinite~loop!
  }
\__kernel_msg_new:nnnn { kernel } { out-of-registers }
  { No~room~for~a~new~#1. }
  {
    TeX~only~supports~\int_use:N \c_max_register_int \ %
    of~each~type.~All~the~#1~registers~have~been~used.~
    This~run~will~be~aborted~now.
  }
\__kernel_msg_new:nnnn { kernel } { non-base-function }
  { Function~'#1'~is~not~a~base~function }
  {
    \c_@@_coding_error_text_tl
    Functions~defined~through~\iow_char:N\\cs_new:Nn~must~have~
    a~signature~consisting~of~only~normal~arguments~'N'~and~'n'.~
    To~define~variants~use~\iow_char:N\\cs_generate_variant:Nn~
    and~to~define~other~functions~use~\iow_char:N\\cs_new:Npn.
  }
\__kernel_msg_new:nnnn { kernel } { missing-colon }
  { Function~'#1'~contains~no~':'. }
  {
    \c_@@_coding_error_text_tl
    Code-level~functions~must~contain~':'~to~separate~the~
    argument~specification~from~the~function~name.~This~is~
    needed~when~defining~conditionals~or~variants,~or~when~building~a~
    parameter~text~from~the~number~of~arguments~of~the~function.
  }
\__kernel_msg_new:nnnn { kernel } { overflow }
  { Integers~larger~than~2^{30}-1~cannot~be~stored~in~arrays. }
  {
    An~attempt~was~made~to~store~#3~
    \tl_if_empty:nF {#2} { at~position~#2~ } in~the~array~'#1'.~
    The~largest~allowed~value~#4~will~be~used~instead.
  }
\__kernel_msg_new:nnnn { kernel } { out-of-bounds }
  { Access~to~an~entry~beyond~an~array's~bounds. }
  {
    An~attempt~was~made~to~access~or~store~data~at~position~#2~of~the~
    array~'#1',~but~this~array~has~entries~at~positions~from~1~to~#3.
  }
\__kernel_msg_new:nnnn { kernel } { protected-predicate }
  { Predicate~'#1'~must~be~expandable. }
  {
    \c_@@_coding_error_text_tl
    LaTeX~has~been~asked~to~define~'#1'~as~a~protected~predicate.~
    Only~expandable~tests~can~have~a~predicate~version.
  }
\__kernel_msg_new:nnn { kernel } { randint-backward-range }
  { Bounds~ordered~backwards~in~\iow_char:N\\int_rand:nn~{#1}~{#2}. }
\__kernel_msg_new:nnnn { kernel } { conditional-form-unknown }
  { Conditional~form~'#1'~for~function~'#2'~unknown. }
  {
    \c_@@_coding_error_text_tl
    LaTeX~has~been~asked~to~define~the~conditional~form~'#1'~of~
    the~function~'#2',~but~only~'TF',~'T',~'F',~and~'p'~forms~exist.
  }
\__kernel_msg_new:nnnn { kernel } { key-no-property }
  { No~property~given~in~definition~of~key~'#1'. }
  {
    \c_@@_coding_error_text_tl
    Inside~\keys_define:nn  each~key~name~
    needs~a~property:  \\ \\
    \iow_indent:n { #1 .<property> } \\ \\
    LaTeX~did~not~find~a~'.'~to~indicate~the~start~of~a~property.
  }
\__kernel_msg_new:nnnn { kernel } { key-property-boolean-values-only }
  { The~property~'#1'~accepts~boolean~values~only. }
  {
    \c_@@_coding_error_text_tl
    The~property~'#1'~only~accepts~the~values~'true'~and~'false'.
  }
\__kernel_msg_new:nnnn { kernel } { key-property-requires-value }
  { The~property~'#1'~requires~a~value. }
  {
    \c_@@_coding_error_text_tl
    LaTeX~was~asked~to~set~property~'#1'~for~key~'#2'.\\
    No~value~was~given~for~the~property,~and~one~is~required.
  }
\__kernel_msg_new:nnnn { kernel } { key-property-unknown }
  { The~key~property~'#1'~is~unknown. }
  {
    \c_@@_coding_error_text_tl
    LaTeX~has~been~asked~to~set~the~property~'#1'~for~key~'#2':~
    this~property~is~not~defined.
  }
\__kernel_msg_new:nnnn { kernel } { scanmark-already-defined }
  { Scan~mark~#1~already~defined. }
  {
    \c_@@_coding_error_text_tl
    LaTeX~has~been~asked~to~create~a~new~scan~mark~'#1'~
    but~this~name~has~already~been~used~for~a~scan~mark.
  }
\__kernel_msg_new:nnnn { kernel } { variable-not-defined }
  { Variable~#1~undefined. }
  {
    \c_@@_coding_error_text_tl
    LaTeX~has~been~asked~to~show~a~variable~#1,~but~this~has~not~
    been~defined~yet.
  }
\__kernel_msg_new:nnnn { kernel } { variant-too-long }
  { Variant~form~'#1'~longer~than~base~signature~of~'#2'. }
  {
    \c_@@_coding_error_text_tl
    LaTeX~has~been~asked~to~create~a~variant~of~the~function~'#2'~
    with~a~signature~starting~with~'#1',~but~that~is~longer~than~
    the~signature~(part~after~the~colon)~of~'#2'.
  }
\__kernel_msg_new:nnnn { kernel } { invalid-variant }
  { Variant~form~'#1'~invalid~for~base~form~'#2'. }
  {
    \c_@@_coding_error_text_tl
    LaTeX~has~been~asked~to~create~a~variant~of~the~function~'#2'~
    with~a~signature~starting~with~'#1',~but~cannot~change~an~argument~
    from~type~'#3'~to~type~'#4'.
  }
\__kernel_msg_new:nnnn { kernel } { invalid-exp-args }
  { Invalid~variant~specifier~'#1'~in~'#2'. }
  {
    \c_@@_coding_error_text_tl
    LaTeX~has~been~asked~to~create~an~\iow_char:N\\exp_args:N...~
    function~with~signature~'N#2'~but~'#1'~is~not~a~valid~argument~
    specifier.
  }
\__kernel_msg_new:nnn { kernel } { deprecated-variant }
  {
    Variant~form~'#1'~deprecated~for~base~form~'#2'.~
    One~should~not~change~an~argument~from~type~'#3'~to~type~'#4'
    \str_case:nnF {#3}
      {
        { n } { :~use~a~'\token_if_eq_charcode:NNTF #4 c v V'~variant? }
        { N } { :~base~form~only~accepts~a~single~token~argument. }
        {#4} { :~base~form~is~already~a~variant. }
      } { . }
  }
%    \end{macrocode}
%
% Some errors are only needed in package mode if debugging is enabled by
% one of the options \texttt{enable-debug}, \texttt{check-declarations},
% \texttt{log-functions}, or on the contrary if debugging is turned off.
% In format mode the error is somewhat different.
%    \begin{macrocode}
%<*package>
\__kernel_if_debug:TF
  {
    \__kernel_msg_new:nnnn { kernel } { debug }
      { The~debugging~option~'#1'~does~not~exist~\msg_line_context:. }
      {
        The~functions~'\iow_char:N\\debug_on:n'~and~
        '\iow_char:N\\debug_off:n'~only~accept~the~arguments~
        'check-declarations',~'deprecation',~'log-functions',~not~'#1'.
      }
    \__kernel_msg_new:nnn { kernel } { expr } { '#2'~in~#1 }
    \__kernel_msg_new:nnnn { kernel } { local-global }
      { Inconsistent~local/global~assignment }
      {
        \c_@@_coding_error_text_tl
        \if:w l #2 Local
        \else:
          \if:w g #2 Global \else: Constant \fi:
        \fi:
        \ %
        assignment~to~a~
        \if:w l #1 local
        \else:
          \if:w g #1 global \else: constant \fi:
        \fi:
        \ %
        variable~'#3'.
      }
    \__kernel_msg_new:nnnn { kernel } { non-declared-variable }
      { The~variable~#1~has~not~been~declared~\msg_line_context:. }
      {
        \c_@@_coding_error_text_tl
        Checking~is~active,~and~you~have~tried~do~so~something~like: \\
        \ \ \tl_set:Nn ~ #1 ~ \{ ~ ... ~ \} \\
        without~first~having: \\
        \ \ \tl_new:N ~ #1  \\
        \\
        LaTeX~will~create~the~variable~and~continue.
      }
  }
  {
    \__kernel_msg_new:nnnn { kernel } { enable-debug }
      { To~use~'#1'~load~expl3~with~the~'enable-debug'~option. }
      {
        The~function~'#1'~will~be~ignored~because~it~can~only~work~if~
        some~internal~functions~in~expl3~have~been~appropriately~
        defined.~This~only~happens~if~one~of~the~options~
        'enable-debug',~'check-declarations'~or~'log-functions'~was~
        given~when~loading~expl3.
      }
  }
%</package>
%<*initex>
\__kernel_msg_new:nnnn { kernel } { enable-debug }
  { '#1'~cannot~be~used~in~format~mode. }
  {
    The~function~'#1'~will~be~ignored~because~it~can~only~work~if~
    some~internal~functions~in~expl3~have~been~appropriately~
    defined.~This~only~happens~in~package~mode~(and~only~if~one~of~
    the~options~'enable-debug',~'check-declarations'~or~'log-functions'~
    was~given~when~loading~expl3.
  }
%</initex>
%    \end{macrocode}
%
% Some errors only appear in expandable settings,
% hence don't need a \enquote{more-text} argument.
%    \begin{macrocode}
\__kernel_msg_new:nnn { kernel } { bad-exp-end-f }
  { Misused~\exp_end_continue_f:w or~:nw }
\__kernel_msg_new:nnn { kernel } { bad-variable }
  { Erroneous~variable~#1 used! }
\__kernel_msg_new:nnn { kernel } { misused-sequence }
  { A~sequence~was~misused. }
\__kernel_msg_new:nnn { kernel } { misused-prop }
  { A~property~list~was~misused. }
\__kernel_msg_new:nnn { kernel } { negative-replication }
  { Negative~argument~for~\iow_char:N\\prg_replicate:nn. }
\__kernel_msg_new:nnn { kernel } { prop-keyval }
  { Missing/extra~'='~in~'#1'~(in~'..._keyval:Nn') }
\__kernel_msg_new:nnn { kernel } { unknown-comparison }
  { Relation~'#1'~unknown:~use~=,~<,~>,~==,~!=,~<=,~>=. }
\__kernel_msg_new:nnn { kernel } { zero-step }
  { Zero~step~size~for~step~function~#1. }
\cs_if_exist:NF \tex_expanded:D
  {
    \__kernel_msg_new:nnn { kernel } { e-type }
      { #1 ~ in~e-type~argument }
  }
%    \end{macrocode}
%
% Messages used by the \enquote{\texttt{show}} functions.
%    \begin{macrocode}
\__kernel_msg_new:nnn { kernel } { show-clist }
  {
    The~comma~list~ \tl_if_empty:nF {#1} { #1 ~ }
    \tl_if_empty:nTF {#2}
      { is~empty \\>~ . }
      { contains~the~items~(without~outer~braces): #2 . }
  }
\__kernel_msg_new:nnn { kernel } { show-intarray }
  { The~integer~array~#1~contains~#2~items: \\ #3 . }
\__kernel_msg_new:nnn { kernel } { show-prop }
  {
    The~property~list~#1~
    \tl_if_empty:nTF {#2}
      { is~empty \\>~ . }
      { contains~the~pairs~(without~outer~braces): #2 . }
  }
\__kernel_msg_new:nnn { kernel } { show-seq }
  {
    The~sequence~#1~
    \tl_if_empty:nTF {#2}
      { is~empty \\>~ . }
      { contains~the~items~(without~outer~braces): #2 . }
  }
\__kernel_msg_new:nnn { kernel } { show-streams }
  {
    \tl_if_empty:nTF {#2} { No~ } { The~following~ }
    \str_case:nn {#1}
      {
        { ior } { input ~ }
        { iow } { output ~ }
      }
    streams~are~
    \tl_if_empty:nTF {#2} { open } { in~use: #2 . }
  }
%    \end{macrocode}
%
% \subsection{Expandable errors}
%
% \begin{macro}{\@@_expandable_error:n}
% \begin{macro}{\@@_expandable_error:w}
%   In expansion only context, we cannot use the normal means of
%   reporting errors. Instead, we feed \TeX{} an undefined control
%   sequence, \cs{LaTeX3 error:}. It is thus interrupted, and shows
%   the context, which thanks to the odd-looking \cs{use:n} is
%   \begin{verbatim}
%     <argument> \LaTeX3 error:
%                               The error message.
%   \end{verbatim}
%   In other words, \TeX{} is processing the argument of \cs{use:n},
%   which is \cs{LaTeX3 error:} \meta{error message}.
%   Then \cs{@@_expandable_error:w} cleans up. In fact, there
%   is an extra subtlety: if the user inserts tokens for error recovery,
%   they should be kept. Thus we also use an odd space character
%   (with category code $7$) and keep tokens until that space character,
%   dropping everything else until \cs{q_stop}. The \cs{exp_end:} prevents
%   losing braces around the user-inserted text if any, and stops the
%   expansion of \cs{exp:w}.  The group is used to prevent
%   |\LaTeX3~error:| from being globally equal to \cs{scan_stop:}.
%    \begin{macrocode}
\group_begin:
\cs_set_protected:Npn \@@_tmp:w #1#2
  {
    \cs_new:Npn \@@_expandable_error:n ##1
      {
        \exp:w
        \exp_after:wN \exp_after:wN
        \exp_after:wN \@@_expandable_error:w
        \exp_after:wN \exp_after:wN
        \exp_after:wN \exp_end:
        \use:n { #1 #2 ##1 } #2
      }
    \cs_new:Npn \@@_expandable_error:w ##1 #2 ##2 #2 {##1}
  }
\exp_args:Ncx \@@_tmp:w { LaTeX3~error: }
  { \char_generate:nn { `\  } { 7 } }
\group_end:
%    \end{macrocode}
% \end{macro}
% \end{macro}
%
% \begin{macro}
%   {
%     \__kernel_msg_expandable_error:nnnnnn,
%     \__kernel_msg_expandable_error:nnnnn,
%     \__kernel_msg_expandable_error:nnnn,
%     \__kernel_msg_expandable_error:nnn,
%     \__kernel_msg_expandable_error:nn,
%     \__kernel_msg_expandable_error:nnffff,
%     \__kernel_msg_expandable_error:nnfff,
%     \__kernel_msg_expandable_error:nnff,
%     \__kernel_msg_expandable_error:nnf,
%   }
%   The command built from the csname
%   \cs{c_@@_text_prefix_tl} |LaTeX / #1 / #2|
%   takes four arguments and builds the error text, which is fed to
%   \cs{@@_expandable_error:n} with appropriate expansion: just as for
%   usual messages the arguments are first turned to strings, then the
%   message is fully expanded.
%    \begin{macrocode}
\exp_args_generate:n { oooo }
\cs_new:Npn \__kernel_msg_expandable_error:nnnnnn #1#2#3#4#5#6
  {
    \exp_args:Ne \@@_expandable_error:n
      {
        \exp_args:Nc \exp_args:Noooo
          { \c_@@_text_prefix_tl LaTeX / #1 / #2 }
          { \tl_to_str:n {#3} }
          { \tl_to_str:n {#4} }
          { \tl_to_str:n {#5} }
          { \tl_to_str:n {#6} }
      }
  }
\cs_new:Npn \__kernel_msg_expandable_error:nnnnn #1#2#3#4#5
  {
    \__kernel_msg_expandable_error:nnnnnn
      {#1} {#2} {#3} {#4} {#5} { }
  }
\cs_new:Npn \__kernel_msg_expandable_error:nnnn #1#2#3#4
  {
    \__kernel_msg_expandable_error:nnnnnn
      {#1} {#2} {#3} {#4} { } { }
  }
\cs_new:Npn \__kernel_msg_expandable_error:nnn #1#2#3
  {
    \__kernel_msg_expandable_error:nnnnnn
      {#1} {#2} {#3} { } { } { }
  }
\cs_new:Npn \__kernel_msg_expandable_error:nn #1#2
  {
    \__kernel_msg_expandable_error:nnnnnn
      {#1} {#2} { } { } { } { }
  }
\cs_generate_variant:Nn \__kernel_msg_expandable_error:nnnnnn { nnffff }
\cs_generate_variant:Nn \__kernel_msg_expandable_error:nnnnn { nnfff }
\cs_generate_variant:Nn \__kernel_msg_expandable_error:nnnn { nnff }
\cs_generate_variant:Nn \__kernel_msg_expandable_error:nnn { nnf }
%    \end{macrocode}
% \end{macro}
%
%    \begin{macrocode}
%</initex|package>
%    \end{macrocode}
%
% \end{implementation}
%
% \PrintIndex
