% \iffalse meta-comment
%
%% File: l3debug.dtx
%
% Copyright (C) 2019-2022 The LaTeX Project
%
% It may be distributed and/or modified under the conditions of the
% LaTeX Project Public License (LPPL), either version 1.3c of this
% license or (at your option) any later version.  The latest version
% of this license is in the file
%
%    https://www.latex-project.org/lppl.txt
%
% This file is part of the "l3kernel bundle" (The Work in LPPL)
% and all files in that bundle must be distributed together.
%
% -----------------------------------------------------------------------
%
% The development version of the bundle can be found at
%
%    https://github.com/latex3/latex3
%
% for those people who are interested.
%
%<*driver>
\documentclass[full,kernel]{l3doc}
\begin{document}
  \DocInput{\jobname.dtx}
\end{document}
%</driver>
% \fi
%
% \title{^^A
%   The \textsf{l3debug} package\\ Debugging support^^A
% }
%
% \author{^^A
%  The \LaTeX{} Project\thanks
%    {^^A
%      E-mail:
%        \href{mailto:latex-team@latex-project.org}
%          {latex-team@latex-project.org}^^A
%    }^^A
% }
%
% \date{Released 2022-10-26}
%
% \maketitle
%
% \begin{documentation}
%
% \begin{function}{\__kernel_chk_var_local:N, \__kernel_chk_var_global:N}
%   \begin{syntax}
%     \cs{__kernel_chk_var_local:N} \meta{var}
%     \cs{__kernel_chk_var_global:N} \meta{var}
%   \end{syntax}
%   Applies \cs{__kernel_chk_var_exist:N} \meta{var} as well as
%   \cs{__kernel_chk_var_scope:NN} \meta{scope} \meta{var}, where
%   \meta{scope} is |l| or~|g|.
% \end{function}
%
% \begin{function}{\__kernel_chk_var_scope:NN}
%   \begin{syntax}
%     \cs{__kernel_chk_var_scope:NN} \meta{scope} \meta{var}
%   \end{syntax}
%   Checks the \meta{var} has the correct \meta{scope}, and if not
%   raises a kernel-level error.  This function is only created if
%   debugging is enabled.  The \meta{scope} is a single letter |l|, |g|,
%   |c| denoting local variables, global variables, or constants.  More
%   precisely, if the variable name starts with a letter and an
%   underscore (normal \pkg{expl3} convention) the function checks that
%   this single letter matches the \meta{scope}.  Otherwise the function
%   cannot know the scope \meta{var} the first time: instead, it defines
%   |\__debug_chk_/|\meta{var name} to store that information for the
%   next call.  Thus, if a given \meta{var} is subject to assignments of
%   different scopes a kernel error will result.
% \end{function}
%
% \begin{function}
%   {
%     \__kernel_chk_cs_exist:N,
%     \__kernel_chk_cs_exist:c,
%     \__kernel_chk_var_exist:N
%   }
%   \begin{syntax}
%     \cs{__kernel_chk_cs_exist:N} \meta{cs}
%     \cs{__kernel_chk_var_exist:N} \meta{var}
%   \end{syntax}
%   These functions are only created if debugging is enabled.  They
%   check that their argument is defined according to the criteria for
%   \cs{cs_if_exist_p:N},
%   and if not raises a kernel-level error.  Error messages are
%   different.
% \end{function}
%
% \begin{function}[EXP]{\__kernel_chk_flag_exist:n}
%   \begin{syntax}
%     \cs{__kernel_chk_flag_exist:n} \Arg{flag}
%   \end{syntax}
%   This function is only created if debugging is enabled.  It checks
%   that the \meta{flag} is defined according to the criterion for
%   \cs{flag_if_exist_p:n}, and if not raises a kernel-level error.
% \end{function}
%
% \begin{function}{\__kernel_debug_log:x}
%   \begin{syntax}
%     \cs{__kernel_debug_log:x} \Arg{message text}
%   \end{syntax}
%   If the \texttt{log-functions} option is active, this function writes
%   the \meta{message text} to the log file using \cs{iow_log:x}.
%   Otherwise, the \meta{message text} is ignored using \cs{use_none:n}.
%   This function is only created if debugging is enabled.
% \end{function}
%
% \end{documentation}
%
% \begin{implementation}
%
% \section{\pkg{l3debug} Implementation}
%
%    \begin{macrocode}
%<*package>
%    \end{macrocode}
%
%    \begin{macrocode}
%<@@=debug>
%    \end{macrocode}
%
% Standard file identification.
%    \begin{macrocode}
\ProvidesExplFile{l3debug.def}{2019-04-06}{}{L3 Debugging support}
%    \end{macrocode}
%
% \begin{variable}{\s_@@_stop}
%   Internal scan marks.
%    \begin{macrocode}
\scan_new:N \s_@@_stop
%    \end{macrocode}
% \end{variable}
%
% \begin{macro}[EXP]{\@@_use_i_delimit_by_s_stop:nw}
%   Functions to gobble up to a scan mark.
%    \begin{macrocode}
\cs_new:Npn \@@_use_i_delimit_by_s_stop:nw #1 #2 \s_@@_stop {#1}
%    \end{macrocode}
% \end{macro}
%
% \begin{variable}{\q_@@_recursion_tail,\q_@@_recursion_stop}
%   Internal quarks.
%    \begin{macrocode}
\quark_new:N \q_@@_recursion_tail
\quark_new:N \q_@@_recursion_stop
%    \end{macrocode}
% \end{variable}
%
% \begin{macro}[EXP]{\@@_if_recursion_tail_stop:N}
%   Functions to query recursion quarks.
%    \begin{macrocode}
\cs_new:Npn \@@_use_none_delimit_by_q_recursion_stop:w
  #1 \q_@@_recursion_stop { }
\__kernel_quark_new_test:N \@@_if_recursion_tail_stop:N
%    \end{macrocode}
% \end{macro}
%
% \begin{macro}{\debug_on:n, \debug_off:n, \@@_all_on:, \@@_all_off:}
%    \begin{macrocode}
\cs_set_protected:Npn \debug_on:n #1
   {
    \exp_args:No \clist_map_inline:nn { \tl_to_str:n {#1} }
       {
        \cs_if_exist_use:cF { @@_ ##1 _on: }
          { \msg_error:nnn { debug } { debug } {##1} }
       }
  }
\cs_set_protected:Npn \debug_off:n #1
  {
    \exp_args:No \clist_map_inline:nn { \tl_to_str:n {#1} }
      {
        \cs_if_exist_use:cF { @@_ ##1 _off: }
          { \msg_error:nnn { debug } { debug } {##1} }
      }
  }
\cs_new_protected:Npn \@@_all_on:
  {
    \debug_on:n
      {
        check-declarations ,
        check-expressions ,
        deprecation ,
        log-functions ,
      }
  }
\cs_new_protected:Npn \@@_all_off:
  {
    \debug_off:n
      {
        check-declarations ,
        check-expressions ,
        deprecation ,
        log-functions ,
      }
  }
%    \end{macrocode}
% \end{macro}
%
% \begin{macro}{\debug_suspend:, \debug_resume:}
% \begin{macro}{\@@_suspended:T}
% \begin{macro}{\l_@@_suspended_tl}
%   Suspend and resume locally all debug-related errors and logging
%   except deprecation errors.  The \cs{debug_suspend:} and \cs{debug_resume:}
%   pairs can be nested.  We keep track of nesting in a token list
%   containing a number of periods.  At first begin with the
%   \enquote{non-suspended} version of \cs{@@_suspended:T}.
%    \begin{macrocode}
\tl_new:N \l_@@_suspended_tl { }
\cs_set_protected:Npn \debug_suspend:
  {
    \tl_put_right:Nn \l_@@_suspended_tl { . }
    \cs_set_eq:NN \@@_suspended:T \use:n
  }
\cs_set_protected:Npn \debug_resume:
  {
    \__kernel_tl_set:Nx \l_@@_suspended_tl
      { \tl_tail:N \l_@@_suspended_tl }
    \tl_if_empty:NT \l_@@_suspended_tl
      {
        \cs_set_eq:NN \@@_suspended:T \use_none:n
      }
  }
\cs_new_eq:NN \@@_suspended:T \use_none:n
%    \end{macrocode}
% \end{macro}
% \end{macro}
% \end{macro}
%
% \begin{macro}
%   {\@@_check-declarations_on:, \@@_check-declarations_off:}
% \begin{macro}
%   {
%     \__kernel_chk_var_exist:N,
%     \__kernel_chk_cs_exist:N,
%     \__kernel_chk_cs_exist:c
%   }
% \begin{macro}[EXP]{\__kernel_chk_flag_exist:n}
% \begin{macro}{\__kernel_chk_var_local:N, \__kernel_chk_var_global:N}
% \begin{macro}{\__kernel_chk_var_scope:NN}
%   When debugging is enabled these two functions set up functions that
%   test their argument (when \texttt{check-declarations} is active)
%   \begin{itemize}
%     \item \cs{__kernel_chk_var_exist:N} and \cs{__kernel_chk_cs_exist:N}, two
%       functions that test that their argument is defined;
%     \item \cs{__kernel_chk_var_scope:NN} that checks that its argument |#2|
%       has scope |#1|.
%     \item \cs{__kernel_chk_var_local:N} and \cs{__kernel_chk_var_global:N} that
%       perform both checks.
%   \end{itemize}
%    \begin{macrocode}
\cs_new_protected:Npn \__kernel_chk_var_exist:N #1 { }
\cs_new_protected:Npn \__kernel_chk_cs_exist:N #1 { }
\cs_generate_variant:Nn \__kernel_chk_cs_exist:N { c }
\cs_new:Npn \__kernel_chk_flag_exist:n #1 { }
\cs_new_protected:Npn \__kernel_chk_var_local:N #1 { }
\cs_new_protected:Npn \__kernel_chk_var_global:N #1 { }
\cs_new_protected:Npn \__kernel_chk_var_scope:NN #1#2 { }
\cs_new_protected:cpn { @@_check-declarations_on: }
  {
    \cs_set_protected:Npn \__kernel_chk_var_exist:N ##1
      {
        \@@_suspended:T \use_none:nnn
        \cs_if_exist:NF ##1
          {
            \msg_error:nnx { debug } { non-declared-variable }
              { \token_to_str:N ##1 }
          }
      }
    \cs_set_protected:Npn \__kernel_chk_cs_exist:N ##1
      {
        \@@_suspended:T \use_none:nnn
        \cs_if_exist:NF ##1
          {
            \msg_error:nnx { kernel } { command-not-defined }
              { \token_to_str:N ##1 }
          }
      }
    \cs_set:Npn \__kernel_chk_flag_exist:n ##1
      {
        \@@_suspended:T \use_none:nnn
        \flag_if_exist:nF {##1}
          {
            \msg_expandable_error:nnn
              { kernel } { bad-variable } { flag~##1~ }
          }
      }
    \cs_set_protected:Npn \__kernel_chk_var_scope:NN
      {
        \@@_suspended:T \use_none:nnn
        \@@_chk_var_scope_aux:NN
      }
    \cs_set_protected:Npn \__kernel_chk_var_local:N ##1
      {
        \@@_suspended:T \use_none:nnnnn
        \__kernel_chk_var_exist:N ##1
        \@@_chk_var_scope_aux:NN l ##1
      }
    \cs_set_protected:Npn \__kernel_chk_var_global:N ##1
      {
        \@@_suspended:T \use_none:nnnnn
        \__kernel_chk_var_exist:N ##1
        \@@_chk_var_scope_aux:NN g ##1
      }
  }
\cs_new_protected:cpn { @@_check-declarations_off: }
  {
    \cs_set_protected:Npn \__kernel_chk_var_exist:N ##1 { }
    \cs_set_protected:Npn \__kernel_chk_cs_exist:N ##1 { }
    \cs_set:Npn \__kernel_chk_flag_exist:N ##1 { }
    \cs_set_protected:Npn \__kernel_chk_var_local:N ##1 { }
    \cs_set_protected:Npn \__kernel_chk_var_global:N ##1 { }
    \cs_set_protected:Npn \__kernel_chk_var_scope:NN ##1##2 { }
  }
%    \end{macrocode}
% \end{macro}
% \end{macro}
% \end{macro}
% \end{macro}
% \end{macro}
%
% \begin{macro}{\@@_chk_var_scope_aux:NN}
% \begin{macro}{\@@_chk_var_scope_aux:Nn}
% \begin{macro}{\@@_chk_var_scope_aux:NNn}
%   First check whether the name of the variable |#2| starts with
%   \meta{letter}|_|.  If it does then pass that letter, the
%   \meta{scope}, and the variable name to
%   \cs{@@_chk_var_scope_aux:NNn}.  That function compares the two
%   letters and triggers an error if they differ (the \cs{scan_stop:}
%   case is not reachable here).  If the second character was not |_|
%   then pass the same data to the same auxiliary, except for its first
%   argument which is now a control sequence.  That control sequence is
%   actually a token list (but to avoid triggering the checking code we
%   manipulate it using \cs{cs_set_nopar:Npn}) containing a single
%   letter \meta{scope} according to what the first assignment to the
%   given variable was.
%    \begin{macrocode}
\cs_new_protected:Npn \@@_chk_var_scope_aux:NN #1#2
  { \exp_args:NNf \@@_chk_var_scope_aux:Nn #1 { \cs_to_str:N #2 } }
\cs_new_protected:Npn \@@_chk_var_scope_aux:Nn #1#2
  {
    \if:w _ \use_i:nn \@@_use_i_delimit_by_s_stop:nw #2 ? ? \s_@@_stop
      \exp_after:wN \@@_chk_var_scope_aux:NNn
        \@@_use_i_delimit_by_s_stop:nw #2 ? \s_@@_stop
        #1 {#2}
    \else:
      \exp_args:Nc \@@_chk_var_scope_aux:NNn
        { @@_chk_/ #2 }
        #1 {#2}
    \fi:
  }
\cs_new_protected:Npn \@@_chk_var_scope_aux:NNn #1#2#3
  {
    \if:w #1 #2
    \else:
      \if:w #1 \scan_stop:
        \cs_gset_nopar:Npn #1 {#2}
      \else:
        \msg_error:nnxxx { debug } { local-global }
          {#1} {#2} { \iow_char:N \\ #3 }
      \fi:
    \fi:
  }
\use:c { @@_check-declarations_off: }
%    \end{macrocode}
% \end{macro}
% \end{macro}
% \end{macro}
%
% \begin{macro}{\@@_log-functions_on:, \@@_log-functions_off:}
% \begin{macro}{\__kernel_debug_log:x}
%   These two functions (corresponding to the \pkg{expl3} option
%   \texttt{log-functions}) control whether \cs{__kernel_debug_log:x}
%   writes to the log file or not.  By default, logging is off.
%    \begin{macrocode}
\cs_new_protected:cpn { @@_log-functions_on: }
  {
    \cs_set_protected:Npn \__kernel_debug_log:x
      { \@@_suspended:T \use_none:nn \iow_log:x }
  }
\cs_new_protected:cpn { @@_log-functions_off: }
  { \cs_set_protected:Npn \__kernel_debug_log:x { \use_none:n } }
\cs_new_protected:Npn \__kernel_debug_log:x { \use_none:n }
%    \end{macrocode}
% \end{macro}
% \end{macro}
%
% \begin{macro}
%   {\@@_check-expressions_on:, \@@_check-expressions_off:}
% \begin{macro}{\__kernel_chk_expr:nNnN}
% \begin{macro}{\@@_chk_expr_aux:nNnN}
%   When debugging is enabled these two functions set
%   \cs{__kernel_chk_expr:nNnN} to test or not whether the given
%   expression is valid.  The idea is to evaluate the expression within
%   a brace group (to catch trailing \cs{use_none:nn} or similar), then
%   test that the result is what we expect.  This is done by turning it
%   to an integer and hitting that with \cs{tex_romannumeral:D} after
%   replacing the first character by |-0|.  If all goes well, that
%   primitive finds a non-positive integer and gives an empty output.
%   If the original expression evaluation stopped early it leaves a
%   trailing \cs{tex_relax:D}, which stops the second evaluation (used
%   to convert to integer) before it encounters the final
%   \cs{tex_relax:D}.  Since \cs{tex_romannumeral:D} does not absorb
%   \cs{tex_relax:D} the output will be nonempty.  Note that |#3| is
%   empty except for mu expressions for which it is \cs{tex_mutoglue:D}
%   to avoid an \enquote{incompatible glue units} error.  Note also that
%   if we had omitted the first \cs{tex_relax:D} then for instance
%   |1+2\relax+3| would incorrectly be accepted as a valid integer
%   expression.
%    \begin{macrocode}
\cs_new_protected:cpn { @@_check-expressions_on: }
  {
    \cs_set:Npn \__kernel_chk_expr:nNnN ##1##2
      {
        \@@_suspended:T { ##1 \use_none:nnnnnnn }
        \exp_after:wN \@@_chk_expr_aux:nNnN
        \exp_after:wN { \tex_the:D ##2 ##1 \scan_stop: }
        ##2
      }
  }
\cs_new_protected:cpn { @@_check-expressions_off: }
  { \cs_set:Npn \__kernel_chk_expr:nNnN ##1##2##3##4 {##1} }
\cs_new:Npn \__kernel_chk_expr:nNnN #1#2#3#4 {#1}
\cs_new:Npn \@@_chk_expr_aux:nNnN #1#2#3#4
  {
    \tl_if_empty:oF
      {
        \tex_romannumeral:D - 0
        \exp_after:wN \use_none:n
        \int_value:w #3 #2 #1 \scan_stop:
      }
      {
        \msg_expandable_error:nnnn
          { debug } { expr } {#4} {#1}
      }
    #1
  }
%    \end{macrocode}
% \end{macro}
% \end{macro}
% \end{macro}
%
% \begin{macro}{\@@_deprecation_on:, \@@_deprecation_off:}
%   Some commands were more recently deprecated and not yet removed;
%   only make these into errors if the user requests it.  This relies on
%   two token lists, filled up in \pkg{l3deprecation} by calls to
%   \cs{__kernel_deprecation_code:nn}.
%    \begin{macrocode}
\cs_new_protected:Npn \@@_deprecation_on:
  { \g_@@_deprecation_on_tl }
\cs_new_protected:Npn \@@_deprecation_off:
  { \g_@@_deprecation_off_tl }
%    \end{macrocode}
% \end{macro}
%
% \begin{variable}{
%     \l_@@_internal_tl,
%     \l_@@_tmpa_tl,
%     \l_@@_tmpb_tl,
%   }
%   For patching.
%    \begin{macrocode}
\tl_new:N \l_@@_internal_tl
\tl_new:N \l_@@_tmpa_tl
\tl_new:N \l_@@_tmpb_tl
%    \end{macrocode}
% \end{variable}
%
% \begin{macro}{
%     \@@_generate_parameter_list:NNN,
%     \@@_build_parm_text:n,
%     \@@_build_arg_list:n,
%     \@@_arg_list_from_signature:nNN,
%     \@@_arg_check_invalid:N,
%     \@@_parm_terminate:w,
%     \@@_arg_if_braced:n,
%     \@@_get_base_form:N,
%     \@@_arg_return:N,
%   }
% \begin{macro}[TF]{\@@_arg_if_braced:N}
%   Some functions don't take the arguments their signature indicates.
%   For instance, \cs{clist_concat:NNN} doesn't take (directly) any
%   argument, so patching it with something that uses |#1|, |#2|, or
%   |#3| results in ``Illegal parameter number in definition of
%   \cs{clist_concat:NNN}''.
%
%   Instead of changing \emph{the} definition of the macros, we'll
%   create a copy of such macros, say, \cs[no-index]{@@_clist_concat:NNN} which
%   will be defined as
%   |<debug code with #1, #2 and #3>\clist_concat:NNN#1#2#3|. For that
%   we need to identify the signature of every function and build the
%   appropriate parameter list.
%
%   \cs{@@_generate_parameter_list:NNN} takes a function in |#1| and
%   returns teo parameter lists: |#2| contains the simple |#1#2#3| as
%   would be used in the \meta{parameter~text} of the definition and
%   |#3| contains the same parameters but with braces where necessary.
%
%   With the current implementation the resulting |#3| is, for example
%   for |\some_function:NnNn|, |#1{#2}#3{#4}|. While this is correct,
%   it might be unnecessary. Bracing everything will usually have the
%   same outcome (unless the function was misused in the first place).
%   What should be done?
%    \begin{macrocode}
\cs_new_protected:Npn \@@_generate_parameter_list:NNN #1#2#3
  {
    \__kernel_tl_set:Nx \l_@@_internal_tl
      { \exp_last_unbraced:Nf \use_ii:nnn \cs_split_function:N #1 }
    \__kernel_tl_set:Nx #2
      { \exp_args:NV \@@_build_parm_text:n \l_@@_internal_tl }
    \__kernel_tl_set:Nx #3
      { \exp_args:NV \@@_build_arg_list:n \l_@@_internal_tl }
  }
\cs_new:Npn \@@_build_parm_text:n #1
  {
    \@@_arg_list_from_signature:nNN { 1 } \c_false_bool #1
    \q_@@_recursion_tail \q_@@_recursion_stop
  }
\cs_new:Npn \@@_build_arg_list:n #1
  {
    \@@_arg_list_from_signature:nNN { 1 } \c_true_bool #1
    \q_@@_recursion_tail \q_@@_recursion_stop
  }
\cs_new:Npn \@@_arg_list_from_signature:nNN #1 #2 #3
  {
    \@@_if_recursion_tail_stop:N #3
    \@@_arg_check_invalid:N #3
    \bool_if:NT #2 { \@@_arg_if_braced:NT #3 { \use_none:n } }
    \use:n { \c_hash_str \int_eval:n {#1} }
    \exp_args:Nf \@@_arg_list_from_signature:nNN
      { \int_eval:n {#1+1} } #2
  }
%    \end{macrocode}
%   Argument types |w|, |p|, |T|, and |F| shouldn't be included in the
%   parameter lists, so we abort the loop if either is found.
%    \begin{macrocode}
\cs_new:Npn \@@_arg_check_invalid:N #1
  {
    \if:w w #1 \@@_parm_terminate:w \else:
      \if:w p #1 \@@_parm_terminate:w \else:
        \if:w T #1 \@@_parm_terminate:w \else:
          \if:w F #1 \@@_parm_terminate:w \else:
            \exp:w
          \fi:
        \fi:
      \fi:
    \fi:
    \exp_end:
  }
\cs_new:Npn \@@_parm_terminate:w
  { \exp_after:wN \@@_use_none_delimit_by_q_recursion_stop:w \exp:w }
\prg_new_conditional:Npnn \@@_arg_if_braced:N #1 { T }
  { \exp_args:Nf \@@_arg_if_braced:n { \@@_get_base_form:N #1 } }
\cs_new:Npn \@@_arg_if_braced:n #1
  {
    \if:w n #1 \prg_return_true: \else:
      \if:w N #1 \prg_return_false: \else:
        \msg_expandable_error:nnn
          { debug } { bad-arg-type } {#1}
      \fi:
    \fi:
  }
\msg_new:nnn { debug } { bad-arg-type }
  { Wrong~argument~type~#1. }
%    \end{macrocode}
%   The macro below gets the base form of an
%   argument type given a variant. It serves only to differentiate
%   arguments which should be braced from ones which shouldn't. If all
%   were to be braced this would be unnecessary. I moved the |n| and |N|
%   variants to the beginning of the test as the are much more common
%   here.
%    \begin{macrocode}
\cs_new:Npn \@@_get_base_form:N #1
  {
    \if:w n #1 \@@_arg_return:N n \else:
      \if:w N #1 \@@_arg_return:N N \else:
        \if:w c #1 \@@_arg_return:N N \else:
          \if:w o #1 \@@_arg_return:N n \else:
            \if:w V #1 \@@_arg_return:N n \else:
              \if:w v #1 \@@_arg_return:N n \else:
                \if:w f #1 \@@_arg_return:N n \else:
                  \if:w e #1 \@@_arg_return:N n \else:
                    \if:w x #1 \@@_arg_return:N n \else:
                      \@@_arg_return:N \scan_stop:
                    \fi:
                  \fi:
                \fi:
              \fi:
            \fi:
          \fi:
        \fi:
      \fi:
    \fi:
    \exp_stop_f:
  }
\cs_new:Npn \@@_arg_return:N #1
  { \exp_after:wN #1 \exp:w \exp_end_continue_f:w }
%    \end{macrocode}
% \end{variable}
%
% \begin{macro}{
%     \__kernel_patch:nnn,
%     \__kernel_patch_aux:nnn,
%     \@@_setup_debug_code:Nnn,
%     \@@_add_to_debug_code:Nnn,
%     \@@_insert_debug_code:Nnn,
%     \__kernel_patch_weird:nnn,
%     \__kernel_patch_weird_aux:nnn,
%     \@@_patch_weird:Nnn,
%   }
%   Simple patching by adding material at the start and end of (a
%   collection of) functions is straight-forward as we know the catcode
%   set up. The approach is essentially that in \pkg{etoolbox}. Notice
%   the need to worry about spaces: those are otherwise lost as normally
%   in \pkg{expl3} code they would be~|~|.
%
%   As discussed above, some functions don't take arguments, so we can't
%   patch something that uses an argument in them. For these functions
%   \cs{__kernel_patch:nnn} is used. It starts by creating a copy of the
%   function (say, \cs{clist_concat:NNN}) with a |__debug_| prefix in
%   the name. This copy won't be changed. The code redefines the
%   original function to take the exact same arguments as advertised in
%   its signature (see \cs{@@_generate_parameter_list:NNN} above).
%   The redefined function also contains the debug code in the proper
%   position. If a function with the same name and the |__debug_| prefix
%   was already defined, then the macro patches that definition by
%   adding more debug code to it.
%    \begin{macrocode}
\group_begin:
  \cs_set_protected:Npn \__kernel_patch:nnn
    {
      \group_begin:
        \char_set_catcode_other:N \#
        \__kernel_patch_aux:nnn
    }
  \cs_set_protected:Npn \__kernel_patch_aux:nnn #1#2#3
    {
        \char_set_catcode_parameter:N \#
        \char_set_catcode_space:N \ %
        \tex_endlinechar:D -1 \scan_stop:
        \tl_map_inline:nn {#3}
          {
            \cs_if_exist:cTF { @@_ \cs_to_str:N ##1 }
              { \@@_add_to_debug_code:Nnn }
              { \@@_setup_debug_code:Nnn }
                ##1 {#1} {#2}
          }
      \group_end:
    }
  \cs_set_protected:Npn \@@_setup_debug_code:Nnn #1#2#3
    {
      \cs_gset_eq:cN { @@_ \cs_to_str:N #1 } #1
      \@@_generate_parameter_list:NNN #1 \l_@@_tmpa_tl \l_@@_tmpb_tl
      \exp_args:Nx \tex_scantokens:D
        {
          \tex_global:D \cs_prefix_spec:N #1
          \tex_def:D \exp_not:N #1
          \tl_use:N \l_@@_tmpa_tl
            {
              \tl_to_str:n {#2}
              \exp_not:c { @@_ \cs_to_str:N #1 }
              \tl_use:N \l_@@_tmpb_tl
              \tl_to_str:n {#3}
            }
        }
    }
  \cs_set_protected:Npn \@@_add_to_debug_code:Nnn #1#2#3
    {
      \use:x
        {
          \cs_set:Npn \exp_not:N \@@_tmp:w
            ####1 \tl_to_str:n { macro: }
            ####2 \tl_to_str:n { -> }
            ####3 \c_backslash_str \tl_to_str:n { @@_ }
                    \cs_to_str:N #1
            ####4 \s_@@_stop
            {
              \exp_not:N \exp_args:Nx \exp_not:N \tex_scantokens:D
                {
                  \tex_global:D ####1
                  \tex_def:D \exp_not:N #1 ####2
                    {
                      ####3 \tl_to_str:n {#2}
                      \c_backslash_str @@_ \cs_to_str:N #1
                      ####4 \tl_to_str:n {#3}
                    }
                }
            }
        }
      \exp_after:wN \@@_tmp:w \cs_meaning:N #1 \s_@@_stop
    }
%    \end{macrocode}
%   Some functions, however, won't work with the signature reading setup
%   above because their signature contains |w|eird arguments. These
%   functions need to be patched using \cs{__kernel_patch_weird:nnn},
%   which won't make a copy of the function, rather it will patch the
%   debug code directly into it. This means that whatever argument the
%   debug code uses must be actually used by the patched function.
%    \begin{macrocode}
  \cs_set_protected:Npn \__kernel_patch_weird:nnn
    {
      \group_begin:
        \char_set_catcode_other:N \#
        \__kernel_patch_weird_aux:nnn
    }
  \cs_set_protected:Npn \__kernel_patch_weird_aux:nnn #1#2#3
    {
        \char_set_catcode_parameter:N \#
        \char_set_catcode_space:N \ %
        \tex_endlinechar:D -1 \scan_stop:
        \tl_map_inline:nn {#3}
          { \@@_patch_weird:Nnn ##1 {#1} {#2} }
      \group_end:
    }
  \cs_set_protected:Npn \@@_patch_weird:Nnn #1#2#3
    {
      \use:x
        {
          \tex_endlinechar:D -1 \scan_stop:
          \exp_not:N \tex_scantokens:D
            {
              \tex_global:D \cs_prefix_spec:N #1
              \tex_def:D \exp_not:N #1
              \cs_parameter_spec:N #1
                {
                  \tl_to_str:n {#2}
                  \cs_replacement_spec:N #1
                  \tl_to_str:n {#3}
                }
            }
        }
    }
%    \end{macrocode}
% \end{macro}
%
% Patching the second argument to ensure it exists. This happens before
% we alter |#1| so the ordering is correct.  For many variable types
% such as \texttt{int} a low-level error occurs when |#2| is unknown, so
% adding a check is not needed.
%    \begin{macrocode}
  \__kernel_patch:nnn
    { \__kernel_chk_var_exist:N #2 }
    { }
    {
      \bool_set_eq:NN
      \bool_gset_eq:NN
      \clist_set_eq:NN
      \clist_gset_eq:NN
      \fp_set_eq:NN
      \fp_gset_eq:NN
      \prop_set_eq:NN
      \prop_gset_eq:NN
      \seq_set_eq:NN
      \seq_gset_eq:NN
      \str_set_eq:NN
      \str_gset_eq:NN
      \tl_set_eq:NN
      \tl_gset_eq:NN
    }
%    \end{macrocode}
%
% Patching both second and third arguments.
%    \begin{macrocode}
% \tracingall
  \__kernel_patch:nnn
    {
      \__kernel_chk_var_exist:N #2
      \__kernel_chk_var_exist:N #3
    }
    { }
    {
      \clist_concat:NNN
      \clist_gconcat:NNN
      \seq_concat:NNN
      \seq_gconcat:NNN
      \str_concat:NNN
      \str_gconcat:NNN
      \tl_concat:NNN
      \tl_gconcat:NNN
    }
% \tracingnone
%    \end{macrocode}
%
%
%    \begin{macrocode}
\cs_gset_protected:Npn \__kernel_tl_set:Nx  { \cs_set_nopar:Npx }
\cs_gset_protected:Npn \__kernel_tl_gset:Nx { \cs_gset_nopar:Npx }
%    \end{macrocode}
%
% Patching where the first argument to a function needs scope-checking:
% either local or global (so two lists).
%    \begin{macrocode}
  \__kernel_patch:nnn
    { \__kernel_chk_var_local:N #1 }
    { }
    {
      \bool_set:Nn
      \bool_set_eq:NN
      \bool_set_true:N
      \bool_set_false:N
      \box_set_eq:NN
      \box_set_eq_drop:NN
      \box_set_to_last:N
      \clist_clear:N
      \clist_set_eq:NN
      \dim_zero:N
      \dim_set:Nn
      \dim_set_eq:NN
      \dim_add:Nn
      \dim_sub:Nn
      \fp_set_eq:NN
      \int_zero:N
      \int_set_eq:NN
      \int_add:Nn
      \int_sub:Nn
      \int_incr:N
      \int_decr:N
      \int_set:Nn
      \hbox_set:Nn
      \hbox_set_to_wd:Nnn
      \hbox_set:Nw
      \hbox_set_to_wd:Nnw
      \muskip_zero:N
      \muskip_set:Nn
      \muskip_add:Nn
      \muskip_sub:Nn
      \muskip_set_eq:NN
      \seq_set_eq:NN
      \skip_zero:N
      \skip_set:Nn
      \skip_set_eq:NN
      \skip_add:Nn
      \skip_sub:Nn
      \__kernel_tl_set:Nx
      \tl_clear:N
      \tl_set_eq:NN
      \tl_put_left:Nn
      \tl_put_left:NV
      \tl_put_left:No
      \tl_put_left:Nx
      \tl_put_right:Nn
      \tl_put_right:NV
      \tl_put_right:No
      \tl_put_right:Nx
      \tl_build_begin:N
      \tl_build_put_right:Nn
      \tl_build_put_right:Nx
      \tl_build_put_left:Nn
      \vbox_set:Nn
      \vbox_set_top:Nn
      \vbox_set_to_ht:Nnn
      \vbox_set:Nw
      \vbox_set_to_ht:Nnw
      \vbox_set_split_to_ht:NNn
    }
  \__kernel_patch:nnn
    { \__kernel_chk_var_global:N #1 }
    { }
    {
      \bool_gset:Nn
      \bool_gset_eq:NN
      \bool_gset_true:N
      \bool_gset_false:N
      \box_gset_eq:NN
      \box_gset_eq_drop:NN
      \box_gset_to_last:N
      \cctab_gset:Nn
      \clist_gclear:N
      \clist_gset_eq:NN
      \dim_gset_eq:NN
      \dim_gzero:N
      \dim_gset:Nn
      \dim_gadd:Nn
      \dim_gsub:Nn
      \fp_gset_eq:NN
      \int_gzero:N
      \int_gset_eq:NN
      \int_gadd:Nn
      \int_gsub:Nn
      \int_gincr:N
      \int_gdecr:N
      \int_gset:Nn
      \hbox_gset:Nn
      \hbox_gset_to_wd:Nnn
      \hbox_gset:Nw
      \hbox_gset_to_wd:Nnw
      \muskip_gzero:N
      \muskip_gset:Nn
      \muskip_gadd:Nn
      \muskip_gsub:Nn
      \muskip_gset_eq:NN
      \seq_gset_eq:NN
      \skip_gzero:N
      \skip_gset:Nn
      \skip_gset_eq:NN
      \skip_gadd:Nn
      \skip_gsub:Nn
      \__kernel_tl_gset:Nx
      \tl_gclear:N
      \tl_gset_eq:NN
      \tl_gput_left:Nn
      \tl_gput_left:NV
      \tl_gput_left:No
      \tl_gput_left:Nx
      \tl_gput_right:Nn
      \tl_gput_right:NV
      \tl_gput_right:No
      \tl_gput_right:Nx
      \tl_build_gbegin:N
      \tl_build_gput_right:Nn
      \tl_build_gput_right:Nx
      \tl_build_gput_left:Nn
      \vbox_gset:Nn
      \vbox_gset_top:Nn
      \vbox_gset_to_ht:Nnn
      \vbox_gset:Nw
      \vbox_gset_to_ht:Nnw
      \vbox_gset_split_to_ht:NNn
    }
%    \end{macrocode}
%
% Scoping for constants.
%    \begin{macrocode}
  \__kernel_patch:nnn
    { \__kernel_chk_var_scope:NN c #1 }
    { }
    {
      \bool_const:Nn
      \cctab_const:Nn
      \dim_const:Nn
      \int_const:Nn
      \intarray_const_from_clist:Nn
      \muskip_const:Nn
      \skip_const:Nn
      \tl_const:Nn
      \tl_const:Nx
    }
%    \end{macrocode}
%
% Flag functions.
%    \begin{macrocode}
  \__kernel_patch:nnn
    { \__kernel_chk_flag_exist:n {#1} }
    { }
    {
      \flag_if_raised:nT
      \flag_if_raised:nF
      \flag_if_raised:nTF
      \flag_if_raised_p:n
      \flag_height:n
      \flag_raise_if_clear:n
      \flag_clear:n
    }
%    \end{macrocode}
%
% Various one-offs.
%    \begin{macrocode}
  \__kernel_patch:nnn
    { \__kernel_chk_cs_exist:N #1 }
    { }
    { \cs_generate_variant:Nn }
  \__kernel_patch:nnn
    { \__kernel_chk_var_scope:NN g #1 }
    { }
    { \cctab_new:N }
  \__kernel_patch:nnn
    { \__kernel_chk_var_scope:NN g #1 }
    { }
    { \intarray_new:Nn }
  \__kernel_patch:nnn
    { \__kernel_chk_var_scope:NN q #1 }
    { }
    { \quark_new:N }
  \__kernel_patch:nnn
    { \__kernel_chk_var_scope:NN s #1 }
    { }
    { \scan_new:N }
%    \end{macrocode}
%
% Patch various internal commands to log definitions of functions.
% First, a kernel internal.  Then internals from the \pkg{cs},
% \pkg{keys} and \pkg{msg} modules.
%    \begin{macrocode}
  \__kernel_patch:nnn
    { }
    {
      \__kernel_debug_log:x
        { Defining~\token_to_str:N #1~ \msg_line_context: }
    }
    { \__kernel_chk_if_free_cs:N }
%<@@=cs>
  \__kernel_patch_weird:nnn
    {
      \cs_if_free:NF #4
        {
          \__kernel_debug_log:x
            {
              Variant~\token_to_str:N #4~%
              already~defined;~ not~ changing~ it~ \msg_line_context:
            }
        }
    }
    { }
    { \@@_generate_variant:wwNN  }
%<@@=keys>
  \__kernel_patch:nnn
    {
      \cs_if_exist:cF { \c_@@_code_root_str #1 }
        { \__kernel_debug_log:x { Defining~key~#1~\msg_line_context: } }
    }
    { }
    { \@@_cmd_set_direct:nn }
%<@@=msg>
  \__kernel_patch:nnn
    { }
    {
      \__kernel_debug_log:x
        { Defining~message~ #1 / #2 ~\msg_line_context: }
    }
    { \@@_chk_free:nn }
%    \end{macrocode}
%
%    \begin{macrocode}
%<@@=prg>
%    \end{macrocode}
% Internal functions from \pkg{prg} module.
%    \begin{macrocode}
  \__kernel_patch_weird:nnn
    { \__kernel_chk_cs_exist:c { #5 _p : #6 } }
    { }
    { \@@_set_eq_conditional_p_form:wNnnnn }
  \__kernel_patch_weird:nnn
    { \__kernel_chk_cs_exist:c { #5    : #6 TF } }
    { }
    { \@@_set_eq_conditional_TF_form:wNnnnn }
  \__kernel_patch_weird:nnn
    { \__kernel_chk_cs_exist:c { #5    : #6 T } }
    { }
    { \@@_set_eq_conditional_T_form:wNnnnn }
  \__kernel_patch_weird:nnn
    { \__kernel_chk_cs_exist:c { #5    : #6 F } }
    { }
    { \@@_set_eq_conditional_F_form:wNnnnn }
%    \end{macrocode}
%
%    \begin{macrocode}
%<@@=regex>
%    \end{macrocode}
% Internal functions from \pkg{regex} module.
%    \begin{macrocode}
  \__kernel_patch:nnn
    {
      \@@_trace_push:nnN { regex } { 1 } \@@_escape_use:nnnn
      \group_begin:
        \__kernel_tl_set:Nx \l_@@_internal_a_tl
          { \@@_trace_pop:nnN { regex } { 1 } \@@_escape_use:nnnn }
        \use_none:nnn
    }
    { }
    { \@@_escape_use:nnn }
  \__kernel_patch:nnn
    { \@@_trace_push:nnN { regex } { 1 } \@@_build:N }
    {
      \@@_trace_states:n { 2 }
      \@@_trace_pop:nnN { regex } { 1 } \@@_build:N
    }
    { \@@_build:N }
  \__kernel_patch:nnn
    { \@@_trace_push:nnN { regex } { 1 } \@@_build_for_cs:n }
    {
      \@@_trace_states:n { 2 }
      \@@_trace_pop:nnN { regex } { 1 } \@@_build_for_cs:n
    }
    { \@@_build_for_cs:n }
  \__kernel_patch:nnn
    {
      \@@_trace:nnx { regex } { 2 }
        {
          regex~new~state~
          L=\int_use:N \l_@@_left_state_int ~ -> ~
          R=\int_use:N \l_@@_right_state_int ~ -> ~
          M=\int_use:N \l_@@_max_state_int ~ -> ~
          \int_eval:n { \l_@@_max_state_int + 1 }
        }
    }
    { }
    { \@@_build_new_state: }
  \__kernel_patch:nnn
    { \@@_trace_push:nnN { regex } { 1 } \@@_group_aux:nnnnN }
    { \@@_trace_pop:nnN { regex } { 1 } \@@_group_aux:nnnnN }
    { \@@_group_aux:nnnnN }
  \__kernel_patch:nnn
    { \@@_trace_push:nnN { regex } { 1 } \@@_branch:n }
    { \@@_trace_pop:nnN { regex } { 1 } \@@_branch:n }
    { \@@_branch:n }
  \__kernel_patch:nnn
    {
      \@@_trace_push:nnN { regex } { 1 } \@@_match:n
      \@@_trace:nnx { regex } { 1 } { analyzing~query~token~list }
    }
    { \@@_trace_pop:nnN { regex } { 1 } \@@_match:n }
    { \@@_match:n }
  \__kernel_patch:nnn
    {
      \@@_trace_push:nnN { regex } { 1 } \@@_match_cs:n
      \@@_trace:nnx { regex } { 1 } { analyzing~query~token~list }
    }
    { \@@_trace_pop:nnN { regex } { 1 } \@@_match_cs:n }
    { \@@_match_cs:n }
  \__kernel_patch:nnn
    { \@@_trace:nnx { regex } { 1 } { initializing } }
    { }
    { \@@_match_init: }
  \__kernel_patch:nnn
    {
      \@@_trace:nnx { regex } { 2 }
        { state~\int_use:N \l_@@_curr_state_int }
    }
    { }
    { \@@_use_state: }
  \__kernel_patch:nnn
    { \@@_trace_push:nnN { regex } { 1 } \@@_replacement:n }
    { \@@_trace_pop:nnN { regex } { 1 } \@@_replacement:n }
    { \@@_replacement:n }
%    \end{macrocode}
%
%    \begin{macrocode}
\group_end:
%    \end{macrocode}
%
%    \begin{macrocode}
%<@@=debug>
%    \end{macrocode}
%
% Patching arguments is a bit more involved: we do these one at a time.
% The basic idea is the same, using a |#| token that is a string.
%    \begin{macrocode}
\group_begin:
  \cs_set_protected:Npn \__kernel_patch:Nn #1
    {
      \group_begin:
        \char_set_catcode_other:N \#
        \__kernel_patch_aux:Nn #1
    }
  \cs_set_protected:Npn \__kernel_patch_aux:Nn #1#2
    {
        \char_set_catcode_parameter:N \#
        \tex_endlinechar:D -1 \scan_stop:
        \exp_args:Nx \tex_scantokens:D
          {
            \tex_global:D \cs_prefix_spec:N #1 \tex_def:D \exp_not:N #1
              \cs_parameter_spec:N #1
              { \exp_args:No \tl_to_str:n { #1 #2 } }
          }
      \group_end:
    }
%    \end{macrocode}
%
% The functions here can get a bit repetitive, so we define a helper
% which can re-use the same patch code repeatedly. The main part of the
% patch is the same, so we just have to deal with the part which varies
% depending on the type of expression.
%    \begin{macrocode}
  \cs_set_protected:Npn \__kernel_patch_eval:nn #1#2
    {
      \tl_map_inline:nn {#1}
        {
          \exp_args:NNx \__kernel_patch:Nn ##1
            {
              { \c_hash_str 1 }
              {
                \exp_not:N \__kernel_chk_expr:nNnN { \c_hash_str 2 }
                \exp_not:n {#2}
                \exp_not:N ##1
              }
            }
        }
    }
%<@@=dim>
  \__kernel_patch_eval:nn
    {
      \dim_set:Nn
      \dim_gset:Nn
      \dim_add:Nn
      \dim_gadd:Nn
      \dim_sub:Nn
      \dim_gsub:Nn
      \dim_const:Nn
    }
    { \@@_eval:w { } }
%<@@=int>
  \__kernel_patch_eval:nn
    {
      \int_set:Nn
      \int_gset:Nn
      \int_add:Nn
      \int_gadd:Nn
      \int_sub:Nn
      \int_gsub:Nn
      \int_const:Nn
    }
    { \@@_eval:w { } }
  \__kernel_patch_eval:nn
    {
      \muskip_set:Nn
      \muskip_gset:Nn
      \muskip_add:Nn
      \muskip_gadd:Nn
      \muskip_sub:Nn
      \muskip_gsub:Nn
      \muskip_const:Nn
    }
    { \tex_muexpr:D { \tex_mutoglue:D } }
  \__kernel_patch_eval:nn
    {
      \skip_set:Nn
      \skip_gset:Nn
      \skip_add:Nn
      \skip_gadd:Nn
      \skip_sub:Nn
      \skip_gsub:Nn
      \skip_const:Nn
    }
    { \tex_glueexpr:D { } }
%    \end{macrocode}
%
% Patching expandable expressions, first the one-argument versions,
% then the two-argument ones.
%    \begin{macrocode}
  \cs_set_protected:Npn \__kernel_patch_eval:nn #1#2
    {
      \tl_map_inline:nn {#1}
        {
          \exp_args:NNx \__kernel_patch:Nn ##1
            {
              {
                \exp_not:N \__kernel_chk_expr:nNnN { \c_hash_str 1 }
                \exp_not:n {#2}
                \exp_not:N ##1
              }
            }
        }
    }
%<@@=box>
  \__kernel_patch_eval:nn
    { \@@_dim_eval:n }
    { \@@_dim_eval:w { } }
%<@@=dim>
  \__kernel_patch_eval:nn
    {
      \dim_eval:n
      \dim_to_decimal:n
      \dim_to_decimal_in_sp:n
      \dim_abs:n
      \dim_sign:n
    }
    { \@@_eval:w { } }
%<@@=int>
  \__kernel_patch_eval:nn
    {
      \int_eval:n
      \int_abs:n
      \int_sign:n
    }
    { \@@_eval:w { } }
  \__kernel_patch_eval:nn
    {
      \skip_eval:n
      \skip_horizontal:n
      \skip_vertical:n
    }
    { \tex_glueexpr:D { } }
  \__kernel_patch_eval:nn
    {
      \muskip_eval:n
    }
    { \tex_muexpr:D { \tex_mutoglue:D } }
  \cs_set_protected:Npn \__kernel_patch_eval:nn #1#2
    {
      \tl_map_inline:nn {#1}
        {
          \exp_args:NNx \__kernel_patch:Nn ##1
            {
              {
                \exp_not:N \__kernel_chk_expr:nNnN { \c_hash_str 1 }
                \exp_not:n {#2}
                \exp_not:N ##1
              }
              {
                \exp_not:N \__kernel_chk_expr:nNnN { \c_hash_str 2 }
                \exp_not:n {#2}
                \exp_not:N ##1
              }
            }
        }
    }
%<@@=dim>
  \__kernel_patch_eval:nn
    {
      \dim_max:nn
      \dim_min:nn
    }
    { \@@_eval:w { } }
%<@@=int>
  \__kernel_patch_eval:nn
    {
      \int_max:nn
      \int_min:nn
      \int_div_truncate:nn
      \int_mod:nn
    }
    { \@@_eval:w { } }
%    \end{macrocode}
%
%  Conditionals: three argument ones then one argument ones
%    \begin{macrocode}
  \cs_set_protected:Npn \__kernel_patch_cond:nn #1#2
    {
      \clist_map_inline:nn { :nNnT , :nNnF , :nNnTF , _p:nNn }
        {
          \exp_args:Ncx \__kernel_patch:Nn { #1 ##1 }
            {
              {
                \exp_not:N \__kernel_chk_expr:nNnN { \c_hash_str 1 }
                \exp_not:n {#2}
                \exp_not:c { #1 ##1 }
              }
              { \c_hash_str 2 }
              {
                \exp_not:N \__kernel_chk_expr:nNnN { \c_hash_str 3 }
                \exp_not:n {#2}
                \exp_not:c { #1 ##1 }
              }
            }
        }
    }
%<@@=dim>
  \__kernel_patch_cond:nn { dim_compare } { \@@_eval:w { } }
%<@@=int>
  \__kernel_patch_cond:nn { int_compare } { \@@_eval:w { } }
  \cs_set_protected:Npn \__kernel_patch_cond:nn #1#2
    {
      \clist_map_inline:nn { :nT , :nF , :nTF , _p:n }
        {
          \exp_args:Ncx \__kernel_patch:Nn { #1 ##1 }
            {
              {
                \exp_not:N \__kernel_chk_expr:nNnN { \c_hash_str 1 }
                \exp_not:n {#2}
                \exp_not:c { #1 ##1 }
              }
            }
        }
    }
%<@@=int>
  \__kernel_patch_cond:nn { int_if_even } { \@@_eval:w { } }
  \__kernel_patch_cond:nn { int_if_odd } { \@@_eval:w { } }
%    \end{macrocode}
%
% Step functions.
%    \begin{macrocode}
%<@@=dim>
  \__kernel_patch:Nn \dim_step_function:nnnN
    {
      {
        \__kernel_chk_expr:nNnN {#1} \@@_eval:w { }
          \dim_step_function:nnnN
      }
      {
        \__kernel_chk_expr:nNnN {#2} \@@_eval:w { }
          \dim_step_function:nnnN
      }
      {
        \__kernel_chk_expr:nNnN {#3} \@@_eval:w { }
          \dim_step_function:nnnN
      }
    }
%<@@=int>
  \__kernel_patch:Nn \int_step_function:nnnN
    {
      {
        \__kernel_chk_expr:nNnN {#1} \@@_eval:w { }
          \int_step_function:nnnN
      }
      {
        \__kernel_chk_expr:nNnN {#2} \@@_eval:w { }
          \int_step_function:nnnN
      }
      {
        \__kernel_chk_expr:nNnN {#3} \@@_eval:w { }
          \int_step_function:nnnN
      }
    }
%    \end{macrocode}
%
%  Odds and ends
%    \begin{macrocode}
  \__kernel_patch:Nn \dim_to_fp:n { { (#1) } }
%    \end{macrocode}
%
%    \begin{macrocode}
\group_end:
%    \end{macrocode}
%
%    \begin{macrocode}
%<@@=skip>
%    \end{macrocode}
% This one has catcode changes so must be done by hand.
%    \begin{macrocode}
\cs_set_protected:Npn \@@_tmp:w #1
  {
    \prg_set_conditional:Npnn \skip_if_finite:n ##1 { p , T , F , TF }
      {
        \exp_after:wN \@@_if_finite:wwNw
        \skip_use:N \tex_glueexpr:D
          \__kernel_chk_expr:nNnN
            {##1} \tex_glueexpr:D { } \skip_if_finite:n
        ; \prg_return_false:
        #1 ; \prg_return_true: \s_@@_stop
      }
  }
\exp_args:No \@@_tmp:w { \tl_to_str:n { fil } }
%    \end{macrocode}
%
%    \begin{macrocode}
%<@@=msg>
%    \end{macrocode}
%
% Messages.
% \begin{macrocode}
\msg_new:nnnn { debug } { debug }
  { The~debugging~option~'#1'~does~not~exist~\msg_line_context:. }
  {
    The~functions~'\iow_char:N\\debug_on:n'~and~
    '\iow_char:N\\debug_off:n'~only~accept~the~arguments~
    'all',~'check-declarations',~'check-expressions',~
    'deprecation',~'log-functions',~not~'#1'.
  }
\msg_new:nnn { debug } { expr } { '#2'~in~#1 }
\msg_new:nnnn { debug } { local-global }
  { Inconsistent~local/global~assignment }
  {
    \c_@@_coding_error_text_tl
    \if:w l #2 Local
    \else:
      \if:w g #2 Global \else: Constant \fi:
    \fi:
    \ %
    assignment~to~a~
    \if:w l #1 local
    \else:
      \if:w g #1 global \else: constant \fi:
    \fi:
    \ %
    variable~'#3'.
  }
\msg_new:nnnn { debug } { non-declared-variable }
  { The~variable~#1~has~not~been~declared~\msg_line_context:. }
  {
    \c_@@_coding_error_text_tl
    Checking~is~active,~and~you~have~tried~do~so~something~like: \\
    \ \ \tl_set:Nn ~ #1 ~ \{ ~ ... ~ \} \\
    without~first~having: \\
    \ \ \tl_new:N ~ #1  \\
    \\
    LaTeX~will~create~the~variable~and~continue.
  }
% \end{macrocode}
%
% \begin{macro}{\__kernel_if_debug:TF}
%   Flip the switch for deprecated code.
%    \begin{macrocode}
\cs_set_protected:Npn \__kernel_if_debug:TF #1#2 {#1}
%    \end{macrocode}
% \end{macro}
%
%    \begin{macrocode}
%</package>
%    \end{macrocode}
%
% \end{implementation}
%
% \PrintIndex
