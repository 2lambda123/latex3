% \iffalse meta-comment
%
%% File: l3coffins.dtx
%
% Copyright (C) 2010-2020 The LaTeX3 Project
%
% It may be distributed and/or modified under the conditions of the
% LaTeX Project Public License (LPPL), either version 1.3c of this
% license or (at your option) any later version.  The latest version
% of this license is in the file
%
%    https://www.latex-project.org/lppl.txt
%
% This file is part of the "l3kernel bundle" (The Work in LPPL)
% and all files in that bundle must be distributed together.
%
% -----------------------------------------------------------------------
%
% The development version of the bundle can be found at
%
%    https://github.com/latex3/latex3
%
% for those people who are interested.
%
%<*driver>
\documentclass[full,kernel]{l3doc}
\begin{document}
  \DocInput{\jobname.dtx}
\end{document}
%</driver>
% \fi
%
% \title{^^A
%   The \textsf{l3coffins} package\\ Coffin code layer^^A
% }
%
% \author{^^A
%  The \LaTeX3 Project\thanks
%    {^^A
%      E-mail:
%        \href{mailto:latex-team@latex-project.org}
%          {latex-team@latex-project.org}^^A
%    }^^A
% }
%
% \date{Released 2020-02-11}
%
% \maketitle
%
% \begin{documentation}
%
% The material in this module provides the low-level support system
% for coffins. For details about the design concept of a coffin, see
% the \pkg{xcoffins} module (in the \pkg{l3experimental} bundle).
%
% \section{Creating and initialising coffins}
%
% \begin{function}[added = 2011-08-17]{\coffin_new:N, \coffin_new:c}
%   \begin{syntax}
%     \cs{coffin_new:N} \meta{coffin}
%   \end{syntax}
%   Creates a new \meta{coffin} or raises an error if the name is
%   already taken. The declaration is global. The \meta{coffin} is
%   initially empty.
% \end{function}
%
% \begin{function}[added = 2011-08-17, updated = 2019-01-21]
%   {
%     \coffin_clear:N, \coffin_clear:c,
%     \coffin_gclear:N, \coffin_gclear:c
%   }
%   \begin{syntax}
%     \cs{coffin_clear:N} \meta{coffin}
%   \end{syntax}
%   Clears the content of the \meta{coffin}.
% \end{function}
%
% \begin{function}[added = 2011-08-17, updated = 2019-01-21]
%  {
%    \coffin_set_eq:NN, \coffin_set_eq:Nc,
%    \coffin_set_eq:cN, \coffin_set_eq:cc,
%    \coffin_gset_eq:NN, \coffin_gset_eq:Nc,
%    \coffin_gset_eq:cN, \coffin_gset_eq:cc
%  }
%   \begin{syntax}
%    \cs{coffin_set_eq:NN} \meta{coffin_1} \meta{coffin_2}
%   \end{syntax}
%   Sets both the content and poles of \meta{coffin_1} equal to those
%   of \meta{coffin_2}.
% \end{function}
%
% \begin{function}[EXP, pTF, added = 2012-06-20]
%   {\coffin_if_exist:N, \coffin_if_exist:c}
%   \begin{syntax}
%     \cs{coffin_if_exist_p:N} \meta{box}
%     \cs{coffin_if_exist:NTF} \meta{box} \Arg{true code} \Arg{false code}
%   \end{syntax}
%   Tests whether the \meta{coffin} is currently defined.
% \end{function}
%
% \section{Setting coffin content and poles}
%
% \begin{function}[added = 2011-08-17, updated = 2019-01-21]
%   {
%     \hcoffin_set:Nn, \hcoffin_set:cn,
%     \hcoffin_gset:Nn, \hcoffin_gset:cn
%   }
%   \begin{syntax}
%     \cs{hcoffin_set:Nn} \meta{coffin} \Arg{material}
%   \end{syntax}
%   Typesets the \meta{material} in horizontal mode, storing the result
%   in the \meta{coffin}. The standard poles for the \meta{coffin} are
%   then set up based on the size of the typeset material.
% \end{function}
%
% \begin{function}[added = 2011-09-10, updated = 2019-01-21]
%   {
%     \hcoffin_set:Nw, \hcoffin_set:cw, \hcoffin_set_end:,
%     \hcoffin_gset:Nw, \hcoffin_gset:cw, \hcoffin_gset_end:
%   }
%   \begin{syntax}
%     \cs{hcoffin_set:Nw} \meta{coffin} \meta{material} \cs{hcoffin_set_end:}
%   \end{syntax}
%   Typesets the \meta{material} in horizontal mode, storing the result
%   in the \meta{coffin}. The standard poles for the \meta{coffin} are
%   then set up based on the size of the typeset material.
%   These functions are useful for setting the entire contents of an
%   environment in a coffin.
% \end{function}
%
% \begin{function}[added = 2011-08-17, updated = 2019-01-21]
%   {
%     \vcoffin_set:Nnn, \vcoffin_set:cnn,
%     \vcoffin_gset:Nnn, \vcoffin_gset:cnn
%   }
%   \begin{syntax}
%     \cs{vcoffin_set:Nnn} \meta{coffin} \Arg{width} \Arg{material}
%   \end{syntax}
%   Typesets the \meta{material} in vertical mode constrained to the
%   given \meta{width} and stores the result in the \meta{coffin}. The
%   standard poles for the \meta{coffin} are then set up based on the
%   size of the typeset material.
% \end{function}
%
% \begin{function}[added = 2011-09-10, updated = 2019-01-21]
%   {
%     \vcoffin_set:Nnw, \vcoffin_set:cnw, \vcoffin_set_end:,
%     \vcoffin_gset:Nnw, \vcoffin_gset:cnw, \vcoffin_gset_end:
%   }
%   \begin{syntax}
%     \cs{vcoffin_set:Nnw} \meta{coffin} \Arg{width} \meta{material} \cs{vcoffin_set_end:}
%   \end{syntax}
%   Typesets the \meta{material} in vertical mode constrained to the
%   given \meta{width} and stores the result in the \meta{coffin}. The
%   standard poles for the \meta{coffin} are then set up based on the
%   size of the typeset material.
%   These functions are useful for setting the entire contents of an
%   environment in a coffin.
% \end{function}
%
% \begin{function}[added = 2012-07-20, updated = 2019-01-21]
%   {
%     \coffin_set_horizontal_pole:Nnn, \coffin_set_horizontal_pole:cnn,
%     \coffin_gset_horizontal_pole:Nnn, \coffin_gset_horizontal_pole:cnn
%   }
%   \begin{syntax}
%     \cs{coffin_set_horizontal_pole:Nnn} \meta{coffin}
%     ~~\Arg{pole} \Arg{offset}
%   \end{syntax}
%   Sets the \meta{pole} to run horizontally through the \meta{coffin}.
%   The \meta{pole} is placed at the \meta{offset} from the
%   bottom edge of the bounding box of the \meta{coffin}. The
%   \meta{offset} should be given as a dimension expression.
% \end{function}
%
% \begin{function}[added = 2012-07-20, updated = 2019-01-21]
%   {
%     \coffin_set_vertical_pole:Nnn, \coffin_set_vertical_pole:cnn,
%     \coffin_gset_vertical_pole:Nnn, \coffin_gset_vertical_pole:cnn
%   }
%   \begin{syntax}
%     \cs{coffin_set_vertical_pole:Nnn} \meta{coffin} \Arg{pole} \Arg{offset}
%   \end{syntax}
%   Sets the \meta{pole} to run vertically through the \meta{coffin}.
%   The \meta{pole} is placed at the \meta{offset} from the
%   left-hand edge of the bounding box of the \meta{coffin}. The
%   \meta{offset} should be given as a dimension expression.
% \end{function}
%
% \section{Coffin affine transformations}
%
% \begin{function}[updated = 2019-01-23]
%   {
%     \coffin_resize:Nnn, \coffin_resize:cnn,
%     \coffin_gresize:Nnn, \coffin_gresize:cnn
%   }
%   \begin{syntax}
%     \cs{coffin_resize:Nnn} \meta{coffin} \Arg{width} \Arg{total-height}
%   \end{syntax}
%   Resized the \meta{coffin} to \meta{width} and \meta{total-height},
%   both of which should be given as dimension expressions.
% \end{function}
%
% \begin{function}
%   {
%     \coffin_rotate:Nn, \coffin_rotate:cn,
%     \coffin_grotate:Nn, \coffin_grotate:cn
%   }
%   \begin{syntax}
%     \cs{coffin_rotate:Nn} \meta{coffin} \Arg{angle}
%   \end{syntax}
%   Rotates the \meta{coffin} by the given \meta{angle} (given in
%   degrees counter-clockwise). This process rotates both the
%   coffin content and poles. Multiple rotations do not result in
%   the bounding box of the coffin growing unnecessarily.
% \end{function}
%
% \begin{function}[updated = 2019-01-23]
%   {
%     \coffin_scale:Nnn, \coffin_scale:cnn,
%     \coffin_gscale:Nnn, \coffin_gscale:cnn
%   }
%   \begin{syntax}
%     \cs{coffin_scale:Nnn} \meta{coffin} \Arg{x-scale} \Arg{y-scale}
%   \end{syntax}
%   Scales the \meta{coffin} by a factors \meta{x-scale} and
%   \meta{y-scale} in the horizontal and vertical directions,
%   respectively. The two scale factors should be given as real numbers.
% \end{function}
%
% \section{Joining and using coffins}
%
% \begin{function}[updated = 2019-01-22]
%   {
%     \coffin_attach:NnnNnnnn, \coffin_attach:cnnNnnnn,
%     \coffin_attach:Nnncnnnn, \coffin_attach:cnncnnnn,
%     \coffin_gattach:NnnNnnnn, \coffin_gattach:cnnNnnnn,
%     \coffin_gattach:Nnncnnnn, \coffin_gattach:cnncnnnn
%   }
%   \begin{syntax}
%     \cs{coffin_attach:NnnNnnnn}
%     ~~\meta{coffin_1} \Arg{coffin_1-pole_1} \Arg{coffin_1-pole_2}
%     ~~\meta{coffin_2} \Arg{coffin_2-pole_1} \Arg{coffin_2-pole_2}
%     ~~\Arg{x-offset} \Arg{y-offset}
%   \end{syntax}
%   This function attaches \meta{coffin_2} to \meta{coffin_1} such that the bounding box
%   of \meta{coffin_1} is not altered, \emph{i.e.}~\meta{coffin_2} can
%   protrude outside of the bounding box of the coffin. The alignment
%   is carried out by first calculating \meta{handle_1}, the
%   point of intersection of \meta{coffin_1-pole_1} and
%   \meta{coffin_1-pole_2}, and \meta{handle_2}, the point of intersection
%   of \meta{coffin_2-pole_1} and \meta{coffin_2-pole_2}. \meta{coffin_2} is
%   then attached to \meta{coffin_1} such that the relationship between
%   \meta{handle_1} and \meta{handle_2} is described by the \meta{x-offset}
%   and \meta{y-offset}. The two offsets should be given as dimension
%   expressions.
% \end{function}
%
% \begin{function}[updated = 2019-01-22]
%   {
%     \coffin_join:NnnNnnnn, \coffin_join:cnnNnnnn,
%     \coffin_join:Nnncnnnn, \coffin_join:cnncnnnn,
%     \coffin_gjoin:NnnNnnnn, \coffin_gjoin:cnnNnnnn,
%     \coffin_gjoin:Nnncnnnn, \coffin_gjoin:cnncnnnn
%   }
%   \begin{syntax}
%     \cs{coffin_join:NnnNnnnn}
%     ~~\meta{coffin_1} \Arg{coffin_1-pole_1} \Arg{coffin_1-pole_2}
%     ~~\meta{coffin_2} \Arg{coffin_2-pole_1} \Arg{coffin_2-pole_2}
%     ~~\Arg{x-offset} \Arg{y-offset}
%   \end{syntax}
%   This function joins \meta{coffin_2} to \meta{coffin_1} such that the bounding box
%   of \meta{coffin_1} may expand. The new bounding
%   box covers the area containing the bounding boxes of the two
%   original coffins. The alignment is carried out by first calculating
%   \meta{handle_1}, the point of intersection of \meta{coffin_1-pole_1} and
%   \meta{coffin_1-pole_2}, and \meta{handle_2}, the point of intersection
%   of \meta{coffin_2-pole_1} and \meta{coffin_2-pole_2}. \meta{coffin_2} is
%   then attached to \meta{coffin_1} such that the relationship between
%   \meta{handle_1} and \meta{handle_2} is described by the \meta{x-offset}
%   and \meta{y-offset}. The two offsets should be given as dimension
%   expressions.
% \end{function}
%
% \begin{function}[updated = 2012-07-20]
%   {\coffin_typeset:Nnnnn, \coffin_typeset:cnnnn}
%   \begin{syntax}
%     \cs{coffin_typeset:Nnnnn} \meta{coffin} \Arg{pole_1} \Arg{pole_2}
%     ~~\Arg{x-offset} \Arg{y-offset}
%   \end{syntax}
%   Typesetting is carried out by first calculating \meta{handle}, the
%   point of intersection of \meta{pole_1} and \meta{pole_2}. The coffin
%   is then typeset in horizontal mode such that the relationship between the
%   current reference point in the document and the \meta{handle} is described
%   by the \meta{x-offset} and \meta{y-offset}. The two offsets should
%   be given as dimension expressions. Typesetting a coffin is
%   therefore analogous to carrying out an alignment where the
%   \enquote{parent} coffin is the current insertion point.
% \end{function}
%
% \section{Measuring coffins}
%
% \begin{function}{\coffin_dp:N, \coffin_dp:c}
%   \begin{syntax}
%     \cs{coffin_dp:N} \meta{coffin}
%   \end{syntax}
%   Calculates the depth (below the baseline) of the \meta{coffin}
%   in a form suitable for use in a \meta{dimension expression}.
% \end{function}
%
% \begin{function}{\coffin_ht:N, \coffin_ht:c}
%   \begin{syntax}
%     \cs{coffin_ht:N} \meta{coffin}
%   \end{syntax}
%   Calculates the height (above the baseline) of the \meta{coffin}
%   in a form suitable for use in a \meta{dimension expression}.
% \end{function}
%
% \begin{function}{\coffin_wd:N, \coffin_wd:c}
%   \begin{syntax}
%     \cs{coffin_wd:N} \meta{coffin}
%   \end{syntax}
%   Calculates the width of the \meta{coffin} in a form
%   suitable for use in a \meta{dimension expression}.
% \end{function}
%
% \section{Coffin diagnostics}
%
% \begin{function}[updated = 2011-09-02]
%   {\coffin_display_handles:Nn, \coffin_display_handles:cn}
%   \begin{syntax}
%     \cs{coffin_display_handles:Nn} \meta{coffin} \Arg{color}
%   \end{syntax}
%   This function first calculates the intersections between all of
%   the \meta{poles} of the \meta{coffin} to give a set of
%   \meta{handles}. It then prints the  \meta{coffin} at the current
%   location in the source, with the  position of the \meta{handles}
%   marked on the coffin. The \meta{handles} are labelled as part
%   of this process: the locations of the \meta{handles} and the labels
%   are both printed in the \meta{color} specified.
% \end{function}
%
% \begin{function}[updated = 2011-09-02]
%   {\coffin_mark_handle:Nnnn, \coffin_mark_handle:cnnn}
%   \begin{syntax}
%     \cs{coffin_mark_handle:Nnnn} \meta{coffin} \Arg{pole_1} \Arg{pole_2} \Arg{color}
%   \end{syntax}
%   This function first calculates the \meta{handle} for the
%   \meta{coffin} as defined by the intersection of \meta{pole_1} and
%   \meta{pole_2}. It then marks the position of the \meta{handle}
%   on the \meta{coffin}. The \meta{handle} are labelled as part of
%   this process: the location of the \meta{handle} and the label are
%   both printed in the \meta{color} specified.
% \end{function}
%
% \begin{function}[updated = 2015-08-01]
%   {\coffin_show_structure:N, \coffin_show_structure:c}
%   \begin{syntax}
%     \cs{coffin_show_structure:N} \meta{coffin}
%   \end{syntax}
%   This function shows the structural information about the
%   \meta{coffin} in the terminal. The width, height and depth of the
%   typeset material are given, along with the location of all of the
%   poles of the coffin.
%
%   Notice that the poles of a coffin are defined by four values:
%   the $x$ and $y$ co-ordinates of a point that the pole
%   passes through and the $x$- and $y$-components of a
%   vector denoting the direction of the pole. It is the ratio between
%   the later, rather than the absolute values, which determines the
%   direction of the pole.
% \end{function}
%
% \begin{function}[added = 2014-08-22, updated = 2015-08-01]
%   {\coffin_log_structure:N, \coffin_log_structure:c}
%   \begin{syntax}
%     \cs{coffin_log_structure:N} \meta{coffin}
%   \end{syntax}
%   This function writes the structural information about the
%   \meta{coffin} in the log file. See also \cs{coffin_show_structure:N}
%   which displays the result in the terminal.
% \end{function}
%
% \section{Constants and variables}
%
% \begin{variable}{\c_empty_coffin}
%   A permanently empty coffin.
% \end{variable}
%
% \begin{variable}[added = 2012-06-19]{\l_tmpa_coffin, \l_tmpb_coffin}
%   Scratch coffins for local assignment. These are never used by
%   the kernel code, and so are safe for use with any \LaTeX3-defined
%   function. However, they may be overwritten by other non-kernel
%   code and so should only be used for short-term storage.
% \end{variable}
%
% \begin{variable}[added = 2019-01-24]{\g_tmpa_coffin, \g_tmpb_coffin}
%   Scratch coffins for global assignment. These are never used by
%   the kernel code, and so are safe for use with any \LaTeX3-defined
%   function. However, they may be overwritten by other non-kernel
%   code and so should only be used for short-term storage.
% \end{variable}
%
% \end{documentation}
%
% \begin{implementation}
%
% \section{\pkg{l3coffins} Implementation}
%
%    \begin{macrocode}
%<*initex|package>
%    \end{macrocode}
%
%    \begin{macrocode}
%<@@=coffin>
%    \end{macrocode}
%
% \subsection{Coffins: data structures and general variables}
%
% \begin{variable}{\l_@@_internal_box}
% \begin{variable}{\l_@@_internal_dim}
% \begin{variable}{\l_@@_internal_tl}
%   Scratch variables.
%    \begin{macrocode}
\box_new:N \l_@@_internal_box
\dim_new:N \l_@@_internal_dim
\tl_new:N  \l_@@_internal_tl
%    \end{macrocode}
% \end{variable}
% \end{variable}
% \end{variable}
%
% \begin{variable}{\c_@@_corners_prop}
%   The \enquote{corners}; of a coffin define the real content, as
%   opposed to the \TeX{} bounding box. They all start off in the same
%   place, of course.
%    \begin{macrocode}
\prop_const_from_keyval:Nn \c_@@_corners_prop
  {
    tl = { 0pt } { 0pt } ,
    tr = { 0pt } { 0pt } ,
    bl = { 0pt } { 0pt } ,
    br = { 0pt } { 0pt } ,
  }
%    \end{macrocode}
% \end{variable}
%
% \begin{variable}{\c_@@_poles_prop}
%   Pole positions are given for horizontal, vertical and reference-point
%   based values.
%    \begin{macrocode}
\prop_const_from_keyval:Nn \c_@@_poles_prop
  {
    l  = { 0pt } { 0pt } { 0pt } { 1000pt } ,
    hc = { 0pt } { 0pt } { 0pt } { 1000pt } ,
    r  = { 0pt } { 0pt } { 0pt } { 1000pt } ,
    b  = { 0pt } { 0pt } { 1000pt } { 0pt } ,
    vc = { 0pt } { 0pt } { 1000pt } { 0pt } ,
    t  = { 0pt } { 0pt } { 1000pt } { 0pt } ,
    B  = { 0pt } { 0pt } { 1000pt } { 0pt } ,
    H  = { 0pt } { 0pt } { 1000pt } { 0pt } ,
    T  = { 0pt } { 0pt } { 1000pt } { 0pt } ,
  }
%    \end{macrocode}
% \end{variable}
%
% \begin{variable}{\l_@@_slope_A_fp}
% \begin{variable}{\l_@@_slope_B_fp}
%   Used for calculations of intersections.
%    \begin{macrocode}
\fp_new:N \l_@@_slope_A_fp
\fp_new:N \l_@@_slope_B_fp
%    \end{macrocode}
% \end{variable}
% \end{variable}
%
% \begin{variable}{\l_@@_error_bool}
%   For propagating errors so that parts of the code can work around them.
%    \begin{macrocode}
\bool_new:N \l_@@_error_bool
%    \end{macrocode}
% \end{variable}
%
% \begin{variable}{\l_@@_offset_x_dim}
% \begin{variable}{\l_@@_offset_y_dim}
%   The offset between two sets of coffin handles when typesetting. These
%   values are corrected from those requested in an alignment for the
%   positions of the handles.
%    \begin{macrocode}
\dim_new:N \l_@@_offset_x_dim
\dim_new:N \l_@@_offset_y_dim
%    \end{macrocode}
% \end{variable}
% \end{variable}
%
% \begin{variable}{\l_@@_pole_a_tl}
% \begin{variable}{\l_@@_pole_b_tl}
%   Needed for finding the intersection of two poles.
%    \begin{macrocode}
\tl_new:N \l_@@_pole_a_tl
\tl_new:N \l_@@_pole_b_tl
%    \end{macrocode}
% \end{variable}
% \end{variable}
%
% \begin{variable}{\l_@@_x_dim}
% \begin{variable}{\l_@@_y_dim}
% \begin{variable}{\l_@@_x_prime_dim}
% \begin{variable}{\l_@@_y_prime_dim}
%   For calculating intersections and so forth.
%    \begin{macrocode}
\dim_new:N \l_@@_x_dim
\dim_new:N \l_@@_y_dim
\dim_new:N \l_@@_x_prime_dim
\dim_new:N \l_@@_y_prime_dim
%    \end{macrocode}
% \end{variable}
% \end{variable}
% \end{variable}
% \end{variable}
%
% \subsection{Basic coffin functions}
%
% There are a number of basic functions needed for creating coffins and
% placing material in them. This all relies on the following data
% structures.
%
% \begin{macro}[EXP]{\@@_to_value:N}
%   Coffins are a two-part structure and we rely on the internal nature of
%   box allocation to make everything work. As such, we need an interface
%   to turn coffin identifiers into numbers. For the purposes here, the
%   signature allowed is |N| despite the nature of the underlying primitive.
%    \begin{macrocode}
\cs_new_eq:NN \@@_to_value:N \tex_number:D
%    \end{macrocode}
% \end{macro}
%
% \begin{macro}[EXP, pTF]{\coffin_if_exist:N, \coffin_if_exist:c}
%   Several of the higher-level coffin functions would give multiple
%   errors if the coffin does not exist. A cleaner way to handle this
%   is provided here: both the box and the coffin structure are
%   checked.
%    \begin{macrocode}
\prg_new_conditional:Npnn \coffin_if_exist:N #1 { p , T , F , TF }
  {
    \cs_if_exist:NTF #1
      {
        \cs_if_exist:cTF { coffin ~ \@@_to_value:N #1 ~ poles }
          { \prg_return_true: }
          { \prg_return_false: }
      }
      { \prg_return_false: }
  }
\prg_generate_conditional_variant:Nnn \coffin_if_exist:N
  { c } { p , T , F , TF }
%    \end{macrocode}
% \end{macro}
%
% \begin{macro}{\@@_if_exist:NT}
%   Several of the higher-level coffin functions would give multiple
%   errors if the coffin does not exist. So a wrapper is provided to deal
%   with this correctly, issuing an error on erroneous use.
%    \begin{macrocode}
\cs_new_protected:Npn \@@_if_exist:NT #1#2
  {
    \coffin_if_exist:NTF #1
      { #2 }
      {
        \__kernel_msg_error:nnx { kernel } { unknown-coffin }
          { \token_to_str:N #1 }
      }
  }
%    \end{macrocode}
% \end{macro}
%
% \begin{macro}
%   {
%     \coffin_clear:N, \coffin_clear:c,
%     \coffin_gclear:N, \coffin_gclear:c
%   }
%   Clearing coffins means emptying the box and resetting all of the
%   structures.
%    \begin{macrocode}
\cs_new_protected:Npn \coffin_clear:N #1
  {
    \@@_if_exist:NT #1
      {
        \box_clear:N #1
        \@@_reset_structure:N #1
      }
  }
\cs_generate_variant:Nn \coffin_clear:N { c }
\cs_new_protected:Npn \coffin_gclear:N #1
  {
    \@@_if_exist:NT #1
      {
        \box_gclear:N #1
        \@@_greset_structure:N #1
      }
  }
\cs_generate_variant:Nn \coffin_gclear:N { c }
%    \end{macrocode}
% \end{macro}
%
% \begin{macro}{\coffin_new:N, \coffin_new:c}
%   Creating a new coffin means making the underlying box and adding the
%   data structures.  The \cs{debug_suspend:} and \cs{debug_resume:} functions
%   prevent \cs{prop_gclear_new:c} from writing useless information to
%   the log file.
%    \begin{macrocode}
\cs_new_protected:Npn \coffin_new:N #1
  {
    \box_new:N #1
    \debug_suspend:
    \prop_gclear_new:c { coffin ~ \@@_to_value:N #1 ~ corners }
    \prop_gclear_new:c { coffin ~ \@@_to_value:N #1 ~ poles }
    \prop_gset_eq:cN { coffin ~ \@@_to_value:N #1 ~ corners }
      \c_@@_corners_prop
    \prop_gset_eq:cN { coffin ~ \@@_to_value:N #1 ~ poles }
      \c_@@_poles_prop
    \debug_resume:
  }
\cs_generate_variant:Nn \coffin_new:N { c }
%    \end{macrocode}
% \end{macro}
%
% \begin{macro}
%   {
%     \hcoffin_set:Nn, \hcoffin_set:cn,
%     \hcoffin_gset:Nn, \hcoffin_gset:cn
%   }
%   Horizontal coffins are relatively easy: set the appropriate box,
%   reset the structures then update the handle positions.
%    \begin{macrocode}
\cs_new_protected:Npn \hcoffin_set:Nn #1#2
  {
    \@@_if_exist:NT #1
      {
        \hbox_set:Nn #1
          {
            \color_ensure_current:
            #2
          }
        \@@_update:N #1
      }
  }
\cs_generate_variant:Nn \hcoffin_set:Nn { c }
\cs_new_protected:Npn \hcoffin_gset:Nn #1#2
  {
    \@@_if_exist:NT #1
      {
        \hbox_gset:Nn #1
          {
            \color_ensure_current:
            #2
          }
        \@@_gupdate:N #1
      }
  }
\cs_generate_variant:Nn \hcoffin_gset:Nn { c }
%    \end{macrocode}
% \end{macro}
%
% \begin{macro}
%   {
%     \vcoffin_set:Nnn, \vcoffin_set:cnn,
%     \vcoffin_gset:Nnn, \vcoffin_gset:cnn
%   }
%  \begin{macro}{\@@_set_vertical:NnnNN}
%   Setting vertical coffins is more complex. First, the material is
%   typeset with a given width. The default handles and poles are set as
%   for a horizontal coffin, before finding the top baseline using a
%   temporary box. No \cs{color_ensure_current:} here as that would add a
%   whatsit to the start of the vertical box and mess up the location of the
%   \texttt{T}~pole (see \emph{\TeX{} by Topic} for discussion of the
%   \tn{vtop} primitive, used to do the measuring).
%    \begin{macrocode}
\cs_new_protected:Npn \vcoffin_set:Nnn #1#2#3
  {
    \@@_set_vertical:NnnNN #1 {#2} {#3}
      \vbox_set:Nn \@@_update:N
  }
\cs_generate_variant:Nn \vcoffin_set:Nnn { c }
\cs_new_protected:Npn \vcoffin_gset:Nnn #1#2#3
  {
    \@@_set_vertical:NnnNN #1 {#2} {#3}
      \vbox_gset:Nn \@@_gupdate:N
  }
\cs_generate_variant:Nn \vcoffin_gset:Nnn { c }
\cs_new_protected:Npn \@@_set_vertical:NnnNN #1#2#3#4#5
  {
    \@@_if_exist:NT #1
      {
        #4 #1
          {
            \dim_set:Nn \tex_hsize:D {#2}
%<*package>
            \dim_set_eq:NN \linewidth   \tex_hsize:D
            \dim_set_eq:NN \columnwidth \tex_hsize:D
%</package>
            #3
          }
        #5 #1
        \vbox_set_top:Nn \l_@@_internal_box { \vbox_unpack:N #1 }
        \@@_set_pole:Nnx #1 { T }
          {
            { 0pt }
            {
              \dim_eval:n
                { \box_ht:N #1 - \box_ht:N \l_@@_internal_box }
            }
            { 1000pt }
            { 0pt }
          }
        \box_clear:N \l_@@_internal_box
      }
  }
%    \end{macrocode}
% \end{macro}
% \end{macro}
%
% \begin{macro}
%   {\hcoffin_set:Nw, \hcoffin_set:cw, \hcoffin_gset:Nw, \hcoffin_gset:cw}
% \begin{macro}{\hcoffin_set_end:, \hcoffin_gset_end:}
% These are the \enquote{begin}/\enquote{end} versions of the above:
% watch the grouping!
%    \begin{macrocode}
\cs_new_protected:Npn \hcoffin_set:Nw #1
  {
    \@@_if_exist:NT #1
      {
        \hbox_set:Nw #1 \color_ensure_current:
          \cs_set_protected:Npn \hcoffin_set_end:
            {
              \hbox_set_end:
              \@@_update:N #1
            }
      }
  }
\cs_generate_variant:Nn \hcoffin_set:Nw { c }
\cs_new_protected:Npn \hcoffin_gset:Nw #1
  {
    \@@_if_exist:NT #1
      {
        \hbox_gset:Nw #1 \color_ensure_current:
          \cs_set_protected:Npn \hcoffin_gset_end:
            {
              \hbox_gset_end:
              \@@_gupdate:N #1
            }
      }
  }
\cs_generate_variant:Nn \hcoffin_gset:Nw { c }
\cs_new_protected:Npn \hcoffin_set_end: { }
\cs_new_protected:Npn \hcoffin_gset_end: { }
%    \end{macrocode}
% \end{macro}
% \end{macro}
%
% \begin{macro}
%   {\vcoffin_set:Nnw, \vcoffin_set:cnw, \vcoffin_gset:Nnw, \vcoffin_gset:cnw}
% \begin{macro}{\@@_set_vertical:NnNNNNw}
% \begin{macro}{\vcoffin_set_end:, \vcoffin_gset_end:}
%   The same for vertical coffins.
%    \begin{macrocode}
\cs_new_protected:Npn \vcoffin_set:Nnw #1#2
  {
    \@@_set_vertical:NnNNNNw #1 {#2} \vbox_set:Nw
      \vcoffin_set_end:
      \vbox_set_end: \@@_update:N
  }
\cs_generate_variant:Nn \vcoffin_set:Nnw { c }
\cs_new_protected:Npn \vcoffin_gset:Nnw #1#2
  {
    \@@_set_vertical:NnNNNNw #1 {#2} \vbox_gset:Nw
      \vcoffin_gset_end:
      \vbox_gset_end: \@@_gupdate:N
  }
\cs_generate_variant:Nn \vcoffin_gset:Nnw { c }
\cs_new_protected:Npn \@@_set_vertical:NnNNNNw #1#2#3#4#5#6
  {
    \@@_if_exist:NT #1
      {
        #3 #1
          \dim_set:Nn \tex_hsize:D {#2}
%<*package>
            \dim_set_eq:NN \linewidth   \tex_hsize:D
            \dim_set_eq:NN \columnwidth \tex_hsize:D
%</package>
          \cs_set_protected:Npn #4
            {
              #5
              #6 #1
              \vbox_set_top:Nn \l_@@_internal_box { \vbox_unpack:N #1 }
              \@@_set_pole:Nnx #1 { T }
                {
                  { 0pt }
                  {
                    \dim_eval:n
                      { \box_ht:N #1 - \box_ht:N \l_@@_internal_box }
                  }
                  { 1000pt }
                  { 0pt }
                }
              \box_clear:N \l_@@_internal_box
            }
      }
  }
\cs_new_protected:Npn \vcoffin_set_end: { }
\cs_new_protected:Npn \vcoffin_gset_end: { }
%    \end{macrocode}
% \end{macro}
% \end{macro}
% \end{macro}
%
% \begin{macro}
%   {
%     \coffin_set_eq:NN, \coffin_set_eq:Nc,
%     \coffin_set_eq:cN, \coffin_set_eq:cc,
%     \coffin_gset_eq:NN, \coffin_gset_eq:Nc,
%     \coffin_gset_eq:cN, \coffin_gset_eq:cc
%   }
%   Setting two coffins equal is just a wrapper around other functions.
%    \begin{macrocode}
\cs_new_protected:Npn \coffin_set_eq:NN #1#2
  {
    \@@_if_exist:NT #1
      {
        \box_set_eq:NN #1 #2
        \prop_set_eq:cc { coffin ~ \@@_to_value:N #1 ~ corners }
          { coffin ~ \@@_to_value:N #2 ~ corners }
         \prop_set_eq:cc { coffin ~ \@@_to_value:N #1 ~ poles }
          { coffin ~ \@@_to_value:N #2 ~ poles }
      }
  }
\cs_generate_variant:Nn \coffin_set_eq:NN { c , Nc , cc }
\cs_new_protected:Npn \coffin_gset_eq:NN #1#2
  {
    \@@_if_exist:NT #1
      {
        \box_gset_eq:NN #1 #2
        \prop_gset_eq:cc { coffin ~ \@@_to_value:N #1 ~ corners }
          { coffin ~ \@@_to_value:N #2 ~ corners }
         \prop_gset_eq:cc { coffin ~ \@@_to_value:N #1 ~ poles }
          { coffin ~ \@@_to_value:N #2 ~ poles }
      }
  }
\cs_generate_variant:Nn \coffin_gset_eq:NN { c , Nc , cc }
%    \end{macrocode}
% \end{macro}
%
% \begin{variable}{\c_empty_coffin}
% \begin{variable}{\l_@@_aligned_coffin}
% \begin{variable}{\l_@@_aligned_internal_coffin}
%   Special coffins: these cannot be set up earlier as they need
%   \cs{coffin_new:N}. The empty coffin is set as a box as the full
%   coffin-setting system needs some material which is not yet available.
%   The empty coffin is creted entirely by hand: not everything is in place
%   yet.
%    \begin{macrocode}
\coffin_new:N \c_empty_coffin
\coffin_new:N \l_@@_aligned_coffin
\coffin_new:N \l_@@_aligned_internal_coffin
%    \end{macrocode}
% \end{variable}
% \end{variable}
% \end{variable}
%
% \begin{variable}
%   {\l_tmpa_coffin, \l_tmpb_coffin, \g_tmpa_coffin, \g_tmpb_coffin}
%   The usual scratch space.
%    \begin{macrocode}
\coffin_new:N \l_tmpa_coffin
\coffin_new:N \l_tmpb_coffin
\coffin_new:N \g_tmpa_coffin
\coffin_new:N \g_tmpb_coffin
%    \end{macrocode}
% \end{variable}
%
% \subsection{Measuring coffins}
%
% \begin{macro}
%   {
%     \coffin_dp:N, \coffin_dp:c, \coffin_ht:N, \coffin_ht:c,
%     \coffin_wd:N, \coffin_wd:c
%   }
%   Coffins are just boxes when it comes to measurement. However, semantically
%   a separate set of functions are required.
%    \begin{macrocode}
\cs_new_eq:NN \coffin_dp:N \box_dp:N
\cs_new_eq:NN \coffin_dp:c \box_dp:c
\cs_new_eq:NN \coffin_ht:N \box_ht:N
\cs_new_eq:NN \coffin_ht:c \box_ht:c
\cs_new_eq:NN \coffin_wd:N \box_wd:N
\cs_new_eq:NN \coffin_wd:c \box_wd:c
%    \end{macrocode}
% \end{macro}
%
% \subsection{Coffins: handle and pole management}
%
% \begin{macro}{\@@_get_pole:NnN}
%   A simple wrapper around the recovery of a coffin pole, with some
%   error checking and recovery built-in.
%    \begin{macrocode}
\cs_new_protected:Npn \@@_get_pole:NnN #1#2#3
  {
    \prop_get:cnNF
      { coffin ~ \@@_to_value:N #1 ~ poles } {#2} #3
      {
        \__kernel_msg_error:nnxx { kernel } { unknown-coffin-pole }
          { \exp_not:n {#2} } { \token_to_str:N #1 }
        \tl_set:Nn #3 { { 0pt } { 0pt } { 0pt } { 0pt } }
      }
  }
%    \end{macrocode}
% \end{macro}
%
% \begin{macro}{\@@_reset_structure:N, \@@_greset_structure:N}
%   Resetting the structure is a simple copy job.
%    \begin{macrocode}
\cs_new_protected:Npn \@@_reset_structure:N #1
  {
    \prop_set_eq:cN { coffin ~ \@@_to_value:N #1 ~ corners }
      \c_@@_corners_prop
    \prop_set_eq:cN { coffin ~ \@@_to_value:N #1 ~ poles }
      \c_@@_poles_prop
  }
\cs_new_protected:Npn \@@_greset_structure:N #1
  {
    \prop_gset_eq:cN { coffin ~ \@@_to_value:N #1 ~ corners }
      \c_@@_corners_prop
    \prop_gset_eq:cN { coffin ~ \@@_to_value:N #1 ~ poles }
      \c_@@_poles_prop
  }
%    \end{macrocode}
% \end{macro}
%
% \begin{macro}
%   {
%     \coffin_set_horizontal_pole:Nnn, \coffin_set_horizontal_pole:cnn,
%     \coffin_gset_horizontal_pole:Nnn, \coffin_gset_horizontal_pole:cnn
%   }
% \begin{macro}{\@@_set_horizontal_pole:NnnN}
% \begin{macro}
%   {
%     \coffin_set_vertical_pole:Nnn, \coffin_set_vertical_pole:cnn,
%     \coffin_gset_vertical_pole:Nnn, \coffin_gset_vertical_pole:cnn
%   }
% \begin{macro}{\@@_set_vertical_pole:NnnN}
% \begin{macro}{\@@_set_pole:Nnn, \@@_set_pole:Nnx}
%   Setting the pole of a coffin at the user/designer level requires a
%   bit more care. The idea here is to provide a reasonable interface to
%   the system, then to do the setting with full expansion. The
%   three-argument version is used internally to do a direct setting.
%    \begin{macrocode}
\cs_new_protected:Npn \coffin_set_horizontal_pole:Nnn #1#2#3
  { \@@_set_horizontal_pole:NnnN #1 {#2} {#3} \prop_put:cnx }
\cs_generate_variant:Nn \coffin_set_horizontal_pole:Nnn { c }
\cs_new_protected:Npn \coffin_gset_horizontal_pole:Nnn #1#2#3
  { \@@_set_horizontal_pole:NnnN #1 {#2} {#3} \prop_gput:cnx }
\cs_generate_variant:Nn \coffin_gset_horizontal_pole:Nnn { c }
\cs_new_protected:Npn \@@_set_horizontal_pole:NnnN #1#2#3#4
  {
    \@@_if_exist:NT #1
      {
        #4 { coffin ~ \@@_to_value:N #1 ~ poles }
          {#2}
          {
            { 0pt } { \dim_eval:n {#3} }
            { 1000pt } { 0pt }
          }
      }
  }
\cs_new_protected:Npn \coffin_set_vertical_pole:Nnn #1#2#3
  { \@@_set_vertical_pole:NnnN #1 {#2} {#3} \prop_put:cnx }
\cs_generate_variant:Nn \coffin_set_vertical_pole:Nnn { c }
\cs_new_protected:Npn \coffin_gset_vertical_pole:Nnn #1#2#3
  { \@@_set_vertical_pole:NnnN #1 {#2} {#3} \prop_gput:cnx }
  \cs_generate_variant:Nn \coffin_gset_vertical_pole:Nnn { c }
\cs_new_protected:Npn \@@_set_vertical_pole:NnnN #1#2#3#4
  {
    \@@_if_exist:NT #1
      {
        #4 { coffin ~ \@@_to_value:N #1 ~ poles }
          {#2}
          {
            { \dim_eval:n {#3} } { 0pt }
            { 0pt } { 1000pt }
          }
      }
  }
\cs_new_protected:Npn \@@_set_pole:Nnn #1#2#3
  {
    \prop_put:cnn { coffin ~ \@@_to_value:N #1 ~ poles }
      {#2} {#3}
  }
\cs_generate_variant:Nn \@@_set_pole:Nnn { Nnx }
%    \end{macrocode}
% \end{macro}
% \end{macro}
% \end{macro}
% \end{macro}
% \end{macro}
%
% \begin{macro}{\@@_update:N, \@@_gupdate:N}
%   Simple shortcuts.
%    \begin{macrocode}
\cs_new_protected:Npn \@@_update:N #1
  {
    \@@_reset_structure:N #1
    \@@_update_corners:N #1
    \@@_update_poles:N #1
  }
\cs_new_protected:Npn \@@_gupdate:N #1
  {
    \@@_greset_structure:N #1
    \@@_gupdate_corners:N #1
    \@@_gupdate_poles:N #1
  }
%    \end{macrocode}
% \end{macro}
%
% \begin{macro}{\@@_update_corners:N, \@@_gupdate_corners:N}
% \begin{macro}{\@@_update_corners:NN}
% \begin{macro}{\@@_update_corners:NNN}
%   Updating the corners of a coffin is straight-forward as at this stage
%   there can be no rotation. So the corners of the content are just those
%   of the underlying \TeX{} box.
%    \begin{macrocode}
\cs_new_protected:Npn \@@_update_corners:N #1
  { \@@_update_corners:NN #1 \prop_put:Nnx }
\cs_new_protected:Npn \@@_gupdate_corners:N #1
  { \@@_update_corners:NN #1 \prop_gput:Nnx }
\cs_new_protected:Npn \@@_update_corners:NN #1#2
  {
    \exp_args:Nc \@@_update_corners:NNN
      { coffin ~ \@@_to_value:N #1 ~ corners }
      #1 #2
  }
\cs_new_protected:Npn \@@_update_corners:NNN #1#2#3
  {
    #3 #1
      { tl }
      { { 0pt } { \dim_eval:n { \box_ht:N #2 } } }
    #3 #1
      { tr }
      {
        { \dim_eval:n { \box_wd:N #2 } }
        { \dim_eval:n { \box_ht:N #2 } }
      }
    #3 #1
      { bl }
      { { 0pt } { \dim_eval:n { -\box_dp:N #2 } } }
    #3 #1
      { br }
      {
        { \dim_eval:n { \box_wd:N #2 } }
        { \dim_eval:n { -\box_dp:N #2 } }
      }
  }
%    \end{macrocode}
% \end{macro}
% \end{macro}
% \end{macro}
%
% \begin{macro}{\@@_update_poles:N, \@@_gupdate_poles:N}
% \begin{macro}{\@@_update_poles:NN}
% \begin{macro}{\@@_update_poles:NNN}
%   This function is called when a coffin is set, and updates the poles to
%   reflect the nature of size of the box. Thus this function only alters
%   poles where the default position is dependent on the size of the box.
%   It also does not set poles which are relevant only to vertical
%   coffins.
%    \begin{macrocode}
\cs_new_protected:Npn \@@_update_poles:N #1
  { \@@_update_poles:NN #1 \prop_put:Nnx }
\cs_new_protected:Npn \@@_gupdate_poles:N #1
  { \@@_update_poles:NN #1 \prop_gput:Nnx }
\cs_new_protected:Npn \@@_update_poles:NN #1#2
  {
    \exp_args:Nc \@@_update_poles:NNN
      { coffin ~ \@@_to_value:N #1 ~ poles }
      #1 #2
  }
\cs_new_protected:Npn \@@_update_poles:NNN #1#2#3
  {
    #3 #1 { hc }
      {
        { \dim_eval:n { 0.5 \box_wd:N #2 } }
        { 0pt } { 0pt } { 1000pt }
      }
    #3 #1 { r }
      {
        { \dim_eval:n { \box_wd:N #2 } }
        { 0pt } { 0pt } { 1000pt }
      }
    #3 #1 { vc }
      {
        { 0pt }
        { \dim_eval:n { ( \box_ht:N #2 - \box_dp:N #2 ) / 2 } }
        { 1000pt }
        { 0pt }
      }
    #3 #1 { t }
      {
        { 0pt }
        { \dim_eval:n { \box_ht:N #2 } }
        { 1000pt }
        { 0pt }
      }
    #3 #1 { b }
      {
        { 0pt }
        { \dim_eval:n { -\box_dp:N #2 } }
        { 1000pt }
        { 0pt }
      }
  }
%    \end{macrocode}
% \end{macro}
% \end{macro}
% \end{macro}
%
% \subsection{Coffins: calculation of pole intersections}
%
% \begin{macro}{\@@_calculate_intersection:Nnn}
% \begin{macro}{\@@_calculate_intersection:nnnnnnnn}
% \begin{macro}{\@@_calculate_intersection:nnnnnn}
%   The lead off in finding intersections is to recover the two poles
%   and then hand off to the auxiliary for the actual calculation. There
%   may of course not be an intersection, for which an error trap is
%   needed.
%    \begin{macrocode}
\cs_new_protected:Npn \@@_calculate_intersection:Nnn #1#2#3
  {
    \@@_get_pole:NnN #1 {#2} \l_@@_pole_a_tl
    \@@_get_pole:NnN #1 {#3} \l_@@_pole_b_tl
    \bool_set_false:N \l_@@_error_bool
    \exp_last_two_unbraced:Noo
      \@@_calculate_intersection:nnnnnnnn
        \l_@@_pole_a_tl \l_@@_pole_b_tl
    \bool_if:NT \l_@@_error_bool
      {
        \__kernel_msg_error:nn { kernel } { no-pole-intersection }
        \dim_zero:N \l_@@_x_dim
        \dim_zero:N \l_@@_y_dim
      }
  }
%    \end{macrocode}
%   The two poles passed here each have four values (as dimensions),
%   ($a$, $b$, $c$, $d$) and
%   ($a'$, $b'$, $c'$, $d'$). These are arguments
%   $1$--$4$ and $5$--$8$, respectively. In both
%   cases $a$ and  $b$ are the co-ordinates of a point on the
%   pole and $c$ and $d$ define the direction of the pole. Finding
%   the intersection depends on the directions of the poles, which are
%   given by $d / c$ and $d' / c'$. However, if one of the poles
%   is either horizontal or vertical then one or more of $c$, $d$,
%   $c'$ and $d'$ are zero and a special case is needed.
%    \begin{macrocode}
\cs_new_protected:Npn \@@_calculate_intersection:nnnnnnnn
  #1#2#3#4#5#6#7#8
  {
    \dim_compare:nNnTF {#3} = \c_zero_dim
%    \end{macrocode}
%   The case where the first pole is vertical.  So the $x$-component
%   of the interaction is at $a$. There is then a test on the
%   second pole: if it is also vertical then there is an error.
%    \begin{macrocode}
      {
        \dim_set:Nn \l_@@_x_dim {#1}
        \dim_compare:nNnTF {#7} = \c_zero_dim
          { \bool_set_true:N \l_@@_error_bool }
%    \end{macrocode}
%   The second pole may still be horizontal, in which case the
%   $y$-component of the intersection is $b'$. If not,
%   \[
%     y = \frac{d'}{c'} \left ( a - a' \right ) + b'
%   \]
%   with the $x$-component already known to be |#1|.
%    \begin{macrocode}
          {
            \dim_set:Nn \l_@@_y_dim
              {
                \dim_compare:nNnTF {#8} = \c_zero_dim
                  {#6}
                  {
                    \fp_to_dim:n
                      {
                          ( \dim_to_fp:n {#8} / \dim_to_fp:n {#7} )
                        * ( \dim_to_fp:n {#1} - \dim_to_fp:n {#5} )
                        + \dim_to_fp:n {#6}
                      }
                  }
              }
          }
      }
%    \end{macrocode}
%   If the first pole is not vertical then it may be horizontal. If so,
%   then the procedure is essentially the same as that already done but
%   with the $x$- and $y$-components interchanged.
%    \begin{macrocode}
      {
        \dim_compare:nNnTF {#4} = \c_zero_dim
          {
            \dim_set:Nn \l_@@_y_dim {#2}
            \dim_compare:nNnTF {#8} = { \c_zero_dim }
              { \bool_set_true:N \l_@@_error_bool }
              {
%    \end{macrocode}
%   Now we deal with the case where the second pole may be vertical, or
%   if not we have
%   \[
%     x = \frac{c'}{d'} \left ( b - b' \right ) + a'
%   \]
%   which is again handled by the same auxiliary.
%    \begin{macrocode}
                \dim_set:Nn \l_@@_x_dim
                  {
                    \dim_compare:nNnTF {#7} = \c_zero_dim
                      {#5}
                      {
                        \fp_to_dim:n
                          {
                              ( \dim_to_fp:n {#7} / \dim_to_fp:n {#8} )
                            * ( \dim_to_fp:n {#4} - \dim_to_fp:n {#6} )
                             + \dim_to_fp:n {#5}
                          }
                      }
                  }
              }
          }
%    \end{macrocode}
%   The first pole is neither horizontal nor vertical. To avoid even
%   more complexity, we now work out both slopes and pass to an auxiliary.
%    \begin{macrocode}
          {
            \use:x
              {
                \@@_calculate_intersection:nnnnnn
                  { \dim_to_fp:n {#4} / \dim_to_fp:n {#3} }
                  { \dim_to_fp:n {#8} / \dim_to_fp:n {#7} }
              }
                {#1} {#2} {#5} {#6}
          }
      }
  }
%    \end{macrocode}
%   Assuming the two poles are not parallel, then the intersection point is
%   found in two steps. First we find the $x$-value with
%   \[
%     x = \frac { sa - s'a'  - b + b' }{ s - s' }
%   \]
%   and then finding the $y$-value with
%   \[
%     y = s ( x - a ) + b
%   \]
%    \begin{macrocode}
\cs_set_protected:Npn \@@_calculate_intersection:nnnnnn #1#2#3#4#5#6
  {
    \fp_compare:nNnTF {#1} = {#2}
      { \bool_set_true:N \l_@@_error_bool }
      {
        \dim_set:Nn \l_@@_x_dim
          {
            \fp_to_dim:n
              {
                (
                    #1 * \dim_to_fp:n {#3}
                  - #2 * \dim_to_fp:n {#5}
                  - \dim_to_fp:n {#4}
                  + \dim_to_fp:n {#6}
                )
                /
                ( #1 - #2 )
              }
          }
        \dim_set:Nn \l_@@_y_dim
          {
            \fp_to_dim:n
              {
                  #1 * ( \l_@@_x_dim - \dim_to_fp:n {#3} )
                + \dim_to_fp:n {#4}
              }
          }
      }
  }
%    \end{macrocode}
% \end{macro}
% \end{macro}
% \end{macro}
%
% \subsection{Affine transformations}
%
% \begin{variable}{\l_@@_sin_fp}
% \begin{variable}{\l_@@_cos_fp}
%   Used for rotations to get the sine and cosine values.
%    \begin{macrocode}
\fp_new:N \l_@@_sin_fp
\fp_new:N \l_@@_cos_fp
%    \end{macrocode}
% \end{variable}
% \end{variable}
%
% \begin{variable}{\l_@@_bounding_prop}
%   A property list for the bounding box of a coffin. This is only needed
%   during the rotation, so there is just the one.
%    \begin{macrocode}
\prop_new:N \l_@@_bounding_prop
%    \end{macrocode}
% \end{variable}
%
% \begin{variable}{\l_@@_corners_prop, \l_@@_poles_prop}
%   Used to avoid needing to track scope for intermediate steps.
%    \begin{macrocode}
\prop_new:N \l_@@_corners_prop
\prop_new:N \l_@@_poles_prop
%    \end{macrocode}
% \end{variable}
%
% \begin{variable}{\l_@@_bounding_shift_dim}
%   The shift of the bounding box of a coffin from the real content.
%    \begin{macrocode}
\dim_new:N \l_@@_bounding_shift_dim
%    \end{macrocode}
% \end{variable}
%
% \begin{variable}{\l_@@_left_corner_dim}
% \begin{variable}{\l_@@_right_corner_dim}
% \begin{variable}{\l_@@_bottom_corner_dim}
% \begin{variable}{\l_@@_top_corner_dim}
%   These are used to hold maxima for the various corner values: these
%   thus define the minimum size of the bounding box after rotation.
%    \begin{macrocode}
\dim_new:N \l_@@_left_corner_dim
\dim_new:N \l_@@_right_corner_dim
\dim_new:N \l_@@_bottom_corner_dim
\dim_new:N \l_@@_top_corner_dim
%    \end{macrocode}
% \end{variable}
% \end{variable}
% \end{variable}
% \end{variable}
%
% \begin{macro}
%   {
%     \coffin_rotate:Nn, \coffin_rotate:cn,
%     \coffin_grotate:Nn, \coffin_grotate:cn
%   }
% \begin{macro}{\@@_rotate:NnNNN}
%   Rotating a coffin requires several steps which can be conveniently
%   run together. The sine and cosine of the angle in degrees are
%   computed.  This is then used to set \cs{l_@@_sin_fp} and
%   \cs{l_@@_cos_fp}, which are carried through unchanged for the rest
%   of the procedure.
%    \begin{macrocode}
\cs_new_protected:Npn \coffin_rotate:Nn #1#2
  { \@@_rotate:NnNNN #1 {#2} \box_rotate:Nn \prop_set_eq:cN \hbox_set:Nn }
\cs_generate_variant:Nn \coffin_rotate:Nn { c }
\cs_new_protected:Npn \coffin_grotate:Nn #1#2
  { \@@_rotate:NnNNN #1 {#2} \box_grotate:Nn \prop_gset_eq:cN \hbox_gset:Nn }
\cs_generate_variant:Nn \coffin_grotate:Nn { c }
\cs_new_protected:Npn \@@_rotate:NnNNN #1#2#3#4#5
  {
    \fp_set:Nn \l_@@_sin_fp { sind ( #2 ) }
    \fp_set:Nn \l_@@_cos_fp { cosd ( #2 ) }
%    \end{macrocode}
%   Use a local copy of the property lists to avoid needing to pass the
%   name and scope around.
%    \begin{macrocode}
    \prop_set_eq:Nc \l_@@_corners_prop
      { coffin ~ \@@_to_value:N #1 ~ corners }
    \prop_set_eq:Nc \l_@@_poles_prop
      { coffin ~ \@@_to_value:N #1 ~ poles }
%    \end{macrocode}
%   The corners and poles of the coffin can now be rotated around the
%    origin. This is best achieved using mapping functions.
%    \begin{macrocode}
    \prop_map_inline:Nn \l_@@_corners_prop
      { \@@_rotate_corner:Nnnn #1 {##1} ##2 }
    \prop_map_inline:Nn \l_@@_poles_prop
      { \@@_rotate_pole:Nnnnnn #1 {##1} ##2 }
%    \end{macrocode}
%   The bounding box of the coffin needs to be rotated, and to do this
%   the corners have to be found first. They are then rotated in the same
%   way as the corners of the coffin material itself.
%    \begin{macrocode}
    \@@_set_bounding:N #1
    \prop_map_inline:Nn \l_@@_bounding_prop
      { \@@_rotate_bounding:nnn {##1} ##2 }
%    \end{macrocode}
%   At this stage, there needs to be a calculation to find where the
%   corners of the content and the box itself will end up.
%    \begin{macrocode}
    \@@_find_corner_maxima:N #1
    \@@_find_bounding_shift:
    #3 #1 {#2}
%    \end{macrocode}
%   The correction of the box position itself takes place here. The idea
%   is that the bounding box for a coffin is tight up to the content, and
%   has the reference point at the bottom-left. The $x$-direction is
%   handled by moving the content by the difference in the positions of
%   the bounding box and the content left edge. The $y$-direction is
%   dealt with by moving the box down by any depth it has acquired. The
%   internal box is used here to allow for the next step.
%    \begin{macrocode}
    \hbox_set:Nn \l_@@_internal_box
      {
        \tex_kern:D
          \dim_eval:n
            { \l_@@_bounding_shift_dim - \l_@@_left_corner_dim }
          \exp_stop_f:
        \box_move_down:nn { \l_@@_bottom_corner_dim }
          { \box_use:N #1 }
      }
%    \end{macrocode}
%   If there have been any previous rotations then the size of the
%   bounding box will be bigger than the contents. This can be corrected
%   easily by setting the size of the box to the height and width of the
%   content. As this operation requires setting box dimensions and these
%   transcend grouping, the safe way to do this is to use the internal box
%   and to reset the result into the target box.
%    \begin{macrocode}
    \box_set_ht:Nn \l_@@_internal_box
      { \l_@@_top_corner_dim - \l_@@_bottom_corner_dim }
    \box_set_dp:Nn \l_@@_internal_box { 0pt }
    \box_set_wd:Nn \l_@@_internal_box
      { \l_@@_right_corner_dim - \l_@@_left_corner_dim }
    #5 #1 { \box_use_drop:N \l_@@_internal_box }
%    \end{macrocode}
%   The final task is to move the poles and corners such that they are
%   back in alignment with the box reference point.
%    \begin{macrocode}
    \prop_map_inline:Nn \l_@@_corners_prop
      { \@@_shift_corner:Nnnn #1 {##1} ##2 }
    \prop_map_inline:Nn \l_@@_poles_prop
      { \@@_shift_pole:Nnnnnn #1 {##1} ##2 }
%    \end{macrocode}
%   Update the coffin data.
%    \begin{macrocode}
    #4 { coffin ~ \@@_to_value:N #1 ~ corners }
      \l_@@_corners_prop
    #4 { coffin ~ \@@_to_value:N #1 ~ poles }
      \l_@@_poles_prop
  }
%    \end{macrocode}
% \end{macro}
% \end{macro}
%
% \begin{macro}{\@@_set_bounding:N}
%   The bounding box corners for a coffin are easy enough to find: this
%   is the same code as for the corners of the material itself, but
%   using a dedicated property list.
%    \begin{macrocode}
\cs_new_protected:Npn \@@_set_bounding:N #1
  {
    \prop_put:Nnx \l_@@_bounding_prop { tl }
      { { 0pt } { \dim_eval:n { \box_ht:N #1 } } }
    \prop_put:Nnx \l_@@_bounding_prop { tr }
      {
        { \dim_eval:n { \box_wd:N #1 } }
        { \dim_eval:n { \box_ht:N #1 } }
      }
    \dim_set:Nn \l_@@_internal_dim { -\box_dp:N #1 }
    \prop_put:Nnx \l_@@_bounding_prop { bl }
      { { 0pt } { \dim_use:N \l_@@_internal_dim } }
    \prop_put:Nnx \l_@@_bounding_prop { br }
      {
        { \dim_eval:n { \box_wd:N #1 } }
        { \dim_use:N \l_@@_internal_dim }
      }
  }
%    \end{macrocode}
% \end{macro}
%
% \begin{macro}{\@@_rotate_bounding:nnn}
% \begin{macro}{\@@_rotate_corner:Nnnn}
%   Rotating the position of the corner of the coffin is just a case
%   of treating this as a vector from the reference point. The same
%   treatment is used for the corners of the material itself and the
%   bounding box.
%    \begin{macrocode}
\cs_new_protected:Npn \@@_rotate_bounding:nnn #1#2#3
  {
    \@@_rotate_vector:nnNN {#2} {#3} \l_@@_x_dim \l_@@_y_dim
    \prop_put:Nnx \l_@@_bounding_prop {#1}
      { { \dim_use:N \l_@@_x_dim } { \dim_use:N \l_@@_y_dim } }
  }
\cs_new_protected:Npn \@@_rotate_corner:Nnnn #1#2#3#4
  {
    \@@_rotate_vector:nnNN {#3} {#4} \l_@@_x_dim \l_@@_y_dim
    \prop_put:Nnx \l_@@_corners_prop {#2}
      { { \dim_use:N \l_@@_x_dim } { \dim_use:N \l_@@_y_dim } }
  }
%    \end{macrocode}
% \end{macro}
% \end{macro}
%
% \begin{macro}{\@@_rotate_pole:Nnnnnn}
%   Rotating a single pole simply means shifting the co-ordinate of
%   the pole and its direction. The rotation here is about the bottom-left
%   corner of the coffin.
%    \begin{macrocode}
\cs_new_protected:Npn \@@_rotate_pole:Nnnnnn #1#2#3#4#5#6
  {
    \@@_rotate_vector:nnNN {#3} {#4} \l_@@_x_dim \l_@@_y_dim
    \@@_rotate_vector:nnNN {#5} {#6}
      \l_@@_x_prime_dim \l_@@_y_prime_dim
    \prop_put:Nnx \l_@@_poles_prop {#2}
      {
        { \dim_use:N \l_@@_x_dim } { \dim_use:N \l_@@_y_dim }
        { \dim_use:N \l_@@_x_prime_dim }
        { \dim_use:N \l_@@_y_prime_dim }
      }
  }
%    \end{macrocode}
% \end{macro}
%
% \begin{macro}{\@@_rotate_vector:nnNN}
%   A rotation function, which needs only an input vector (as dimensions)
%   and an output space. The values \cs{l_@@_cos_fp} and
%   \cs{l_@@_sin_fp} should previously have been set up correctly.
%   Working this way means that the floating point work is kept to a
%   minimum: for any given rotation the sin and cosine values do no
%   change, after all.
%    \begin{macrocode}
\cs_new_protected:Npn \@@_rotate_vector:nnNN #1#2#3#4
  {
    \dim_set:Nn #3
      {
        \fp_to_dim:n
          {
              \dim_to_fp:n {#1} * \l_@@_cos_fp
            - \dim_to_fp:n {#2} * \l_@@_sin_fp
          }
      }
    \dim_set:Nn #4
      {
        \fp_to_dim:n
          {
              \dim_to_fp:n {#1} * \l_@@_sin_fp
            + \dim_to_fp:n {#2} * \l_@@_cos_fp
          }
      }
  }
%    \end{macrocode}
% \end{macro}
%
% \begin{macro}{\@@_find_corner_maxima:N}
% \begin{macro}{\@@_find_corner_maxima_aux:nn}
%   The idea here is to find the extremities of the content of the
%   coffin. This is done by looking for the smallest values for the bottom
%   and left corners, and the largest values for the top and right
%   corners. The values start at the maximum dimensions so that the
%   case where all are positive or all are negative works out correctly.
%    \begin{macrocode}
\cs_new_protected:Npn \@@_find_corner_maxima:N #1
  {
    \dim_set:Nn \l_@@_top_corner_dim   { -\c_max_dim }
    \dim_set:Nn \l_@@_right_corner_dim { -\c_max_dim }
    \dim_set:Nn \l_@@_bottom_corner_dim { \c_max_dim }
    \dim_set:Nn \l_@@_left_corner_dim   { \c_max_dim }
    \prop_map_inline:Nn \l_@@_corners_prop
      { \@@_find_corner_maxima_aux:nn ##2 }
  }
\cs_new_protected:Npn \@@_find_corner_maxima_aux:nn #1#2
  {
    \dim_set:Nn \l_@@_left_corner_dim
     { \dim_min:nn { \l_@@_left_corner_dim } {#1} }
    \dim_set:Nn \l_@@_right_corner_dim
     { \dim_max:nn { \l_@@_right_corner_dim } {#1} }
    \dim_set:Nn \l_@@_bottom_corner_dim
     { \dim_min:nn { \l_@@_bottom_corner_dim } {#2} }
    \dim_set:Nn \l_@@_top_corner_dim
     { \dim_max:nn { \l_@@_top_corner_dim } {#2} }
  }
%    \end{macrocode}
% \end{macro}
% \end{macro}
%
% \begin{macro}{\@@_find_bounding_shift:}
% \begin{macro}{\@@_find_bounding_shift_aux:nn}
%   The approach to finding the shift for the bounding box is similar to
%   that for the corners. However, there is only one value needed here and
%   a fixed input property list, so things are a bit clearer.
%    \begin{macrocode}
\cs_new_protected:Npn \@@_find_bounding_shift:
  {
    \dim_set:Nn \l_@@_bounding_shift_dim { \c_max_dim }
    \prop_map_inline:Nn \l_@@_bounding_prop
      { \@@_find_bounding_shift_aux:nn ##2 }
  }
\cs_new_protected:Npn \@@_find_bounding_shift_aux:nn #1#2
  {
    \dim_set:Nn \l_@@_bounding_shift_dim
      { \dim_min:nn { \l_@@_bounding_shift_dim } {#1} }
  }
%    \end{macrocode}
% \end{macro}
% \end{macro}
%
% \begin{macro}{\@@_shift_corner:Nnnn}
% \begin{macro}{\@@_shift_pole:Nnnnnn}
%   Shifting the corners and poles of a coffin means subtracting the
%   appropriate values from the $x$- and $y$-components. For
%   the poles, this means that the direction vector is unchanged.
%    \begin{macrocode}
\cs_new_protected:Npn \@@_shift_corner:Nnnn #1#2#3#4
  {
    \prop_put:Nnx \l_@@_corners_prop {#2}
      {
        { \dim_eval:n { #3 - \l_@@_left_corner_dim } }
        { \dim_eval:n { #4 - \l_@@_bottom_corner_dim } }
      }
  }
\cs_new_protected:Npn \@@_shift_pole:Nnnnnn #1#2#3#4#5#6
  {
    \prop_put:Nnx \l_@@_poles_prop {#2}
      {
        { \dim_eval:n { #3 - \l_@@_left_corner_dim } }
        { \dim_eval:n { #4 - \l_@@_bottom_corner_dim } }
        {#5} {#6}
      }
  }
%    \end{macrocode}
% \end{macro}
% \end{macro}
%
% \begin{variable}{\l_@@_scale_x_fp}
% \begin{variable}{\l_@@_scale_y_fp}
%   Storage for the scaling factors in $x$ and $y$, respectively.
%    \begin{macrocode}
\fp_new:N \l_@@_scale_x_fp
\fp_new:N \l_@@_scale_y_fp
%    \end{macrocode}
% \end{variable}
% \end{variable}
%
% \begin{variable}{\l_@@_scaled_total_height_dim}
% \begin{variable}{\l_@@_scaled_width_dim}
%   When scaling, the values given have to be turned into absolute values.
%    \begin{macrocode}
\dim_new:N \l_@@_scaled_total_height_dim
\dim_new:N \l_@@_scaled_width_dim
%    \end{macrocode}
% \end{variable}
% \end{variable}
%
% \begin{macro}
%   {
%     \coffin_resize:Nnn, \coffin_resize:cnn,
%     \coffin_gresize:Nnn, \coffin_gresize:cnn
%   }
% \begin{macro}{\@@_resize:NnnNN}
%   Resizing a coffin begins by setting up the user-friendly names for
%   the dimensions of the coffin box. The new sizes are then turned into
%   scale factor. This is the same operation as takes place for the
%   underlying box, but that operation is grouped and so the same
%   calculation is done here.
%    \begin{macrocode}
\cs_new_protected:Npn \coffin_resize:Nnn #1#2#3
  {
    \@@_resize:NnnNN #1 {#2} {#3}
      \box_resize_to_wd_and_ht_plus_dp:Nnn
      \prop_set_eq:cN
  }
\cs_generate_variant:Nn \coffin_resize:Nnn { c }
\cs_new_protected:Npn \coffin_gresize:Nnn #1#2#3
  {
    \@@_resize:NnnNN #1 {#2} {#3}
      \box_gresize_to_wd_and_ht_plus_dp:Nnn
      \prop_gset_eq:cN
  }
\cs_generate_variant:Nn \coffin_gresize:Nnn { c }
\cs_new_protected:Npn \@@_resize:NnnNN #1#2#3#4#5
  {  
    \fp_set:Nn \l_@@_scale_x_fp
      { \dim_to_fp:n {#2} / \dim_to_fp:n { \coffin_wd:N #1 } }
    \fp_set:Nn \l_@@_scale_y_fp
      {
          \dim_to_fp:n {#3}
        / \dim_to_fp:n { \coffin_ht:N #1 + \coffin_dp:N #1 }
      }
    #4 #1 {#2} {#3}
    \@@_resize_common:NnnN #1 {#2} {#3} #5
  }
%    \end{macrocode}
% \end{macro}
% \end{macro}
%
% \begin{macro}{\@@_resize_common:NnnN}
%   The poles and corners of the coffin are scaled to the appropriate
%   places before actually resizing the underlying box.
%    \begin{macrocode}
\cs_new_protected:Npn \@@_resize_common:NnnN #1#2#3#4
  {
    \prop_set_eq:Nc \l_@@_corners_prop
      { coffin ~ \@@_to_value:N #1 ~ corners }
    \prop_set_eq:Nc \l_@@_poles_prop
      { coffin ~ \@@_to_value:N #1 ~ poles }
    \prop_map_inline:Nn \l_@@_corners_prop
      { \@@_scale_corner:Nnnn #1 {##1} ##2 }
    \prop_map_inline:Nn \l_@@_poles_prop
      { \@@_scale_pole:Nnnnnn #1 {##1} ##2 }
%    \end{macrocode}
%   Negative $x$-scaling values place the poles in the wrong
%   location: this is corrected here.
%    \begin{macrocode}
    \fp_compare:nNnT \l_@@_scale_x_fp < \c_zero_fp
      {
        \prop_map_inline:Nn \l_@@_corners_prop
          { \@@_x_shift_corner:Nnnn #1 {##1} ##2 }
        \prop_map_inline:Nn \l_@@_poles_prop
          { \@@_x_shift_pole:Nnnnnn #1 {##1} ##2 }
      }
    #4 { coffin ~ \@@_to_value:N #1 ~ corners }
      \l_@@_corners_prop
    #4 { coffin ~ \@@_to_value:N #1 ~ poles }
      \l_@@_poles_prop  
  }
%    \end{macrocode}
% \end{macro}
%
% \begin{macro}
%   {
%     \coffin_scale:Nnn, \coffin_scale:cnn,
%     \coffin_gscale:Nnn, \coffin_gscale:cnn
%   }
% \begin{macro}{\coffin_scale:NnnNN}
%   For scaling, the opposite calculation is done to find the new
%   dimensions for the coffin. Only the total height is needed, as this
%   is the shift required for corners and poles. The scaling is done
%   the \TeX{} way as this works properly with floating point values
%   without needing to use the \texttt{fp} module.
%    \begin{macrocode}
\cs_new_protected:Npn \coffin_scale:Nnn #1#2#3
  { \@@_scale:NnnNN #1 {#2} {#3} \box_scale:Nnn \prop_set_eq:cN }
\cs_generate_variant:Nn \coffin_scale:Nnn { c }
\cs_new_protected:Npn \coffin_gscale:Nnn #1#2#3
  { \@@_scale:NnnNN #1 {#2} {#3} \box_gscale:Nnn \prop_gset_eq:cN }
\cs_generate_variant:Nn \coffin_gscale:Nnn { c }
\cs_new_protected:Npn \@@_scale:NnnNN #1#2#3#4#5
  {
    \fp_set:Nn \l_@@_scale_x_fp {#2}
    \fp_set:Nn \l_@@_scale_y_fp {#3}
    #4 #1 { \l_@@_scale_x_fp } { \l_@@_scale_y_fp }
    \dim_set:Nn \l_@@_internal_dim
      { \coffin_ht:N #1 + \coffin_dp:N #1 }
    \dim_set:Nn \l_@@_scaled_total_height_dim
      { \fp_abs:n { \l_@@_scale_y_fp } \l_@@_internal_dim }
    \dim_set:Nn \l_@@_scaled_width_dim
      { -\fp_abs:n { \l_@@_scale_x_fp  } \coffin_wd:N #1 }
    \@@_resize_common:NnnN #1
      { \l_@@_scaled_width_dim } { \l_@@_scaled_total_height_dim }
      #5
  }
%    \end{macrocode}
% \end{macro}
% \end{macro}
%
% \begin{macro}{\@@_scale_vector:nnNN}
%   This functions scales a vector from the origin using the pre-set scale
%   factors in $x$ and $y$. This is a much less complex operation
%   than rotation, and as a result the code is a lot clearer.
%    \begin{macrocode}
\cs_new_protected:Npn \@@_scale_vector:nnNN #1#2#3#4
  {
    \dim_set:Nn #3
      { \fp_to_dim:n { \dim_to_fp:n {#1} * \l_@@_scale_x_fp } }
    \dim_set:Nn #4
      { \fp_to_dim:n { \dim_to_fp:n {#2} * \l_@@_scale_y_fp } }
  }
%    \end{macrocode}
% \end{macro}
%
% \begin{macro}{\@@_scale_corner:Nnnn}
% \begin{macro}{\@@_scale_pole:Nnnnnn}
%   Scaling both corners and poles is a simple calculation using the
%   preceding vector scaling.
%    \begin{macrocode}
\cs_new_protected:Npn \@@_scale_corner:Nnnn #1#2#3#4
  {
    \@@_scale_vector:nnNN {#3} {#4} \l_@@_x_dim \l_@@_y_dim
    \prop_put:Nnx \l_@@_corners_prop {#2}
      { { \dim_use:N \l_@@_x_dim } { \dim_use:N \l_@@_y_dim } }
  }
\cs_new_protected:Npn \@@_scale_pole:Nnnnnn #1#2#3#4#5#6
  {
    \@@_scale_vector:nnNN {#3} {#4} \l_@@_x_dim \l_@@_y_dim
    \prop_put:Nnx \l_@@_poles_prop {#2}
      {
        { \dim_use:N \l_@@_x_dim } { \dim_use:N \l_@@_y_dim }
        {#5} {#6}
      }
  }
%    \end{macrocode}
% \end{macro}
% \end{macro}
%
% \begin{macro}{\@@_x_shift_corner:Nnnn}
% \begin{macro}{\@@_x_shift_pole:Nnnnnn}
%   These functions correct for the $x$ displacement that takes
%   place with a negative horizontal scaling.
%    \begin{macrocode}
\cs_new_protected:Npn \@@_x_shift_corner:Nnnn #1#2#3#4
  {
    \prop_put:Nnx \l_@@_corners_prop {#2}
      {
        { \dim_eval:n { #3 + \box_wd:N #1 } } {#4}
      }
  }
\cs_new_protected:Npn \@@_x_shift_pole:Nnnnnn #1#2#3#4#5#6
  {
    \prop_put:Nnx \l_@@_poles_prop {#2}
      {
        { \dim_eval:n { #3 + \box_wd:N #1 } } {#4}
        {#5} {#6}
      }
  }
%    \end{macrocode}
% \end{macro}
% \end{macro}
%
% \subsection{Aligning and typesetting of coffins}
%
% \begin{macro}
%   {
%     \coffin_join:NnnNnnnn, \coffin_join:cnnNnnnn,
%     \coffin_join:Nnncnnnn, \coffin_join:cnncnnnn,
%     \coffin_gjoin:NnnNnnnn, \coffin_gjoin:cnnNnnnn,
%     \coffin_gjoin:Nnncnnnn, \coffin_gjoin:cnncnnnn
%   }
% \begin{macro}{\@@_join:NnnNnnnnN}
%   This command joins two coffins, using a horizontal and vertical pole
%   from each coffin and making an offset between the two. The result
%   is stored as the as a third coffin, which has all of its handles
%   reset to standard values. First, the more basic alignment function is
%   used to get things started.
%    \begin{macrocode}
\cs_new_protected:Npn \coffin_join:NnnNnnnn #1#2#3#4#5#6#7#8
  {
    \@@_join:NnnNnnnnN #1 {#2} {#3} #4 {#5} {#6} {#7} {#8}
      \coffin_set_eq:NN
  }
\cs_generate_variant:Nn \coffin_join:NnnNnnnn { c , Nnnc , cnnc }
\cs_new_protected:Npn \coffin_gjoin:NnnNnnnn #1#2#3#4#5#6#7#8
  {
    \@@_join:NnnNnnnnN #1 {#2} {#3} #4 {#5} {#6} {#7} {#8}
      \coffin_gset_eq:NN
  }
\cs_generate_variant:Nn \coffin_gjoin:NnnNnnnn { c , Nnnc , cnnc }
\cs_new_protected:Npn \@@_join:NnnNnnnnN #1#2#3#4#5#6#7#8#9
  {
    \@@_align:NnnNnnnnN
      #1 {#2} {#3} #4 {#5} {#6} {#7} {#8} \l_@@_aligned_coffin
%    \end{macrocode}
%   Correct the placement of the reference point. If the $x$-offset
%   is negative then the reference point of the second box is to the left
%   of that of the first, which is corrected using a kern. On the right
%   side the first box might stick out, which would show up if it is wider
%   than the sum of the $x$-offset and the width of the second box.
%   So a second kern may be needed.
%    \begin{macrocode}
    \hbox_set:Nn \l_@@_aligned_coffin
      {
        \dim_compare:nNnT { \l_@@_offset_x_dim } < \c_zero_dim
          { \tex_kern:D -\l_@@_offset_x_dim }
        \hbox_unpack:N \l_@@_aligned_coffin
        \dim_set:Nn \l_@@_internal_dim
          { \l_@@_offset_x_dim - \box_wd:N #1 + \box_wd:N #4 }
        \dim_compare:nNnT \l_@@_internal_dim < \c_zero_dim
          { \tex_kern:D -\l_@@_internal_dim }
      }
%    \end{macrocode}
%   The coffin structure is reset, and the corners are cleared: only
%   those from the two parent coffins are needed.
%    \begin{macrocode}
   \@@_reset_structure:N \l_@@_aligned_coffin
   \prop_clear:c
     {
       coffin ~ \@@_to_value:N \l_@@_aligned_coffin
       \c_space_tl corners
     }
   \@@_update_poles:N \l_@@_aligned_coffin
%    \end{macrocode}
%   The structures of the parent coffins are now transferred to the new
%   coffin, which requires that the appropriate offsets are applied. That
%   then depends on whether any shift was needed.
%    \begin{macrocode}
    \dim_compare:nNnTF \l_@@_offset_x_dim < \c_zero_dim
      {
        \@@_offset_poles:Nnn #1 { -\l_@@_offset_x_dim } { 0pt }
        \@@_offset_poles:Nnn #4 { 0pt } { \l_@@_offset_y_dim }
        \@@_offset_corners:Nnn #1 { -\l_@@_offset_x_dim } { 0pt }
        \@@_offset_corners:Nnn #4 { 0pt } { \l_@@_offset_y_dim }
      }
      {
        \@@_offset_poles:Nnn #1 { 0pt } { 0pt }
        \@@_offset_poles:Nnn #4
          { \l_@@_offset_x_dim } { \l_@@_offset_y_dim }
        \@@_offset_corners:Nnn #1 { 0pt } { 0pt }
        \@@_offset_corners:Nnn #4
          { \l_@@_offset_x_dim } { \l_@@_offset_y_dim }
      }
    \@@_update_vertical_poles:NNN #1 #4 \l_@@_aligned_coffin
    #9 #1 \l_@@_aligned_coffin
  }
%    \end{macrocode}
% \end{macro}
% \end{macro}
%
% \begin{macro}
%   {
%     \coffin_attach:NnnNnnnn, \coffin_attach:cnnNnnnn,
%     \coffin_attach:Nnncnnnn, \coffin_attach:cnncnnnn,
%     \coffin_gattach:NnnNnnnn, \coffin_gattach:cnnNnnnn,
%     \coffin_gattach:Nnncnnnn, \coffin_gattach:cnncnnnn
%   }
% \begin{macro}{\@@_attach:NnnNnnnnN}
% \begin{macro}{\@@_attach_mark:NnnNnnnn}
%   A more simple version of the above, as it simply uses the size of the
%   first coffin for the new one. This means that the work here is rather
%   simplified compared to the above code. The function used when marking
%   a position is hear also as it is similar but without the structure
%   updates.
%    \begin{macrocode}
\cs_new_protected:Npn \coffin_attach:NnnNnnnn #1#2#3#4#5#6#7#8
  {
    \@@_attach:NnnNnnnnN #1 {#2} {#3} #4 {#5} {#6} {#7} {#8}
      \coffin_set_eq:NN
  }
\cs_generate_variant:Nn \coffin_attach:NnnNnnnn { c , Nnnc , cnnc }
\cs_new_protected:Npn \coffin_gattach:NnnNnnnn #1#2#3#4#5#6#7#8
  {
    \@@_attach:NnnNnnnnN #1 {#2} {#3} #4 {#5} {#6} {#7} {#8}
      \coffin_gset_eq:NN
  }
\cs_generate_variant:Nn \coffin_gattach:NnnNnnnn { c , Nnnc , cnnc }
\cs_new_protected:Npn \@@_attach:NnnNnnnnN #1#2#3#4#5#6#7#8#9
  {
    \@@_align:NnnNnnnnN
      #1 {#2} {#3} #4 {#5} {#6} {#7} {#8} \l_@@_aligned_coffin
    \box_set_ht:Nn \l_@@_aligned_coffin { \box_ht:N #1 }
    \box_set_dp:Nn \l_@@_aligned_coffin { \box_dp:N #1 }
    \box_set_wd:Nn \l_@@_aligned_coffin { \box_wd:N #1 }
    \@@_reset_structure:N \l_@@_aligned_coffin
    \prop_set_eq:cc
      {
        coffin ~ \@@_to_value:N \l_@@_aligned_coffin 
        \c_space_tl corners
      }
      { coffin ~ \@@_to_value:N #1 ~ corners }
    \@@_update_poles:N  \l_@@_aligned_coffin
    \@@_offset_poles:Nnn #1 { 0pt } { 0pt }
    \@@_offset_poles:Nnn #4
      { \l_@@_offset_x_dim } { \l_@@_offset_y_dim }
    \@@_update_vertical_poles:NNN #1 #4 \l_@@_aligned_coffin
    \coffin_set_eq:NN #1 \l_@@_aligned_coffin
  }
\cs_new_protected:Npn \@@_attach_mark:NnnNnnnn #1#2#3#4#5#6#7#8
  {
    \@@_align:NnnNnnnnN
      #1 {#2} {#3} #4 {#5} {#6} {#7} {#8} \l_@@_aligned_coffin
    \box_set_ht:Nn \l_@@_aligned_coffin { \box_ht:N #1 }
    \box_set_dp:Nn \l_@@_aligned_coffin { \box_dp:N #1 }
    \box_set_wd:Nn \l_@@_aligned_coffin { \box_wd:N #1 }
    \box_set_eq:NN #1 \l_@@_aligned_coffin
  }
%    \end{macrocode}
% \end{macro}
% \end{macro}
% \end{macro}
%
% \begin{macro}{\@@_align:NnnNnnnnN}
%   The internal function aligns the two coffins into a third one, but
%   performs no corrections on the resulting coffin poles. The process
%   begins by finding the points of intersection for the poles for each
%   of the input coffins. Those for the first coffin are worked out after
%   those for the second coffin, as this allows the `primed'
%   storage area to be used for the second coffin. The `real' box
%   offsets are then calculated, before using these to re-box the
%   input coffins. The default poles are then set up, but the final result
%   depends on how the bounding box is being handled.
%    \begin{macrocode}
\cs_new_protected:Npn \@@_align:NnnNnnnnN #1#2#3#4#5#6#7#8#9
  {
    \@@_calculate_intersection:Nnn #4 {#5} {#6}
    \dim_set:Nn \l_@@_x_prime_dim { \l_@@_x_dim }
    \dim_set:Nn \l_@@_y_prime_dim { \l_@@_y_dim }
    \@@_calculate_intersection:Nnn #1 {#2} {#3}
    \dim_set:Nn \l_@@_offset_x_dim
      { \l_@@_x_dim - \l_@@_x_prime_dim + #7 }
    \dim_set:Nn \l_@@_offset_y_dim
      { \l_@@_y_dim - \l_@@_y_prime_dim + #8 }
    \hbox_set:Nn \l_@@_aligned_internal_coffin
      {
        \box_use:N #1
        \tex_kern:D -\box_wd:N #1
        \tex_kern:D \l_@@_offset_x_dim
        \box_move_up:nn { \l_@@_offset_y_dim } { \box_use:N #4 }
      }
    \coffin_set_eq:NN #9 \l_@@_aligned_internal_coffin
  }
%    \end{macrocode}
% \end{macro}
%
% \begin{macro}{\@@_offset_poles:Nnn}
% \begin{macro}{\@@_offset_pole:Nnnnnnn}
%   Transferring structures from one coffin to another requires that the
%   positions are updated by the offset between the two coffins. This is
%   done by mapping to the property list of the source coffins, moving
%   as appropriate and saving to the new coffin data structures. The
%   test for a |-| means that the structures from the parent coffins
%   are uniquely labelled and do not depend on the order of alignment.
%   The pay off for this is that |-| should not be used in coffin pole
%   or handle names, and that multiple alignments do not result in a
%   whole set of values.
%    \begin{macrocode}
\cs_new_protected:Npn \@@_offset_poles:Nnn #1#2#3
  {
    \prop_map_inline:cn { coffin ~ \@@_to_value:N #1 ~ poles }
      { \@@_offset_pole:Nnnnnnn #1 {##1} ##2 {#2} {#3} }
  }
\cs_new_protected:Npn \@@_offset_pole:Nnnnnnn #1#2#3#4#5#6#7#8
  {
    \dim_set:Nn \l_@@_x_dim { #3 + #7 }
    \dim_set:Nn \l_@@_y_dim { #4 + #8 }
    \tl_if_in:nnTF {#2} { - }
      { \tl_set:Nn \l_@@_internal_tl { {#2} } }
      { \tl_set:Nn \l_@@_internal_tl { { #1 - #2 } } }
    \exp_last_unbraced:NNo \@@_set_pole:Nnx \l_@@_aligned_coffin
      { \l_@@_internal_tl }
      {
        { \dim_use:N \l_@@_x_dim } { \dim_use:N \l_@@_y_dim }
        {#5} {#6}
      }
  }
%    \end{macrocode}
% \end{macro}
% \end{macro}
%
% \begin{macro}{\@@_offset_corners:Nnn}
% \begin{macro}{\@@_offset_corner:Nnnnn}
%   Saving the offset corners of a coffin is very similar, except that
%   there is no need to worry about naming: every corner can be saved
%   here as order is unimportant.
%    \begin{macrocode}
\cs_new_protected:Npn \@@_offset_corners:Nnn #1#2#3
  {
    \prop_map_inline:cn { coffin ~ \@@_to_value:N #1 ~ corners }
      { \@@_offset_corner:Nnnnn #1 {##1} ##2 {#2} {#3} }
  }
\cs_new_protected:Npn \@@_offset_corner:Nnnnn #1#2#3#4#5#6
  {
    \prop_put:cnx
      {
        coffin ~ \@@_to_value:N \l_@@_aligned_coffin
        \c_space_tl corners
      }
      { #1 - #2 }
      {
        { \dim_eval:n { #3 + #5 } }
        { \dim_eval:n { #4 + #6 } }
      }
  }
%    \end{macrocode}
% \end{macro}
% \end{macro}
%
% \begin{macro}{\@@_update_vertical_poles:NNN}
% \begin{macro}{\@@_update_T:nnnnnnnnN}
% \begin{macro}{\@@_update_B:nnnnnnnnN}
%   The \texttt{T} and \texttt{B} poles need to be recalculated
%   after alignment. These functions find the larger absolute value for
%   the poles, but this is of course only logical when the poles are
%   horizontal.
%    \begin{macrocode}
\cs_new_protected:Npn \@@_update_vertical_poles:NNN #1#2#3
  {
    \@@_get_pole:NnN #3 { #1 -T } \l_@@_pole_a_tl
    \@@_get_pole:NnN #3 { #2 -T } \l_@@_pole_b_tl
    \exp_last_two_unbraced:Noo \@@_update_T:nnnnnnnnN
      \l_@@_pole_a_tl \l_@@_pole_b_tl #3
    \@@_get_pole:NnN #3 { #1 -B } \l_@@_pole_a_tl
    \@@_get_pole:NnN #3 { #2 -B } \l_@@_pole_b_tl
    \exp_last_two_unbraced:Noo \@@_update_B:nnnnnnnnN
      \l_@@_pole_a_tl \l_@@_pole_b_tl #3
  }
\cs_new_protected:Npn \@@_update_T:nnnnnnnnN #1#2#3#4#5#6#7#8#9
  {
    \dim_compare:nNnTF {#2} < {#6}
      {
        \@@_set_pole:Nnx #9 { T }
          { { 0pt } {#6} { 1000pt } { 0pt } }
      }
      {
        \@@_set_pole:Nnx #9 { T }
          { { 0pt } {#2} { 1000pt } { 0pt } }
      }
  }
\cs_new_protected:Npn \@@_update_B:nnnnnnnnN #1#2#3#4#5#6#7#8#9
  {
    \dim_compare:nNnTF {#2} < {#6}
      {
        \@@_set_pole:Nnx #9 { B }
          { { 0pt } {#2}  { 1000pt } { 0pt } }
      }
      {
        \@@_set_pole:Nnx #9 { B }
          { { 0pt } {#6} { 1000pt } { 0pt } }
      }
  }
%    \end{macrocode}
% \end{macro}
% \end{macro}
% \end{macro}
%
% \begin{variable}{\c_@@_empty_coffin}
%   An empty-but-horizontal coffin.
%    \begin{macrocode}
\coffin_new:N \c_@@_empty_coffin
\tex_setbox:D \c_@@_empty_coffin = \tex_hbox:D { }
%    \end{macrocode}
% \end{variable}
%
% \begin{macro}{\coffin_typeset:Nnnnn, \coffin_typeset:cnnnn}
%   Typesetting a coffin means aligning it with the current position,
%   which is done using a coffin with no content at all. As well as aligning to
%   the empty coffin, there is also a need to leave vertical mode, if necessary.
%    \begin{macrocode}
\cs_new_protected:Npn \coffin_typeset:Nnnnn #1#2#3#4#5
  {
    \mode_leave_vertical:
    \@@_align:NnnNnnnnN \c_@@_empty_coffin { H } { l }
      #1 {#2} {#3} {#4} {#5} \l_@@_aligned_coffin
    \box_use_drop:N \l_@@_aligned_coffin
  }
\cs_generate_variant:Nn \coffin_typeset:Nnnnn { c }
%    \end{macrocode}
% \end{macro}
%
% \subsection{Coffin diagnostics}
%
% \begin{variable}{\l_@@_display_coffin}
% \begin{variable}{\l_@@_display_coord_coffin}
% \begin{variable}{\l_@@_display_pole_coffin}
%   Used for printing coffins with data structures attached.
%    \begin{macrocode}
\coffin_new:N \l_@@_display_coffin
\coffin_new:N \l_@@_display_coord_coffin
\coffin_new:N \l_@@_display_pole_coffin
%    \end{macrocode}
% \end{variable}
% \end{variable}
% \end{variable}
%
% \begin{variable}{\l_@@_display_handles_prop}
%   This property list is used to print coffin handles at suitable
%   positions. The offsets are expressed as multiples of the basic offset
%   value, which therefore acts as a scale-factor.
%    \begin{macrocode}
\prop_new:N \l_@@_display_handles_prop
\prop_put:Nnn \l_@@_display_handles_prop { tl }
  { { b } { r } { -1 } { 1 } }
\prop_put:Nnn \l_@@_display_handles_prop { thc }
  { { b } { hc } { 0 } { 1 } }
\prop_put:Nnn \l_@@_display_handles_prop { tr }
  { { b } { l } { 1 } { 1 } }
\prop_put:Nnn \l_@@_display_handles_prop { vcl }
  { { vc } { r } { -1 } { 0 } }
\prop_put:Nnn \l_@@_display_handles_prop { vchc }
  { { vc } { hc } { 0 } { 0 } }
\prop_put:Nnn \l_@@_display_handles_prop { vcr }
  { { vc } { l } { 1 } { 0 } }
\prop_put:Nnn \l_@@_display_handles_prop { bl }
  { { t } { r } { -1 } { -1 } }
\prop_put:Nnn \l_@@_display_handles_prop { bhc }
  { { t } { hc } { 0 } { -1 } }
\prop_put:Nnn \l_@@_display_handles_prop { br }
  { { t } { l } { 1 } { -1 } }
\prop_put:Nnn \l_@@_display_handles_prop { Tl }
  { { t } { r } { -1 } { -1 } }
\prop_put:Nnn \l_@@_display_handles_prop { Thc }
  { { t } { hc } { 0 } { -1 } }
\prop_put:Nnn \l_@@_display_handles_prop { Tr }
  { { t } { l } { 1 } { -1 } }
\prop_put:Nnn \l_@@_display_handles_prop { Hl }
  { { vc } { r } { -1 } { 1 } }
\prop_put:Nnn \l_@@_display_handles_prop { Hhc }
  { { vc } { hc } { 0 } { 1 } }
\prop_put:Nnn \l_@@_display_handles_prop { Hr }
  { { vc } { l } { 1 } { 1 } }
\prop_put:Nnn \l_@@_display_handles_prop { Bl }
  { { b } { r } { -1 } { -1 } }
\prop_put:Nnn \l_@@_display_handles_prop { Bhc }
  { { b } { hc } { 0 } { -1 } }
\prop_put:Nnn \l_@@_display_handles_prop { Br }
  { { b } { l } { 1 } { -1 } }
%    \end{macrocode}
% \end{variable}
%
% \begin{variable}{\l_@@_display_offset_dim}
%   The standard offset for the label from the handle position when
%   displaying handles.
%    \begin{macrocode}
\dim_new:N  \l_@@_display_offset_dim
\dim_set:Nn \l_@@_display_offset_dim { 2pt }
%    \end{macrocode}
% \end{variable}
%
% \begin{variable}{\l_@@_display_x_dim}
% \begin{variable}{\l_@@_display_y_dim}
%   As the intersections of poles have to be calculated to find which
%   ones to print, there is a need to avoid repetition. This is done
%   by saving the intersection into two dedicated values.
%    \begin{macrocode}
\dim_new:N \l_@@_display_x_dim
\dim_new:N \l_@@_display_y_dim
%    \end{macrocode}
% \end{variable}
% \end{variable}
%
% \begin{variable}{\l_@@_display_poles_prop}
%   A property list for printing poles: various things need to be deleted
%   from this to get a \enquote{nice} output.
%    \begin{macrocode}
\prop_new:N \l_@@_display_poles_prop
%    \end{macrocode}
% \end{variable}
%
% \begin{variable}{\l_@@_display_font_tl}
%   Stores the settings used to print coffin data: this keeps things
%   flexible.
%    \begin{macrocode}
\tl_new:N  \l_@@_display_font_tl
%<*initex>
\tl_set:Nn \l_@@_display_font_tl { } % TODO
%</initex>
%<*package>
\tl_set:Nn \l_@@_display_font_tl { \sffamily \tiny }
%</package>
%    \end{macrocode}
% \end{variable}
%
% \begin{macro}{\@@_color:n}
%   Calls \tn{color}, and otherwise does nothing if \tn{color} is not defined.
%    \begin{macrocode}
\cs_new_protected:Npn \@@_color:n #1
  { \cs_if_exist:NT \color { \color {#1} } }
%    \end{macrocode}
% \end{macro}
%
% \begin{macro}{\coffin_mark_handle:Nnnn, \coffin_mark_handle:cnnn}
% \begin{macro}{\@@_mark_handle_aux:nnnnNnn}
%   Marking a single handle is relatively easy. The standard attachment
%   function is used, meaning that there are two calculations for the
%   location. However, this is likely to be okay given the load expected.
%   Contrast with the more optimised version for showing all handles which
%   comes next.
%    \begin{macrocode}
\cs_new_protected:Npn \coffin_mark_handle:Nnnn #1#2#3#4
  {
    \hcoffin_set:Nn \l_@@_display_pole_coffin
      {
%<*initex>
        \hbox:n { \tex_vrule:D width 1pt height 1pt \scan_stop: } % TODO
%</initex>
%<*package>
        \@@_color:n {#4}
        \rule { 1pt } { 1pt }
%</package>
      }
    \@@_attach_mark:NnnNnnnn #1 {#2} {#3}
      \l_@@_display_pole_coffin { hc } { vc } { 0pt } { 0pt }
    \hcoffin_set:Nn \l_@@_display_coord_coffin
      {
%<*initex>
        % TODO
%</initex>
%<*package>
        \@@_color:n {#4}
%</package>
        \l_@@_display_font_tl
        ( \tl_to_str:n { #2 , #3 } )
      }
    \prop_get:NnN \l_@@_display_handles_prop
      { #2 #3 } \l_@@_internal_tl
    \quark_if_no_value:NTF \l_@@_internal_tl
      {
        \prop_get:NnN \l_@@_display_handles_prop
          { #3 #2 } \l_@@_internal_tl
        \quark_if_no_value:NTF \l_@@_internal_tl
          {
            \@@_attach_mark:NnnNnnnn #1 {#2} {#3}
              \l_@@_display_coord_coffin { l } { vc }
                { 1pt } { 0pt }
          }
          {
            \exp_last_unbraced:No \@@_mark_handle_aux:nnnnNnn
              \l_@@_internal_tl #1 {#2} {#3}
          }
      }
      {
        \exp_last_unbraced:No \@@_mark_handle_aux:nnnnNnn
          \l_@@_internal_tl #1 {#2} {#3}
      }
  }
\cs_new_protected:Npn \@@_mark_handle_aux:nnnnNnn #1#2#3#4#5#6#7
  {
    \@@_attach_mark:NnnNnnnn #5 {#6} {#7}
      \l_@@_display_coord_coffin {#1} {#2}
      { #3 \l_@@_display_offset_dim }
      { #4 \l_@@_display_offset_dim }
  }
\cs_generate_variant:Nn \coffin_mark_handle:Nnnn { c }
%    \end{macrocode}
% \end{macro}
% \end{macro}
%
% \begin{macro}{\coffin_display_handles:Nn, \coffin_display_handles:cn}
% \begin{macro}{\@@_display_handles_aux:nnnnnn}
% \begin{macro}{\@@_display_handles_aux:nnnn}
% \begin{macro}{\@@_display_attach:Nnnnn}
%   Printing the poles starts by removing any duplicates, for which the
%   \texttt{H} poles is used as the definitive version for the baseline
%   and bottom. Two loops are then used to find the combinations of
%   handles for all of these poles. This is done such that poles are
%   removed during the loops to avoid duplication.
%    \begin{macrocode}
\cs_new_protected:Npn \coffin_display_handles:Nn #1#2
  {
    \hcoffin_set:Nn \l_@@_display_pole_coffin
      {
%<*initex>
        \hbox:n { \tex_vrule:D width 1pt height 1pt \scan_stop: } % TODO
%</initex>
%<*package>
        \@@_color:n {#2}
        \rule { 1pt } { 1pt }
%</package>
      }
    \prop_set_eq:Nc \l_@@_display_poles_prop
      { coffin ~ \@@_to_value:N #1 ~ poles }
    \@@_get_pole:NnN #1 { H } \l_@@_pole_a_tl
    \@@_get_pole:NnN #1 { T } \l_@@_pole_b_tl
    \tl_if_eq:NNT \l_@@_pole_a_tl \l_@@_pole_b_tl
      { \prop_remove:Nn \l_@@_display_poles_prop { T } }
    \@@_get_pole:NnN #1 { B } \l_@@_pole_b_tl
    \tl_if_eq:NNT \l_@@_pole_a_tl \l_@@_pole_b_tl
      { \prop_remove:Nn \l_@@_display_poles_prop { B } }
    \coffin_set_eq:NN \l_@@_display_coffin #1
    \prop_map_inline:Nn \l_@@_display_poles_prop
      {
        \prop_remove:Nn \l_@@_display_poles_prop {##1}
        \@@_display_handles_aux:nnnnnn {##1} ##2 {#2}
      }
    \box_use_drop:N \l_@@_display_coffin
  }
%    \end{macrocode}
%   For each pole there is a check for an intersection, which here does
%   not give an error if none is found. The successful values are stored
%   and used to align the pole coffin with the main coffin for output.
%   The positions are recovered from the preset list if available.
%    \begin{macrocode}
\cs_new_protected:Npn \@@_display_handles_aux:nnnnnn #1#2#3#4#5#6
  {
    \prop_map_inline:Nn \l_@@_display_poles_prop
      {
        \bool_set_false:N \l_@@_error_bool
        \@@_calculate_intersection:nnnnnnnn {#2} {#3} {#4} {#5} ##2
        \bool_if:NF \l_@@_error_bool
          {
            \dim_set:Nn \l_@@_display_x_dim { \l_@@_x_dim }
            \dim_set:Nn \l_@@_display_y_dim { \l_@@_y_dim }
            \@@_display_attach:Nnnnn
              \l_@@_display_pole_coffin { hc } { vc }
              { 0pt } { 0pt }
            \hcoffin_set:Nn \l_@@_display_coord_coffin
              {
%<*initex>
                % TODO
%</initex>
%<*package>
                \@@_color:n {#6}
%</package>
                \l_@@_display_font_tl
                ( \tl_to_str:n { #1 , ##1 } )
              }
            \prop_get:NnN \l_@@_display_handles_prop
              { #1 ##1 } \l_@@_internal_tl
            \quark_if_no_value:NTF \l_@@_internal_tl
              {
                \prop_get:NnN \l_@@_display_handles_prop
                  { ##1 #1 } \l_@@_internal_tl
                \quark_if_no_value:NTF \l_@@_internal_tl
                  {
                    \@@_display_attach:Nnnnn
                      \l_@@_display_coord_coffin { l } { vc }
                      { 1pt } { 0pt }
                  }
                  {
                    \exp_last_unbraced:No
                      \@@_display_handles_aux:nnnn
                      \l_@@_internal_tl
                  }
              }
              {
                \exp_last_unbraced:No \@@_display_handles_aux:nnnn
                  \l_@@_internal_tl
              }
          }
      }
  }
\cs_new_protected:Npn \@@_display_handles_aux:nnnn #1#2#3#4
  {
    \@@_display_attach:Nnnnn
      \l_@@_display_coord_coffin {#1} {#2}
      { #3 \l_@@_display_offset_dim }
      { #4 \l_@@_display_offset_dim }
  }
\cs_generate_variant:Nn \coffin_display_handles:Nn { c }
%    \end{macrocode}
%   This is a dedicated version of \cs{coffin_attach:NnnNnnnn} with
%    a hard-wired first coffin. As the intersection is already known
%   and stored for the display coffin the code simply uses it directly,
%   with no calculation.
%    \begin{macrocode}
\cs_new_protected:Npn \@@_display_attach:Nnnnn #1#2#3#4#5
  {
    \@@_calculate_intersection:Nnn #1 {#2} {#3}
    \dim_set:Nn \l_@@_x_prime_dim { \l_@@_x_dim }
    \dim_set:Nn \l_@@_y_prime_dim { \l_@@_y_dim }
    \dim_set:Nn \l_@@_offset_x_dim
      { \l_@@_display_x_dim - \l_@@_x_prime_dim + #4 }
    \dim_set:Nn \l_@@_offset_y_dim
      { \l_@@_display_y_dim - \l_@@_y_prime_dim + #5 }
    \hbox_set:Nn \l_@@_aligned_coffin
      {
        \box_use:N \l_@@_display_coffin
        \tex_kern:D -\box_wd:N \l_@@_display_coffin
        \tex_kern:D \l_@@_offset_x_dim
        \box_move_up:nn { \l_@@_offset_y_dim } { \box_use:N #1 }
      }
    \box_set_ht:Nn \l_@@_aligned_coffin
      { \box_ht:N \l_@@_display_coffin }
    \box_set_dp:Nn \l_@@_aligned_coffin
      { \box_dp:N \l_@@_display_coffin }
    \box_set_wd:Nn \l_@@_aligned_coffin
      { \box_wd:N \l_@@_display_coffin }
    \box_set_eq:NN \l_@@_display_coffin \l_@@_aligned_coffin
  }
%    \end{macrocode}
% \end{macro}
% \end{macro}
% \end{macro}
% \end{macro}
%
% \begin{macro}
%   {
%     \coffin_show_structure:N, \coffin_show_structure:c,
%     \coffin_log_structure:N, \coffin_log_structure:c,
%     \@@_show_structure:NN
%   }
%   For showing the various internal structures attached to a coffin in
%   a way that keeps things relatively readable. If there is no apparent
%   structure then the code complains.
%    \begin{macrocode}
\cs_new_protected:Npn \coffin_show_structure:N
  { \@@_show_structure:NN \msg_show:nnxxxx }
\cs_generate_variant:Nn \coffin_show_structure:N { c }
\cs_new_protected:Npn \coffin_log_structure:N
  { \@@_show_structure:NN \msg_log:nnxxxx }
\cs_generate_variant:Nn \coffin_log_structure:N { c }
\cs_new_protected:Npn \@@_show_structure:NN #1#2
  {
    \@@_if_exist:NT #2
      {
        #1 { LaTeX / kernel } { show-coffin }
          { \token_to_str:N #2 }
          {
            \iow_newline: >~ ht ~=~ \dim_eval:n { \coffin_ht:N #2 }
            \iow_newline: >~ dp ~=~ \dim_eval:n { \coffin_dp:N #2 }
            \iow_newline: >~ wd ~=~ \dim_eval:n { \coffin_wd:N #2 }
          }
          {
            \prop_map_function:cN
              { coffin ~ \@@_to_value:N #2 ~ poles }
              \msg_show_item_unbraced:nn
          }
          { }
      }
  }
%    \end{macrocode}
% \end{macro}
%
% \subsection{Messages}
%
%    \begin{macrocode}
\__kernel_msg_new:nnnn { kernel } { no-pole-intersection }
  { No~intersection~between~coffin~poles. }
  {
    LaTeX~was~asked~to~find~the~intersection~between~two~poles,~
    but~they~do~not~have~a~unique~meeting~point:~
    the~value~(0pt,~0pt)~will~be~used.
  }
\__kernel_msg_new:nnnn { kernel } { unknown-coffin }
  { Unknown~coffin~'#1'. }
  { The~coffin~'#1'~was~never~defined. }
\__kernel_msg_new:nnnn { kernel } { unknown-coffin-pole }
  { Pole~'#1'~unknown~for~coffin~'#2'. }
  {
    LaTeX~was~asked~to~find~a~typesetting~pole~for~a~coffin,~
    but~either~the~coffin~does~not~exist~or~the~pole~name~is~wrong.
  }
\__kernel_msg_new:nnn { kernel } { show-coffin }
  {
    Size~of~coffin~#1 : #2 \\
    Poles~of~coffin~#1 : #3 .
  }
%    \end{macrocode}
%
%    \begin{macrocode}
%</initex|package>
%    \end{macrocode}
%
% \end{implementation}
%
% \PrintIndex
