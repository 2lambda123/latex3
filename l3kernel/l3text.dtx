% \iffalse meta-comment
%
%% File: l3text.dtx
%
% Copyright (C) 2020-2022 The LaTeX Project
%
% It may be distributed and/or modified under the conditions of the
% LaTeX Project Public License (LPPL), either version 1.3c of this
% license or (at your option) any later version.  The latest version
% of this license is in the file
%
%    https://www.latex-project.org/lppl.txt
%
% This file is part of the "l3kernel bundle" (The Work in LPPL)
% and all files in that bundle must be distributed together.
%
% -----------------------------------------------------------------------
%
% The development version of the bundle can be found at
%
%    https://github.com/latex3/latex3
%
% for those people who are interested.
%
%<*driver>
\documentclass[full,kernel]{l3doc}
\begin{document}
  \DocInput{\jobname.dtx}
\end{document}
%</driver>
% \fi
%
% \title{^^A
%   The \textsf{l3text} package: text processing^^A
% }
%
% \author{^^A
%  The \LaTeX{} Project\thanks
%    {^^A
%      E-mail:
%        \href{mailto:latex-team@latex-project.org}
%          {latex-team@latex-project.org}^^A
%    }^^A
% }
%
% \date{Released 2022-05-04}
%
% \maketitle
%
% \begin{documentation}
%
% This module deals with manipulation of (formatted) text; such material is
% comprised of a restricted set of token list content. The functions provided
% here concern conversion of textual content for example in case changing,
% generation of bookmarks and extraction to tags. All of the major functions
% operate by expansion. Begin-group and end-group tokens in the \meta{text}
% are normalized and become |{| and |}|, respectively.
%
% \section{Expanding text}
%
% \begin{function}[EXP, added = 2020-01-02]{\text_expand:n}
%   \begin{syntax}
%     \cs{text_expand:n} \Arg{text}
%   \end{syntax}
%   Takes user input \meta{text} and expands the content.
%   Protected commands (typically
%   formatting) are left in place, and no processing takes place of
%   math mode material (as delimited by pairs given in
%   \cs{l_text_math_delims_tl} or as the argument to commands listed
%   in \cs{l_text_math_arg_tl}). Commands which are neither engine-
%   nor \LaTeX{} protected are expanded exhaustively.
%   Any commands listed in \cs{l_text_expand_exclude_tl},
%   \cs{l_text_accents_tl} and \cs{l_text_letterlike_tl} are excluded from
%   expansion.
% \end{function}
%
% \begin{function}[added = 2020-01-22]
%   {
%     \text_declare_expand_equivalent:Nn ,
%     \text_declare_expand_equivalent:cn
%   }
%   \begin{syntax}
%     \cs{text_declare_expand_equivalent:Nn} \meta{cmd} \Arg{replacement}
%   \end{syntax}
%   Declares that the \meta{replacement} tokens should be used whenever the
%   \meta{cmd} (a single token) is encountered. The \meta{replacement} tokens
%   should be expandable.
% \end{function}
%
% \section{Case changing}
%
% \begin{function}[EXP, added = 2019-11-20, updated = 2020-02-24]
%   {
%     \text_lowercase:n,  \text_uppercase:n,  \text_titlecase:n,
%       \text_titlecase_first:n,
%     \text_lowercase:nn, \text_uppercase:nn, \text_titlecase:nn,
%       \text_titlecase_first:nn
%   }
%   \begin{syntax}
%     \cs{text_uppercase:n}  \Arg{tokens}
%     \cs{text_uppercase:nn} \Arg{language} \Arg{tokens}
%   \end{syntax}
%   Takes user input \meta{text} first applies \cs{text_expand}, then
%   transforms the case of character tokens as specified by the
%   function name. The category code of letters are not changed by this
%   process (at least where they can be represented by the engine as a single
%   token: $8$-bit engines may require active characters).
% \end{function}
%
%   Upper- and lowercase have the obvious meanings. Titlecasing may be regarded
%   informally as converting the first character of the \meta{tokens} to
%   uppercase and the rest to lowercase. However, the process is more complex
%   than this as there are some situations where a single lowercase character
%   maps to a special form, for example \texttt{ij} in Dutch which becomes
%   \texttt{IJ}. The \texttt{titlecase_first} variant does not attempt
%   any case changing at all after the first letter has been processed.
%
%   Importantly, notice that these functions are intended for working with
%   user \emph{text for typesetting}. For case changing programmatic data see
%   the \pkg{l3str} module and discussion there of \cs{str_lowercase:n},
%   \cs{str_uppercase:n} and \cs{str_foldcase:n}.
%
%   Case changing does not take place within math mode material so for example
%   \begin{verbatim}
%     \text_uppercase:n { Some~text~$y = mx + c$~with~{Braces} }
%   \end{verbatim}
%   becomes
%   \begin{verbatim}
%     SOME TEXT $y = mx + c$ WITH {BRACES}
%   \end{verbatim}
%
%   The arguments of commands listed in \cs{l_text_case_exclude_arg_tl}
%   are excluded from case changing; the latter are entirely non-textual
%   content (such as labels).
%
%   As is generally true for \pkg{expl3}, these functions are designed to
%   work with Unicode input only. As such, UTF-8 input is assumed for
%   \emph{all} engines. When used with \XeTeX{} or \LuaTeX{} a full range of
%   Unicode transformations are enabled. Specifically, the standard mappings
%   here follow those defined by the \href{http://www.unicode.org}^^A
%   {Unicode Consortium} in \texttt{UnicodeData.txt} and
%   \texttt{SpecialCasing.txt}. In the case of $8$-bit engines, mappings
%   are provided for characters which can be represented in output typeset
%   using the |T1|, |T2| and |LGR| font encodings.
%   Thus for example \texttt{ä} can be
%   case-changed using \pdfTeX{}.  For \pTeX{} only the ASCII range is
%   covered as the engine treats input outside of this range as east Asian.
%
%   Language-sensitive conversions are enabled using the \meta{language}
%   argument, and follow Unicode Consortium guidelines. Currently, the
%   languages recognised for special handling are as follows.
%   \begin{itemize}
%     \item Armenian (\texttt{hy} and \texttt{hy-x-yiwn})
%       The setting \texttt{hy} maps the codepoint U+0587, the ligature of
%       letters ech and yiwn, to the codepoints for capital ech and vew
%       when uppercasing: this follows the spelling reform which is used
%       in Armenia. The alternative \texttt{hy-x-yiwn} maps U+0587 to
%       capital ech and yiwn on uppercasing (also the output if Armenian
%       is not selected at all).
%     \item Azeri and Turkish (\texttt{az} and \texttt{tr}).
%       The case pairs I/i-dotless and I-dot/i are activated for these
%       languages. The combining dot mark is removed when lowercasing
%       I-dot and introduced when upper casing i-dotless.
%     \item German (\texttt{de-x-eszett}).
%       An alternative mapping for German in which the lowercase
%       \emph{Eszett} maps to a \emph{gro\ss{}es Eszett}. Since there is
%       a |T1| slot for the \emph{gro\ss{}es Eszett} in |T1|, this
%       tailoring \emph{is} available with \pdfTeX{} as well as in the
%       Unicode \TeX{} engines.
%     \item Greek (\texttt{el}).
%       Removes accents from Greek letters when uppercasing; titlecasing
%       leaves accents in place. (At present this is implemented only
%       for Unicode engines.) A variant \texttt{el-x-iota} is available
%       which retains the \textit{ypogegrammeni} (subscript muted iota)
%       when uppercasing: the standard version converts these to a following
%       capital iota.
%     \item Lithuanian (\texttt{lt}).
%       The lowercase letters i and j should retain a dot above when the
%       accents grave, acute or tilde are present. This is implemented for
%       lowercasing of the relevant uppercase letters both when input as
%       single Unicode codepoints and when using combining accents. The
%       combining dot is removed when uppercasing in these cases. Note that
%       \emph{only} the accents used in Lithuanian are covered: the behaviour
%       of other accents are not modified.
%     \item Dutch (\texttt{nl}).
%       Capitalisation of \texttt{ij} at the beginning of titlecased
%       input produces \texttt{IJ} rather than \texttt{Ij}. The output
%       retains two separate letters, thus this transformation \emph{is}
%       available using \pdfTeX{}.
%   \end{itemize}
%
%  For titlecasing, note that there are two functions available. The
%  function \cs{text_titlecase:n} applies (broadly) uppercasing to the first
%  letter of the input, then lowercasing to the remainder. In contrast,
%  \cs{text_titlecase_first:n} \emph{only} carries out the uppercasing operation,
%  and leaves the balance of the input unchanged. Determining whether
%  non-letter characters at the start of text should switch from upper- to
%  lowercasing is controllable. When \cs{l_text_titlecase_check_letter_bool} is
%  \texttt{true}, characters which are not letters (category code~$11$) are
%  left unchanged and \enquote{skipped}: the first \emph{letter} is uppercased.
%  (With $8$-bit engines, this is extended to active characters which form
%  part of a multi-byte letter codepoint.) When
%  \cs{l_text_titlecase_check_letter_bool} is \texttt{false}, the first
%  character is uppercased, and the rest lowercased, irrespective of the nature
%  of the character.
%
% \section{Removing formatting from text}
%
% \begin{function}[EXP, added = 2020-03-05, updated = 2020-05-14]{\text_purify:n}
%   \begin{syntax}
%     \cs{text_purify:n} \Arg{text}
%   \end{syntax}
%   Takes user input \meta{text} and expands as described for
%   \cs{text_expand:n}, then removes all functions from the resulting
%   text. Math mode material (as delimited by pairs given in
%   \cs{l_text_math_delims_tl} or as the argument to commands listed in
%   \cs{l_text_math_arg_tl}) is left contained in a pair of |$| delimiters.
%   Non-expandable functions present in the \meta{text} must either have a
%   defined equivalent (see \cs{text_declare_purify_equivalent:Nn}) or will
%   be removed from the result. Implicit tokens are converted to their
%   explicit equivalent.
% \end{function}
%
% \begin{function}[added = 2020-03-05]
%   {
%     \text_declare_purify_equivalent:Nn ,
%     \text_declare_purify_equivalent:Nx
%   }
%   \begin{syntax}
%     \cs{text_declare_purify_equivalent:Nn} \meta{cmd} \Arg{replacement}
%   \end{syntax}
%   Declares that the \meta{replacement} tokens should be used whenever the
%   \meta{cmd} (a single token) is encountered. The \meta{replacement} tokens
%   should be expandable.
% \end{function}
%
% \section{Control variables}
%
% \begin{variable}{\l_text_accents_tl}
%   Lists commands which represent accents, and which are left unchanged
%   by expansion. (Defined only for the \LaTeXe{} package.)
% \end{variable}
%
% \begin{variable}{\l_text_letterlike_tl}
%   Lists commands which represent  letters; these are left unchanged by
%   expansion. (Defined only for the \LaTeXe{} package.)
% \end{variable}
%
% \begin{variable}{\l_text_math_arg_tl}
%   Lists commands present in the \meta{text} where the argument of the
%   command should be treated as math mode material. The treatment here is
%   similar to \cs{l_text_math_delims_tl} but for a command rather than
%   paired delimiters.
% \end{variable}
%
% \begin{variable}{\l_text_math_delims_tl}
%   Lists pairs of tokens which delimit (in-line) math mode content; such
%   content \emph{may} be excluded from processing.
% \end{variable}
%
% \begin{variable}{\l_text_case_exclude_arg_tl}
%   Lists commands which are excluded from case changing. This protection
%   includes everything up to and including their first braced argument.
% \end{variable}
%
% \begin{variable}{\l_text_expand_exclude_tl}
%   Lists commands which are excluded from expansion. This protection
%   includes everything up to and including their first braced argument.
% \end{variable}
%
% \begin{variable}{\l_text_titlecase_check_letter_bool}
%   Controls how the start of titlecasing is handled: when \texttt{true}, the
%   first \emph{letter} in text is considered. The standard setting is
%   \texttt{true}.
% \end{variable}
%
% \end{documentation}
%
% \begin{implementation}
%
% \section{\pkg{l3text} implementation}
%
%    \begin{macrocode}
%<*package>
%    \end{macrocode}
%
%    \begin{macrocode}
%<@@=text>
%    \end{macrocode}
%
% \subsection{Internal auxiliaries}
%
% \begin{variable}{\s_@@_stop}
%   Internal scan marks.
%    \begin{macrocode}
\scan_new:N \s_@@_stop
%    \end{macrocode}
% \end{variable}
%
% \begin{variable}{\q_@@_nil}
%   Internal quarks.
%    \begin{macrocode}
\quark_new:N \q_@@_nil
%    \end{macrocode}
% \end{variable}
%
% \begin{macro}[pTF]{\@@_quark_if_nil:n}
%   Branching quark conditional.
%    \begin{macrocode}
\__kernel_quark_new_conditional:Nn \@@_quark_if_nil:n { TF }
%    \end{macrocode}
% \end{macro}
%
% \begin{variable}{\q_@@_recursion_tail,\q_@@_recursion_stop}
%   Internal recursion quarks.
%    \begin{macrocode}
\quark_new:N \q_@@_recursion_tail
\quark_new:N \q_@@_recursion_stop
%    \end{macrocode}
% \end{variable}
%
% \begin{macro}[EXP]{\@@_use_i_delimit_by_q_recursion_stop:nw}
%   Functions to gobble up to a quark.
%    \begin{macrocode}
\cs_new:Npn \@@_use_i_delimit_by_q_recursion_stop:nw
  #1 #2 \q_@@_recursion_stop {#1}
%    \end{macrocode}
% \end{macro}
%
% \begin{macro}[EXP]{\@@_if_recursion_tail_stop_do:Nn}
%   Functions to query recursion quarks.
%    \begin{macrocode}
\__kernel_quark_new_test:N \@@_if_recursion_tail_stop_do:Nn
%    \end{macrocode}
% \end{macro}
%
% \subsection{Utilities}
%
% \begin{macro}[EXP]
%   {
%     \@@_token_to_explicit:N      ,
%     \@@_token_to_explicit_char:N ,
%     \@@_token_to_explicit_cs:N   ,
%     \@@_token_to_explicit_cs_aux:N
%   }
% \begin{macro}[EXP]{\@@_token_to_explicit:n}
% \begin{macro}[EXP]
%   {
%     \@@_token_to_explicit_auxi:w  ,
%     \@@_token_to_explicit_auxii:w ,
%     \@@_token_to_explicit_auxiii:w
%   }
%   The idea here is to take a token and ensure that if it's an implicit
%   char, we output the explicit version. Otherwise, the token needs to be
%   unchanged. First, we have to split between control sequences and everything
%   else.
%    \begin{macrocode}
\group_begin:
  \char_set_catcode_active:n { 0 }
  \cs_new:Npn \@@_token_to_explicit:N #1
    {
      \if_catcode:w \exp_not:N #1
        \if_catcode:w \scan_stop: \exp_not:N #1
          \scan_stop:
        \else:
          \exp_not:N ^^@
        \fi:
        \exp_after:wN \@@_token_to_explicit_cs:N
      \else:
        \exp_after:wN \@@_token_to_explicit_char:N
      \fi:
      #1
    }
\group_end:
%    \end{macrocode}
%   For control sequences, we can check for macros versus other cases using
%   \cs{if_meaning:w}, then explicitly check for \tn{chardef} and
%   \tn{mathchardef}.
%    \begin{macrocode}
\cs_new:Npn \@@_token_to_explicit_cs:N #1
  {
    \exp_after:wN \if_meaning:w \exp_not:N #1 #1
      \exp_after:wN \use:nn \exp_after:wN
        \@@_token_to_explicit_cs_aux:N
    \else:
      \exp_after:wN \exp_not:n
    \fi:
      {#1}
  }
\cs_new:Npn \@@_token_to_explicit_cs_aux:N #1
  {
    \bool_lazy_or:nnTF
      { \token_if_chardef_p:N #1 }
      { \token_if_mathchardef_p:N #1 }
      {
        \char_generate:nn {#1}
          { \char_value_catcode:n {#1} }
      }
      {#1}
  }
%    \end{macrocode}
%   For character tokens, we need to filter out the implicit characters from
%   those that are explicit. That's done here, then if necessary we work out
%   the category code and generate the char. To avoid issues with alignment
%   tabs, that one is done by elimination rather than looking up the code
%   explicitly. The trick with finding the charcode is that the \TeX{}
%   messages are either \texttt{the \meta{something} character \meta{char}}
%   or \texttt{the \meta{type} \meta{char}}.
%    \begin{macrocode}
\cs_new:Npn \@@_token_to_explicit_char:N #1
  {
    \if:w
      \if_catcode:w ^ \exp_args:No \str_tail:n { \token_to_str:N #1 } ^
        \token_to_str:N #1 #1
        \else:
        AB
      \fi:
      \exp_after:wN \exp_not:n
    \else:
      \exp_after:wN \@@_token_to_explicit:n
    \fi:
      {#1}
  }
\cs_new:Npn \@@_token_to_explicit:n #1
  {
    \exp_after:wN \@@_token_to_explicit_auxi:w
      \int_value:w
        \if_catcode:w \c_group_begin_token #1 1 \else:
        \if_catcode:w \c_group_end_token #1 2 \else:
        \if_catcode:w \c_math_toggle_token #1 3 \else:
        \if_catcode:w ## #1 6 \else:
        \if_catcode:w ^ #1 7 \else:
        \if_catcode:w \c_math_subscript_token #1 8 \else:
        \if_catcode:w \c_space_token #1 10 \else:
        \if_catcode:w A #1 11 \else:
        \if_catcode:w + #1 12 \else:
        4 \fi: \fi: \fi: \fi: \fi: \fi: \fi: \fi: \fi:
    \exp_after:wN ;
    \token_to_meaning:N #1 \s_@@_stop
  }
\cs_new:Npn \@@_token_to_explicit_auxi:w #1 ; #2 \s_@@_stop
  {
    \char_generate:nn
      {
        \if_int_compare:w #1 < 9 \exp_stop_f:
          \exp_after:wN \@@_token_to_explicit_auxii:w
        \else:
          \exp_after:wN \@@_token_to_explicit_auxiii:w
        \fi:
        #2
      }
      {#1}
  }
\exp_last_unbraced:NNNNo \cs_new:Npn \@@_token_to_explicit_auxii:w
  #1 { \tl_to_str:n { character ~ } } { ` }
\cs_new:Npn \@@_token_to_explicit_auxiii:w #1 ~ #2 ~ { ` }
%    \end{macrocode}
% \end{macro}
% \end{macro}
% \end{macro}
%
% \begin{macro}[EXP]{\@@_char_catcode:N}
%   An idea from \pkg{l3char}: we need to get the category code of
%   a specific token, not the general case.
%    \begin{macrocode}
\cs_new:Npn \@@_char_catcode:N #1
  {
    \if_catcode:w \exp_not:N #1 \c_math_toggle_token
      3
    \else:
      \if_catcode:w \exp_not:N #1 \c_alignment_token
        4
      \else:
        \if_catcode:w \exp_not:N #1 \c_math_superscript_token
          7
        \else:
          \if_catcode:w \exp_not:N #1 \c_math_subscript_token
            8
          \else:
            \if_catcode:w \exp_not:N #1 \c_space_token
              10
            \else:
             \if_catcode:w \exp_not:N #1 \c_catcode_letter_token
               11
             \else:
               \if_catcode:w \exp_not:N #1 \c_catcode_other_token
                 12
               \else:
                 13
               \fi:
             \fi:
            \fi:
          \fi:
        \fi:
      \fi:
    \fi:
  }
%    \end{macrocode}
% \end{macro}
%
% \begin{macro}[EXP, TF]{\@@_if_expandable:N}
%   Test for tokens that make sense to expand here: that is more
%   restrictive than the engine view.
%    \begin{macrocode}
\prg_new_conditional:Npnn \@@_if_expandable:N #1 { T , F , TF }
  {
    \token_if_expandable:NTF #1
      {
        \bool_lazy_any:nTF
          {
            { \token_if_protected_macro_p:N      #1 }
            { \token_if_protected_long_macro_p:N #1 }
            { \token_if_eq_meaning_p:NN \q_@@_recursion_tail #1 }
          }
          { \prg_return_false: }
          { \prg_return_true: }
      }
      { \prg_return_false: }
  }
%    \end{macrocode}
% \end{macro}
%
% \subsection{Configuration variables}
%
% \begin{variable}{\l_text_accents_tl, \l_text_letterlike_tl}
%   Special cases for accents and letter-like symbols, which in some cases will
%   need to be converted further.
%    \begin{macrocode}
\tl_new:N \l_text_accents_tl
\tl_set:Nn \l_text_accents_tl
  { \` \' \^ \~ \= \u \. \" \r \H \v \d \c \k \b \t }
\tl_new:N \l_text_letterlike_tl
\tl_set:Nn \l_text_letterlike_tl
  {
    \AA \aa
    \AE \ae
    \DH \dh
    \DJ \dj
    \IJ \ij
    \L  \l
    \NG \ng
    \O  \o
    \OE \oe
    \SS \ss
    \TH \th
  }
%    \end{macrocode}
% \end{variable}
%
% \begin{variable}{\l_text_case_exclude_arg_tl}
%   Non-text arguments.
%    \begin{macrocode}
\tl_new:N \l_text_case_exclude_arg_tl
\tl_set:Nn \l_text_case_exclude_arg_tl { \begin \cite \end \label \ref }
%    \end{macrocode}
% \end{variable}
%
% \begin{variable}{\l_text_math_arg_tl}
%   Math mode as arguments.
%    \begin{macrocode}
\tl_new:N \l_text_math_arg_tl
\tl_set:Nn \l_text_math_arg_tl { \ensuremath }
%    \end{macrocode}
% \end{variable}
%
% \begin{variable}{\l_text_math_delims_tl}
%   Paired math mode delimiters.
%    \begin{macrocode}
\tl_new:N \l_text_math_delims_tl
\tl_set:Nn \l_text_math_delims_tl { $ $ \( \) }
%    \end{macrocode}
% \end{variable}
%
% \begin{variable}{\l_text_expand_exclude_tl}
%   Commands which need not to expand. We start with a somewhat historical
%   list, and tidy up if possible.
%    \begin{macrocode}
\tl_new:N \l_text_expand_exclude_tl
\tl_set:Nn \l_text_expand_exclude_tl
  { \begin \cite \end \label \ref }
\bool_lazy_and:nnT
  { \str_if_eq_p:Vn \fmtname { LaTeX2e } }
  { \tl_if_exist_p:N \@expl@finalise@setup@@@@ }
  {
    \tl_gput_right:Nn \@expl@finalise@setup@@@@
      {
        \tl_gput_right:Nn \@kernel@after@begindocument
          {
            \group_begin:
              \cs_set_protected:Npn \@@_tmp:w #1
                {
                  \tl_clear:N \l_text_expand_exclude_tl
                  \tl_map_inline:nn {#1}
                    {
                      \bool_lazy_any:nF
                        {
                          { \token_if_protected_macro_p:N ##1 }
                          { \token_if_protected_long_macro_p:N ##1 }
                          {
                            \str_if_eq_p:ee
                              { \cs_replacement_spec:N ##1 }
                              { \exp_not:n { \protect ##1 } \c_space_tl }
                          }
                        }
                        { \tl_put_right:Nn \l_text_expand_exclude_tl {##1} }
                    }
                }
              \exp_args:NV \@@_tmp:w \l_text_expand_exclude_tl
            \exp_args:NNNV \group_end:
            \tl_set:Nn \l_text_expand_exclude_tl \l_text_expand_exclude_tl
          }
      }
  }
%    \end{macrocode}
% \end{variable}
%
% \begin{macro}{\l_@@_math_mode_tl}
%   Used to control math mode output: internal as there is a dedicated
%   setter.
%    \begin{macrocode}
\tl_new:N \l_@@_math_mode_tl
%    \end{macrocode}
% \end{macro}
%
% \subsection{Expansion to formatted text}
%
% \begin{variable}{\c_@@_chardef_space_token, \c_@@_mathchardef_space_token}
% \begin{variable}
%   {\c_@@_chardef_group_begin_token, \c_@@_mathchardef_group_begin_token}
% \begin{variable}
%   {\c_@@_chardef_group_end_token, \c_@@_mathchardef_group_end_token}
%   Markers for implict char handling.
%    \begin{macrocode}
\tex_chardef:D \c_@@_chardef_space_token = `\  %
\tex_mathchardef:D \c_@@_mathchardef_space_token = `\  %
\tex_chardef:D \c_@@_chardef_group_begin_token = `\{ % `\}
\tex_mathchardef:D \c_@@_mathchardef_group_begin_token = `\{ % `\} `\{
\tex_chardef:D \c_@@_chardef_group_end_token = `\} % `\{
\tex_mathchardef:D \c_@@_mathchardef_group_end_token = `\} %
%    \end{macrocode}
% \end{variable}
% \end{variable}
% \end{variable}
%
% \begin{macro}[EXP]{\text_expand:n, \@@_expand:n}
% \begin{macro}[EXP]{\@@_expand_result:n}
% \begin{macro}[EXP]{\@@_expand_store:n, \@@_expand_store:o}
% \begin{macro}[EXP]{\@@_expand_store:nw}
% \begin{macro}[EXP]{\@@_expand_end:w}
% \begin{macro}[EXP]{\@@_expand_loop:w}
% \begin{macro}[EXP]{\@@_expand_group:n}
% \begin{macro}[EXP]{\@@_expand_space:w}
% \begin{macro}[EXP]
%   {
%    \@@_expand_N_type:N       ,
%    \@@_expand_N_type_auxi:N  ,
%    \@@_expand_N_type_auxii:N ,
%    \@@_expand_N_type_auxiii:N
%   }
% \begin{macro}[EXP]{\@@_expand_math_search:NNN}
% \begin{macro}[EXP]{\@@_expand_math_loop:Nw}
% \begin{macro}[EXP]{\@@_expand_math_N_type:NN}
% \begin{macro}[EXP]{\@@_expand_math_group:Nn}
% \begin{macro}[EXP]{\@@_expand_math_space:Nw}
% \begin{macro}[EXP]
%   {
%     \@@_expand_implicit:N ,
%     \@@_expand_explicit:N ,
%     \@@_expand_exclude:N
%   }
% \begin{macro}[EXP]{\@@_expand_exclude:nN}
% \begin{macro}[EXP]{\@@_expand_exclude:NN}
% \begin{macro}[EXP]{\@@_expand_exclude:Nw}
% \begin{macro}[EXP]{\@@_expand_exclude:Nnn}
% \begin{macro}[EXP]{\@@_expand_accent:N}
% \begin{macro}[EXP]{\@@_expand_accent:NN}
% \begin{macro}[EXP]{\@@_expand_letterlike:N}
% \begin{macro}[EXP]{\@@_expand_letterlike:NN}
% \begin{macro}[EXP]{\@@_expand_cs:N}
% \begin{macro}[EXP]{\@@_expand_protect:w}
% \begin{macro}[EXP]{\@@_expand_protect:N}
% \begin{macro}[EXP]{\@@_expand_protect:nN}
% \begin{macro}[EXP]{\@@_expand_protect:Nw}
% \begin{macro}[EXP]{\@@_expand_testopt:N}
% \begin{macro}[EXP]{\@@_expand_testopt:NNn}
% \begin{macro}[EXP]{\@@_expand_encoding:N, \@@_expand_encoding_escape:N}
% \begin{macro}[EXP]{\@@_expand_replace:N}
% \begin{macro}[EXP]{\@@_expand_replace:n}
% \begin{macro}[EXP]{\@@_expand_cs_expand:N}
% \begin{macro}[EXP]{\@@_expand_unexpanded:w}
% \begin{macro}[EXP]{\@@_expand_unexpanded_test:w}
% \begin{macro}[EXP]{\@@_expand_unexpanded:N}
% \begin{macro}[EXP]{\@@_expand_unexpanded:n}
%   After precautions against |&| tokens, start a simple loop: that of
%   course means that \enquote{text} cannot contain the two recursion
%   quarks. The loop here must be \texttt{f}-type expandable; we have
%   arbitrary user commands which might be protected \emph{and} take
%   arguments, and if the expansion code is used in a typesetting
%   context, that will otherwise explode. (The same issue applies more
%   clearly to case changing: see the example there.)
%    \begin{macrocode}
\cs_new:Npn \text_expand:n #1
  {
    \__kernel_exp_not:w \exp_after:wN
      {
        \exp:w
        \@@_expand:n {#1}
      }
  }
\cs_new:Npn \@@_expand:n #1
  {
    \group_align_safe_begin:
    \@@_expand_loop:w #1
      \q_@@_recursion_tail \q_@@_recursion_stop
    \@@_expand_result:n { }
  }
%    \end{macrocode}
%   The approach to making the code \texttt{f}-type expandable is to usee
%   a marker result token and to shuffle the collected tokens
%    \begin{macrocode}
\cs_new:Npn \@@_expand_store:n #1
  { \@@_expand_store:nw {#1} }
\cs_generate_variant:Nn \@@_expand_store:n { o }
\cs_new:Npn \@@_expand_store:nw #1#2 \@@_expand_result:n #3
  { #2 \@@_expand_result:n { #3 #1 } }
\cs_new:Npn \@@_expand_end:w #1 \@@_expand_result:n #2
  {
    \group_align_safe_end:
    \exp_end:
    #2
  }
%    \end{macrocode}
%   The main loop is a standard \enquote{tl action}; groups are handled
%   recursively, while spaces are just passed through. Thus all of the
%   action is in handling \texttt{N}-type tokens.
%    \begin{macrocode}
\cs_new:Npn \@@_expand_loop:w #1 \q_@@_recursion_stop
  {
    \tl_if_head_is_N_type:nTF {#1}
      { \@@_expand_N_type:N }
      {
        \tl_if_head_is_group:nTF {#1}
          { \@@_expand_group:n }
          { \@@_expand_space:w }
      }
    #1 \q_@@_recursion_stop
  }
\cs_new:Npn \@@_expand_group:n #1
  {
    \@@_expand_store:o
      {
        \exp_after:wN
          {
            \exp:w
            \@@_expand:n {#1}
          }
      }
    \@@_expand_loop:w
  }
\exp_last_unbraced:NNo \cs_new:Npn \@@_expand_space:w \c_space_tl
  {
    \@@_expand_store:n { ~ }
    \@@_expand_loop:w
  }
%    \end{macrocode}
%   Before we get into the real work, we have to watch out for problematic
%   implicit characters: spaces and grouping tokens. Converting these to
%   explicit characters later would lead to real issues as they are \emph{not}
%   \texttt{N}-type. A space is the easy case, so it's dealt with first:
%   just insert the explicit token and continue the loop.
%    \begin{macrocode}
\cs_new:Npx \@@_expand_N_type:N #1
  {
    \exp_not:N \@@_if_recursion_tail_stop_do:Nn #1
      { \exp_not:N \@@_expand_end:w }
    \exp_not:N \bool_lazy_any:nTF
      {
        { \exp_not:N \token_if_eq_meaning_p:NN #1 \c_space_token }
        {
          \exp_not:N \token_if_eq_meaning_p:NN #1
            \c_@@_chardef_space_token
        }
        {
          \exp_not:N \token_if_eq_meaning_p:NN #1
            \c_@@_mathchardef_space_token
        }
      }
      { \exp_not:N \@@_expand_space:w \c_space_tl }
      { \exp_not:N \@@_expand_N_type_auxi:N #1 }
  }
%    \end{macrocode}
%  Implicit |{|/|}| offer two issues. First, the token could be an implicit
%  brace character: we need to avoid turning that into a brace group, so filter
%  out the cases manually. Then we handle the case where an implicit group is
%  present. That is done in an \enquote{open-ended} way: there's the possibility
%  the closing token is hidden somewhere.
%    \begin{macrocode}
\cs_new:Npn \@@_expand_N_type_auxi:N #1
  {
    \bool_lazy_or:nnTF
      { \token_if_eq_meaning_p:NN #1 \c_@@_chardef_group_begin_token }
      { \token_if_eq_meaning_p:NN #1 \c_@@_mathchardef_group_begin_token }
      {
        \@@_expand_store:o \c_left_brace_str
        \@@_expand_loop:w
      }
      {
        \bool_lazy_or:nnTF
          { \token_if_eq_meaning_p:NN #1 \c_@@_chardef_group_end_token }
          { \token_if_eq_meaning_p:NN #1 \c_@@_mathchardef_group_end_token }
          {
            \@@_expand_store:o \c_right_brace_str
            \@@_expand_loop:w
          }
          { \@@_expand_N_type_auxii:N #1 }
      }
  }
\cs_new:Npn \@@_expand_N_type_auxii:N #1
  {
    \token_if_eq_meaning:NNTF #1 \c_group_begin_token
      {
        { \if_false: } \fi:
        \@@_expand_loop:w
      }
      {
        \token_if_eq_meaning:NNTF #1 \c_group_end_token
          {
            \if_false: { \fi: }
            \@@_expand_loop:w
          }
          { \@@_expand_N_type_auxiii:N #1 }
      }
  }
%    \end{macrocode}
%   The first step in dealing with \texttt{N}-type tokens is to look for
%   math mode material: that needs to be left alone. The starting function
%   has to be split into two as we need \cs{quark_if_recursion_tail_stop:N}
%   first before we can trigger the search. We then look for matching
%   pairs of delimiters, allowing for the case where math mode starts
%   but does not end. Within math mode, we simply pass all the tokens
%   through unchanged, just checking the \texttt{N}-type ones against the
%   end marker.
%    \begin{macrocode}
\cs_new:Npn \@@_expand_N_type_auxiii:N #1
  {
    \exp_after:wN \@@_expand_math_search:NNN
      \exp_after:wN #1 \l_text_math_delims_tl
      \q_@@_recursion_tail \q_@@_recursion_tail
      \q_@@_recursion_stop
  }
\cs_new:Npn \@@_expand_math_search:NNN #1#2#3
  {
    \@@_if_recursion_tail_stop_do:Nn #2
      { \@@_expand_explicit:N #1 }
    \token_if_eq_meaning:NNTF #1 #2
      {
        \@@_use_i_delimit_by_q_recursion_stop:nw
           {
             \@@_expand_store:n {#1}
             \@@_expand_math_loop:Nw #3
           }
      }
      { \@@_expand_math_search:NNN #1 }
  }
\cs_new:Npn \@@_expand_math_loop:Nw #1#2 \q_@@_recursion_stop
  {
    \tl_if_head_is_N_type:nTF {#2}
      { \@@_expand_math_N_type:NN }
      {
        \tl_if_head_is_group:nTF {#2}
          { \@@_expand_math_group:Nn }
          { \@@_expand_math_space:Nw }
      }
    #1#2 \q_@@_recursion_stop
  }
\cs_new:Npn \@@_expand_math_N_type:NN #1#2
  {
    \@@_if_recursion_tail_stop_do:Nn #2
      { \@@_expand_end:w }
    \@@_expand_store:n {#2}
    \token_if_eq_meaning:NNTF #2 #1
      { \@@_expand_loop:w }
      { \@@_expand_math_loop:Nw #1 }
  }
\cs_new:Npn \@@_expand_math_group:Nn #1#2
  {
    \@@_expand_store:n { {#2} }
    \@@_expand_math_loop:Nw #1
  }
\exp_after:wN \cs_new:Npn \exp_after:wN \@@_expand_math_space:Nw
  \exp_after:wN # \exp_after:wN 1 \c_space_tl
  {
    \@@_expand_store:n { ~ }
    \@@_expand_math_loop:Nw #1
  }
%    \end{macrocode}
%   At this stage, either we have a control sequence or a simple character:
%   split and handle.
%    \begin{macrocode}
\cs_new:Npn \@@_expand_explicit:N #1
  {
    \token_if_cs:NTF #1
      { \@@_expand_exclude:N #1 }
      {
        \@@_expand_store:n {#1}
        \@@_expand_loop:w
      }
  }
%    \end{macrocode}
%   Next we exclude math commands: this is mainly as there \emph{might} be an
%   \cs{ensuremath}.
%    \begin{macrocode}
\cs_new:Npn \@@_expand_exclude:N #1
  {
    \exp_args:Ne \@@_expand_exclude:nN
      {
        \exp_not:V \l_text_math_arg_tl
        \exp_not:V \l_text_expand_exclude_tl
      }
    #1
  }
\cs_new:Npn \@@_expand_exclude:nN #1#2
  {
    \@@_expand_exclude:NN #2 #1
      \q_@@_recursion_tail \q_@@_recursion_stop
  }
\cs_new:Npn \@@_expand_exclude:NN #1#2
  {
    \@@_if_recursion_tail_stop_do:Nn #2
      { \@@_expand_accent:N #1 }
    \str_if_eq:nnTF {#1} {#2}
      {
        \@@_use_i_delimit_by_q_recursion_stop:nw
          { \@@_expand_exclude:Nw #1 }
      }
      { \@@_expand_exclude:NN #1 }
  }
\cs_new:Npn \@@_expand_exclude:Nw #1#2#
  { \@@_expand_exclude:Nnn #1 {#2} }
\cs_new:Npn \@@_expand_exclude:Nnn #1#2#3
  {
    \@@_expand_store:n { #1#2 {#3} }
    \@@_expand_loop:w
  }
%    \end{macrocode}
%   Accents.
%    \begin{macrocode}
\cs_new:Npn \@@_expand_accent:N #1
  {
    \exp_after:wN \@@_expand_accent:NN \exp_after:wN
      #1 \l_text_accents_tl
      \q_@@_recursion_tail \q_@@_recursion_stop
  }
\cs_new:Npn \@@_expand_accent:NN #1#2
  {
    \@@_if_recursion_tail_stop_do:Nn #2
      { \@@_expand_letterlike:N #1 }
    \cs_if_eq:NNTF #2 #1
      {
        \@@_use_i_delimit_by_q_recursion_stop:nw
          {
            \@@_expand_store:n {#1}
            \@@_expand_loop:w
          }
      }
      { \@@_expand_accent:NN #1 }
  }
%    \end{macrocode}
%   Another list of exceptions: these ones take no arguments so are
%   easier to handle.
%    \begin{macrocode}
\cs_new:Npn \@@_expand_letterlike:N #1
  {
    \exp_after:wN \@@_expand_letterlike:NN \exp_after:wN
      #1 \l_text_letterlike_tl
      \q_@@_recursion_tail \q_@@_recursion_stop
  }
\cs_new:Npn \@@_expand_letterlike:NN #1#2
  {
    \@@_if_recursion_tail_stop_do:Nn #2
      { \@@_expand_cs:N #1 }
    \cs_if_eq:NNTF #2 #1
      {
        \@@_use_i_delimit_by_q_recursion_stop:nw
          {
            \@@_expand_store:n {#1}
            \@@_expand_loop:w
          }
      }
      { \@@_expand_letterlike:NN #1 }
  }
%    \end{macrocode}
%   \LaTeXe{}'s \cs{protect} makes life interesting. Where possible, we
%   simply remove it and replace with the \enquote{parent} command; of course,
%   the \cs{protect} might be explicit, in which case we need to leave it alone.
%   That includes the case where it's not even followed by an \texttt{N}-type
%   token. There is also the case of a straight \tn{@protected@testopt} to
%   cover.
%    \begin{macrocode}
\cs_new:Npx \@@_expand_cs:N #1
  {
    \exp_not:N \str_if_eq:nnTF {#1} { \exp_not:N \protect }
      { \exp_not:N \@@_expand_protect:w }
      {
        \bool_lazy_and:nnTF
          { \cs_if_exist_p:N \fmtname }
          { \str_if_eq_p:Vn \fmtname { LaTeX2e } }
          { \exp_not:N \@@_expand_testopt:N #1 }
          { \exp_not:N \@@_expand_replace:N #1 }
      }
  }
\cs_new:Npn \@@_expand_protect:w #1 \q_@@_recursion_stop
  {
    \tl_if_head_is_N_type:nTF {#1}
      { \@@_expand_protect:N }
      {
        \@@_expand_store:n { \protect }
        \@@_expand_loop:w
      }
        #1 \q_@@_recursion_stop
  }
\cs_new:Npn \@@_expand_protect:N #1
  {
    \@@_if_recursion_tail_stop_do:Nn #1
      {
        \@@_expand_store:n { \protect }
        \@@_expand_end:w
      }
    \exp_args:Ne \@@_expand_protect:nN
      { \cs_to_str:N #1 } #1
  }
\cs_new:Npn \@@_expand_protect:nN #1#2
  { \@@_expand_protect:Nw #2 #1 \q_@@_nil #1 ~ \q_@@_nil \q_@@_nil \s_@@_stop }
\cs_new:Npn \@@_expand_protect:Nw #1 #2 ~ \q_@@_nil #3 \q_@@_nil #4 \s_@@_stop
  {
    \@@_quark_if_nil:nTF {#4}
      {
        \cs_if_exist:cTF {#2}
          { \exp_args:Ne \@@_expand_store:n { \exp_not:c {#2} } }
          { \@@_expand_store:n { \protect #1 } }
      }
      { \@@_expand_store:n { \protect #1 } }
    \@@_expand_loop:w
  }
\cs_new:Npn \@@_expand_testopt:N #1
  {
    \token_if_eq_meaning:NNTF #1 \@protected@testopt
      { \@@_expand_testopt:NNn }
      { \@@_expand_encoding:N #1 }
  }
\cs_new:Npn \@@_expand_testopt:NNn #1#2#3
  {
    \@@_expand_store:n {#1}
    \@@_expand_loop:w
  }
%    \end{macrocode}
%   Deal with encoding-specific commands
%    \begin{macrocode}
\cs_new:Npn \@@_expand_encoding:N #1
  {
    \bool_lazy_or:nnTF
      { \cs_if_eq_p:NN  #1 \@current@cmd }
      { \cs_if_eq_p:NN  #1 \@changed@cmd }
      { \exp_after:wN \@@_expand_loop:w \@@_expand_encoding_escape:NN }
      { \@@_expand_replace:N #1 }
  }
\cs_new:Npn \@@_expand_encoding_escape:NN #1#2 { \exp_not:n {#1} }
%    \end{macrocode}
%   See if there is a dedicated replacement, and if there is, insert it.
%    \begin{macrocode}
\cs_new:Npn \@@_expand_replace:N #1
  {
    \bool_lazy_and:nnTF
      { \cs_if_exist_p:c { l_@@_expand_ \token_to_str:N #1 _tl } }
      {
        \bool_lazy_or_p:nn
          { \token_if_cs_p:N #1 }
          { \token_if_active_p:N #1 }
      }
      {
        \exp_args:Nv \@@_expand_replace:n
          { l_@@_expand_ \token_to_str:N #1 _tl }
      }
      { \@@_expand_cs_expand:N #1 }
  }
\cs_new:Npn \@@_expand_replace:n #1 { \@@_expand_loop:w #1 }
%    \end{macrocode}
%   Finally, expand any macros which can be: this then loops back around to
%   deal with what they produce. The only issue is if the token is
%   \cs{exp_not:n}, as that must apply to the following balanced text.
%    \begin{macrocode}
\cs_new:Npn \@@_expand_cs_expand:N #1
  {
    \@@_if_expandable:NTF #1
      {
        \token_if_eq_meaning:NNTF #1 \exp_not:n
          { \@@_expand_unexpanded:w }
          { \exp_after:wN \@@_expand_loop:w #1 }
      }
      {
        \@@_expand_store:n {#1}
        \@@_expand_loop:w
      }
  }
%    \end{macrocode}
%   Since \cs{exp_not:n} is actually a primitive, it allows a strange syntax
%   and it particular the primitive expands what follows and discards spaces
%   and \cs{scan_stop:} until finding a braced argument (the opening brace
%   can be implicit but we will not support this here).  Here, we repeatedly
%   |f|-expand after such an \cs{exp_not:n}, and test what follows.  If
%   it is a brace group, then we found the intended argument of
%   \cs{exp_not:n}.  If it is a space, then the next |f|-expansion will
%   eliminate it.  If it is an |N|-type token then
%   \cs{@@_expand_unexpanded:N} leaves the token to be expanded if it is
%   expandable, and otherwise removes it, assuming that it is
%   \cs{scan_stop:}.  This silently hides errors when \cs{exp_not:n} is
%   incorrectly followed by some non-expandable token other than
%   \cs{scan_stop:}, but this should be pretty rare, and there is no good
%   error recovery anyways.
%    \begin{macrocode}
\cs_new:Npn \@@_expand_unexpanded:w
  {
    \exp_after:wN \@@_expand_unexpanded_test:w
    \exp:w \exp_end_continue_f:w
  }
\cs_new:Npn \@@_expand_unexpanded_test:w #1 \q_@@_recursion_stop
  {
    \tl_if_head_is_group:nTF {#1}
      { \@@_expand_unexpanded:n }
      {
        \@@_expand_unexpanded:w
        \tl_if_head_is_N_type:nT {#1} { \@@_expand_unexpanded:N }
      }
    #1 \q_@@_recursion_stop
  }
\cs_new:Npn \@@_expand_unexpanded:N #1
  {
    \exp_after:wN \if_meaning:w \exp_not:N #1 #1
    \else:
      \exp_after:wN #1
    \fi:
  }
\cs_new:Npn \@@_expand_unexpanded:n #1
  {
    \@@_expand_store:n {#1}
    \@@_expand_loop:w
  }
%    \end{macrocode}
% \end{macro}
% \end{macro}
% \end{macro}
% \end{macro}
% \end{macro}
% \end{macro}
% \end{macro}
% \end{macro}
% \end{macro}
% \end{macro}
% \end{macro}
% \end{macro}
% \end{macro}
% \end{macro}
% \end{macro}
% \end{macro}
% \end{macro}
% \end{macro}
% \end{macro}
% \end{macro}
% \end{macro}
% \end{macro}
% \end{macro}
% \end{macro}
% \end{macro}
% \end{macro}
% \end{macro}
% \end{macro}
% \end{macro}
% \end{macro}
% \end{macro}
% \end{macro}
% \end{macro}
% \end{macro}
% \end{macro}
% \end{macro}
% \end{macro}
% \end{macro}
%
% \begin{macro}
%   {
%     \text_declare_expand_equivalent:Nn ,
%     \text_declare_expand_equivalent:cn
%   }
%  Create equivalents to allow replacement.
%    \begin{macrocode}
\cs_new_protected:Npn \text_declare_expand_equivalent:Nn #1#2
  {
    \tl_clear_new:c { l_@@_expand_ \token_to_str:N #1 _tl }
    \tl_set:cn { l_@@_expand_ \token_to_str:N #1 _tl } {#2}
  }
\cs_generate_variant:Nn \text_declare_expand_equivalent:Nn { c }
%    \end{macrocode}
% \end{macro}
%
%    \begin{macrocode}
%</package>
%    \end{macrocode}
%
% \end{implementation}
%
% \PrintIndex
