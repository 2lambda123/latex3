% \iffalse
%
%% File l3dosctrip.dtx
%
% Copyright (C) 2012,2014-2021 The LaTeX Project
%
% It may be distributed and/or modified under the conditions of the
% LaTeX Project Public License (LPPL), either version 1.3c of this
% license or (at your option) any later version.  The latest version
% of this license is in the file
%
%    https://www.latex-project.org/lppl.txt
%
% This file is part of the "l3kernel bundle" (The Work in LPPL)
% and all files in that bundle must be distributed together.
%
% -----------------------------------------------------------------------
%
% The development version of the bundle can be found at
%
%    https://github.com/latex3/latex3
%
% for those people who are interested.
%
%<*driver|program>
%</driver|program>
%<*driver>
% The same approach as used in \textsf{DocStrip}: if \cs{documentclass}
% is undefined then skip the driver, allowing the file to be used to extract
% \texttt{l3docstrip.tex} from \texttt{l3docstrip.dtx} directly. This works
% as the \cs{fi} is only seen if \LaTeX{} is not in use. The odd \cs{jobname}
% business allows the extraction to work with \LaTeX{} provided an appropriate
% \texttt{.ins} file is set up.
%<*gobble>
\ifx\jobname\relax\let\documentclass\undefined\fi
\ifx\documentclass\undefined
\else \csname fi\endcsname
%</gobble>
  \def\filename{docstrip.dtx}
  \documentclass[full,kernel]{l3doc}
  \ExplSyntaxOn
  \cs_set_eq:NN \__codedoc_replace_at_at:N \use_none:n
  \ExplSyntaxOff
  \begin{document}
    \DocInput{\jobname.dtx}
  \end{document}
%<*gobble>
\fi
%</gobble>
%</driver>
% \fi
%
% \title{^^A
%   The \pkg{l3docstrip} package\\ Code extraction and manipulation^^A
% }
%
% \author{^^A
%  The \LaTeX{} Project\thanks
%    {^^A
%      E-mail:
%        \href{mailto:latex-team@latex-project.org}
%          {latex-team@latex-project.org}^^A
%    }^^A
% }
%
% \date{Released 2021-11-12}
%
% \maketitle
%
% \begin{documentation}
%
% This is a stub file to allow extraction of \texttt{l3docstrip}: all
% functionality has been moved to the main DocStrip program.
%
% \end{documentation}
%
% \begin{implementation}
%
% \section{\pkg{l3docstrip} implementation}
%
%    \begin{macrocode}
%<*program>
%    \end{macrocode}
%
% Simply input DocStrip.
%    \begin{macrocode}
\input docstrip %
%    \end{macrocode}
%
%    \begin{macrocode}
%</program>
%    \end{macrocode}
%
% \end{implementation}
%
% \PrintIndex
