% \iffalse meta-comment
%
%% File: l3fp-assign.dtx
%
% Copyright (C) 2011-2022 The LaTeX Project
%
% It may be distributed and/or modified under the conditions of the
% LaTeX Project Public License (LPPL), either version 1.3c of this
% license or (at your option) any later version.  The latest version
% of this license is in the file
%
%    https://www.latex-project.org/lppl.txt
%
% This file is part of the "l3kernel bundle" (The Work in LPPL)
% and all files in that bundle must be distributed together.
%
% -----------------------------------------------------------------------
%
% The development version of the bundle can be found at
%
%    https://github.com/latex3/latex3
%
% for those people who are interested.
%
%<*driver>
\documentclass[full,kernel]{l3doc}
\begin{document}
  \DocInput{\jobname.dtx}
\end{document}
%</driver>
% \fi
%
% \title{The \textsf{l3fp-assign} package\\
%   Floating point expressions}
% \author{^^A
%  The \LaTeX{} Project\thanks
%    {^^A
%      E-mail:
%        \href{mailto:latex-team@latex-project.org}
%          {latex-team@latex-project.org}^^A
%    }^^A
% }
% \date{Released 2022-02-05}
% \maketitle
%
% \begin{documentation}
%
% \end{documentation}
%
% \begin{implementation}
%
% \section{\pkg{l3fp-assign} implementation}
%
%    \begin{macrocode}
%<*package>
%    \end{macrocode}
%
%    \begin{macrocode}
%<@@=fp>
%    \end{macrocode}
%
% \subsection{Assigning values}
%
% \begin{macro}{\fp_new:N}
%   Floating point variables are initialized to be $+0$.
%    \begin{macrocode}
\cs_new_protected:Npn \fp_new:N #1
  { \cs_new_eq:NN #1 \c_zero_fp }
\cs_generate_variant:Nn \fp_new:N {c}
%    \end{macrocode}
% \end{macro}
%
% \begin{macro}
%   {
%     \fp_set:Nn, \fp_set:cn,
%     \fp_gset:Nn, \fp_gset:cn,
%     \fp_const:Nn, \fp_const:cn
%   }
%   Simply use \cs{@@_parse:n} within various \texttt{f}-expanding
%   assignments.
%    \begin{macrocode}
\cs_new_protected:Npn \fp_set:Nn   #1#2
  { \__kernel_tl_set:Nx #1 { \exp_not:f { \@@_parse:n {#2} } } }
\cs_new_protected:Npn \fp_gset:Nn  #1#2
  { \__kernel_tl_gset:Nx #1 { \exp_not:f { \@@_parse:n {#2} } } }
\cs_new_protected:Npn \fp_const:Nn #1#2
  { \tl_const:Nx #1 { \exp_not:f { \@@_parse:n {#2} } } }
\cs_generate_variant:Nn \fp_set:Nn {c}
\cs_generate_variant:Nn \fp_gset:Nn {c}
\cs_generate_variant:Nn \fp_const:Nn {c}
%    \end{macrocode}
% \end{macro}
%
% \begin{macro}
%   {
%     \fp_set_eq:NN , \fp_set_eq:cN , \fp_set_eq:Nc , \fp_set_eq:cc,
%     \fp_gset_eq:NN, \fp_gset_eq:cN, \fp_gset_eq:Nc, \fp_gset_eq:cc
%   }
%   Copying a floating point is the same as copying the underlying token
%   list.
%    \begin{macrocode}
\cs_new_eq:NN \fp_set_eq:NN  \tl_set_eq:NN
\cs_new_eq:NN \fp_gset_eq:NN \tl_gset_eq:NN
\cs_generate_variant:Nn \fp_set_eq:NN  { c , Nc , cc }
\cs_generate_variant:Nn \fp_gset_eq:NN { c , Nc , cc }
%    \end{macrocode}
% \end{macro}
%
% \begin{macro}{\fp_zero:N, \fp_zero:c, \fp_gzero:N, \fp_gzero:c}
%   Setting a floating point to zero: copy \cs{c_zero_fp}.
%    \begin{macrocode}
\cs_new_protected:Npn \fp_zero:N #1 { \fp_set_eq:NN #1 \c_zero_fp }
\cs_new_protected:Npn \fp_gzero:N #1 { \fp_gset_eq:NN #1 \c_zero_fp }
\cs_generate_variant:Nn \fp_zero:N  { c }
\cs_generate_variant:Nn \fp_gzero:N { c }
%    \end{macrocode}
% \end{macro}
%
% \begin{macro}
%   {\fp_zero_new:N, \fp_zero_new:c, \fp_gzero_new:N, \fp_gzero_new:c}
%   Set the floating point to zero, or define it if needed.
%    \begin{macrocode}
\cs_new_protected:Npn \fp_zero_new:N #1
  { \fp_if_exist:NTF #1 { \fp_zero:N #1 } { \fp_new:N #1 } }
\cs_new_protected:Npn \fp_gzero_new:N #1
  { \fp_if_exist:NTF #1 { \fp_gzero:N #1 } { \fp_new:N #1 } }
\cs_generate_variant:Nn \fp_zero_new:N  { c }
\cs_generate_variant:Nn \fp_gzero_new:N { c }
%    \end{macrocode}
% \end{macro}
%
% \subsection{Updating values}
%
% These match the equivalent functions in \pkg{l3int} and \pkg{l3skip}.
%
% \begin{macro}
%   {
%     \fp_add:Nn, \fp_add:cn, \fp_gadd:Nn, \fp_gadd:cn,
%     \fp_sub:Nn, \fp_sub:cn, \fp_gsub:Nn, \fp_gsub:cn,
%   }
% \begin{macro}{\@@_add:NNNn}
%   For the sake of error recovery we should not simply set |#1| to
%   $|#1| \pm (|#2|)$: for instance, if |#2| is % ^^A(
%   |0)+2|, the parsing error would be raised at the last closing
%   parenthesis rather than at the closing parenthesis in the user
%   argument.  Thus we evaluate |#2| instead of just putting
%   parentheses.  As an optimization we use \cs{@@_parse:n} rather than
%   \cs{fp_eval:n}, which would convert the result away from the
%   internal representation and back.
%    \begin{macrocode}
\cs_new_protected:Npn \fp_add:Nn  { \@@_add:NNNn \fp_set:Nn  + }
\cs_new_protected:Npn \fp_gadd:Nn { \@@_add:NNNn \fp_gset:Nn + }
\cs_new_protected:Npn \fp_sub:Nn  { \@@_add:NNNn \fp_set:Nn  - }
\cs_new_protected:Npn \fp_gsub:Nn { \@@_add:NNNn \fp_gset:Nn - }
\cs_new_protected:Npn \@@_add:NNNn #1#2#3#4
  { #1 #3 { #3 #2 \@@_parse:n {#4} } }
\cs_generate_variant:Nn \fp_add:Nn  { c }
\cs_generate_variant:Nn \fp_gadd:Nn { c }
\cs_generate_variant:Nn \fp_sub:Nn  { c }
\cs_generate_variant:Nn \fp_gsub:Nn { c }
%    \end{macrocode}
% \end{macro}
% \end{macro}
%
% \subsection{Showing values}
%
% \begin{macro}{\fp_show:N, \fp_show:c, \fp_log:N, \fp_log:c, \@@_show:NN}
% \begin{macro}[EXP]{\@@_show_validate:w}
%   This shows the result of computing its argument by
%   passing the right data to \cs{tl_show:n} or \cs{tl_log:n}.
%    \begin{macrocode}
\cs_new_protected:Npn \fp_show:N { \@@_show:NN \tl_show:n }
\cs_generate_variant:Nn \fp_show:N { c }
\cs_new_protected:Npn \fp_log:N { \@@_show:NN \tl_log:n }
\cs_generate_variant:Nn \fp_log:N { c }
\cs_new_protected:Npn \@@_show:NN #1#2
  {
    \__kernel_chk_tl_type:NnnT #2 { fp }
      {
        \str_if_eq:eeTF { \tl_head:N #2 } { \s_@@_tuple } { \exp_not:o #2 }
          {
            \exp_after:wN \@@_show_validate:w #2
            \s_@@ \@@_chk:w ??? ; \s_@@_stop
          }
      }
      { \exp_args:Nx #1 { \token_to_str:N #2 = \fp_to_tl:N #2 } }
  }
\cs_new:Npn \@@_show_validate:w
    #1 \s_@@ \@@_chk:w #2#3#4#5 ; #6 \s_@@_stop
  {
    \token_if_eq_meaning:NNTF #2 1
      { \s_@@ \@@_chk:w #2 #3 {#4} #5 ; }
      { \s_@@ \@@_chk:w #2 #3 #4 #5 ; }
  }
%    \end{macrocode}
% \end{macro}
% \end{macro}
%
% \begin{macro}{\fp_show:n, \fp_log:n}
%   Use general tools.
%    \begin{macrocode}
\cs_new_protected:Npn \fp_show:n
  { \msg_show_eval:Nn \fp_to_tl:n }
\cs_new_protected:Npn \fp_log:n
  { \msg_log_eval:Nn \fp_to_tl:n }
%    \end{macrocode}
% \end{macro}
%
% \subsection{Some useful constants and scratch variables}
%
% \begin{variable}{\c_one_fp, \c_e_fp}
%   Some constants.
%    \begin{macrocode}
\fp_const:Nn \c_e_fp          { 2.718 2818 2845 9045 }
\fp_const:Nn \c_one_fp        { 1 }
%    \end{macrocode}
% \end{variable}
%
% \begin{variable}{\c_pi_fp, \c_one_degree_fp}
%   We simply round $\pi$ to and $\pi/180$ to $16$ significant digits.
%    \begin{macrocode}
\fp_const:Nn \c_pi_fp         { 3.141 5926 5358 9793 }
\fp_const:Nn \c_one_degree_fp { 0.0 1745 3292 5199 4330 }
%    \end{macrocode}
% \end{variable}
%
% \begin{variable}{\l_tmpa_fp, \l_tmpb_fp, \g_tmpa_fp, \g_tmpb_fp}
%   Scratch variables are simply initialized there.
%    \begin{macrocode}
\fp_new:N \l_tmpa_fp
\fp_new:N \l_tmpb_fp
\fp_new:N \g_tmpa_fp
\fp_new:N \g_tmpb_fp
%    \end{macrocode}
% \end{variable}
%
%    \begin{macrocode}
%</package>
%    \end{macrocode}
%
% \end{implementation}
%
% \PrintChanges
%
% \PrintIndex
