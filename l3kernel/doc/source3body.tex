\iffalse meta-comment

File: source3body.tex

Copyright (C) 1990-2012,2014-2021 The LaTeX Project

It may be distributed and/or modified under the conditions of the
LaTeX Project Public License (LPPL), either version 1.3c of this
license or (at your option) any later version.  The latest version
of this license is in the file

   https://www.latex-project.org/lppl.txt

This file is part of the "l3kernel bundle" (The Work in LPPL)
and all files in that bundle must be distributed together.

The released version of this bundle is available from CTAN.

-----------------------------------------------------------------------

The development version of the bundle can be found at

   https://github.com/latex3/latex3

for those people who are interested.

\fi

%%%%%%%%%%%%%%%%%%%%%%%%%%%%%%%%%%%%%%%%%%%%%%%%%%%%%%%%%%%%%%%%%%%%%%%%%%%
%
% This file is used by
%
%  source3.tex            % documentation including implementation
%
%  interface3.tex         % only interface documentation
%
%
%%%%%%%%%%%%%%%%%%%%%%%%%%%%%%%%%%%%%%%%%%%%%%%%%%%%%%%%%%%%%%%%%%%%%%%%%%%


\begin{abstract}

\setlength\parindent{0pt}
\setlength\parskip{\baselineskip}

\noindent
This is the reference documentation for the \pkg{expl3}
programming environment. The \pkg{expl3} modules set up an experimental
naming scheme for \LaTeX{} commands, which allow the \LaTeX{} programmer
to systematically name functions and variables, and specify the argument
types of functions.

The \TeX{} and \eTeX{} primitives are all given a new name according to
these conventions. However, in the main direct use of the primitives is
not required or encouraged: the \pkg{expl3} modules define an
independent low-level \LaTeX3 programming language.

he \pkg{expl3} modules are designed to be loaded on top of
\LaTeXe{}. With an up-to-date \LaTeXe{} kernel, this material is loaded
as part of the format. The fundamental programming code can also be loaded
with other \TeX{} formats, subject ot restrictions on the full range of
functionality.

\end{abstract}

%%%%%%%%%%%%%%%%%%%%%%%%%%%%%%%%%%%%%%%%%%%%%%%%%%%%%%%%%%%%
%
% Each of the following \DocInput lines includes a file with extension
% .dtx. Each of these files may be typeset separately. For instance
%   pdflatex l3box.dtx
% will typeset the source of the LaTeX3 box commands. If you use the
% Makefile, the index will be generated automatically; e.g.,
%   make doc F=l3box
%
% If this file is processed, each of these separate dtx files will be
% contained as a part of a single document.
%
%%%%%%%%%%%%%%%%%%%%%%%%%%%%%%%%%%%%%%%%%%%%%%%%%%%%%%%%%%%%

\makeatletter
% l3doc is based on article, but for these very large documents we need
% chapters; the code is based on the standard classes but somewhat simplified
\renewcommand\part{%
    \clearpage
  \thispagestyle{plain}%
  \@tempswafalse
  \null\vfil
  \secdef\@part\@spart}
\newcounter {chapter}
\numberwithin{section}{chapter}
\renewcommand \thechapter {\@arabic\c@chapter}
\renewcommand \thesection {\thechapter.\@arabic\c@section}
\newcommand*\chaptermark[1]{}
\setcounter{secnumdepth}{2}
\newcommand\@chapapp{\chaptername}
\newcommand\chaptername{Chapter}
  \def\ps@headings{%
    \let\@oddfoot\@empty
    \def\@oddhead{{\slshape\rightmark}\hfil\thepage}%
    \let\@mkboth\markboth
    \def\chaptermark##1{%
      \markright {\MakeUppercase{%
        \ifnum \c@secnumdepth >\m@ne
            \@chapapp\ \thechapter. \ %
        \fi
        ##1}}}}
\newcommand\chapter{\clearpage
                    \thispagestyle{plain}%
                    \global\@topnum\z@
                    \@afterindentfalse
                    \secdef\@chapter\@schapter}
\def\@chapter[#1]#2{\refstepcounter{chapter}%
                    \typeout{\@chapapp\space\thechapter.}%
                    \addcontentsline{toc}{chapter}%
                      {\protect\numberline{\thechapter}#1}%
                    \chaptermark{#1}%
                    \addtocontents{lof}{\protect\addvspace{10\p@}}%
                    \addtocontents{lot}{\protect\addvspace{10\p@}}%
                    \@makechapterhead{#2}%
                    \@afterheading}
\def\@makechapterhead#1{%
  \vspace*{50\p@}%
  {\parindent \z@ \raggedright \normalfont
   \huge\bfseries \@chapapp\space \thechapter
   \par\nobreak
   \vskip 20\p@
    \interlinepenalty\@M
    \Huge \bfseries #1\par\nobreak
    \vskip 40\p@
  }}
\newcommand*\l@chapter[2]{%
  \ifnum \c@tocdepth >\m@ne
    \addpenalty{-\@highpenalty}%
    \vskip 1.0em \@plus\p@
    \setlength\@tempdima{1.5em}%
    \begingroup
      \parindent \z@ \rightskip \@pnumwidth
      \parfillskip -\@pnumwidth
      \leavevmode \bfseries
      \advance\leftskip\@tempdima
      \hskip -\leftskip
      #1\nobreak\hfil
      \nobreak\hb@xt@\@pnumwidth{\hss #2%
                                 \kern-\p@\kern\p@}\par
      \penalty\@highpenalty
    \endgroup
  \fi}
\renewcommand*\l@section{\@dottedtocline{1}{1.5em}{2.8em}}
\renewcommand*\l@subsection{\@dottedtocline{2}{3.8em}{3.2em}}
\renewcommand*\l@subsubsection{\@dottedtocline{3}{7.0em}{4.1em}}
\def\partname{Part}
\def\toclevel@part{-1}
\def\maketitle{\chapter{\@title}}
\let\thanks\@gobble
\let\DelayPrintIndex\PrintIndex
\let\PrintIndex\@empty
\providecommand*{\hexnum}[1]{\text{\texttt{\char`\"}#1}}
\makeatother

\clearpage

{%
  \def\\{:}% fix "newlines" in the ToC
  \tableofcontents
}

\clearpage
\pagenumbering{arabic}

\part{Introduction}

\chapter{Introduction to \pkg{expl3} and this document}

This document is intended to act as a comprehensive reference manual
for the \pkg{expl3} language. A general guide to the \LaTeX3
programming language is found in \href{expl3.pdf}{expl3.pdf}.

\section{Naming functions and variables}

\LaTeX3 does not use \texttt{@} as a \enquote{letter} for defining
internal macros.  Instead, the symbols |_| and \texttt{:}
are used in internal macro names to provide structure. The name of
each \emph{function} is divided into logical units using \texttt{_},
while \texttt{:} separates the \emph{name} of the function from the
\emph{argument specifier} (\enquote{arg-spec}). This describes the arguments
expected by the function. In most cases, each argument is represented
by a single letter. The complete list of arg-spec letters for a function
is referred to as the \emph{signature} of the function.

Each function name starts with the \emph{module} to which it belongs.
Thus apart from a small number of very basic functions, all \pkg{expl3}
function names contain at least one underscore to divide the module
name from the descriptive name of the function. For example, all
functions concerned with comma lists are in module \texttt{clist} and
begin |\clist_|.

Every function must include an argument specifier. For functions which
take no arguments, this will be blank and the function name will end
\texttt{:}. Most functions take one or more arguments, and use the
following argument specifiers:
\begin{description}
  \item[\texttt{N} and \texttt{n}] These mean \emph{no manipulation},
    of a single token for \texttt{N} and of a set of tokens given in
    braces for \texttt{n}. Both pass the argument through exactly as
    given. Usually, if you use a single token for an \texttt{n} argument,
    all will be well.
  \item[\texttt{c}] This means \emph{csname}, and indicates that the
    argument will be turned into a csname before being used. So
    |\foo:c| |{ArgumentOne}| will act in the same way as |\foo:N|
    |\ArgumentOne|.
  \item[\texttt{V} and \texttt{v}] These mean \emph{value
    of variable}. The \texttt{V} and \texttt{v} specifiers are used to
    get the content of a variable without needing to worry about the
    underlying \TeX{} structure containing the data. A \texttt{V}
    argument will be a single token (similar to \texttt{N}), for example
    |\foo:V| |\MyVariable|; on the other hand, using \texttt{v} a
    csname is constructed first, and then the value is recovered, for
    example |\foo:v| |{MyVariable}|.
  \item[\texttt{o}] This means \emph{expansion once}. In general, the
    \texttt{V} and \texttt{v} specifiers are favoured over \texttt{o}
    for recovering stored information. However, \texttt{o} is useful
    for correctly processing information with delimited arguments.
  \item[\texttt{x}] The \texttt{x} specifier stands for \emph{exhaustive
    expansion}: every token in the argument is fully expanded until only
    unexpandable ones remain. The \TeX{} \tn{edef} primitive carries out
    this type of expansion. Functions which feature an \texttt{x}-type
    argument are \emph{not} expandable.
  \item[\texttt{e}] The \texttt{e} specifier is in many respects
    identical to \texttt{x}, but with a very different implementation.
    Functions which feature an \texttt{e}-type argument may be
    expandable.  The drawback is that \texttt{e} is extremely slow
    (often more than $200$ times slower) in older engines, more
    precisely in non-\LuaTeX{} engines older than 2019.
  \item[\texttt{f}] The \texttt{f} specifier stands for \emph{full
    expansion}, and in contrast to \texttt{x} stops at the first
    non-expandable token (reading the argument from left to right) without
    trying to expand it. If this token is a \meta{space token}, it is gobbled,
    and thus won't be part of the resulting argument. For example, when
    setting a token list variable (a macro used for storage), the sequence
    \begin{verbatim}
      \tl_set:Nn \l_mya_tl { A }
      \tl_set:Nn \l_myb_tl { B }
      \tl_set:Nf \l_mya_tl { \l_mya_tl \l_myb_tl }
    \end{verbatim}
    will leave |\l_mya_tl| with the content |A\l_myb_tl|, as |A| cannot
    be expanded and so terminates expansion before |\l_myb_tl| is considered.
  \item[\texttt{T} and \texttt{F}] For logic tests, there are the branch
    specifiers \texttt{T} (\emph{true}) and \texttt{F} (\emph{false}).
    Both specifiers treat the input in the same way as \texttt{n} (no
    change), but make the logic much easier to see.
  \item[\texttt{p}] The letter \texttt{p} indicates \TeX{}
    \emph{parameters}. Normally this will be used for delimited
    functions as \pkg{expl3} provides better methods for creating simple
    sequential arguments.
  \item[\texttt{w}] Finally, there is the \texttt{w} specifier for
    \emph{weird} arguments. This covers everything else, but mainly
    applies to delimited values (where the argument must be terminated
    by some specified string).
  \item[\texttt{D}] The \texttt{D} stands for \textbf{Do not use}.
    All of the \TeX{} primitives are initially \cs{let} to a \texttt{D}
    name, and some are then given a second name.
    These functions have no standardized syntax, they are engine
    dependent and their name can change without warning, thus their
    use is \emph{strongly discouraged} in package code: programmers
    should instead use the interfaces documented in
    \href{interface3.pdf}{interface3.pdf}%^^A
    \footnote{If a primitive offers a functionality not yet in the
      kernel, programmers and users are encouraged to write to the
      \texttt{LaTeX-L} mailing list
      (\url{mailto:LATEX-L@listserv.uni-heidelberg.de}) describing
      their use-case and intended behaviour, so that a possible
      interface can be discussed.  Temporarily, while an interface is
      not provided, programmers may use the procedure described in the
      \href{l3styleguide.pdf}{l3styleguide.pdf}.}.
\end{description}
Notice that the argument specifier describes how the argument is
processed prior to being passed to the underlying function. For example,
|\foo:c| will take its argument, convert it to a control sequence and
pass it to |\foo:N|.

Variables are named in a similar manner to functions, but begin with
a single letter to define the type of variable:
\begin{description}
  \item[\texttt{c}] Constant: global parameters whose value should not
    be changed.
  \item[\texttt{g}] Parameters whose value should only be set globally.
  \item[\texttt{l}] Parameters whose value should only be set locally.
\end{description}
Each variable name is then build up in a similar way to that of a
function, typically starting with the module\footnote{The module names are
  not used in case of generic scratch registers defined in the data
  type modules, e.g., the
  \texttt{int} module contains some scratch variables called \cs{l_tmpa_int},
  \cs{l_tmpb_int}, and so on. In such a case adding the module name up front
  to denote the module
  and in the back to indicate the type, as in
  \texttt{\string\l_int_tmpa_int} would be very unreadable.}  name
and then a descriptive part.
Variables end with a short identifier to show the variable type:
\begin{description}
  \item[\texttt{clist}] Comma separated list.
  \item[\texttt{dim}] \enquote{Rigid} lengths.
  \item[\texttt{fp}] Floating-point values;
  \item[\texttt{int}] Integer-valued count register.
  \item[\texttt{muskip}] \enquote{Rubber} lengths for use in
    mathematics.
  \item[\texttt{seq}] \enquote{Sequence}: a data-type used to implement
    lists (with access at both ends) and stacks.
  \item[\texttt{skip}] \enquote{Rubber} lengths.
  \item[\texttt{str}] String variables: contain character data.
  \item[\texttt{tl}] Token list variables: placeholder for a token list.
\end{description}
Applying \texttt{V}-type or \texttt{v}-type expansion to variables of
one of the above types is supported, while it is not supported for the
following variable types:
\begin{description}
  \item[\texttt{bool}] Either true or false.
  \item[\texttt{box}] Box register.
  \item[\texttt{coffin}] A \enquote{box with handles} --- a higher-level
    data type for carrying out \texttt{box} alignment operations.
  \item[\texttt{flag}] Integer that can be incremented expandably.
  \item[\texttt{fparray}] Fixed-size array of floating point values.
  \item[\texttt{intarray}] Fixed-size array of integers.
  \item[\texttt{ior}/\texttt{iow}] An input or output stream, for
    reading from or writing to, respectively.
  \item[\texttt{prop}] Property list: analogue of dictionary or
    associative arrays in other languages.
  \item[\texttt{regex}] Regular expression.
\end{description}

\subsection{Terminological inexactitude}

A word of warning. In this document, and others referring to the \pkg{expl3}
programming modules, we often refer to \enquote{variables} and
\enquote{functions} as if they
were actual constructs from a real programming language.  In truth, \TeX{}
is a macro processor, and functions are simply macros that may or may not take
arguments and expand to their replacement text.  Many of the common variables
are \emph{also} macros, and if placed into the input stream will simply expand
to their definition as well~--- a \enquote{function} with no arguments and a
\enquote{token list variable} are almost the same.\footnote{\TeX{}nically,
functions with no arguments are \tn{long} while token list variables are not.}
On the other
hand, some \enquote{variables} are actually registers that must be
initialised and their values set and retrieved with specific functions.

The conventions of the \pkg{expl3} code are designed to clearly separate the
ideas of \enquote{macros that contain data} and
\enquote{macros that contain code}, and a
consistent wrapper is applied to all forms of \enquote{data} whether they be
macros or
actually registers.  This means that sometimes we will use phrases like
\enquote{the function returns a value}, when actually we just mean
\enquote{the macro expands to something}. Similarly, the term
\enquote{execute} might be used in place of \enquote{expand}
or it might refer to the more specific case of
\enquote{processing in \TeX's stomach}
(if you are familiar with the \TeX{}book parlance).

If in doubt, please ask; chances are we've been hasty in writing certain
definitions and need to be told to tighten up our terminology.

\section{Documentation conventions}

This document is typeset with the experimental \pkg{l3doc} class;
several conventions are used to help describe the features of the code.
A number of conventions are used here to make the documentation clearer.

Each group of related functions is given in a box. For a function with
a \enquote{user} name, this might read:
\begin{function}[label = ]{\ExplSyntaxOn, \ExplSyntaxOff}
  \begin{syntax}
    |\ExplSyntaxOn| \dots{} |\ExplSyntaxOff|
  \end{syntax}
  The textual description of how the function works would appear here. The
  syntax of the function is shown in mono-spaced text to the right of
  the box. In this example, the function takes no arguments and so the
  name of the function is simply reprinted.
\end{function}

For programming functions, which use \texttt{_} and \texttt{:} in their name
there are a few additional conventions: If two related functions are given
with identical names but different argument specifiers, these are termed
\emph{variants} of each other, and the latter functions are printed in grey to
show this more clearly. They will carry out the same function but will take
different types of argument:
\begin{function}[label = ]{\seq_new:N, \seq_new:c}
  \begin{syntax}
    |\seq_new:N| \meta{sequence}
  \end{syntax}
  When a number of variants are described, the arguments are usually
  illustrated only for the base function. Here, \meta{sequence} indicates
  that |\seq_new:N| expects the name of a sequence. From the argument
  specifier, |\seq_new:c| also expects a sequence name, but as a
  name rather than as a control sequence. Each argument given in the
  illustration should be described in the following text.
\end{function}

\paragraph{Fully expandable functions}
\hypertarget{expstar}{Some functions are fully expandable},
which allows them to be used within
an \texttt{x}-type or \texttt{e}-type argument (in plain \TeX{} terms, inside an \tn{edef} or \tn{expanded}),
as well as within an \texttt{f}-type argument.
These fully expandable functions are indicated in the documentation by
a star:
\begin{function}[EXP, label = ]{\cs_to_str:N}
  \begin{syntax}
    |\cs_to_str:N| \meta{cs}
  \end{syntax}
  As with other functions, some text should follow which explains how
  the function works. Usually, only the star will indicate that the
  function is expandable. In this case, the function expects a \meta{cs},
  shorthand for a \meta{control sequence}.
\end{function}

\paragraph{Restricted expandable functions}
\hypertarget{rexpstar}{A few functions are fully expandable} but cannot be fully expanded within
an \texttt{f}-type argument. In this case a hollow star is used to indicate
this:
\begin{function}[rEXP, label = ]{\seq_map_function:NN}
  \begin{syntax}
    |\seq_map_function:NN| \meta{seq} \meta{function}
  \end{syntax}
\end{function}

\paragraph{Conditional functions}
\hypertarget{explTF}{Conditional (\texttt{if}) functions}
are normally defined in three variants, with
\texttt{T}, \texttt{F} and \texttt{TF} argument specifiers. This allows
them to be used for different \enquote{true}/\enquote{false} branches,
depending on
which outcome the conditional is being used to test. To indicate this
without repetition, this information is given in a shortened form:
\begin{function}[EXP,TF, label = ]{\sys_if_engine_xetex:}
  \begin{syntax}
    |\sys_if_engine_xetex:TF| \Arg{true code} \Arg{false code}
  \end{syntax}
  The underlining and italic of \texttt{TF} indicates that three functions
  are available:
  \begin{itemize}
    \item |\sys_if_engine_xetex:T|
    \item |\sys_if_engine_xetex:F|
    \item |\sys_if_engine_xetex:TF|
  \end{itemize}
  Usually, the illustration
  will use the \texttt{TF} variant, and so both \meta{true code}
  and \meta{false code} will be shown. The two variant forms \texttt{T} and
  \texttt{F} take only \meta{true code} and \meta{false code}, respectively.
  Here, the star also shows that this function is expandable.
  With some minor exceptions, \emph{all} conditional functions in the
  \pkg{expl3} modules should be defined in this way.
\end{function}

Variables, constants and so on are described in a similar manner:
\begin{variable}[label = ]{\l_tmpa_tl}
  A short piece of text will describe the variable: there is no
  syntax illustration in this case.
\end{variable}

In some cases, the function is similar to one in \LaTeXe{} or plain \TeX{}.
In these cases, the text will include an extra \enquote{\textbf{\TeX{}hackers
note}} section:
\begin{function}[EXP, label = ]{\token_to_str:N}
  \begin{syntax}
    |\token_to_str:N| \meta{token}
  \end{syntax}
  The normal description text.
  \begin{texnote}
    Detail for the experienced \TeX{} or \LaTeXe\ programmer. In this
    case, it would point out that this function is the \TeX{} primitive
    |\string|.
  \end{texnote}
\end{function}

\paragraph{Changes to behaviour}
When new functions are added to \pkg{expl3}, the date of first inclusion is
given in the documentation. Where the documented behaviour of a function
changes after it is first introduced, the date of the update will also be
given. This means that the programmer can be sure that any release of
\pkg{expl3} after the date given will contain the function of interest with
expected behaviour as described. Note that changes to code internals, including
bug fixes, are not recorded in this way \emph{unless} they impact on the
expected behaviour.

\section{Formal language conventions which apply generally}

As this is a formal reference guide for \LaTeX3 programming, the descriptions
of functions are intended to be reasonably \enquote{complete}. However, there
is also a need to avoid repetition. Formal ideas which apply to general
classes of function are therefore summarised here.

For tests which have a \texttt{TF} argument specification, the test if
evaluated to give a logically \texttt{TRUE} or \texttt{FALSE} result.
Depending on this result, either the \meta{true code} or the \meta{false code}
will be left in the input stream. In the case where the test is expandable,
and a predicate (|_p|) variant is available, the logical value determined by
the test is left in the input stream: this will typically be part of a larger
logical construct.

\section{\TeX{} concepts not supported by \LaTeX3{}}

The \TeX{} concept of an \enquote{\cs{outer}} macro is \emph{not supported}
at all by \LaTeX3{}. As such, the functions provided here may break when
used on top of \LaTeXe{} if \cs{outer} tokens are used in the arguments.

\DisableImplementation

\part{Bootstrapping}

\DocInput{l3bootstrap.dtx}
\DocInput{l3names.dtx}
\ExplSyntaxOn
\clist_gput_right:Nn \g_docinput_clist { l3kernel-functions.dtx }
\ExplSyntaxOff

\part{Programming Flow}

\DocInput{l3basics.dtx}
\DocInput{l3expan.dtx}
\DocInput{l3sort.dtx}
\DocInput{l3tl-analysis.dtx}
\DocInput{l3regex.dtx}
\DocInput{l3prg.dtx}
\DocInput{l3sys.dtx}
\DocInput{l3msg.dtx}
\DocInput{l3file.dtx}

\DocInput{l3luatex.dtx}

\DocInput{l3legacy.dtx}


\part{Data types}

\DocInput{l3tl.dtx}
\DocInput{l3str.dtx}
\DocInput{l3str-convert.dtx}
\DocInput{l3quark.dtx}
\DocInput{l3seq.dtx}
\DocInput{l3int.dtx}
\DocInput{l3flag.dtx}
\DocInput{l3clist.dtx}
\DocInput{l3token.dtx}
\DocInput{l3prop.dtx}
\DocInput{l3skip.dtx}
\DocInput{l3keys.dtx}
\DocInput{l3intarray.dtx}
\DocInput{l3fp.dtx}
% To get the various submodules of l3fp to appear in the implementation
% part only, they have to be added to the documentation list after typesetting
% the 'user' part just for the main module.
\ExplSyntaxOn
\clist_gput_right:Nn \g_docinput_clist
  {
    l3fp-aux.dtx ,
    l3fp-traps.dtx ,
    l3fp-round.dtx ,
    l3fp-parse.dtx ,
    l3fp-assign.dtx ,
    l3fp-logic.dtx ,
    l3fp-basics.dtx ,
    l3fp-extended.dtx ,
    l3fp-expo.dtx ,
    l3fp-trig.dtx ,
    l3fp-convert.dtx ,
    l3fp-random.dtx ,
  }
\ExplSyntaxOff

\DocInput{l3fparray.dtx}
\DocInput{l3cctab.dtx}

\part{Text manipulation}

\DocInput{l3unicode.dtx}
\DocInput{l3text.dtx}
\ExplSyntaxOn
\clist_gput_right:Nn \g_docinput_clist
  {
    l3text-case.dtx ,
    l3text-purify.dtx
  }
\ExplSyntaxOff

\part{Typesetting}

\DocInput{l3box.dtx}
\DocInput{l3coffins.dtx}
\DocInput{l3color-base.dtx}

\part{Additions and removals}

\DocInput{l3candidates.dtx}

\ExplSyntaxOn
\clist_gput_right:Nn \g_docinput_clist { l3deprecation.dtx }
\ExplSyntaxOff

\endinput
