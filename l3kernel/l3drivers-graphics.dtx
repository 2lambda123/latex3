% \iffalse meta-comment
%
%% File: l3drivers-graphics.dtx
%
% Copyright (C) 2019 The LaTeX3 Project
%
% It may be distributed and/or modified under the conditions of the
% LaTeX Project Public License (LPPL), either version 1.3c of this
% license or (at your option) any later version.  The latest version
% of this license is in the file
%
%    https://www.latex-project.org/lppl.txt
%
% This file is part of the "l3kernel bundle" (The Work in LPPL)
% and all files in that bundle must be distributed together.
%
% -----------------------------------------------------------------------
%
% The development version of the bundle can be found at
%
%    https://github.com/latex3/latex3
%
% for those people who are interested.
%
%<*driver>
\documentclass[full,kernel]{l3doc}
\begin{document}
  \DocInput{\jobname.dtx}
\end{document}
%</driver>
% \fi
%
% \title{^^A
%   The \textsf{l3drivers-graphics} package\\ Driver graphics support^^A
% }
%
% \author{^^A
%  The \LaTeX3 Project\thanks
%    {^^A
%      E-mail:
%        \href{mailto:latex-team@latex-project.org}
%          {latex-team@latex-project.org}^^A
%    }^^A
% }
%
% \date{Released 2019-05-09}
%
% \maketitle
%
% \begin{documentation}
%
% \end{documentation}
%
% \begin{implementation}
%
% \section{\pkg{l3drivers-graphics} Implementation}
%
%    \begin{macrocode}
%<*initex|package>
%<@@=driver>
%    \end{macrocode}
%
% \subsection{\texttt{dvips} driver}
%
%    \begin{macrocode}
%<*dvips>
%    \end{macrocode}
%
% \begin{macro}{\driver_graphics_getbb_eps:n}
%   Simply use the generic function.
%    \begin{macrocode}
%<*initex>
\use:n
%</initex>
%<*package>
\AtBeginDocument
%</package>
  { \cs_new_eq:NN \driver_graphics_getbb_eps:n \graphics_read_bb:n }
%    \end{macrocode}
% \end{macro}
%
% \begin{macro}{\driver_graphics_include_eps:n}
%  The special syntax is relatively clear here: remember we need PostScript
%  sizes here.
%    \begin{macrocode}
\cs_new_protected:Npn \driver_graphics_include_eps:n #1
  {
    \@@_literal:x
      {
        PSfile = #1 \c_space_tl
        llx = \dim_to_decimal_in_bp:n \l_graphics_llx_dim \c_space_tl
        lly = \dim_to_decimal_in_bp:n \l_graphics_lly_dim \c_space_tl
        urx = \dim_to_decimal_in_bp:n \l_graphics_urx_dim \c_space_tl
        ury = \dim_to_decimal_in_bp:n \l_graphics_ury_dim
      }
  }
%    \end{macrocode}
% \end{macro}
%
%    \begin{macrocode}
%</dvips>
%    \end{macrocode}
%
% \subsection{\texttt{pdfmode} driver}
%
%    \begin{macrocode}
%<*pdfmode>
%    \end{macrocode}
%
% \begin{variable}{\l_@@_graphics_attr_tl}
%   In PDF mode, additional attributes of an graphic (such as page number) are
%   needed both to obtain the bounding box and when inserting the graphic: this
%   occurs as the graphic dictionary approach means they are read as part of
%   the bounding box operation. As such, it is easier to track additional
%   attributes using a dedicated |tl| rather than build up the same data
%   twice.
%    \begin{macrocode}
\tl_new:N \l_@@_graphics_attr_tl
%    \end{macrocode}
% \end{variable}
%
% \begin{macro}
%   {
%     \driver_graphics_getbb_jpg:n,
%     \driver_graphics_getbb_pdf:n,
%     \driver_graphics_getbb_png:n
%   }
% \begin{macro}
%   {\@@_graphics_getbb_auxi:n, \@@_graphics_getbb_auxii:n}
%   Getting the bounding box here requires us to box up the graphic and
%   measure it. To deal with the difference in feature support in bitmap
%   and vector graphics but keeping the common parts, there is a little work
%   to do in terms of auxiliaries. The key here is to notice that we need
%   two forms of the attributes: a \enquote{short} set to allow us to
%   track for caching, and the full form to pass to the primitive.
%    \begin{macrocode}
\cs_new_protected:Npn \driver_graphics_getbb_jpg:n #1
  {
    \int_zero:N \l_graphics_page_int
    \tl_clear:N \l_graphics_pagebox_tl
    \tl_set:Nx \l_@@_graphics_attr_tl
      {
        \tl_if_empty:NF \l_graphics_decodearray_tl
          { :D \l_graphics_decodearray_tl }
        \bool_if:NT \l_graphics_interpolate_bool
          { :I }
      }
    \tl_clear:N \l_@@_graphics_attr_tl
    \@@_graphics_getbb_auxi:n {#1}
  }
\cs_new_eq:NN \driver_graphics_getbb_png:n \driver_graphics_getbb_jpg:n
\cs_new_protected:Npn \driver_graphics_getbb_pdf:n #1
  {
    \tl_clear:N \l_graphics_decodearray_tl
    \bool_set_false:N \l_graphics_interpolate_bool
    \tl_set:Nx \l_@@_graphics_attr_tl
      {
        : \l_graphics_pagebox_tl
        \int_compare:nNnT \l_graphics_page_int > 1
          { :P \int_use:N \l_graphics_page_int }
      }
    \@@_graphics_getbb_auxi:n {#1}
  }
\cs_new_protected:Npn \@@_graphics_getbb_auxi:n #1
  {
    \graphics_bb_restore:xF { #1 \l_@@_graphics_attr_tl }
      { \@@_graphics_getbb_auxii:n {#1} }
  }
%    \begin{macrocode}
%   Measuring the graphic is done by boxing up: for PDF graphics we could
%   use |\tex_pdfximagebbox:D|, but if doesn't work for other types.
%   As the box always starts at $(0,0)$ there is no need to worry about
%   the lower-left position.
%    \begin{macrocode}
\cs_new_protected:Npn \@@_graphics_getbb_auxii:n #1
  {
    \tex_immediate:D \tex_pdfximage:D
      \bool_lazy_or:nnT
        { \l_graphics_interpolate_bool }
        { ! \tl_if_empty_p:N \l_graphics_decodearray_tl }
        {
          attr ~
            {
              \tl_if_empty:NF \l_graphics_decodearray_tl
                { /Decode~[ \l_graphics_decodearray_tl ] }
              \bool_if:NT \l_graphics_interpolate_bool
                { /Interpolate~true }
            }
        }
      \int_compare:nNnT \l_graphics_page_int > 0
        { page ~ \int_use:N \l_graphics_page_int }
      \tl_if_empty:NF \l_graphics_pagebox_tl
        { \l_graphics_pagebox_tl }
      {#1}
    \hbox_set:Nn \l_@@_internal_box
      { \tex_pdfrefximage:D \tex_pdflastximage:D }
    \dim_set:Nn \l_graphics_urx_dim { \box_wd:N \l_@@_internal_box }
    \dim_set:Nn \l_graphics_ury_dim { \box_ht:N \l_@@_internal_box }
    \int_const:cn { c_@@_graphics_ #1 \l_@@_graphics_attr_tl _int }
      { \tex_the:D \tex_pdflastximage:D }
    \graphics_bb_save:x { #1 \l_@@_graphics_attr_tl }
  }
%    \end{macrocode}
% \end{macro}
% \end{macro}
%
% \begin{macro}
%   {
%     \driver_graphics_include_jpg:n,
%     \driver_graphics_include_pdf:n,
%     \driver_graphics_include_png:n
%   }
%   Images are already loaded for the measurement part of the code, so
%   inclusion is straight-forward, with only any attributes to worry about. The
%   latter carry through from determination of the bounding box.
%    \begin{macrocode}
\cs_new_protected:Npn \driver_graphics_include_jpg:n #1
  {
    \tex_pdfrefximage:D
      \int_use:c { c_@@_graphics_ #1 \l_@@_graphics_attr_tl _int }
  }
\cs_new_eq:NN \driver_graphics_include_pdf:n \driver_graphics_include_jpg:n
\cs_new_eq:NN \driver_graphics_include_png:n \driver_graphics_include_jpg:n
%    \end{macrocode}
% \end{macro}
%
% \begin{macro}{\driver_graphics_getbb_eps:n}
% \begin{macro}{\@@_graphics_getbb_eps:nm}
% \begin{macro}{\driver_graphics_include_eps:n}
% \begin{variable}{\l_@@_dir_str, \l_@@_name_str, \l_@@_ext_str}
%   EPS graphics may be included in \texttt{pdfmode} by conversion to
%   PDF: this requires restricted shell escape. Modelled on the \pkg{epstopdf}
%   \LaTeXe{} package, but simplified, conversion takes place here if we have
%   shell access.
%    \begin{macrocode}
\sys_if_shell:T
  {
    \str_new:N \l_@@_dir_str
    \str_new:N \l_@@_name_str
    \str_new:N \l_@@_ext_str
    \cs_new_protected:Npn \driver_graphics_getbb_eps:n #1
      {
        \file_parse_full_name:nNNN {#1}
          \l_@@_dir_str
          \l_@@_name_str
          \l_@@_ext_str
        \exp_args:Nx \@@_graphics_getbb_eps:nn
          {
            \l_@@_name_str - \str_tail:N \l_@@_ext_str
            -converted-to.pdf
          }
          {#1}
     }
    \cs_new_protected:Npn \@@_graphics_getbb_eps:nn #1#2
      {
        \file_compare_timestamp:nNnT {#2} > {#1}
          {
            \sys_shell_now:n
              { repstopdf ~ #2 ~ #1 }
          }
        \tl_set:Nn \l_graphics_name_tl {#1}
        \driver_graphics_getbb_pdf:n {#1}
      }
    \cs_new_protected:Npn \driver_graphics_include_eps:n #1
      {
        \file_parse_full_name:nNNN {#1}
          \l_@@_dir_str \l_@@_name_str \l_@@_ext_str
        \exp_args:Nx \driver_graphics_include_pdf:n
          {
            \l_@@_name_str - \str_tail:N \l_@@_ext_str
            -converted-to.pdf
          }
      }
  }
%    \end{macrocode}
% \end{variable}
% \end{macro}
% \end{macro}
% \end{macro}
%
%    \begin{macrocode}
%</pdfmode>
%    \end{macrocode}
%
% \subsection{\texttt{dvipdfmx} driver}
%
%    \begin{macrocode}
%<*dvipdfmx|xdvipdfmx>
%    \end{macrocode}
%
% \begin{macro}
%   {
%     \driver_graphics_getbb_eps:n, \driver_graphics_getbb_jpg:n,
%     \driver_graphics_getbb_pdf:n, \driver_graphics_getbb_png:n
%   }
%   Simply use the generic functions: only for \texttt{dvipdfmx} in the
%   extraction cases.
%    \begin{macrocode}
%<*initex>
\use:n
%</initex>
%<*package>
\AtBeginDocument
%</package>
  { \cs_new_eq:NN \driver_graphics_getbb_eps:n \graphics_read_bb:n }
%<*dvipdfmx>
\cs_new_protected:Npn \driver_graphics_getbb_jpg:n #1
  {
    \int_zero:N \l_graphics_page_int
    \tl_clear:N \l_graphics_pagebox_tl
    \graphics_extract_bb:n {#1}
  }
\cs_new_eq:NN \driver_graphics_getbb_png:n \driver_graphics_getbb_jpg:n
\cs_new_protected:Npn \driver_graphics_getbb_pdf:n #1
  {
    \tl_clear:N \l_graphics_decodearray_tl
    \bool_set_false:N \l_graphics_interpolate_bool
    \graphics_extract_bb:n {#1}
  }
%</dvipdfmx>
%    \end{macrocode}
% \end{macro}
%
% \begin{variable}{\g_@@_graphics_int}
%   Used to track the object number associated with each graphic.
%    \begin{macrocode}
\int_new:N \g_@@_graphics_int
%    \end{macrocode}
% \end{variable}
%
% \begin{macro}
%   {
%     \driver_graphics_include_eps:n, \driver_graphics_include_jpg:n,
%     \driver_graphics_include_pdf:n, \driver_graphics_include_png:n
%   }
%  \begin{macro}{\@@_graphics_include_auxi:nn}
%  \begin{macro}{\@@_graphics_include_auxii:nnn, \@@_graphics_include_auxii:xnn}
%  \begin{macro}{\@@_graphics_include_auxiii:nnn}
%   The special syntax depends on the file type. There is a difference in
%   how PDF graphics are best handled between |dvipdfmx| and |xdvipdfmx|: for
%   the latter it is better to use the primitive route. The relevant code for
%   that is included later in this file.
%    \begin{macrocode}
\cs_new_protected:Npn \driver_graphics_include_eps:n #1
  {
    \@@_literal:x
      {
        PSfile = #1 \c_space_tl
        llx = \dim_to_decimal_in_bp:n \l_graphics_llx_dim \c_space_tl
        lly = \dim_to_decimal_in_bp:n \l_graphics_lly_dim \c_space_tl
        urx = \dim_to_decimal_in_bp:n \l_graphics_urx_dim \c_space_tl
        ury = \dim_to_decimal_in_bp:n \l_graphics_ury_dim
      }
  }
\cs_new_protected:Npn \driver_graphics_include_jpg:n #1
  { \@@_graphics_include_auxi:nn {#1} { image } }
\cs_new_eq:NN \driver_graphics_include_png:n \driver_graphics_include_jpg:n
%<*dvipdfmx>
\cs_new_protected:Npn \driver_graphics_include_pdf:n #1
  { \@@_graphics_include_auxi:nn {#1} { epdf } }
%</dvipdfmx>
%    \end{macrocode}
%   Graphic inclusion is set up to use the fact that each image is stored in
%   the PDF as an XObject. This means that we can include repeated images
%   only once and refer to them. To allow that, track the nature of each
%   image: much the same as for the direct PDF mode case.
%    \begin{macrocode}
\cs_new_protected:Npn \@@_graphics_include_auxi:nn #1#2
  {
    \@@_graphics_include_auxii:xnn
      {
        \tl_if_empty:NF \l_graphics_pagebox_tl
          { : \l_graphics_pagebox_tl }
        \int_compare:nNnT \l_graphics_page_int > 1
          { :P \int_use:N \l_graphics_page_int }
        \tl_if_empty:NF \l_graphics_decodearray_tl
          { :D \l_graphics_decodearray_tl }
        \bool_if:NT \l_graphics_interpolate_bool
           { :I }
      }
      {#1} {#2}
  }
\cs_new_protected:Npn \@@_graphics_include_auxii:nnn #1#2#3
  {
    \int_if_exist:cTF { c_@@_graphics_ #2#1 _int }
      {
        \@@_literal:x
          { pdf:usexobj~@graphic \int_use:c { c_@@_graphics_ #2#1 _int } }
      }
      { \@@_graphics_include_auxiii:nnn {#2} {#1} {#3} }
  }
\cs_generate_variant:Nn \@@_graphics_include_auxii:nnn { x }
%    \end{macrocode}
%  Inclusion using the specials is relatively straight-forward, but there
%  is one wrinkle. To get the |pagebox| correct for PDF graphics in all cases,
%  it is necessary to provide both that information and the |bbox| argument:
%  odd things happen otherwise!
%    \begin{macrocode}
\cs_new_protected:Npn \@@_graphics_include_auxiii:nnn #1#2#3
  {
    \int_gincr:N \g_@@_graphics_int
    \int_const:cn { c_@@_graphics_ #1#2 _int } { \g_@@_graphics_int }
    \@@_literal:x
      {
        pdf:#3~
        @graphic \int_use:c { c_@@_graphics_ #1#2 _int } ~
        \int_compare:nNnT \l_graphics_page_int > 1
          { page ~ \int_use:N \l_graphics_page_int \c_space_tl }
        \tl_if_empty:NF \l_graphics_pagebox_tl
          {
            pagebox ~ \l_graphics_pagebox_tl \c_space_tl
            bbox ~
              \dim_to_decimal_in_bp:n \l_graphics_llx_dim \c_space_tl
              \dim_to_decimal_in_bp:n \l_graphics_lly_dim \c_space_tl
              \dim_to_decimal_in_bp:n \l_graphics_urx_dim \c_space_tl
              \dim_to_decimal_in_bp:n \l_graphics_ury_dim \c_space_tl
          }
        (#1)
        \bool_lazy_or:nnT
          { \l_graphics_interpolate_bool }
          { ! \tl_if_empty_p:N \l_graphics_decodearray_tl }
          {
            <<
              \tl_if_empty:NF \l_graphics_decodearray_tl
                { /Decode~[ \l_graphics_decodearray_tl ] }
              \bool_if:NT \l_graphics_interpolate_bool
                { /Interpolate~true> }
            >>
          }
      }
  }
%    \end{macrocode}
% \end{macro}
% \end{macro}
% \end{macro}
% \end{macro}
%
%    \begin{macrocode}
%</dvipdfmx|xdvipdfmx>
%    \end{macrocode}
%
% \subsection{\texttt{xdvipdfmx} driver}
%
%    \begin{macrocode}
%<*xdvipdfmx>
%    \end{macrocode}
%
% \subsubsection{Images}
%
% \begin{macro}
%   {
%     \driver_graphics_getbb_jpg:n,
%     \driver_graphics_getbb_pdf:n,
%     \driver_graphics_getbb_png:n
%   }
% \begin{macro}{\@@_graphics_getbb_auxi:nN}
% \begin{macro}{\@@_graphics_getbb_auxii:nnN, \@@_graphics_getbb_auxii:VnN}
% \begin{macro}{\@@_graphics_getbb_auxiii:nNnn}
% \begin{macro}{\@@_graphics_getbb_auxiv:nnNnn, \@@_graphics_getbb_auxiv:VnNnn}
% \begin{macro}{\@@_graphics_getbb_auxv:nNnn, \@@_graphics_getbb_auxv:nNnn}
% \begin{macro}[EXP]{\@@_graphics_getbb_pagebox:w}
%   For \texttt{xdvipdfmx}, there are two primitives that allow us to obtain
%   the bounding box without needing \texttt{extractbb}. The only complexity
%   is passing the various minor variations to a common core process. The
%   \XeTeX{} primitive omits the text |box| from the page box specification,
%   so there is also some \enquote{trimming} to do here.
%    \begin{macrocode}
\cs_new_protected:Npn \driver_graphics_getbb_jpg:n #1
  {
    \int_zero:N \l_graphics_page_int
    \tl_clear:N \l_graphics_pagebox_tl
    \@@_graphics_getbb_auxi:nN {#1} \tex_XeTeXpicfile:D
  }
\cs_new_eq:NN \driver_graphics_getbb_png:n \driver_graphics_getbb_jpg:n
\cs_new_protected:Npn \driver_graphics_getbb_pdf:n #1
  {
    \tl_clear:N \l_graphics_decodearray_tl
    \bool_set_false:N \l_graphics_interpolate_bool
    \@@_graphics_getbb_auxi:nN {#1} \tex_XeTeXpdffile:D
  }
\cs_new_protected:Npn \@@_graphics_getbb_auxi:nN #1#2
  {
    \int_compare:nNnTF \l_graphics_page_int > 1
      { \@@_graphics_getbb_auxii:VnN \l_graphics_page_int {#1} #2  }
      { \@@_graphics_getbb_auxiii:nNnn {#1} #2 { :P 1 } { page 1 } }
  }
\cs_new_protected:Npn \@@_graphics_getbb_auxii:nnN #1#2#3
  { \@@_graphics_getbb_auxiii:nNnn {#2} #3 { :P #1 } { page #1 } }
\cs_generate_variant:Nn \@@_graphics_getbb_auxii:nnN { V }
\cs_new_protected:Npn \@@_graphics_getbb_auxiii:nNnn #1#2#3#4
  {
    \tl_if_empty:NTF \l_graphics_pagebox_tl
      { \@@_graphics_getbb_auxiv:VnNnn \l_graphics_pagebox_tl }
      { \@@_graphics_getbb_auxv:nNnn }
      {#1} #2 {#3} {#4}
  }
\cs_new_protected:Npn \@@_graphics_getbb_auxiv:nnNnn #1#2#3#4#5
  {
    \use:x
      {
        \@@_graphics_getbb_auxv:nNnn {#2} #3 { : #1 #4 }
          { #5 ~ \@@_graphics_getbb_pagebox:w #1 }
      }
  }
\cs_generate_variant:Nn \@@_graphics_getbb_auxiv:nnNnn { V }
\cs_new_protected:Npn \@@_graphics_getbb_auxv:nNnn #1#2#3#4
  {
    \graphics_bb_restore:nF {#1#3}
      { \@@_graphics_getbb_auxvi:nNnn {#1} #2 {#3} {#4} }
  }
\cs_new_protected:Npn \@@_graphics_getbb_auxvi:nNnn #1#2#3#4
  {
    \hbox_set:Nn \l_@@_internal_box { #2 #1 ~ #4 }
    \dim_set:Nn \l_graphics_urx_dim { \box_wd:N \l_@@_internal_box }
    \dim_set:Nn \l_graphics_ury_dim { \box_ht:N \l_@@_internal_box }
    \graphics_bb_save:n {#1#3}
  }
\cs_new:Npn \@@_graphics_getbb_pagebox:w #1 box {#1}
%    \end{macrocode}
% \end{macro}
% \end{macro}
% \end{macro}
% \end{macro}
% \end{macro}
% \end{macro}
% \end{macro}
%
% \begin{macro}{\driver_graphics_include_pdf:n}
% \begin{macro}{\@@_graphics_include_bitmap_quote:w}
%   For PDF graphics, properly supporting the |pagebox| concept in \XeTeX{}
%   is best done using the |\tex_XeTeXpdffile:D| primitive. The syntax here
%   is the same as for the graphic measurement part, although we know at this
%   stage that there must be some valid setting for \cs{l_graphics_pagebox_tl}.
%    \begin{macrocode}
\cs_new_protected:Npn \driver_graphics_include_pdf:n #1
  {
    \tex_XeTeXpdffile:D
      \@@_graphics_include_pdf_quote:w #1 "#1" \q_stop \c_space_tl
      \int_compare:nNnT \l_graphics_page_int > 0
        { page ~ \int_use:N \l_graphics_page_int \c_space_tl }
        \exp_after:wN \@@_graphics_getbb_pagebox:w \l_graphics_pagebox_tl
  }
\cs_new:Npn \@@_graphics_include_pdf_quote:w #1 " #2 " #3 \q_stop
  { " #2 " }
%    \end{macrocode}
% \end{macro}
% \end{macro}
%
%    \begin{macrocode}
%</xdvipdfmx>
%    \end{macrocode}
%
% \subsection{\texttt{dvisvgm} driver}
%
%    \begin{macrocode}
%<*dvisvgm>
%    \end{macrocode}
%
% \begin{macro}{\driver_graphics_getbb_png:n, \driver_graphics_getbb_jpg:n}
%   These can be included by extracting the bounding box data.
%    \begin{macrocode}
%<*initex>
\use:n
%</initex>
%<*package>
\AtBeginDocument
%</package>
  {
    \cs_new_eq:NN \driver_graphics_getbb_png:n \graphics_extract_bb:n
    \cs_new_eq:NN \driver_graphics_getbb_jpg:n \graphics_extract_bb:n
  }
%    \end{macrocode}
% \end{macro}
%
% \begin{macro}{\driver_graphics_include_png:n, \driver_graphics_include_jpg:n}
% \begin{macro}{\@@_graphics_include_bitmap_quote:w}
%   The driver here has built-in support for basic graphic inclusion (see
%   \texttt{dvisvgm.def} for a more complex approach, needed if clipping,
%   \emph{etc.}, is covered at the graphic driver level). The only issue is
%   that |#1| must be quote-corrected. The \texttt{dvisvgm:img} operation
%   quotes the file name, but if it is already quoted (contains spaces)
%   then we have an issue: we simply strip off any quotes as a result.
%    \begin{macrocode}
\cs_new_protected:Npn \driver_graphics_include_png:n #1
  {
     \@@_literal:x
       {
         dvisvgm:img~
         \dim_to_decimal:n { \l_graphics_ury_dim } ~
         \dim_to_decimal:n { \l_graphics_ury_dim } ~
         \@@_graphics_include_bitmap_quote:w #1 " #1 " \q_stop
       }
  }
\cs_new_eq:NN \driver_graphics_include_jpg:n \driver_graphics_include_png:n
\cs_new:Npn \@@_graphics_include_bitmap_quote:w #1 " #2 " #3 \q_stop
  { " #2 " }
%    \end{macrocode}
% \end{macro}
% \end{macro}
%
%    \begin{macrocode}
%</dvisvgm>
%    \end{macrocode}
%
%    \begin{macrocode}
%</initex|package>
%    \end{macrocode}
%
% \end{implementation}
%
% \PrintIndex
