% \iffalse meta-comment
%
%% File: l3cctab.dtx
%
% Copyright (C) 2018-2020 The LaTeX3 Project
%
% It may be distributed and/or modified under the conditions of the
% LaTeX Project Public License (LPPL), either version 1.3c of this
% license or (at your option) any later version.  The latest version
% of this license is in the file
%
%    http://www.latex-project.org/lppl.txt
%
% This file is part of the "l3kernel bundle" (The Work in LPPL)
% and all files in that bundle must be distributed together.
%
% -----------------------------------------------------------------------
%
% The development version of the bundle can be found at
%
%    https://github.com/latex3/latex3
%
% for those people who are interested.
%
%<*driver>
\documentclass[full]{l3doc}
\begin{document}
  \DocInput{\jobname.dtx}
\end{document}
%</driver>
% \fi
%
% \title{^^A
%   The \pkg{l3cctab} package\\ Category code tables^^A
% }
%
% \author{^^A
%  The \LaTeX3 Project\thanks
%    {^^A
%      E-mail:
%        \href{mailto:latex-team@latex-project.org}
%          {latex-team@latex-project.org}^^A
%    }^^A
% }
%
% \date{Released 2020-06-18}
%
% \maketitle
%
% \begin{documentation}
%
% A category code table enables rapid switching of all category codes in
% one operation. For \LuaTeX{}, this is possible over the entire Unicode
% range. For other engines, only the $8$-bit range ($0$--$255$) is covered by
% such tables.
%
% \section{Creating and initialising category code tables}
%
% \begin{function}{\cctab_new:N,\cctab_new:c}
%   \begin{syntax}
%     \cs{cctab_new:N} \meta{category code table}
%   \end{syntax}
%   Creates a new \meta{category code table} variable or raises an error if
%   the name is already taken. The declaration is global.  The
%   \meta{category code table} is initialised with the codes
%   as used by \IniTeX{}.
% \end{function}
%
% \begin{function}{\cctab_const:Nn,\cctab_const:cn}
%   \begin{syntax}
%     \cs{cctab_const:Nn} \meta{category code table} \Arg{category code set up}
%   \end{syntax}
%   Creates a new \meta{category code table}, applies (in a group) the
%   \meta{category code set up} on top of prevailing settings at the
%   time the function is called, then saves them globally as a constant
%   table.  The \meta{category code set up} can include a call to
%   \cs{cctab_select:N}.
% \end{function}
%
% \begin{function}{\cctab_gset:Nn,\cctab_gset:cn}
%   \begin{syntax}
%     \cs{cctab_gset:Nn} \meta{category code table} \Arg{category code set up}
%   \end{syntax}
%   Applies (in a group) the \meta{category code set up} on top of
%   prevailing category code settings, then saves them globally in the
%   \meta{category code table}.  The \meta{category code set up} can
%   include a call to \cs{cctab_select:N}.  Within a standard code block
%   for instance, the starting point will be the code applied by
%   \cs{c_code_cctab}.  The assignment of the table is global: the
%   underlying primitive does not respect grouping.
% \end{function}
%
% \section{Using category code tables}
%
% \begin{function}{\cctab_begin:N,\cctab_begin:c}
%   \begin{syntax}
%     \cs{cctab_begin:N} \meta{category code table}
%   \end{syntax}
%   Switches locally the category codes in force to those stored in the
%   \meta{category code table}.  The prevailing codes before the
%   function is called are added to a stack, for use with
%   \cs{cctab_end:}. This function does not start a \TeX{} group.
% \end{function}
%
% \begin{function}{\cctab_end:}
%   \begin{syntax}
%     \cs{cctab_end:}
%   \end{syntax}
%   Ends the scope of a \meta{category code table} started using
%   \cs{cctab_begin:N}, returning the codes to those in force before the
%   matching \cs{cctab_begin:N} was used.  This must be used within the
%   same \TeX{} group (and at the same \TeX{} group level) as the
%   matching \cs{cctab_begin:N}.
% \end{function}
%
% \begin{function}[added = 2020-05-19]{\cctab_select:N}
%   \begin{syntax}
%     \cs{cctab_select:N} \meta{category code table}
%   \end{syntax}
%   Selects the \meta{category code table} for the scope of the current
%   group.  This is in particular useful in the \meta{setup} arguments
%   of \cs{tl_set_rescan:Nnn}, \cs{tl_rescan:nn}, \cs{cctab_const:Nn},
%   and \cs{cctab_gset:Nn}.
% \end{function}
%
% \section{Category code table conditionals}
%
% \begin{function}[pTF]{\cctab_if_exist:N,\cctab_if_exist:c}
%   \begin{syntax}
%     \cs{cctab_if_exist_p:N} \meta{category code table}
%     \cs{cctab_if_exist:NTF} \meta{category code table} \Arg{true code} \Arg{false code}
%   \end{syntax}
%   Tests whether the \meta{category code table} is currently defined.
%   This does not check that the \meta{category code table} really is a
%   category code table.
% \end{function}
%
% \section{Constant category code tables}
%
% \begin{variable}{\c_code_cctab}
%   Category code table for the code environment. This does not include
%   setting the behaviour of the line-end character, which is only
%   altered by \cs{ExplSyntaxOn}.
% \end{variable}
%
% \begin{variable}{\c_document_cctab}
%   Category code table for a standard \LaTeX{} document. This does not
%   include setting the behaviour of the line-end character, which is
%   only altered by \cs{ExplSyntaxOff}.
% \end{variable}
%
% \begin{variable}{\c_initex_cctab}
%   Category code table as set up by \IniTeX{}.
% \end{variable}
%
% \begin{variable}{\c_other_cctab}
%   Category code table where all characters have category code $12$
%   (other).
% \end{variable}
%
% \begin{variable}{\c_str_cctab}
%   Category code table where all characters have category code $12$
%   (other) with the exception of spaces, which have category code
%   $10$ (space).
% \end{variable}
%
% \end{documentation}
%
% \begin{implementation}
%
% \section{\pkg{l3cctab} implementation}
%
%    \begin{macrocode}
%<*initex|package>
%    \end{macrocode}
%
%    \begin{macrocode}
%<@@=cctab>
%    \end{macrocode}
%
% As \LuaTeX{} offers engine support for category code tables, and this
% is entirely lacking from the other engines, we need two complementary
% approaches. (Some future \XeTeX{} may add support, at which point the
% conditionals below would be different.)
%
% \subsection{Variables}
%
% \begin{variable}{\g_@@_stack_seq, \g_@@_unused_seq}
%   List of catcode tables saved by nested \cs{cctab_begin:N}, to
%   restore catcodes at the matching \cs{cctab_end:}.  When popped from
%   the \cs{g_@@_stack_seq} the table numbers are stored in
%   \cs{g_@@_unused_seq} for later reuse.
%    \begin{macrocode}
\seq_new:N \g_@@_stack_seq
\seq_new:N \g_@@_unused_seq
%    \end{macrocode}
% \end{variable}
%
% \begin{variable}{\g_@@_allocate_int}
%   Integer to keep track of what category code table to allocate.  In
%   \LuaTeX{} it is only used in format mode to implement
%   \cs{cctab_new:N}.  In other engines it is used to make csnames for
%   dynamic tables.
%    \begin{macrocode}
\int_new:N  \g_@@_allocate_int
%    \end{macrocode}
% \end{variable}
%
% \begin{variable}{\l_@@_internal_tl}
%   Scratch space.  For instance, when popping
%   \cs{g_@@_stack_seq}/\cs{g_@@_unused_seq}, consists of the
%   catcodetable number (integer denotation) in \LuaTeX{}, or of an
%   intarray variable (as a single token) in other engines.
%    \begin{macrocode}
\tl_new:N \l_@@_internal_tl
%    \end{macrocode}
% \end{variable}
%
% \begin{variable}{\g_@@_endlinechar_prop}
%   In \LuaTeX{} we store the \tn{endlinechar} associated to each
%   \tn{catcodetable} in a property list, unless it is the default
%   value~$13$.
%    \begin{macrocode}
\prop_new:N \g_@@_endlinechar_prop
%    \end{macrocode}
% \end{variable}
%
% \subsection{Allocating category code tables}
%
% \begin{macro}{\cctab_new:N, \cctab_new:c, \@@_new:N, \@@_gstore:Nnn}
%   The \cs{@@_new:N} auxiliary allocates a new catcode table but does
%   not attempt to set its value consistently across engines.  It is
%   used both in \cs{cctab_new:N}, which sets catcodes to \IniTeX{}
%   values, and in \cs{cctab_begin:N}/\cs{cctab_end:} for dynamically
%   allocated tables.
%
%   First, the \LuaTeX{} case.
%   Creating a new category code table is done like other registers.
%    \begin{macrocode}
\sys_if_engine_luatex:TF
  {
    \cs_new_protected:Npn \cctab_new:N #1
      {
        \__kernel_chk_if_free_cs:N #1
        \@@_new:N #1
      }
%<*initex>
    \cs_new_protected:Npn \@@_new:N #1
      {
        \int_gincr:N \g_@@_allocate_int
        \int_compare:nNnTF
          \g_@@_allocate_int > \c_max_register_int
           {
             \__kernel_msg_fatal:nnx
               { kernel } { out-of-registers } { cctab }
           }
           {
             \tex_global:D \tex_chardef:D #1 \g_@@_allocate_int
             \tex_initcatcodetable:D #1
           }
      }
%</initex>
%<*package>
    \cs_new_eq:NN \@@_new:N \newcatcodetable
%</package>
  }
%    \end{macrocode}
%   Now the case for other engines. Here, each table is an integer
%   array.  Following the \LuaTeX{} pattern, a new table starts with
%   \IniTeX{} codes.  The index base is out-by-one, so we have an
%   internal function to handle that.  The \IniTeX{} \tn{endlinechar} is
%   $13$.
%    \begin{macrocode}
  {
    \cs_new_protected:Npn \@@_new:N #1
      { \intarray_new:Nn #1 { 257 } }
    \cs_new_protected:Npn \@@_gstore:Nnn #1#2#3
      { \intarray_gset:Nnn #1 { \int_eval:n { #2 + 1 } } {#3} }
    \cs_new_protected:Npn \cctab_new:N #1
      {
        \__kernel_chk_if_free_cs:N #1
        \@@_new:N #1
        \int_step_inline:nn { 256 }
          { \__kernel_intarray_gset:Nnn #1 {##1} { 12 } }
        \__kernel_intarray_gset:Nnn #1 { 257 } { 13 }
        \@@_gstore:Nnn #1 { 0 } { 9 }
        \@@_gstore:Nnn #1 { 13 } { 5 }
        \@@_gstore:Nnn #1 { 32 } { 10 }
        \@@_gstore:Nnn #1 { 37 } { 14 }
        \int_step_inline:nnn { 65 } { 90 }
          { \@@_gstore:Nnn #1 {##1} { 11 } }
        \@@_gstore:Nnn #1 { 92 } { 0 }
        \int_step_inline:nnn { 97 } { 122 }
          { \@@_gstore:Nnn #1 {##1} { 11 } }
        \@@_gstore:Nnn #1 { 127 } { 15 }
      }
  }
\cs_generate_variant:Nn \cctab_new:N { c }
%    \end{macrocode}
% \end{macro}
%
% \subsection{Saving category code tables}
%
% \begin{macro}{\@@_gset:n, \@@_gset_aux:n}
%   In various functions we need to save the current catcodes (globally)
%   in a table.  In \LuaTeX{}, saving the catcodes is a primitives, but
%   the \tn{endlinechar} needs more work: to avoid filling
%   \cs{g_@@_endlinechar_prop} with many entries we special-case the
%   default value $13$.  In other engines we store $256$ current
%   catcodes and the \tn{endlinechar} in an intarray variable.
%    \begin{macrocode}
\sys_if_engine_luatex:TF
  {
    \cs_new_protected:Npn \@@_gset:n #1
      { \exp_args:Nf \@@_gset_aux:n { \int_eval:n {#1} } }
    \cs_new_protected:Npn \@@_gset_aux:n #1
      {
        \tex_savecatcodetable:D #1 \scan_stop:
        \int_compare:nNnTF { \tex_endlinechar:D } = { 13 }
          { \prop_gremove:Nn \g_@@_endlinechar_prop {#1} }
          {
            \prop_gput:NnV \g_@@_endlinechar_prop {#1}
              \tex_endlinechar:D
          }
      }
  }
  {
    \cs_new_protected:Npn \@@_gset:n #1
      {
        \int_step_inline:nn { 256 }
          {
            \__kernel_intarray_gset:Nnn #1 {##1}
              { \char_value_catcode:n { ##1 - 1 } }
          }
        \__kernel_intarray_gset:Nnn #1 { 257 }
          { \tex_endlinechar:D }
      }
  }
%    \end{macrocode}
% \end{macro}
%
% \begin{macro}{\cctab_gset:Nn, \cctab_gset:cn}
%   Category code tables are always global, so only one version of
%   assignments is needed.  Simply run the setup in a group and save the
%   result in a category code table~|#1|, provided it is valid.  The
%   internal function is defined above depending on the engine.
%    \begin{macrocode}
\cs_new_protected:Npn \cctab_gset:Nn #1#2
  {
    \@@_chk_if_valid:NT #1
      {
        \group_begin:
          #2 \scan_stop:
          \@@_gset:n {#1}
        \group_end:
      }
  }
\cs_generate_variant:Nn \cctab_gset:Nn { c }
%    \end{macrocode}
% \end{macro}
%
% \subsection{Using category code tables}
%
% \begin{macro}{\cctab_select:N, \cctab_select:c}
% \begin{variable}{\g_@@_internal_cctab}
% \begin{macro}{\@@_select:N}
%   The public function simply checks the \meta{cctab~var} exists before
%   using the engine-dependent \cs{@@_select:N}.  Skipping these checks
%   would result in low-level engine-dependent errors.  First, the
%   \LuaTeX{} case.  The aim here is to ensure that the saved tables are
%   read-only.  This is done by applying the saved table, then switching
%   immediately to a scratch table \cs{g_@@_internal_cctab}.  Any later
%   catcode assignment will affect that scratch table rather than the
%   saved one.  In other engines, selecting a catcode table is a matter
%   of doing $256$ catcode assignments and setting the \tn{endlinechar}.
%    \begin{macrocode}
\cs_new_protected:Npn \cctab_select:N #1
  { \@@_chk_if_valid:NT #1 { \@@_select:N #1 } }
\cs_generate_variant:Nn \cctab_select:N { c }
\sys_if_engine_luatex:TF
  {
    \@@_new:N \g_@@_internal_cctab
    \cs_new_protected:Npn \@@_select:N #1
      {
        \tex_catcodetable:D #1
        \tex_savecatcodetable:D \g_@@_internal_cctab
        \tex_catcodetable:D \g_@@_internal_cctab
        \prop_get:NVNTF \g_@@_endlinechar_prop #1 \l_@@_internal_tl
          { \int_set:Nn \tex_endlinechar:D { \l_@@_internal_tl } }
          { \int_set:Nn \tex_endlinechar:D { 13 } }
      }
  }
  {
    \cs_new_protected:Npn \@@_select:N #1
      {
        \int_step_inline:nn { 256 }
          {
            \char_set_catcode:nn { ##1 - 1 }
              { \__kernel_intarray_item:Nn #1 {##1} }
          }
        \int_set:Nn \tex_endlinechar:D
          { \__kernel_intarray_item:Nn #1 { 257 } }
      }
  }
%    \end{macrocode}
% \end{macro}
% \end{variable}
% \end{macro}
%
% \begin{variable}{\g_@@_next_cctab}
% \begin{macro}{\@@_begin_aux:}
%   For \cs{cctab_begin:N}/\cs{cctab_end:} we will need to allocate
%   dynamic tables.  This is done here by \cs{@@_begin_aux:}, which puts
%   a table number (in \LuaTeX{}) or name (in other engines) into
%   \cs{l_@@_internal_tl}.  In \LuaTeX{} this simply calls \cs{@@_new:N}
%   and uses the resulting catcodetable number; in other engines we need
%   to give a name to the intarray variable and use that.
%    \begin{macrocode}
\sys_if_engine_luatex:TF
  {
    \cs_new_protected:Npn \@@_begin_aux:
      {
        \@@_new:N \g_@@_next_cctab
        \tl_set:NV \l_@@_internal_tl \g_@@_next_cctab
        \cs_undefine:N \g_@@_next_cctab
      }
  }
  {
    \cs_new_protected:Npn \@@_begin_aux:
      {
        \int_gincr:N \g_@@_allocate_int
        \exp_args:Nc \@@_new:N
          { g_@@_ \int_use:N \g_@@_allocate_int _cctab }
        \exp_args:NNc \tl_set:Nn \l_@@_internal_tl
          { g_@@_ \int_use:N \g_@@_allocate_int _cctab }
      }
  }
%    \end{macrocode}
% \end{macro}
% \end{variable}
%
% \begin{macro}{\cctab_begin:N, \cctab_begin:c}
%   Check the \meta{cctab~var} exists, to avoid low-level errors.  Get
%   in \cs{l_@@_internal_tl} the number/name of a dynamic table, either
%   from \cs{g_@@_unused_seq} where we save tables that are not
%   currently in use, or from \cs{@@_begin_aux:} if none are available.
%   Then save the current catcodes into the table (pointed to by)
%   \cs{l_@@_internal_tl} and save that table number in a stack before
%   selecting the desired catcodes.
%    \begin{macrocode}
\cs_new_protected:Npn \cctab_begin:N #1
  {
    \@@_chk_if_valid:NT #1
      {
        \seq_gpop:NNF \g_@@_unused_seq \l_@@_internal_tl
          { \@@_begin_aux: }
        \seq_gpush:NV \g_@@_stack_seq \l_@@_internal_tl
        \exp_args:NV \@@_gset:n \l_@@_internal_tl
        \@@_select:N #1
      }
  }
\cs_generate_variant:Nn \cctab_begin:N { c }
%    \end{macrocode}
% \end{macro}
%
% \begin{macro}{\cctab_end:}
%   Make sure a \cs{cctab_begin:N} was used some time earlier, get in
%   \cs{l_@@_internal_tl} the catcode table number/name in which the
%   prevailing catcodes were stored, then restore these catcodes.  The
%   dynamic table is now unused hence stored in \cs{g_@@_unused_seq} for
%   recycling by later \cs{cctab_begin:N}.
%    \begin{macrocode}
\cs_new_protected:Npn \cctab_end:
  {
    \seq_gpop:NNTF \g_@@_stack_seq \l_@@_internal_tl
      {
        \seq_gpush:NV \g_@@_unused_seq \l_@@_internal_tl
        \@@_select:N \l_@@_internal_tl
      }
      { \__kernel_msg_error:nn { kernel } { cctab-extra-end } }
  }
%    \end{macrocode}
% \end{macro}
%
% \subsection{Category code table conditionals}
%
% \begin{macro}{\cctab_if_exist:N,\cctab_if_exist:c}
%   Checks whether a \meta{cctab~var} is defined.
%    \begin{macrocode}
\prg_new_eq_conditional:NNn \cctab_if_exist:N \cs_if_exist:N
  { TF , T , F , p }
\prg_new_eq_conditional:NNn \cctab_if_exist:c \cs_if_exist:c
  { TF , T , F , p }
%    \end{macrocode}
% \end{macro}
%
% \begin{macro}[TF]{\@@_chk_if_valid:N}
% \begin{macro}{\@@_chk_if_valid_aux:NTF}
%   Checks whether the argument is defined and whether it is a valid
%   \meta{cctab~var}. In \LuaTeX{} the validity of the \meta{cctab~var}
%   is checked by the engine, which complains if the argument is not a
%   \cs{chardef}'ed constant. In other engines, check if the given
%   command is an intarray variable (the underlying definition is a copy
%   of the \texttt{cmr10} font).
%    \begin{macrocode}
\prg_new_protected_conditional:Npnn \@@_chk_if_valid:N #1
  { TF , T , F }
  {
    \cctab_if_exist:NTF #1
      {
        \@@_chk_if_valid_aux:NTF #1
          { \prg_return_true: }
          {
            \__kernel_msg_error:nnx { kernel } { invalid-cctab }
              { \token_to_str:N #1 }
            \prg_return_false:
          }
      }
      {
        \__kernel_msg_error:nnx { kernel } { command-not-defined }
          { \token_to_str:N #1 }
        \prg_return_false:
      }
  }
\sys_if_engine_luatex:TF
  {
    \cs_new_protected:Npn \@@_chk_if_valid_aux:NTF #1
      {
%<*initex>
        \bool_lazy_and:nnTF
          { \int_if_odd_p:n {#1} }
          { \int_compare_p:nNn {#1-1} < { \g_@@_allocate_int } }
%</initex>
%<*package>
        \int_compare:nNnTF {#1-1} < { \e@alloc@ccodetable@count }
%</package>
      }
  }
  {
    \cs_new_protected:Npn \@@_chk_if_valid_aux:NTF #1
      {
        \exp_args:Nf \str_if_in:nnTF
          { \cs_meaning:N #1 }
          { select~font~cmr10~at~ }
      }
  }
%    \end{macrocode}
% \end{macro}
% \end{macro}
%
% \subsection{Constant category code tables}
%
% \begin{macro}{\cctab_const:Nn,\cctab_const:cn}
%  Creates a new \meta{cctab~var} then sets it with the current and
%  user-supplied codes.
%    \begin{macrocode}
\cs_new_protected:Npn \cctab_const:Nn #1#2
  {
    \cctab_new:N #1
    \cctab_gset:Nn #1 {#2}
  }
\cs_generate_variant:Nn \cctab_const:Nn { c }
%    \end{macrocode}
% \end{macro}
%
% \begin{variable}
%   {
%     \c_initex_cctab   ,
%     \c_code_cctab     ,
%     \c_document_cctab ,
%     \c_other_cctab    ,
%     \c_str_cctab
%   }  
%   Creating category code tables is easy using the function above.
%   The \texttt{other} and \texttt{string} ones are done by completely
%   ignoring the existing codes as this makes life a lot less complex.
%    \begin{macrocode}
\cctab_new:N \c_initex_cctab
\cctab_const:Nn \c_code_cctab
  {
    \char_set_catcode_letter:n         { 64 }
    \int_set:Nn \tex_endlinechar:D     { 32 }
  }
\cctab_const:Nn \c_document_cctab
  {
    \int_set:Nn \tex_endlinechar:D     { 13 }
    \char_set_catcode_space:n          { 9 }
    \char_set_catcode_space:n          { 32 }
    \char_set_catcode_other:n          { 58 }
    \char_set_catcode_math_subscript:n { 95 }
    \char_set_catcode_active:n         { 126 }
  }
\cctab_const:Nn \c_other_cctab
  {
    \cctab_select:N \c_initex_cctab
    \int_set:Nn \tex_endlinechar:D     { -1 }
    \int_step_inline:nnn { 0 } { 127 }
      { \char_set_catcode_other:n {#1} }
  }
\cctab_const:Nn \c_str_cctab
  {
    \cctab_select:N \c_other_cctab
    \char_set_catcode_space:n { 32 }
  }
%    \end{macrocode}
% \end{variable}
%
% \subsection{Messages}
%
%    \begin{macrocode}
\__kernel_msg_new:nnnn { kernel } { cctab-stack-full }
  { The~category~code~table~stack~is~exhausted. }
  {
    LaTeX~has~been~asked~to~switch~to~a~new~category~code~table,~
    but~there~is~no~more~space~to~do~this!
  }
\__kernel_msg_new:nnnn { kernel } { cctab-extra-end }
  { Extra~\iow_char:N\\cctab_end:~ignored~\msg_line_context:. }
  {
    LaTeX~came~across~a~\iow_char:N\\cctab_end:~without~a~matching~
    \iow_char:N\\cctab_begin:N.~This~command~will~be~ignored.
  }
\__kernel_msg_new:nnnn { kernel } { invalid-cctab }
  { Invalid~\iow_char:N\\catcode~table. }
  {
    You~can~only~switch~to~a~\iow_char:N\\catcode~table~that~is~
    initialized~using~\iow_char:N\\cctab_new:N~or~
    \iow_char:N\\cctab_const:Nn.
  }
%    \end{macrocode}
%
%    \begin{macrocode}
%</initex|package>
%    \end{macrocode}
%
% \end{implementation}
%
%\PrintIndex
