% \iffalse meta-comment
%
%% File: l3drivers-pdf.dtx
%
% Copyright (C) 2011-2019 The LaTeX3 Project
%
% It may be distributed and/or modified under the conditions of the
% LaTeX Project Public License (LPPL), either version 1.3c of this
% license or (at your option) any later version.  The latest version
% of this license is in the file
%
%    https://www.latex-project.org/lppl.txt
%
% This file is part of the "l3kernel bundle" (The Work in LPPL)
% and all files in that bundle must be distributed together.
%
% -----------------------------------------------------------------------
%
% The development version of the bundle can be found at
%
%    https://github.com/latex3/latex3
%
% for those people who are interested.
%
%<*driver>
\documentclass[full,kernel]{l3doc}
\begin{document}
  \DocInput{\jobname.dtx}
\end{document}
%</driver>
% \fi
%
% \title{^^A
%   The \textsf{l3drivers-pdf} package\\ Driver PDF features^^A
% }
%
% \author{^^A
%  The \LaTeX3 Project\thanks
%    {^^A
%      E-mail:
%        \href{mailto:latex-team@latex-project.org}
%          {latex-team@latex-project.org}^^A
%    }^^A
% }
%
% \date{Released 2019-05-28}
%
% \maketitle
%
% \begin{documentation}
%
% \end{documentation}
%
% \begin{implementation}
%
% \section{\pkg{l3drivers-pdf} Implementation}
%
%    \begin{macrocode}
%<*initex|package>
%<@@=driver>
%    \end{macrocode}
%
% Setting up PDF resources is a complex area with only limited documentation
% in the engine manuals. The following code builds heavily on existing ideas
% from \pkg{hyperref} work by Sebastian Rahtz and Heiko Oberdiek, and
% significant contributions by Alexander Grahn, in addition to the specific
% code referenced a various points.
%
% \subsection{Shared code}
%
% A very small number of items that belong at the driver level but which
% are common to all drivers.
%
% \subsection{\texttt{dvips} driver}
%
%    \begin{macrocode}
%<*dvips>
%    \end{macrocode}
%
% \begin{macro}{\@@_pdfmark:n, \@@_pdfmark:x}
%   Used often enough it should be a separate function.
%    \begin{macrocode}
\cs_new_protected:Npn \@@_pdfmark:n #1
  { \@@_postscript:n { mark #1 ~ pdfmark } }
\cs_generate_variant:Nn \@@_pdfmark:n { x }
%    \end{macrocode}
% \end{macro}
%
% \subsubsection{Catalogue entries}
%
% \begin{macro}{\driver_pdf_catalog_gput:nn, \driver_pdf_info_gput:nn}
%    \begin{macrocode}
\cs_new_protected:Npn \driver_pdf_catalog_gput:nn #1#2
  { \@@_pdfmark:n { { Catalog } << /#1 ~ #2 >> /PUT } }
\cs_new_protected:Npn \driver_pdf_info_gput:nn #1#2
  { \@@_pdfmark:n { /#1 ~ #2 /DOCINFO } }
%    \end{macrocode}
% \end{macro}
%
% \subsubsection{Objects}
%
% \begin{variable}{\g_@@_pdf_object_int, \g_@@_pdf_object_prop}
%   For tracking objects to allow finalisation.
%    \begin{macrocode}
\int_new:N \g_@@_pdf_object_int
\prop_new:N \g_@@_pdf_object_prop
%    \end{macrocode}
% \end{variable}
%
% \begin{macro}{\driver_pdf_object_new:nn}
% \begin{macro}[EXP]{\driver_pdf_object_ref:n}
%   Tracking objects is similar to \texttt{dvipdfmx}.
%    \begin{macrocode}
\cs_new_protected:Npn \driver_pdf_object_new:nn #1#2
  {
    \int_gincr:N \g_@@_pdf_object_int
    \int_const:cn
      { c_@@_pdf_object_ \tl_to_str:n {#1} _int }
      { \g_@@_pdf_object_int }
    \prop_gput:Nnn \g_@@_pdf_object_prop {#1} {#2}
  }
\cs_new:Npn \driver_pdf_object_ref:n #1
  { { driver.obj \int_use:c { c_@@_pdf_object_ \tl_to_str:n {#1} _int } } }
%    \end{macrocode}
% \end{macro}
% \end{macro}
%
% \begin{macro}{\driver_pdf_object_write:nn, \driver_pdf_object_write:nx}
% \begin{macro}
%   {
%     \@@_pdf_object_write_array:nn ,
%     \@@_pdf_object_write_dict:nn  ,
%     \@@_pdf_object_write_stream:nn
%   }
% \begin{macro}{\@@_pdf_object_write_stream:nnn}
%   This is where we choose the actual type: some work to get things
%   right.
%    \begin{macrocode}
\cs_new_protected:Npn \driver_pdf_object_write:nn #1#2
  {
    \@@_pdfmark:x
      {
        /_objdef ~ \driver_pdf_object_ref:n {#1}
        /type
        \str_case_e:nn
          { \prop_item:Nn \g_@@_pdf_object_prop {#1} }
          {
            { array }   { /array }
            { dict }    { /dict }
            { fstream } { /stream }
            { stream }  { /stream }
          }
        /OBJ
      }
    \use:c
      { @@_pdf_object_write_ \prop_item:Nn \g_@@_pdf_object_prop {#1} :nn }
      { \driver_pdf_object_ref:n {#1} } {#2}
  }
\cs_generate_variant:Nn \driver_pdf_object_write:nn { nx }
\cs_new_protected:Npn \@@_pdf_object_write_array:nn #1#2
  {
    \@@_pdfmark:x
      { #1 [ ~ \exp_not:n {#2} ~ ] ~ /PUTINTERVAL }
  }
\cs_new_protected:Npn \@@_pdf_object_write_dict:nn #1#2
  {
    \@@_pdfmark:x
      { #1 << \exp_not:n {#2} >> /PUT }
  }
\cs_new_protected:Npn \@@_pdf_object_write_stream:nn #1#2
  {
    \exp_args:Nx
      \@@_pdf_object_write_stream:nnn {#1} #2
  }
\cs_new_protected:Npn \@@_pdf_object_write_stream:nnn #1#2#3
  {
    \@@_postscript:n
      {
        [nobreak]
        mark ~ #1 ~ ( #3 ) /PUT ~ pdfmark ~
        mark ~ #1 ~ << #2 >> /PUT ~ pdfmark
      }
  }
%    \end{macrocode}
% \end{macro}
% \end{macro}
% \end{macro}
%
% \begin{macro}{\driver_pdf_object_now:nn, \driver_pdf_object_now:nx}
%   No anonymous objects, so things are done manually.
%    \begin{macrocode}
\cs_new_protected:Npn \driver_pdf_object_now:nn #1#2
  {
    \int_gincr:N \g_@@_pdf_object_int
    \@@_pdfmark:x
      {
        /_objdef ~ { driver.obj \int_use:N \g_@@_pdf_object_int }
        /type
        \str_case:nn
          {#1}
          {
            { array }   { /array }
            { dict }    { /dict }
            { fstream } { /stream }
            { stream }  { /stream }
          }
        /OBJ
      }
    \exp_args:Nnx \use:c { @@_pdf_object_write_ #1 :nn }
      { { driver.obj \int_use:N \g_@@_pdf_object_int } } {#2}
  }
\cs_generate_variant:Nn \driver_pdf_object_now:nn { nx }
%    \end{macrocode}
% \end{macro}
%
% \begin{macro}[EXP]{\driver_pdf_object_last:}
%   Much like the annotation version.
%    \begin{macrocode}
\cs_new:Npn \driver_pdf_object_last:
  { { driver.obj \int_use:N \g_@@_pdf_object_int } }
%    \end{macrocode}
% \end{macro}
%
% \subsubsection{Annotations}
%
% In \texttt{dvips}, annotations have to be constructed manually. As such,
% we need the object code above for some definitions.
%
% \begin{macro}{driver.globaldict}
%   A small global dictionary for driver use.
%    \begin{macrocode}
\@@_postscript_header:n
  {
    true ~ setglobal ~
    /driver.globaldict ~ 4 ~ dict ~ def ~
    false ~ setglobal
  }
%    \end{macrocode}
% \end{macro}
%
% \begin{macro}
%   {
%     driver.cvs     ,
%     driver.dvi.pt  ,
%     driver.pt.dvi  ,
%     driver.rect.ht
%   }
%   Small utilities for PostScript manipulations. Conversion to DVI dimensions
%   is done here to allow for |Resolution|. The total height of a rectangle
%   (an array) needs a little maths, in contrast to simply extracting a value.
%    \begin{macrocode}
\@@_postscript_header:n
  {
    /driver.cvs { 65534 ~ string ~ cvs } def
    /driver.dvi.pt { 72.27 ~ mul ~ Resolution ~ div } def
    /driver.pt.dvi { 72.27 ~ div ~ Resolution ~ mul } def
    /driver.rect.ht { dup ~ 1 ~ get ~ neg ~ exch ~ 3 ~ get ~ add } def
  }
%    \end{macrocode}
% \end{macro}
%
% \begin{macro}{driver.linkmargin, driver.linkdp.pad, driver.linkht.pad}
%   Settings which are defined up-front in |SDict|.
%    \begin{macrocode}
\@@_postscript_header:n
  {
    /driver.linkmargin { 1 ~ driver.pt.dvi } def
    /driver.linkdp.pad { 0 } def 
    /driver.linkht.pad { 0 } def
  }
%    \end{macrocode}
% \end{macro}
%
% \begin{macro}
%   {
%     driver.rect        ,
%     driver.save.ll     ,
%     driver.save.ur     ,
%     driver.save.linkll ,
%     driver.save.linkur ,
%     driver.llx         ,
%     driver.lly         ,
%     driver.urx         ,
%     driver.ury
%   }
%   Functions for marking the limits of an annotation/link, plus drawing the
%   border. We separate links for generic annotations to support adding a
%   margin and setting a minimal size.
%    \begin{macrocode}
\@@_postscript_header:n
  {
    /driver.rect
      { /Rect [ driver.llx ~ driver.lly ~ driver.urx ~ driver.ury ] } def
    /driver.save.ll
      {
        currentpoint
        /driver.lly ~ exch ~ def
        /driver.llx ~ exch ~ def
      }
        def
    /driver.save.ur
      {
        currentpoint
        /driver.ury ~ exch ~ def
        /driver.urx ~ exch ~ def
      }
        def
    /driver.save.linkll
      {
        currentpoint ~
        driver.linkmargin ~ add ~
        driver.linkdp.pad ~ add
        /driver.lly ~ exch ~ def ~
        driver.linkmargin ~ sub
        /driver.llx ~ exch ~ def
      }
        def
    /driver.save.linkur
      {
        currentpoint ~
        driver.linkmargin ~ sub ~
        driver.linkht.pad ~ sub
        /driver.ury ~ exch ~ def ~
        driver.linkmargin ~ add
        /driver.urx ~ exch ~ def
      }
        def
  }
%    \end{macrocode}
% \end{macro}
%
% \begin{macro}
%   {
%     driver.dest.anchor ,
%     driver.dest.x      ,
%     driver.dest.y      ,
%     driver.dest.point  ,
%     driver.dest2device ,
%     driver.dev.x       ,
%     driver.dev.y       ,
%     driver.tmpa        ,
%     driver.tmpb        ,
%     driver.tmpc        ,
%     driver.tmpd
%   }
%   For finding the anchor point of a destination link. We make the use case
%   a separate function as it comes up a lot, and as this makes it easier to
%   adjust if we need additional effects. We also need a more complex approach
%   to convert a co-ordinate pair correctly when defining a rectangle: this
%   can otherwise be out when using a landscape page. (Thanks to Alexander
%   Grahn for the approach here.)
%    \begin{macrocode}
\@@_postscript_header:n
  {
    /driver.dest.anchor
      {
        currentpoint ~ exch ~
        driver.dvi.pt ~ 72 ~ add ~
        /driver.dest.x ~ exch ~ def ~
        driver.dvi.pt ~
        vsize ~ 72 ~ sub ~ exch ~ sub ~
        /driver.dest.y ~ exch ~ def
      }
        def
    /driver.dest.point
      { driver.dest.x ~ driver.dest.y } def
   /driver.dest2device
     {
       /driver.dest.y ~ exch ~ def
       /driver.dest.x ~ exch ~ def ~
       matrix ~ currentmatrix ~
       matrix ~ defaultmatrix ~
       matrix ~ invertmatrix ~
       matrix ~ concatmatrix ~
       cvx ~ exec
       /driver.dev.y ~ exch ~ def
       /driver.dev.x ~ exch ~ def
       /driver.tmpd ~ exch ~ def
       /driver.tmpc ~ exch ~ def
       /driver.tmpb ~ exch ~ def
       /driver.tmpa ~ exch ~ def ~
       driver.dest.x ~ driver.tmpa ~ mul ~
         driver.dest.y ~ driver.tmpc ~ mul ~ add ~
         driver.dev.x ~ add ~
       driver.dest.x ~ driver.tmpb ~ mul ~
         driver.dest.y ~ driver.tmpd ~ mul ~ add ~
         driver.dev.y ~ add
     }
       def
  }
%    \end{macrocode}
% \end{macro}
%
% \begin{macro}
%   {
%     driver.bordertracking          ,
%     driver.bordertracking.begin    ,
%     driver.bordertracking.end      ,
%     driver.leftboundary            ,
%     driver.rightboundary           ,
%     driver.brokenlink.rect         ,
%     driver.brokenlink.skip         ,
%     driver.brokenlink.dict         ,
%     driver.bordertracking.endpage  ,
%     driver.bordertracking.continue ,
%     driver.originx                 ,
%     driver.originy
%   }
%    To know where a breakable link can go, we need to track the boundary
%    rectangle. That can be done by hooking into |a| and |x| operations:
%    those names have to be retained. The boundary is stored at the end of
%    the operation. Special effort is needed at the start and end of pages
%    (or rather galleys), such that everything works properly.
%    \begin{macrocode}
\@@_postscript_header:n
  {
    /driver.bordertracking ~ false ~ def
    /driver.bordertracking.begin
      {
        SDict ~ /driver.bordertracking ~ true ~ put ~
        SDict ~ /driver.leftboundary ~ undef ~
        SDict ~ /driver.rightboundary ~ undef ~
        /a ~ where
          {
            /a
              {
                currentpoint ~ pop ~
                SDict /driver.rightboundary ~ known ~ dup
                  {
                    SDict /driver.rightboundary ~ get ~ 2 ~ index ~ lt
                      { not }
                    if
                  }
                if
                  { pop }
                  { SDict ~ exch /driver.rightboundary ~ exch ~ put }
                ifelse ~
                moveto ~
                currentpoint ~ pop ~
                SDict /driver.leftboundary ~ known ~ dup
                  {
                    SDict /driver.leftboundary ~ get ~ 2 ~ index ~ gt
                      { not }
                    if
                  }
                if
                  { pop }
                  { SDict ~ exch /driver.leftboundary ~ exch ~ put }
                ifelse
              }
            put
          }
        if
      }
        def
    /driver.bordertracking.end
      {
        /a ~ where { /a { moveto } put } if
        /x ~ where { /x { 0 ~ exch ~ rmoveto } put } if ~
        SDict /driver.leftboundary ~ known
          { driver.outerbox ~ 0 ~ driver.leftboundary ~ put }
        if ~
        SDict /driver.rightboundary ~ known
          { driver.outerbox ~ 2 ~ driver.rightboundary ~ put }
        if ~
        SDict /driver.bordertracking ~ false ~ put
      }
        def
  /driver.bordertracking.endpage
    {
      driver.bordertracking
        {
          driver.bordertracking.end ~
          true ~ setglobal ~
          driver.globaldict
            /driver.brokenlink.rect [ driver.outerbox ~ aload ~ pop ] put ~
          driver.globaldict
            /driver.brokenlink.skip ~ driver.baselineskip ~ put ~
          driver.globaldict
            /driver.brokenlink.dict ~
              driver.link.dict ~ driver.cvs ~ put ~
          false ~ setglobal ~
          mark ~ driver.link.dict ~ cvx ~ exec ~ /Rect
            [
              driver.llx ~
              driver.lly ~
              driver.outerbox ~ 2 ~ get ~ driver.linkmargin ~ add ~
              currentpoint ~ exch ~ pop ~
              driver.outerbox ~ driver.rect.ht ~ sub ~ driver.linkmargin ~ sub
            ]
          /ANN ~ driver.pdfmark
        }
      if
    }
      def
    /driver.bordertracking.continue
      {
        /driver.link.dict ~ driver.globaldict
          /driver.brokenlink.dict ~ get ~ def
        /driver.outerbox ~ driver.globaldict
          /driver.brokenlink.rect ~ get ~ def
        /driver.baselineskip ~ driver.globaldict
          /driver.brokenlink.skip ~ get ~ def ~
        driver.globaldict ~ dup ~ dup
        /driver.brokenlink.dict ~ undef
        /driver.brokenlink.skip ~ undef
        /driver.brokenlink.rect ~ undef ~
        currentpoint
        /driver.originy ~ exch ~ def
        /driver.originx ~ exch ~ def
        /a ~ where
          {
            /a
              {
                moveto ~
                SDict ~
                begin ~
                currentpoint ~ driver.originy ~ ne ~ exch ~
                  driver.originx ~ ne ~ or
                  {
                    driver.save.linkll
                    /driver.lly ~
                      driver.lly ~ driver.outerbox ~ 1 ~ get ~ sub ~ def ~
                    driver.bordertracking.begin
                  }
                if ~
                end
              }
            put
          }
        if
        /x ~ where
          {
            /x
              {
                0 ~ exch ~ rmoveto ~
                SDict~
                begin ~
                currentpoint ~
                driver.originy ~ ne ~ exch ~ driver.originx ~ ne ~ or
                  {
                    driver.save.linkll
                    /driver.lly ~
                      driver.lly ~ driver.outerbox ~ 1 ~ get ~ sub ~ def ~
                    driver.bordertracking.begin
                  }
                if ~
                end
              }
            put
          }
        if
      }
        def
  }
%    \end{macrocode}
% \end{macro}
%
% \begin{macro}
%   {
%     driver.breaklink       ,
%     driver.breaklink.write ,
%     driver.count           ,
%     driver.currentrect
%   }
%   Dealing with link breaking itself has multiple stage. The first step is to
%   find the |Rect| entry in the dictionary, looping over key--value pairs.
%   The first line is handled first, adjusting the rectangle to stay inside the
%   text area. The second phase is a loop over the height of the bulk of the
%   link area, done on the basis of a number of baselines. Finally, the end of
%   the link area is tidied up, again from the boundary of the text area.
%    \begin{macrocode}
\@@_postscript_header:n
  {
    /driver.breaklink
      {
        pop ~
        counttomark ~ 2 ~ mod ~ 0 ~ eq
          {
            counttomark /driver.count ~ exch ~ def
              {
               driver.count ~ 0 ~ eq { exit } if ~
               counttomark ~ 2 ~ roll ~
               1 ~ index ~ /Rect ~ eq
                 {
                   dup ~ 4 ~ array ~ copy ~
                   dup ~ dup ~
                     1 ~ get ~
                     driver.outerbox ~ driver.rect.ht ~
                     driver.linkmargin ~ 2 ~ mul ~ add ~ sub ~
                     3 ~ exch ~ put ~
                   dup ~
                     driver.outerbox ~ 2 ~ get ~
                     driver.linkmargin ~ add ~ 
                     2 ~ exch ~ put ~
                   dup ~ dup ~
                     3 ~ get ~
                     driver.outerbox ~ driver.rect.ht ~
                     driver.linkmargin ~ 2 ~ mul ~ add ~ add ~
                     1 ~ exch ~ put
                   /driver.currentrect ~ exch ~  def ~
                   driver.breaklink.write
                     {
                       driver.currentrect ~
                       dup ~
                         driver.outerbox ~ 0 ~ get ~
                         driver.linkmargin ~ sub ~
                         0 ~ exch ~ put ~
                       dup ~
                         driver.outerbox ~ 2 ~ get ~
                         driver.linkmargin ~ add ~
                         2 ~ exch ~ put ~
                       dup ~ dup ~
                         1 ~ get ~
                         driver.baselineskip ~ add ~
                         1 ~ exch ~ put ~
                       dup ~ dup ~
                         3 ~ get ~
                         driver.baselineskip ~ add ~
                         3 ~ exch ~ put ~
                       /driver.currentrect ~ exch ~ def ~
                       driver.breaklink.write
                      }
                    1 ~ index ~ 3 ~ get ~
                    driver.linkmargin ~ 2 ~ mul ~ add ~
                    driver.outerbox ~ driver.rect.ht ~ add ~
                    2 ~ index ~ 1 ~ get ~ sub ~
                    driver.baselineskip ~ div ~ round ~ cvi ~ 1 ~ sub ~
                    exch ~
                  repeat ~
                  driver.currentrect ~
                  dup ~
                    driver.outerbox ~ 0 ~ get ~
                    driver.linkmargin ~ sub ~
                    0 ~ exch ~ put ~
                  dup ~ dup ~
                    1 ~ get ~
                    driver.baselineskip ~ add ~
                    1 ~ exch ~ put ~
                  dup ~ dup ~
                    3 ~ get ~
                    driver.baselineskip ~ add ~
                    3 ~ exch ~ put ~
                  dup ~ 2 ~ index ~ 2 ~ get ~  2 ~ exch ~ put
                  /driver.currentrect ~ exch ~ def ~
                  driver.breaklink.write ~
                  SDict /driver.pdfmark.good ~ false ~ put ~
                  exit
                }
                { driver.count ~ 2 ~ sub /driver.count ~ exch ~ def }
              ifelse
            }
          loop
        }
      if
      /ANN
    }
      def
    /driver.breaklink.write
      {
        counttomark ~ 1 ~ sub ~
        index /_objdef ~ eq
          {
            counttomark ~ -2 ~ roll ~
            dup ~ wcheck ~
              {
                readonly ~
                counttomark ~ 2 ~ roll
              }
              { pop ~ pop }
            ifelse
          }
        if ~
        counttomark ~ 1 ~ add ~ copy ~
        pop ~ driver.currentrect
        /ANN ~ pdfmark
      }
        def
  }
%    \end{macrocode}
% \end{macro}
%
% \begin{macro}
%   {
%     driver.pdfmark      ,
%     driver.pdfmark.good ,
%     driver.outerbox     ,
%     driver.baselineskip ,
%     driver.pdfmark.dict
%   }
%   The business end of breaking links starts by hooking into |pdfmarks|.
%   Unlike \pkg{hypdvips}, we avoid altering any links we have not created
%   by using a copy of the core |pdfmarks| function. Only mark types which
%   are known are altered. At present, this is purely |ANN| marks, which are
%   measured relative to the size of the baseline skip. If they are
%   more than one apparent line high, breaking is applied.
%    \begin{macrocode}
\@@_postscript_header:n
  {
    /driver.pdfmark
      {
        SDict /driver.pdfmark.good ~ true ~ put ~
        dup /ANN ~ eq
          {
            driver.pdfmark.store ~
            driver.pdfmark.dict ~
              begin ~
                Subtype /Link ~ eq ~
                currentdict /Rect ~ known ~ and ~
                SDict /driver.outerbox ~ known ~ and ~
                SDict /driver.baselineskip ~ known ~ and ~
                  {
                    Rect ~ 3 ~ get ~
                    driver.linkmargin ~ 2 ~ mul ~ add ~
                    driver.outerbox ~ driver.rect.ht ~ add ~
                    Rect ~ 1 ~ get ~ sub ~
                    driver.baselineskip ~ div ~ round ~ cvi ~ 0 ~ gt
                      { driver.breaklink }
                    if
                  }
                if ~
              end ~
            SDict /driver.outerbox ~ undef ~
            SDict /driver.baselineskip ~ undef ~
            currentdict /driver.pdfmark.dict ~ undef ~
          }
        if ~
        driver.pdfmark.good
          { pdfmark }
          { cleartomark }
        ifelse
      }
        def
    /driver.pdfmark.store
      {
        /driver.pdfmark.dict ~ 65534 ~ dict ~ def ~
        counttomark ~ 1 ~ add ~ copy ~
        pop
          {
            dup ~ mark ~ eq
              {
                pop ~
                exit
              }
              {
                driver.pdfmark.dict ~
                begin ~ def ~ end
              }
            ifelse
          }
        loop
    }
      def
  }
%    \end{macrocode}
% \end{macro}
%
% \begin{variable}{\l_@@_pdf_content_box}
%   The content of an annotation.
%    \begin{macrocode}
\box_new:N \l_@@_pdf_content_box
%    \end{macrocode}
% \end{variable}
%
% \begin{variable}{\l_@@_pdf_model_box}
%   For creating model sizing for links.
%    \begin{macrocode}
\box_new:N \l_@@_pdf_model_box
%    \end{macrocode}
% \end{variable}
%
% \begin{variable}{\g_@@_pdf_annotation_int}
%   Needed as objects which are not annotations could be created.
%    \begin{macrocode}
\int_new:N \g_@@_pdf_annotation_int
%    \end{macrocode}
% \end{variable}
%
% \begin{macro}{\driver_pdf_annotation:nnnn, \@@_pdf_annotation:nnnn}
% \begin{macro}{driver.llx, driver.lly, driver.urx, driver.ury}
%   Annotations are objects, but we track them separately. Notably, they are
%   not in the object data lists. Here, to get the co-ordinates of the
%   annotation, we need to have the data collected at the PostScript level.
%   That requires a bit of box trickery (effectively a \LaTeXe{} |picture|
%   of zero size). Once the data is collected, use it to set up the annotation
%   border. There is a split into two parts here to allow an easy way of
%   applying the Adobe Reader fix.
%    \begin{macrocode}
\cs_new_protected:Npn \driver_pdf_annotation:nnnn #1#2#3#4
  {
    \@@_pdf_annotation:nnnn {#1} {#2} {#3} {#4}
    \int_gincr:N \g_@@_pdf_object_int
    \int_gset_eq:NN \g_@@_pdf_annotation_int \g_@@_pdf_object_int
    \@@_pdfmark:x
      {
      
        /_objdef { driver.obj \int_use:N \g_@@_pdf_object_int }
        driver.rect ~
        #4 ~
        /ANN
      }
  }
\cs_new_protected:Npn \@@_pdf_annotation:nnnn #1#2#3#4
  {
    \box_move_down:nn {#3}
      { \hbox:n { \@@_postscript:n { driver.save.ll } } }
    \hbox:n {#4}
    \box_move_up:nn {#2}
      {
        \hbox:n
          {
            \tex_kern:D \dim_eval:n {#1} \scan_stop:
            \@@_postscript:n { driver.save.ur }
          }
      }
    \int_gincr:N \g_@@_pdf_object_int
    \int_gset_eq:NN \g_@@_pdf_annotation_int \g_@@_pdf_object_int
    \@@_pdfmark:x
      {
        /_objdef { driver.obj \int_use:N \g_@@_pdf_object_int }
        driver.rect
        /ANN
      }
  }
%    \end{macrocode}
% \end{macro}
% \end{macro}
%
% \begin{macro}[EXP]{\driver_pdf_annotation_last:}
%   Provide the last annotation we created: could get tricky of course if
%   other packages are loaded.
%    \begin{macrocode}
\cs_new:Npn \driver_pdf_annotation_last:
  { { driver.obj \int_use:N \g_@@_pdf_annotation_int } }
%    \end{macrocode}
% \end{macro}
%
% \begin{variable}{\g_@@_pdf_link_int}
%   To track annotations which are links.
%    \begin{macrocode}
\int_new:N \g_@@_pdf_link_int
%    \end{macrocode}
% \end{variable}
%
% \begin{variable}{\g_@@_pdf_link_dict_tl}
%   To pass information to the end-of-link function.
%    \begin{macrocode}
\tl_new:N \g_@@_pdf_link_dict_tl
%    \end{macrocode}
% \end{variable}
%
% \begin{variable}{\g_@@_pdf_link_sf_int}
%   Needed to save/restore space factor, which is needed to deal with the face
%   we need a box.
%    \begin{macrocode}
\int_new:N \g_@@_pdf_link_sf_int
%    \end{macrocode}
% \end{variable}
%
% \begin{variable}{\g_@@_pdf_link_math_bool}
%   Needed to save/restore math mode.
%    \begin{macrocode}
\bool_new:N \g_@@_pdf_link_math_bool
%    \end{macrocode}
% \end{variable}
%
% \begin{variable}{\g_@@_pdf_link_bool}
%   Track link formation: we cannot nest at all.
%    \begin{macrocode}
\bool_new:N \g_@@_pdf_link_bool
%    \end{macrocode}
% \end{variable}
%
% \begin{variable}{\l_@@_breaklink_pdfmark_tl}
%   Swappable content for link breaking.
%    \begin{macrocode}
\tl_new:N \l_@@_breaklink_pdfmark_tl
\tl_set:Nn \l_@@_breaklink_pdfmark_tl { pdfmark }
%    \end{macrocode}
% \end{variable}
%
% \begin{macro}{\@@_breaklink_postscript:n}
%   To allow dropping material unless link breaking is active.
%    \begin{macrocode}
\cs_new_protected:Npn \@@_breaklink_postscript:n #1 { }
%    \end{macrocode}
% \end{macro}
%
% \begin{macro}{\@@_breaklink_usebox:N}
%   Swappable box unpacking or use.
%    \begin{macrocode}
\cs_new_eq:NN \@@_breaklink_usebox:N \box_use:N
%    \end{macrocode}
% \end{macro}
%
% \begin{macro}{\driver_pdf_link_begin_goto:nnw, \driver_pdf_link_begin_user:nnw}
% \begin{macro}{\@@_pdf_link:nw, \@@_pdf_link_aux:nw}
% \begin{macro}{\driver_pdf_link_end:, \@@_pdf_link_end:}
% \begin{macro}{\@@_pdf_link_minima:}
% \begin{macro}{\@@_pdf_link_outerbox:n}
% \begin{macro}{\@@_pdf_link_sf_save:, \@@_pdf_link_sf_restore:}
% \begin{macro}
%   {
%     driver.linkdp.pad      ,
%     driver.linkht.pad      ,
%     driver.llx, driver.lly ,
%     driver.ury, driver.ury ,
%     driver.link.dict       ,
%     driver.outerbox        ,
%     driver.baselineskip
%   }
%   Links are crated like annotations but with dedicated code to allow for
%   adjusting the size of the rectangle. In contrast to \pkg{hyperref}, we
%   grab the link content as a box which can then unbox: this allows the same
%   interface as for \pdfTeX{}.
%
%   Taking the idea of |evenboxes| from \pkg{hypdvips}, we implement a minimum
%   box height and depth for link placement. This means that \enquote{underlining}
%   with a hyperlink will generally give an even appearance. However, to ensure
%   that the full content is always above the link border, we do not allow
%   this to be negative (contrast \pkg{hypdvips} approach). The result should
%   be similar to \pdfTeX{} in the vast majority of foreseeable cases.
%
%   The object number for a link is saved separately from the rest of the
%   dictionary as this allows us to insert it just once, at either an
%   unbroken link or only in the first line of a broken one. That makes the
%   code clearer but also avoids a low-level PostScript error with the code
%   as taken from \pkg{hypdvips}.
%
%   Getting the outer dimensions of the text area may be better using a two-pass
%   approach and |\tex_savepos:D|. That plus format mode are still to re-examine.
%    \begin{macrocode}
\cs_new_protected:Npn \driver_pdf_link_begin_goto:nnw #1#2
  { \@@_pdf_link_begin:nw { #1 /Subtype /Link /A << /S /GoTo /D ( #2 ) >> } }
\cs_new_protected:Npn \driver_pdf_link_begin_user:nnw #1#2
  { \@@_pdf_link_begin:nw {#1#2} }
\cs_new_protected:Npn \@@_pdf_link_begin:nw #1
  {
    \bool_if:NF \g_@@_pdf_link_bool
      { \@@_pdf_link_begin_aux:nw {#1} }
  }
\cs_new_protected:Npn \@@_pdf_link_begin_aux:nw #1
  {
    \bool_gset_true:N \g_@@_pdf_link_bool
    \@@_postscript:n
      { /driver.link.dict ( #1 ) def }
    \tl_gset:Nn \g_@@_pdf_link_dict_tl {#1}
    \@@_pdf_link_sf_save:
    \mode_if_math:TF
      { \bool_gset_true:N \g_@@_pdf_link_math_bool }
      { \bool_gset_false:N \g_@@_pdf_link_math_bool }
    \hbox_set:Nw \l_@@_pdf_content_box
      \@@_pdf_link_sf_restore:
      \bool_if:NT \g_@@_pdf_link_math_bool
        { \c_math_toggle_token }
  }
\cs_new_protected:Npn \driver_pdf_link_end:
  {
    \bool_if:NT \g_@@_pdf_link_bool
      { \@@_pdf_link_end: }
  }
\cs_new_protected:Npn \@@_pdf_link_end:
  {
      \bool_if:NT \g_@@_pdf_link_math_bool
        { \c_math_toggle_token }
      \@@_pdf_link_sf_save:
    \hbox_set_end:
    \@@_pdf_link_minima:
    \hbox_set:Nn \l_@@_pdf_model_box { Gg }
    \exp_args:Nx \@@_driver_link_outerbox:n
      {
%<*initex>
         \l_galley_total_left_margin_dim
%</initex>
%<*package>
         \int_if_odd:nTF { \value { page } }
           { \oddsidemargin }
           { \evensidemargin }
%</package>
      }
    \box_move_down:nn { \box_dp:N \l_@@_pdf_content_box }
      { \hbox:n { \@@_postscript:n { driver.save.linkll } } }
    \@@_breaklink_postscript:n { driver.bordertracking.begin }
    \@@_breaklink_usebox:N \l_@@_pdf_content_box
    \@@_breaklink_postscript:n { driver.bordertracking.end }
    \box_move_up:nn { \box_ht:N \l_@@_pdf_content_box }
      {
        \hbox:n
          { \@@_postscript:n { driver.save.linkur } }
      }
    \int_gincr:N \g_@@_pdf_object_int
    \int_gset_eq:NN \g_@@_pdf_link_int \g_@@_pdf_object_int
    \@@_postscript:x
      {
        mark
        /_objdef { driver.obj \int_use:N \g_@@_pdf_link_int }
        \g_@@_pdf_link_dict_tl \c_space_tl
        driver.rect
        /ANN ~ \l_@@_breaklink_pdfmark_tl
      }
    \@@_pdf_link_sf_restore:
    \bool_gset_false:N \g_@@_pdf_link_bool
  }
\cs_new_protected:Npn \@@_pdf_link_minima:
  {
    \hbox_set:Nn \l_@@_pdf_model_box { Gg }
    \@@_postscript:x
      {
        /driver.linkdp.pad ~
          \dim_to_decimal:n
            {
              \dim_max:nn
                {
                    \box_dp:N \l_@@_pdf_model_box
                  - \box_dp:N \l_@@_pdf_content_box
                }
                { 0pt }
            } ~
              driver.pt.dvi ~ def
        /driver.linkht.pad ~
          \dim_to_decimal:n
            {
              \dim_max:nn
                {
                    \box_ht:N \l_@@_pdf_model_box
                  - \box_ht:N \l_@@_pdf_content_box
                }
                { 0pt }
            } ~
              driver.pt.dvi ~ def
      }
  }
\cs_new_protected:Npn \@@_driver_link_outerbox:n #1
  {
    \@@_postscript:x
      {
        /driver.outerbox
          [
            \dim_to_decimal:n {#1} ~
            \dim_to_decimal:n { -\box_dp:N \l_@@_pdf_model_box } ~
%<*initex>
            \dim_to_decimal:n { #1 + \l_galley_text_width_dim } ~
%</initex>
%<*package>
            \dim_to_decimal:n { #1 + \textwidth } ~
%</package>
            \dim_to_decimal:n { \box_ht:N \l_@@_pdf_model_box }
          ]
          [ exch { driver.pt.dvi } forall ] def
        /driver.baselineskip ~
          \dim_to_decimal:n { \tex_baselineskip:D } ~ dup ~ 0 ~ gt
            { driver.pt.dvi ~ def }
            { pop ~ pop }
          ifelse 
      }
  }
\cs_new_protected:Npn \@@_pdf_link_sf_save:
  {
    \int_gset:Nn \g_@@_pdf_link_sf_int
      {
        \mode_if_horizontal:TF
          { \tex_spacefactor:D }
          { 0 }
      }
  }
\cs_new_protected:Npn \@@_pdf_link_sf_restore:
  {
    \mode_if_horizontal:T
      {
        \int_compare:nNnT \g_@@_pdf_link_sf_int > { 0 }
          { \int_set_eq:NN \tex_spacefactor:D \g_@@_pdf_link_sf_int }
      }
  }
%    \end{macrocode}
% \end{macro}
% \end{macro}
% \end{macro}
% \end{macro}
% \end{macro}
% \end{macro}
% \end{macro}
%
% \begin{macro}{\@makecol@hook}
%   Hooks to allow link breaking: something will be needed in format mode
%   at some stage. At present this code is disabled as there is an open
%   question about the name of the hook: to be resolved at the \LaTeXe{}
%   end.
%    \begin{macrocode}
%<*package>
\use_none:n
  {
    \cs_if_exist:NT \@makecol@hook
      {
        \tl_put_right:Nn \@makecol@hook
          {
            \box_if_empty:NF \@cclv
              {
                \vbox_set:Nn \@cclv
                  {
                    \@@_postscript:n
                      {
                        driver.globaldict /driver.brokenlink.rect ~ known
                          { driver.bordertracking.continue }
                        if
                      }
                    \vbox_unpack_drop:N \@cclv
                    \@@_postscript:n
                      { driver.bordertracking.endpage }
                  }
              }
          }
        \tl_set:Nn \l_@@_breaklink_pdfmark_tl { driver.pdfmark }
        \cs_set_eq:NN \@@_breaklink_postscript:n \@@_postscript:n
        \cs_set_eq:NN \@@_breaklink_usebox:N \hbox_unpack:N
      }
  }
%</package>
%    \end{macrocode}
% \end{macro}
%
% \begin{macro}{\driver_pdf_link_last:}
%   The same as annotations, but with a custom integer.
%    \begin{macrocode}
\cs_new:Npn \driver_pdf_link_last:
  { { driver.obj \int_use:N \g_@@_pdf_link_int } }
%    \end{macrocode}
% \end{macro}
%
% \begin{macro}{\driver_pdf_link_margin:n}
%   Convert to big points and pass to PostScript.
%    \begin{macrocode}
\cs_new_protected:Npn \driver_pdf_link_margin:n #1
  {
    \@@_postscript:x
      {
        /driver.linkmargin { \dim_to_decimal:n {#1} ~ driver.pt.dvi } def
      }
  }
%    \end{macrocode}
% \end{macro}
%
% \begin{macro}{\driver_pdf_destination:nn, \driver_pdf_destination_rectangle:nn}
%   Here, we need to turn the zoom into a scale. We also need to know where
%   the current anchor point actually is: worked out in PostScript. For the
%   rectangle version, we have a bit more PostScript: we need two points.
%    \begin{macrocode}
\cs_new_protected:Npn \driver_pdf_destination:nn #1#2
  {
    \@@_postscript:n { driver.dest.anchor }
    \@@_pdfmark:x
      {
        /View
        [
          \str_case:nnF {#2}
            {
              { xyz }   { /XYZ ~ driver.dest.point ~ null }
              { fit }   { /Fit }
              { fitb }  { /FitB }
              { fitbh } { /FitBH ~ driver.dest.y }
              { fitbv } { /FitBV ~ driver.dest.x }
              { fith }  { /FitH ~ driver.dest.y }
              { fitv }  { /FitV ~ driver.dest.x }
            }
            {
              /XYZ ~ driver.dest.point ~ \fp_eval:n { (#2) / 100 }
            }
        ]
        /Dest ( \exp_not:n {#1} ) cvn
        /DEST
      }
  }
\cs_new_protected:Npn \driver_pdf_destination_rectangle:nn #1#2
  {
    \group_begin:
      \hbox_set:Nn \l_@@_internal_box {#2}
      \box_move_down:nn
        { \box_dp:N \l_@@_internal_box }
        { \hbox:n { \@@_postscript:n { driver.save.ll } } }
      \box_use:N \l_@@_internal_box
      \box_move_up:nn
        { \box_ht:N \l_@@_internal_box }
        { \hbox:n { \@@_postscript:n { driver.save.ur } } }
      \@@_pdfmark:n
        {
          /View
          [
            /FitR ~
              driver.llx ~ driver.lly ~ driver.dest2device ~
              driver.urx ~ driver.ury ~ driver.dest2device
          ]
          /Dest ( #1 ) cvn
          /DEST
        }
    \group_end:    
  }
%    \end{macrocode}
% \end{macro}
%
% \subsubsection{Structure}
%
% \begin{macro}{\driver_pdf_compresslevel:n}
% \begin{macro}{\driver_pdf_compress_objects:n}
%   These are all no-ops.
%    \begin{macrocode}
\cs_new_protected:Npn \driver_pdf_compresslevel:n #1 { }
\cs_new_protected:Npn \driver_pdf_compress_objects:n #1 { }
%    \end{macrocode}
% \end{macro}
% \end{macro}
%
% \begin{macro}
%   {\driver_pdf_version_major_gset:n, \driver_pdf_version_minor_gset:n}
%   Data not available!
%    \begin{macrocode}
\cs_new_protected:Npn \driver_pdf_version_major_gset:n #1 { }
\cs_new_protected:Npn \driver_pdf_version_minor_gset:n #1 { }
%    \end{macrocode}
% \end{macro}
%
% \begin{macro}[EXP]{\driver_pdf_version_major:, \driver_pdf_version_minor:}
%   Data not available!
%    \begin{macrocode}
\cs_new:Npn \driver_pdf_version_major: { -1 }
\cs_new:Npn \driver_pdf_version_minor: { -1 }
%    \end{macrocode}
% \end{macro}
%
%    \begin{macrocode}
%</dvips>
%    \end{macrocode}
%
% \subsection{\texttt{pdfmode} driver}
%
%    \begin{macrocode}
%<*pdfmode>
%    \end{macrocode}
%
% \subsubsection{Annotations}
%
% \begin{macro}{\driver_pdf_annotation:nnnn}
%   Simply pass the raw data through, just dealing with evaluation of dimensions.
%    \begin{macrocode}
\cs_new_protected:Npx \driver_pdf_annotation:nnnn #1#2#3#4
  {
    \cs_if_exist:NTF \tex_pdfextension:D
      { \tex_pdfextension:D annot ~ }
      { \tex_pdfannot:D }
      width  ~ \exp_not:N \dim_eval:n {#1} ~
      height ~ \exp_not:N \dim_eval:n {#2} ~
      depth  ~ \exp_not:N \dim_eval:n {#3} ~
      {#4}
  }
%    \end{macrocode}
% \end{macro}
%
% \begin{macro}[EXP]{\driver_pdf_annotation_last:}
%   A tiny amount of extra data gets added here.
%    \begin{macrocode}
\cs_new:Npx \driver_pdf_annotation_last:
  {
    \exp_not:N \tex_the:D 
    \cs_if_exist:NTF \tex_pdffeedback:D
      { \exp_not:N \tex_pdffeedback:D annot ~ }
      { \exp_not:N \tex_pdflastannot:D }
      0 ~ R
  }
%    \end{macrocode}
% \end{macro}
%
% \begin{macro}
%   {\driver_pdf_link_begin_goto:nnw, \driver_pdf_link_begin_user:nnw}
% \begin{macro}{\@@_pdf_link_begin:nnnw}
% \begin{macro}{\driver_pdf_link_end:}
%   Links are all created using the same internals.
%    \begin{macrocode}
\cs_new_protected:Npn \driver_pdf_link_begin_goto:nnw #1#2
  { \@@_pdf_link_begin:nnnw {#1} { goto~name } {#2} }
\cs_new_protected:Npn \driver_pdf_link_begin_user:nnw #1#2
  { \@@_pdf_link_begin:nnnw {#1} { user } {#2} }
\cs_new_protected:Npx \@@_pdf_link_begin:nnnw #1#2#3
  {
    \cs_if_exist:NTF \tex_pdfextension:D
      { \tex_pdfextension:D startlink ~ }
      { \tex_pdfstartlink:D }
        attr {#1}
        #2 {#3}
  }
\cs_new_protected:Npx \driver_pdf_link_end:
  {
    \cs_if_exist:NTF \tex_pdfextension:D
      { \tex_pdfextension:D endlink \scan_stop: }
      { \tex_pdfendlink:D }
  }
%    \end{macrocode}
% \end{macro}
% \end{macro}
% \end{macro}
%
% \begin{macro}{\driver_pdf_link_last:}
%   Formatted for direct use.
%    \begin{macrocode}
\cs_new:Npx \driver_pdf_link_last:
  {
    \exp_not:N \tex_the:D
    \cs_if_exist:NTF \tex_pdffeedback:D
      { \exp_not:N \tex_pdffeedback:D lastlink \scan_stop: }
      { \exp_not:N \tex_pdflastlink:D }
      ~ 0 ~ R
  }
%    \end{macrocode}
% \end{macro}
%
% \begin{macro}{\driver_pdf_link_margin:n}
%   A simple task: pass the data to the primitive.
%    \begin{macrocode}
\cs_new_protected:Npx \driver_pdf_link_margin:n #1
  {
    \cs_if_exist:NTF \tex_pdfvariable:D
      { \exp_not:N \tex_pdfvariable:D linkmargin }
      { \exp_not:N \tex_pdflinkmargin:D }
        \exp_not:N \dim_eval:n {#1} \scan_stop:
  }
%    \end{macrocode}
% \end{macro}
%
% \begin{macro}{\driver_pdf_destination:nn, \driver_pdf_destination_rectangle:nn}
%   A simple task: pass the data to the primitive. The |\scan_stop:| deals
%   with the danger of an unterminated keyword. The zoom given here is a
%   percentage, but we need to pass it as \emph{per mille}. The rectangle
%   version is also easy as everything is build in.
%    \begin{macrocode}
\cs_new_protected:Npx \driver_pdf_destination:nn #1#2
  {
    \cs_if_exist:NTF \tex_pdfextension:D
      { \exp_not:N \tex_pdfextension:D dest ~ }
      { \exp_not:N \tex_pdfdest:D }
        name {#1}
        \exp_not:N \str_case:nnF {#2}
          {
            { xyz }   { xyz }
            { fit }   { fit }
            { fitb }  { fitb }
            { fitbh } { fitbh }
            { fitbv } { fitbv }
            { fith }  { fith }
            { fitv }  { fitv }
          }
          { xyz ~ zoom \exp_not:N \fp_eval:n { #2 * 10 } }
        \scan_stop:
  }
\cs_new_protected:Npx \driver_pdf_destination_rectangle:nn #1#2
  {
    \group_begin:
      \hbox_set:Nn \l_@@_internal_box {#2}
     \cs_if_exist:NTF \tex_pdfextension:D
      { \exp_not:N \tex_pdfextension:D dest ~ }
      { \exp_not:N \tex_pdfdest:D }  
      name {#1}
      fitr ~
        width  \exp_not:N \box_wd:N \l_@@_internal_box
        height \exp_not:N \box_ht:N \l_@@_internal_box
        depth  \exp_not:N \box_dp:N \l_@@_internal_box 
      \box_use:N \l_@@_internal_box
    \group_end:    
  }
%    \end{macrocode}
% \end{macro}
%
% \subsubsection{Catalogue entries}
%
% \begin{macro}{\driver_pdf_catalog_gput:nn, \driver_pdf_info_gput:nn}
%    \begin{macrocode}
\cs_new_protected:Npx \driver_pdf_catalog_gput:nn #1#2
  {
    \cs_if_exist:NTF \tex_pdfextension:D
      { \tex_pdfextension:D catalog }
      { \tex_pdfcatalog:D }
        { / #1 ~ #2 }
  }
\cs_new_protected:Npx \driver_pdf_info_gput:nn #1#2
  {
    \cs_if_exist:NTF \tex_pdfextension:D
      { \tex_pdfextension:D info }
      { \tex_pdfinfo:D }
        { / #1 ~ #2 }
  }
%    \end{macrocode}
% \end{macro}
%
% \subsubsection{Objects}
%
% \begin{variable}{\g_@@_pdf_object_prop}
%   For tracking objects to allow finalisation.
%    \begin{macrocode}
\prop_new:N \g_@@_pdf_object_prop
%    \end{macrocode}
% \end{variable}
%
% \begin{macro}{\driver_pdf_object_new:nn}
% \begin{macro}[EXP]{\driver_pdf_object_ref:n}
%   Declaring objects means reserving at the PDF level plus starting
%   tracking.
%    \begin{macrocode}
\group_begin:
  \cs_set_protected:Npn \@@_tmp:w #1#2
    {
      \cs_new_protected:Npx \driver_pdf_object_new:nn ##1##2
        {
          #1 reserveobjnum ~
          \int_const:cn
            { c_@@_pdf_object_ \exp_not:N \tl_to_str:n {##1} _int }
            {#2}
          \prop_gput:Nnn \exp_not:N \g_@@_pdf_object_prop {##1} {##2}
        }
    }
  \cs_if_exist:NTF \tex_pdfextension:D
    {
      \@@_tmp:w
        { \tex_pdfextension:D obj ~ }
        { \exp_not:N \tex_pdffeedback:D lastobj }
    }
    { \@@_tmp:w { \tex_pdfobj:D } { \tex_pdflastobj:D } }
\group_end:
\cs_new:Npn \driver_pdf_object_ref:n #1
  { \int_use:c { c_@@_pdf_object_ \tl_to_str:n {#1} _int } ~ 0 ~ R }
%    \end{macrocode}
% \end{macro}
% \end{macro}
%
% \begin{macro}{\driver_pdf_object_write:nn,\driver_pdf_object_write:nx}
% \begin{macro}[EXP]{\@@_exp_not_i:nn, \@@_exp_not_ii:nn}
%   Writing the data needs a little information about the structure of the
%   object.
%    \begin{macrocode}
\group_begin:
  \cs_set_protected:Npn \@@_tmp:w #1
    {
      \cs_new_protected:Npn \driver_pdf_object_write:nn ##1##2
        {
          \tex_immediate:D #1 useobjnum ~
          \int_use:c
            { c_@@_pdf_object_ \tl_to_str:n {##1} _int }
            \str_case_e:nn
              { \prop_item:Nn \g_@@_pdf_object_prop {##1} }
              {
                { array } { { [ ~ \exp_not:n {##2} ~ ] } }
                { dict }  { { << ~ \exp_not:n {##2} ~ >> } }
                { fstream }
                  {
                    stream ~ attr ~ { \@@_exp_not_i:nn ##2 } ~
                      file ~ { \@@_exp_not_ii:nn ##2 }
                  }
                { stream }
                  {
                    stream ~ attr ~ { \@@_exp_not_i:nn ##2 } ~ 
                      { \@@_exp_not_ii:nn ##2 }
                  }
              }
        }
    }
  \cs_if_exist:NTF \tex_pdfextension:D
    { \@@_tmp:w { \tex_pdfextension:D obj ~ } }
    { \@@_tmp:w { \tex_pdfobj:D } }
\group_end:
\cs_generate_variant:Nn \driver_pdf_object_write:nn { nx }
\cs_new:Npn \@@_exp_not_i:nn #1#2 { \exp_not:n {#1} }
\cs_new:Npn \@@_exp_not_ii:nn #1#2 { \exp_not:n {#2} }
%    \end{macrocode}
% \end{macro}
% \end{macro}
%
% \begin{macro}{\driver_pdf_object_now:nn, \driver_pdf_object_now:nx}
%   Much like writing, but direct creation.
%    \begin{macrocode}
\group_begin:
  \cs_set_protected:Npn \@@_tmp:w #1
    {
      \cs_new_protected:Npn \driver_pdf_object_now:nn ##1##2
        {
          \tex_immediate:D #1
            \str_case:nn
              {##1}
              {
                { array } { { [ ~ \exp_not:n {##2} ~ ] } }
                { dict }  { { << ~ \exp_not:n {##2} ~ >> } }
                { fstream }
                  {
                    stream ~ attr ~ { \@@_exp_not_i:nn ##2 } ~
                      file ~ { \@@_exp_not_ii:nn ##2 }
                  }
                { stream }
                  {
                    stream ~ attr ~ { \@@_exp_not_i:nn ##2 } ~ 
                      { \@@_exp_not_ii:nn ##2 }
                  }
              }
        }
    }
  \cs_if_exist:NTF \tex_pdfextension:D
    { \@@_tmp:w { \tex_pdfextension:D obj ~ } }
    { \@@_tmp:w { \tex_pdfobj:D } }
\group_end:
\cs_generate_variant:Nn \driver_pdf_object_now:nn { nx }
%    \end{macrocode}
% \end{macro}
%
% \begin{macro}[EXP]{\driver_pdf_object_last:}
%   Much like annotation.
%    \begin{macrocode}
\cs_new:Npx \driver_pdf_object_last:
  {
    \exp_not:N \tex_the:D 
    \cs_if_exist:NTF \tex_pdffeedback:D
      { \exp_not:N \tex_pdffeedback:D lastobj }
      { \exp_not:N \tex_pdflastobj:D }
      0 ~ R
  }
%    \end{macrocode}
% \end{macro}
%
% \subsubsection{Structure}
%
% \begin{macro}{\driver_pdf_compresslevel:n}
% \begin{macro}{\driver_pdf_compress_objects:n}
% \begin{macro}{\@@_pdf_objcompresslevel:n}
%   Simply pass data to the engine.
%    \begin{macrocode}
\cs_new_protected:Npx \driver_pdf_compresslevel:n #1
  {
    \exp_not:N \tex_global:D
    \cs_if_exist:NTF \tex_pdfcompresslevel:D
      { \tex_pdfcompresslevel:D }
      { \tex_pdfvariable:D compresslevel }
      \exp_not:N \int_value:w \exp_not:N \int_eval:n {#1} \scan_stop:
  }
\cs_new_protected:Npn \driver_pdf_compress_objects:n #1
  {
    \str_if_eq:nnTF {#1} { true }
      { \@@_pdf_objcompresslevel:n { 2 } }
      { \@@_pdf_objcompresslevel:n { 0 } }
  }
\cs_new_protected:Npx \@@_pdf_objcompresslevel:n #1
  {
    \exp_not:N \tex_global:D
    \cs_if_exist:NTF \tex_pdfobjcompresslevel:D
      { \tex_pdfobjcompresslevel:D }
      { \tex_pdfvariable:D objcompresslevel }
      #1 \scan_stop:
  }
%    \end{macrocode}
% \end{macro}
% \end{macro}
% \end{macro}
%
% \begin{macro}
%   {\driver_pdf_version_major_gset:n, \driver_pdf_version_minor_gset:n}
%   At present, we don't have a primitive for the major version in \pdfTeX{},
%   but we anticipate one \ldots
%    \begin{macrocode}
\cs_new_protected:Npx \driver_pdf_version_major_gset:n #1
  {
    \cs_if_exist:NTF \tex_pdfvariable:D
      {
        \int_compare:nNnT \tex_luatexversion:D > { 106 }
          {
            \exp_not:N \tex_global:D \tex_pdfvariable:D majorversion
              \exp_not:N \int_eval:n {#1} \scan_stop:
          }
      }
      {
        \cs_if_exist:NT \tex_pdfmajorversion:D
          {
            \exp_not:N \tex_global:D \tex_pdfmajorversion:D
              \exp_not:N \int_eval:n {#1} \scan_stop:
          }
      }
  }
\cs_new_protected:Npx \driver_pdf_version_minor_gset:n #1
  {
    \exp_not:N \tex_global:D
    \cs_if_exist:NTF \tex_pdfminorversion:D
      { \exp_not:N \tex_pdfminorversion:D }
      { \tex_pdfvariable:D minorversion }
        \exp_not:N \int_eval:n {#1} \scan_stop:
  }
%    \end{macrocode}
% \end{macro}
%
% \begin{macro}[EXP]{\driver_pdf_version_major:, \driver_pdf_version_minor:}
%   At present, we don't have a primitive for the major version!
%    \begin{macrocode}
\cs_new:Npx \driver_pdf_version_major:
  {
    \cs_if_exist:NTF \tex_pdfvariable:D
      {
        \int_compare:nNnTF \tex_luatexversion:D > { 106 }
          { \exp_not:N \tex_the:D \tex_pdfvariable:D majorversion }
          { 1 }
      }
      {
        \cs_if_exist:NTF \tex_pdfmajorversion:D
          { \exp_not:N \tex_the:D \tex_pdfmajorversion:D }
          { 1 }
      }
  }
\cs_new:Npx \driver_pdf_version_minor:
  {
    \exp_not:N \tex_the:D
    \cs_if_exist:NTF \tex_pdfminorversion:D
      { \exp_not:N \tex_pdfminorversion:D }
      { \tex_pdfvariable:D minorversion }
  }
%    \end{macrocode}
% \end{macro}
%
%    \begin{macrocode}
%</pdfmode>
%    \end{macrocode}
%
% \subsection{\texttt{dvipdfmx} driver}
%
%    \begin{macrocode}
%<*dvipdfmx|xdvipdfmx>
%    \end{macrocode}
%
% \begin{macro}{\@@_pdf:n, \@@_pdf:x}
%   A generic function for the driver PDF specials: used where we can.
%    \begin{macrocode}
\cs_new_protected:Npx \@@_pdf:n #1
  { \@@_literal:n { pdf: #1 } }
\cs_generate_variant:Nn \@@_pdf:n { x }
%    \end{macrocode}
% \end{macro}
%
% \subsubsection{Catalogue entries}
%
% \begin{macro}{\driver_pdf_catalog_gput:nn, \driver_pdf_info_gput:nn}
%    \begin{macrocode}
\cs_new_protected:Npn \driver_pdf_catalog_gput:nn #1#2
  { \@@_pdf:n { put ~ @catalog << /#1 ~ #2 >> } }
\cs_new_protected:Npn \driver_pdf_info_gput:nn #1#2
  { \@@_pdf:n { docinfo << /#1 ~ #2 >> } }
%    \end{macrocode}
% \end{macro}
%
% \subsubsection{Objects}
%
% \begin{variable}{\g_@@_pdf_object_int, \g_@@_pdf_object_prop}
%   For tracking objects to allow finalisation.
%    \begin{macrocode}
\int_new:N \g_@@_pdf_object_int
\prop_new:N \g_@@_pdf_object_prop
%    \end{macrocode}
% \end{variable}
%
% \begin{macro}{\driver_pdf_object_new:nn}
% \begin{macro}[EXP]{\driver_pdf_object_ref:n}
%   Objects are tracked at the macro level, but we don't have to do anything
%   at this stage.
%    \begin{macrocode}
\cs_new_protected:Npn \driver_pdf_object_new:nn #1#2
  {
    \int_gincr:N \g_@@_pdf_object_int
    \int_const:cn
      { g_@@_pdf_object_ \tl_to_str:n {#1} _int }
      { \g_@@_pdf_object_int }
    \prop_gput:Nnn \g_@@_pdf_object_prop {#1} {#2}
  }
\cs_new:Npn \driver_pdf_object_ref:n #1
  { @driver.obj \int_use:c { g_@@_pdf_object_ \tl_to_str:n {#1} _int } }
%    \end{macrocode}
% \end{macro}
% \end{macro}
%
% \begin{macro}{\driver_pdf_object_write:nn, \driver_pdf_object_write:nx}
% \begin{macro}{\@@_pdf_object_write:nnn}
% \begin{macro}
%   {
%     \@@_pdf_object_write_array:nn   ,
%     \@@_pdf_object_write_dict:nn    ,
%     \@@_pdf_object_write_fstream:nn ,
%     \@@_pdf_object_write_stream:nn
%   }
% \begin{macro}{\@@_pdf_object_write_stream:nnnn}
%   This is where we choose the actual type.
%    \begin{macrocode}
\cs_new_protected:Npn \driver_pdf_object_write:nn #1#2
  {
    \exp_args:Nx \@@_pdf_object_write:nnn
      { \prop_item:Nn \g_@@_pdf_object_prop {#1} } {#1} {#2}
  }
\cs_generate_variant:Nn \driver_pdf_object_write:nn { nx }
\cs_new_protected:Npn \@@_pdf_object_write:nnn #1#2#3
  {
    \use:c { @@_pdf_object_write_ #1 :nn }
      { \driver_pdf_object_ref:n {#2} } {#3}
  }
\cs_new_protected:Npn \@@_pdf_object_write_array:nn #1#2
  {
    \@@_pdf:x
      { obj ~ #1 ~ [ ~ \exp_not:n {#2} ~ ] }
  }
\cs_new_protected:Npn \@@_pdf_object_write_dict:nn #1#2
  {
    \@@_pdf:x
      { obj ~ #1 ~ << ~ \exp_not:n {#2} ~ >> }
  }
\cs_new_protected:Npn \@@_pdf_object_write_fstream:nn #1#2
  { \@@_pdf_object_write_stream:nnnn { f } {#1} #2 }
\cs_new_protected:Npn \@@_pdf_object_write_stream:nn #1#2
  { \@@_pdf_object_write_stream:nnnn { } {#1} #2 }
\cs_new_protected:Npn \@@_pdf_object_write_stream:nnnn #1#2#3#4
  {
    \@@_pdf:x
      {
        #1 stream ~ #2 ~
          ( \exp_not:n {#4} ) ~ << \exp_not:n {#3} >>
      }
  }
%    \end{macrocode}
% \end{macro}
% \end{macro}
% \end{macro}
% \end{macro}
%
% \begin{macro}{\driver_pdf_object_now:nn, \driver_pdf_object_now:nx}
%   No anonymous objects with \texttt{dvipdfmx} so we have to give an
%   object name.
%    \begin{macrocode}
\cs_new_protected:Npn \driver_pdf_object_now:nn #1#2
  {
    \int_gincr:N \g_@@_pdf_object_int
    \exp_args:Nnx \use:c { @@_pdf_object_write_ #1 :nn }
      { @driver.obj \int_use:N \g_@@_pdf_object_int }
      {#2}
  }
\cs_generate_variant:Nn \driver_pdf_object_now:nn { nx }
%    \end{macrocode}
% \end{macro}
%
% \begin{macro}{\driver_pdf_object_last:}
%    \begin{macrocode}
\cs_new:Npn \driver_pdf_object_last:
 { @driver.obj \int_use:N \g_@@_pdf_object_int }
%    \end{macrocode}
% \end{macro}
%
% \subsubsection{Annotations}
%
% \begin{variable}{\g_@@_landscape_bool}
%   There is a bug in \texttt{(x)dvipdfmx} which means annotations do
%   not rotate. As such, we need to know if landscape is active.
%    \begin{macrocode}
\bool_new:N \g_@@_landscape_bool
%<*package>
\AtBeginDocument
  {
    \cs_if_exist:NT \landscape
      {
        \tl_put_right:Nn \landscape
          { \bool_gset_true:N \g_@@_landscape_bool }
        \tl_put_left:Nn \endlandscape
          { \bool_gset_false:N \g_@@_landscape_bool }
      }
  }
%</package>
%    \end{macrocode}
% \end{variable}
%
% \begin{variable}{\g_@@_pdf_annotation_int}
%   Needed as objects which are not annotations could be created.
%    \begin{macrocode}
\int_new:N \g_@@_pdf_annotation_int
%    \end{macrocode}
% \end{variable}
%
% \begin{macro}{\driver_pdf_annotation:nnnn, \@@_pdf_annotation:nnnn}
%   Simply pass the raw data through, just dealing with evaluation of dimensions.
%   The only wrinkle is landscape: we have to adjust by hand.
%    \begin{macrocode}
\cs_new_protected:Npn \driver_pdf_annotation:nnnn #1#2#3#4
  {
    \bool_if:NTF \g_@@_landscape_bool
      {
         \box_move_up:nn {#2}
           {
             \vbox:n
               {
                 \@@_pdf_annotation:nnnn
                   { #2 + #3 } {#1} { 0pt } {#4}
               }
           }
      }
      { \@@_pdf_annotation:nnnn {#1} {#2} {#3} {#4} }
  }
\cs_new_protected:Npn \@@_pdf_annotation:nnnn #1#2#3#4
  {
    \int_gincr:N \g_@@_pdf_object_int
    \int_gset_eq:NN \g_@@_pdf_annotation_int \g_@@_pdf_object_int
    \@@_pdf:x
      {
        ann ~ @driver.obj \int_use:N \g_@@_pdf_object_int \c_space_tl
        width  ~ \dim_eval:n {#1} ~
        height ~ \dim_eval:n {#2} ~
        depth  ~ \dim_eval:n {#3} ~
        << #4 >>
      }
  }
%    \end{macrocode}
% \end{macro}
%
% \begin{macro}{\driver_pdf_annotation_last:}
%    \begin{macrocode}
\cs_new:Npn \driver_pdf_annotation_last:
 { @driver.obj \int_use:N \g_@@_pdf_annotation_int }
%    \end{macrocode}
% \end{macro}
%
% \begin{macro}
%   {\driver_pdf_link_begin_goto:nnw, \driver_pdf_link_begin_user:nnw}
% \begin{macro}{\@@_pdf_link_begin:n}
% \begin{macro}{\driver_pdf_link_end:}
%   All created using the same internals.
%    \begin{macrocode}
\cs_new_protected:Npn \driver_pdf_link_begin_goto:nnw #1#2
  { \@@_pdf_link_begin:n { #1 /Subtype /Link /A << /S /GoTo /D ( #2 ) >> } }
\cs_new_protected:Npn \driver_pdf_link_begin_user:nnw #1#2
  { \@@_pdf_link_begin:n {#1#2} }
\cs_new_protected:Npn \@@_pdf_link_begin:n #1
  {
    \@@_pdf:n
      {
         bann
         <<
           /Type /Annot
           #1
         >>
      }
  }
\cs_new_protected:Npn \driver_pdf_link_end:
  { \@@_pdf:n { eann } }
%    \end{macrocode}
% \end{macro}
% \end{macro}
% \end{macro}
%
% \begin{macro}{\driver_pdf_link_last:}
%   Data not available.
%    \begin{macrocode}
\cs_new:Npn \driver_pdf_link_last: { }
%    \end{macrocode}
% \end{macro}
%
% \begin{macro}{\driver_pdf_link_margin:n}
%   Pass to \texttt{dvipdfmx}.
%    \begin{macrocode}
\cs_new_protected:Npn \driver_pdf_link_margin:n #1
  { \@@_literal:x { dvipdfmx:config~g~ \dim_eval:n {#1} } }
%    \end{macrocode}
% \end{macro}
%
% \begin{macro}{\driver_pdf_destination:nn, \driver_pdf_destination_rectangle:nn}
%   Here, we need to turn the zoom into a scale. The method for \texttt{FitR}
%   is from Alexander Grahn: the idea is to avoid needing to do any calculations
%   in \TeX{} by using the driver data for \texttt{@xpos} and \texttt{@ypos}.
%    \begin{macrocode}
\cs_new_protected:Npn \driver_pdf_destination:nn #1#2
  {
    \@@_pdf:x
      {
        dest ~ ( \exp_not:n {#1} )
        [
          @thispage
          \str_case:nnF {#2}
            {
              { xyz }   { /XYZ ~ @xpos ~ @ypos ~ null }
              { fit }   { /Fit }
              { fitb }  { /FitB }
              { fitbh } { /FitBH }
              { fitbv } { /FitBV ~ @xpos }
              { fith }  { /FitH ~ @ypos }
              { fitv }  { /FitV ~ @xpos }
            }
            { /XYZ ~ @xpos ~ @ypos ~ \fp_eval:n { (#2) / 100 } }
        ]
      }
  }
\cs_new_protected:Npn \driver_pdf_destination_rectangle:nn #1#2
  {
    \group_begin:
      \hbox_set:Nn \l_@@_internal_box {#2}
      \box_move_down:nn { \box_dp:N \l_@@_internal_box }
        {
          \hbox:n
            {
              \@@_pdf:n { obj ~ @driver_#1_llx ~ @xpos }
              \@@_pdf:n { obj ~ @driver_#1_lly ~ @ypos }
            }
        }
      \box_use:N \l_@@_internal_box
      \box_move_up:nn { \box_ht:N \l_@@_internal_box }
        {
          \hbox:n
            {
              \@@_pdf:n
                {
                  dest ~ (#1)
                  [
                    @thispage
                    /FitR ~
                      @driver_#1_llx ~ @driver_#1_lly ~
                      @xpos ~ @ypos
                  ]
                }
            }
        }
    \group_end:    
  }
%    \end{macrocode}
% \end{macro}
%
% \subsubsection{Structure}
%
% \begin{macro}{\driver_pdf_compresslevel:n}
% \begin{macro}{\driver_pdf_compress_objects:n}
%   Pass data to the driver: these are a one-shot.
%    \begin{macrocode}
\cs_new_protected:Npn \driver_pdf_compresslevel:n #1
  { \@@_literal:x { dvipdfmx:config~z~ \int_eval:n {#1} } }
\cs_new_protected:Npn \driver_pdf_compress_objects:n #1
  {
    \str_if_eq:nnF {#1} { true }
      { \@@_literal:n { dvipdfmx:config~C~0x40 } }
  }
%    \end{macrocode}
% \end{macro}
% \end{macro}
%
% \begin{macro}
%   {\driver_pdf_version_major_gset:n, \driver_pdf_version_minor_gset:n}
%   We start with the assumption that the default is active.
%    \begin{macrocode}
\cs_new_protected:Npn \driver_pdf_version_major:n #1
  {
    \cs_gset:Npx \driver_pdf_version_major: { \int_eval:n {#1} }
    \@@_literal:x { pdf:majorversion \driver_pdf_version_major: }
  }
\cs_new_protected:Npn \driver_pdf_version_minor:n #1
  {
    \cs_gset:Npx \driver_pdf_version_minor: { \int_eval:n {#1} }
    \@@_literal:x { pdf:minorversion \driver_pdf_version_minor: }
  }
%    \end{macrocode}
% \end{macro}
%
% \begin{macro}[EXP]{\driver_pdf_version_major:, \driver_pdf_version_minor:}
%   We start with the assumption that the default is active.
%    \begin{macrocode}
\cs_new:Npn \driver_pdf_version_major: { 1 }
\cs_new:Npn \driver_pdf_version_minor: { 5 }
%    \end{macrocode}
% \end{macro}
%
%    \begin{macrocode}
%</dvipdfmx|xdvipdfmx>
%    \end{macrocode}
%
% \subsection{\texttt{dvisvgm} driver}
%
%    \begin{macrocode}
%<*dvisvgm>
%    \end{macrocode}
%
% \subsubsection{Catalogue entries}
%
% \begin{macro}{\driver_pdf_catalog_gput:nn, \driver_pdf_info_gput:nn}
%   No-op.
%    \begin{macrocode}
\cs_new_protected:Npn \driver_pdf_catalog_gput:nn #1#2 { }
\cs_new_protected:Npn \driver_pdf_info_gput:nn #1#2 { }
%    \end{macrocode}
% \end{macro}
%
% \subsubsection{Objects}
%
% \begin{macro}{\driver_pdf_object_new:nn}
% \begin{macro}[EXP]{\driver_pdf_object_ref:n}
% \begin{macro}{\driver_pdf_object_write:nn, , \driver_pdf_object_write:nx}
% \begin{macro}{\driver_pdf_object_now:nn, , \driver_pdf_object_now:nx}
% \begin{macro}{\driver_pdf_object_last:}
%   All no-ops here.
%    \begin{macrocode}
\cs_new_protected:Npn \driver_pdf_object_new:nn #1#2 { }
\cs_new:Npn \driver_pdf_object_ref:n #1 { }
\cs_new_protected:Npn \driver_pdf_object_write:nn #1#2 { }
\cs_new_protected:Npn \driver_pdf_object_write:nx #1#2 { }
\cs_new_protected:Npn \driver_pdf_object_now:nn #1#2 { }
\cs_new_protected:Npn \driver_pdf_object_now:nx #1#2 { }
\cs_new:Npn \driver_pdf_object_last: { }
%    \end{macrocode}
% \end{macro}
% \end{macro}
% \end{macro}
% \end{macro}
%
% \subsubsection{Structure}
%
% \begin{macro}{\driver_pdf_compresslevel:n}
% \begin{macro}{\driver_pdf_compress_objects:n}
%   These are all no-ops.
%    \begin{macrocode}
\cs_new_protected:Npn \driver_pdf_compresslevel:n #1 { }
\cs_new_protected:Npn \driver_pdf_compress_objects:n #1 { }
%    \end{macrocode}
% \end{macro}
% \end{macro}
%
% \begin{macro}
%   {\driver_pdf_version_major_gset:n, \driver_pdf_version_minor_gset:n}
%   Data not available!
%    \begin{macrocode}
\cs_new_protected:Npn \driver_pdf_version_major_gset:n #1 { }
\cs_new_protected:Npn \driver_pdf_version_minor_gset:n #1 { }
%    \end{macrocode}
% \end{macro}
%
% \begin{macro}[EXP]{\driver_pdf_version_major:, \driver_pdf_version_minor:}
%   Data not available!
%    \begin{macrocode}
\cs_new:Npn \driver_pdf_version_major: { -1 }
\cs_new:Npn \driver_pdf_version_minor: { -1 }
%    \end{macrocode}
% \end{macro}
%
%    \begin{macrocode}
%</dvisvgm>
%    \end{macrocode}
%
%    \begin{macrocode}
%</initex|package>
%    \end{macrocode}
%
% \end{implementation}
%
% \PrintIndex
