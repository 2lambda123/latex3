% \iffalse meta-comment
%
%% File: l3flag.dtx
%
% Copyright (C) 2011-2022 The LaTeX Project
%
% It may be distributed and/or modified under the conditions of the
% LaTeX Project Public License (LPPL), either version 1.3c of this
% license or (at your option) any later version.  The latest version
% of this license is in the file
%
%    https://www.latex-project.org/lppl.txt
%
% This file is part of the "l3kernel bundle" (The Work in LPPL)
% and all files in that bundle must be distributed together.
%
% -----------------------------------------------------------------------
%
% The development version of the bundle can be found at
%
%    https://github.com/latex3/latex3
%
% for those people who are interested.
%
%<*driver>
\documentclass[full,kernel]{l3doc}
\begin{document}
  \DocInput{\jobname.dtx}
\end{document}
%</driver>
% \fi
%
% \title{^^A
%   The \textsf{l3flag} package: Expandable flags^^A
% }
%
% \author{^^A
%  The \LaTeX{} Project\thanks
%    {^^A
%      E-mail:
%        \href{mailto:latex-team@latex-project.org}
%          {latex-team@latex-project.org}^^A
%    }^^A
% }
%
% \date{Released 2022-07-15}
%
% \maketitle
%
% \begin{documentation}
%
% Flags are the only data-type that can be modified in expansion-only
% contexts. This module is meant mostly for kernel use: in almost all
% cases, booleans or integers should be preferred to flags because they
% are very significantly faster.
%
% A flag can hold any non-negative value, which we call its
% \meta{height}. In expansion-only contexts, a flag can only be
% \enquote{raised}: this increases the \meta{height} by $1$. The \meta{height}
% can also be queried expandably. However, decreasing it, or setting it
% to zero requires non-expandable assignments.
%
% Flag variables are always local. They are referenced by a \meta{flag
% name} such as \texttt{str_missing}.  The \meta{flag name} is used as
% part of \cs{use:c} constructions hence is expanded at point of use.
% It must expand to character tokens only, with no spaces.
%
% A typical use case of flags would be to keep track of whether an
% exceptional condition has occurred during expandable processing, and
% produce a meaningful (non-expandable) message after the end of the
% expandable processing.  This is exemplified by \pkg{l3str-convert},
% which for performance reasons performs conversions of individual
% characters expandably and for readability reasons produces a single
% error message describing incorrect inputs that were encountered.
%
% Flags should not be used without carefully considering the fact that
% raising a flag takes a time and memory proportional to its height.
% Flags should not be used unless unavoidable.
%
% \section{Setting up flags}
%
% \begin{function}{\flag_new:n}
%   \begin{syntax}
%     \cs{flag_new:n} \Arg{flag name}
%   \end{syntax}
%   Creates a new flag with a name given by \meta{flag name}, or raises
%   an error if the name is already taken. The \meta{flag name} may not
%   contain spaces. The declaration is global, but flags are always
%   local variables. The \meta{flag} initially has zero height.
% \end{function}
%
% \begin{function}{\flag_clear:n}
%   \begin{syntax}
%     \cs{flag_clear:n} \Arg{flag name}
%   \end{syntax}
%   The \meta{flag}'s height is set to zero. The assignment is local.
% \end{function}
%
% \begin{function}{\flag_clear_new:n}
%   \begin{syntax}
%     \cs{flag_clear_new:n} \Arg{flag name}
%   \end{syntax}
%   Ensures that the \meta{flag} exists globally by applying
%   \cs{flag_new:n} if necessary, then applies \cs{flag_clear:n}, setting
%   the height to zero locally.
% \end{function}
%
% \begin{function}{\flag_show:n}
%   \begin{syntax}
%     \cs{flag_show:n} \Arg{flag name}
%   \end{syntax}
%   Displays the \meta{flag}'s height in the terminal.
% \end{function}
%
% \begin{function}{\flag_log:n}
%   \begin{syntax}
%     \cs{flag_log:n} \Arg{flag name}
%   \end{syntax}
%   Writes the \meta{flag}'s height to the log file.
% \end{function}
%
% \section{Expandable flag commands}
%
% \begin{function}[EXP,pTF]{\flag_if_exist:n}
%   \begin{syntax}
%     \cs{flag_if_exist:n} \Arg{flag name}
%   \end{syntax}
%   This function returns \texttt{true} if the \meta{flag name}
%   references a flag that has been defined previously, and
%   \texttt{false} otherwise.
% \end{function}
%
% \begin{function}[EXP,pTF]{\flag_if_raised:n}
%   \begin{syntax}
%     \cs{flag_if_raised:n} \Arg{flag name}
%   \end{syntax}
%   This function returns \texttt{true} if the \meta{flag} has non-zero
%   height, and \texttt{false} if the \meta{flag} has zero height.
% \end{function}
%
% \begin{function}[EXP]{\flag_height:n}
%   \begin{syntax}
%     \cs{flag_height:n} \Arg{flag name}
%   \end{syntax}
%   Expands to the height of the \meta{flag} as an integer denotation.
% \end{function}
%
% \begin{function}[EXP]{\flag_raise:n}
%   \begin{syntax}
%     \cs{flag_raise:n} \Arg{flag name}
%   \end{syntax}
%   The \meta{flag}'s height is increased by $1$ locally.
% \end{function}
%
% \end{documentation}
%
% \begin{implementation}
%
% \section{\pkg{l3flag} implementation}
%
%    \begin{macrocode}
%<*package>
%    \end{macrocode}
%
%    \begin{macrocode}
%<@@=flag>
%    \end{macrocode}
%
% \TestFiles{m3flag001}
%
% \subsection{Non-expandable flag commands}
%
% The height $h$ of a flag (initially zero) is stored by setting control
% sequences of the form \cs[no-index]{flag \meta{name} \meta{integer}}
% to \tn{relax} for $0\leq\meta{integer}<h$.  When a flag is raised, a
% \enquote{trap} function \cs[no-index]{flag \meta{name}} is called.
% The existence of this function is also used to test for the existence
% of a flag.
%
% \begin{macro}{\flag_new:n}
%   For each flag, we define a \enquote{trap} function, which by default
%   simply increases the flag by $1$ by letting the appropriate control
%   sequence to \tn{relax}.  This can be done expandably!
%   \begin{macrocode}
\cs_new_protected:Npn \flag_new:n #1
  {
    \cs_new:cpn { flag~#1 } ##1 ;
      { \exp_after:wN \use_none:n \cs:w flag~#1~##1 \cs_end: }
  }
%    \end{macrocode}
% \end{macro}
%
% \begin{macro}{\flag_clear:n}
% \begin{macro}{\@@_clear:wn}
%   Undefine control sequences, starting from the |0| flag, upwards,
%   until reaching an undefined control sequence.  We don't use
%   \cs{cs_undefine:c} because that would act globally.
%   When the option \texttt{check-declarations} is used, check for the
%   function defined by \cs{flag_new:n}.
%    \begin{macrocode}
\cs_new_protected:Npn \flag_clear:n #1 { \@@_clear:wn 0 ; {#1} }
\cs_new_protected:Npn \@@_clear:wn #1 ; #2
  {
    \if_cs_exist:w flag~#2~#1 \cs_end:
      \cs_set_eq:cN { flag~#2~#1 } \tex_undefined:D
      \exp_after:wN \@@_clear:wn
      \int_value:w \int_eval:w 1 + #1
    \else:
      \use_i:nnn
    \fi:
    ; {#2}
  }
%    \end{macrocode}
% \end{macro}
% \end{macro}
%
% \begin{macro}{\flag_clear_new:n}
%   As for other datatypes, clear the \meta{flag} or create a new one,
%   as appropriate.
%    \begin{macrocode}
\cs_new_protected:Npn \flag_clear_new:n #1
  { \flag_if_exist:nTF {#1} { \flag_clear:n } { \flag_new:n } {#1} }
%    \end{macrocode}
% \end{macro}
%
% \begin{macro}{\flag_show:n, \flag_log:n, \@@_show:Nn}
%   Show the height (terminal or log file) using appropriate \pkg{l3msg}
%   auxiliaries.
%    \begin{macrocode}
\cs_new_protected:Npn \flag_show:n { \@@_show:Nn \tl_show:n }
\cs_new_protected:Npn \flag_log:n { \@@_show:Nn \tl_log:n }
\cs_new_protected:Npn \@@_show:Nn #1#2
  {
    \exp_args:Nc \__kernel_chk_defined:NT { flag~#2 }
      {
        \exp_args:Nx #1
          { \tl_to_str:n { flag~#2~height } = \flag_height:n {#2} }
      }
  }
%    \end{macrocode}
% \end{macro}
%
% \subsection{Expandable flag commands}
%
% \begin{macro}[EXP, pTF]{\flag_if_exist:n}
%   A flag exist if the corresponding trap \cs[no-index]{flag \meta{flag
%   name}:n} is defined.
%    \begin{macrocode}
\prg_new_conditional:Npnn \flag_if_exist:n #1 { p , T , F , TF }
  {
    \cs_if_exist:cTF { flag~#1 }
      { \prg_return_true: } { \prg_return_false: }
  }
%    \end{macrocode}
% \end{macro}
%
% \begin{macro}[EXP, pTF]{\flag_if_raised:n}
%   Test if the flag has a non-zero height, by checking the |0| control sequence.
%    \begin{macrocode}
\prg_new_conditional:Npnn \flag_if_raised:n #1 { p , T , F , TF }
  {
    \if_cs_exist:w flag~#1~0 \cs_end:
      \prg_return_true:
    \else:
      \prg_return_false:
    \fi:
  }
%    \end{macrocode}
% \end{macro}
%
% \begin{macro}[EXP]{\flag_height:n}
% \begin{macro}[EXP]{\@@_height_loop:wn, \@@_height_end:wn}
%   Extract the value of the flag by going through all of the
%   control sequences starting from |0|.
%    \begin{macrocode}
\cs_new:Npn \flag_height:n #1 { \@@_height_loop:wn 0; {#1} }
\cs_new:Npn \@@_height_loop:wn #1 ; #2
  {
    \if_cs_exist:w flag~#2~#1 \cs_end:
      \exp_after:wN \@@_height_loop:wn \int_value:w \int_eval:w 1 +
    \else:
      \exp_after:wN \@@_height_end:wn
    \fi:
    #1 ; {#2}
  }
\cs_new:Npn \@@_height_end:wn #1 ; #2 {#1}
%    \end{macrocode}
% \end{macro}
% \end{macro}
%
% \begin{macro}[EXP]{\flag_raise:n}
%   Simply apply the trap to the height, after expanding the latter.
%    \begin{macrocode}
\cs_new:Npn \flag_raise:n #1
  {
    \cs:w flag~#1 \exp_after:wN \cs_end:
    \int_value:w \flag_height:n {#1} ;
  }
%    \end{macrocode}
% \end{macro}
%
%    \begin{macrocode}
%</package>
%    \end{macrocode}
%
% \end{implementation}
%
% \PrintIndex
