% \iffalse meta-comment
%
%% File: l3precom.dtx Copyright (C) 2012,2013,2015-2018,2021 The LaTeX Project
%%
%% It may be distributed and/or modified under the conditions of the
%% LaTeX Project Public License (LPPL), either version 1.3c of this
%% license or (at your option) any later version.  The latest version
%% of this license is in the file
%%
%%    http://www.latex-project.org/lppl.txt
%%
%% This file is part of the "l3trial bundle" (The Work in LPPL)
%% and all files in that bundle must be distributed together.
%%
%% The released version of this bundle is available from CTAN.
%%
%% -----------------------------------------------------------------------
%%
%% The development version of the bundle can be found at
%%
%%    http://www.latex-project.org/svnroot/experimental/trunk/
%%
%% for those people who are interested.
%%
%%%%%%%%%%%
%% NOTE: %%
%%%%%%%%%%%
%%
%%   Snapshots taken from the repository represent work in progress and may
%%   not work or may contain conflicting material!  We therefore ask
%%   people _not_ to put them into distributions, archives, etc. without
%%   prior consultation with The LaTeX Project.
%%
%% -----------------------------------------------------------------------
%
%<*driver|package>
\RequirePackage{expl3}
%</driver|package>
%<*driver>
\documentclass[full]{l3doc}
\usepackage{amsmath}
\begin{document}
  \DocInput{\jobname.dtx}
\end{document}
%</driver>
% \fi
%
% \title{^^A
%   The \textsf{l3precom} package: precompiling functions^^A
% }
%
% \author{^^A
%  The \LaTeX{} Project\thanks
%    {^^A
%      E-mail:
%        \href{mailto:latex-team@latex-project.org}
%          {latex-team@latex-project.org}^^A
%    }^^A
% }
%
% \date{Released 2018-04-30}
%
% \maketitle
%
% \begin{documentation}
%
% \section{\pkg{l3precom} documentation}
%
% It might speed up the processing of documents when certain parts of
% the document style file are `precompiled' and stored in a separate
% file. How much of this is still useful these days is open to some
% experiments I guess. If a large number of LDB settings are being used
% then perhaps this will indeed safe some considerable time as all the
% parsing isn't necessary (some initial test show 1 second extra user
% time per 1000 additional LDB entries).
%
% I've largely added it to have \cs{dumpLaTeXstate} available while
% debugging the LDB pointer structure.
%
% \end{documentation}
%
% \begin{implementation}
%
% \section{\pkg{l3precom} implementation}
%
%    \begin{macrocode}
%<*package>
\RequirePackage{expl3}
\ProvidesExplPackage{l3precom}{2018-04-30}{}
  {L3 Experimental precompilation of functions}
%    \end{macrocode}
%
%    \begin{macrocode}
\tex_message:D { LaTeX:~ precompiling~ code, }
%    \end{macrocode}
%
%    \begin{macrocode}
%<*precompile>
%    \end{macrocode}
%
% \begin{variable}{\c_cs_dump_iow}
%   We need to allocate an output stream for the dump in order to be
%   able to write the precompiled code out.
%    \begin{macrocode}
\iow_new:N \c_cs_dump_iow
%    \end{macrocode}
% \end{variable}
%
% \begin{variable}{\g_cs_dump_name_tl}
%   This \meta{tl} is used to store the name of the file.
%    \begin{macrocode}
\tl_new:N  \g_cs_dump_name_tl
\tl_gset:Nn \g_cs_dump_name_tl { \c_sys_jobname_str -default.dmp }
%    \end{macrocode}
% \end{variable}
%
% \begin{variable}{\g_cs_dump_seq}
%   While processing the document class we build up a list of
%   control sequence names to be dumped. For this purpose
%   we use the |\g_cs_dump_seq| sequence.
%    \begin{macrocode}
\seq_new:N \g_cs_dump_seq
%    \end{macrocode}
% \end{variable}
%
% \begin{macro}{\cs_record_name:N, \cs_record_name:c}
%   These functions mark a control sequence for dumping into a
%   precompiled class.  When the \textsf{trace} `module' is included in
%   the code we also write information about the control sequence into a
%   |.dmp| file.
%    \begin{macrocode}
\cs_new_protected:Npn \cs_record_name:N #1
  {
    \seq_gput_left:Nn \g_cs_dump_seq {#1}
%<trace>    \seq_gput_left:Nn \g_cs_trace_seq {#1}
  }
\cs_generate_variant:Nn \cs_record_name:N { c }
%    \end{macrocode}
% \end{macro}
%
% \begin{macro}{\cs_dump:}
%   When a document styleclass calls \cs{cs_dump:} it triggers this code
%   to write all the precompilation information out to a file.
%
%   Before dumping, we write a message to the terminal informing the
%   \enquote{user} of this fact.
%    \begin{macrocode}
\cs_new_protected:Npn \cs_dump:
  {
    \iow_term:x { Precompiling~class~into~(\g_cs_dump_name_tl) }
    \iow_open:Nn \c_cs_dump_iow { \g_cs_dump_name_tl }
%    \end{macrocode}
%   The first thing we write on a `dump' file is a command that allows
%   us to use |*| in control sequences. We also need to be able to write
%   to (and read from) the file internal control sequences, containing
%   |_| and |:|.
%    \begin{macrocode}
    \iow_now:Nn \c_cs_dump_iow
      {
        \group_begin:
          \char_set_catcode_letter:N \*
          \ExplSyntaxOn
      }
%    \end{macrocode}
%   We use a direct |\cs_gset:Npn| to disable any type of local/global
%   check on the pointers.
%
%   Such dumping should probably ensure that the line length is not going
%   to be too long or else it may not be possible to read such a file
%   back in (not done).
%
%   One other thing: dumping of the LDB doesn't work of course because
%   dumping of property lists doesn't work (in all cases) as the keys
%   these days are specially handled and become strings.
%    \begin{macrocode}
    \seq_map_inline:Nn \g_cs_dump_seq
      {
        \tex_message:D {.}
        \iow_now:Nx \c_cs_dump_iow
          {
            \token_to_str:N \cs_gset:Npn
            \token_to_str:N ##1
              { \tl_to_str:N ##1 }
          }
      }
    \iow_now:Nn \c_cs_dump_iow { \group_end: }
    \iow_close:N \c_cs_dump_iow
    \tex_message:D { ~ finished }
  }
%</precompile>
%    \end{macrocode}
% \end{macro}
%
% \begin{macro}{\cs_load_dump:n}
%   A function to read a precompiled file into memory and skip until a
%   \cs{cs_dump:} command is found. If no such file is found, processing
%   continues and a subsequent \cs{cs_dump:} command will then create
%   the dump file.
%    \begin{macrocode}
\cs_new_protected:Npn \cs_load_dump:n #1
  {
    \file_if_exist:nTF { #1.cmp }
      {
        \iow_term:x { Loading~ precompiled~ format~ in~ #1.cmp ~... }
        \file_input:n { #1.cmp }
        \cs_set_eq:NN \cs_dump: \fi:
        \if_false:
      }
%<precompile>      { \tl_gset:Nn \g_cs_dump_name_tl { #1.cmp } }
%<-precompile>      { \tex_errmessage:D { Cannot~ dump~ with~ this~ format } }
  }
%    \end{macrocode}
% \end{macro}
%
% \begin{variable}{\g_gen_sym_int, \g_ggen_sym_int}
%    Two counters to make up new local or global \textsl{short} names
%    in pointer structures like the \textsc{ldb}.
%    \begin{macrocode}
\int_new:N   \g_gen_sym_int
\int_new:N   \g_ggen_sym_int
%    \end{macrocode}
% \end{variable}
%
% \begin{macro}{\cs_gen_sym:N, \cs_ggen_sym:N}
%   We need to be able to generate control sequences on the fly.  They
%   will exist of a prefix, either |l*| or |g*|, followed by the value
%   of the counter \cs{g_gen_sym_int} (\cs{g_ggen_sym_int}) in roman
%   numeral representation. The generated control sequence is locally
%   stored in the token that was passed in |#1|.
%
%    \begin{macrocode}
\cs_new_protected:Npn \cs_gen_sym:N #1
  {
    \int_gincr:N \g_gen_sym_int
    \tl_set:Nx #1 { \exp_not:c { l*\__int_to_roman:w \g_gen_sym_int } }
%<precompile>    \exp_after:wN \cs_record_name:N #1
%    \end{macrocode}
%   We still want to define the initial value for the new symbol
%   globally to make sure that during compilation something is written
%   to the output file.
%    \begin{macrocode}
    \exp_after:wN \tl_clear_new:N #1
  }
%    \end{macrocode}
%   The global variant %^^A todo: why so different?
%    \begin{macrocode}
\cs_new_protected:Npn \cs_ggen_sym:N #1
  {
%<trace>    \trace_ldb_internal:x { \exp_not:n { \cs_ggen_sym:N { #1 } } }
    \int_gincr:N \g_ggen_sym_int
    \tl_set:Nx #1 { \exp_not:c { g*\__int_to_roman:w \g_ggen_sym_int } }
%<trace>    \trace_ldb_internal:x { \token_to_meaning:N #1 ~ ... ~ allocated }
%<precompile>    \exp_after:wN \cs_record_name:N #1
    \exp_after:wN \tl_clear_new:N #1
  }
%    \end{macrocode}
% \end{macro}
%
%
% \begin{macro}{\cs_ggen_prop:N}
%   We really need to allocate property lists these days for nodes of
%   the LDB.
%    \begin{macrocode}
\cs_new_protected:Npn \cs_ggen_prop:N #1
  {
    \cs_ggen_sym:N #1
    \exp_after:wN \prop_gclear:N #1
  }
%    \end{macrocode}
% \end{macro}
%
%
%
% \subsection{Dumping data structure states for tracing}
%
% \begin{variable}{\g_cs_trace_seq}
%   A sequence which holds the control sequence names that are to be
%   dumped. They are stored together with their meaning.
%    \begin{macrocode}
%<*trace>
\seq_new:N \g_cs_trace_seq
%    \end{macrocode}
% \end{variable}
%
% \begin{variable}{\g_register_trace_seq}
%   Sequence holding the register names to be dumped with their
%   corresponding values.
%    \begin{macrocode}
\seq_new:N \g_register_trace_seq
%    \end{macrocode}
% \end{variable}
%
% \begin{macro}{\cs_record_meaning:N}
%   Function marking a control sequence for dumping with meaning.
%    \begin{macrocode}
\cs_new_protected:Npn \cs_record_meaning:N #1
  { \seq_gput_left:Nn \g_cs_trace_seq {#1} }
%    \end{macrocode}
% \end{macro}
%
% \begin{macro}{\register_record_name:N}
%   Function marking a register for dumping with value.
%    \begin{macrocode}
\cs_new_protected:Npn \register_record_name:N #1
  { \seq_gput_left:Nn \g_register_trace_seq {#1} }
%    \end{macrocode}
% \end{macro}
%
%
% \begin{macro}{\dumpLaTeXstate}
%   The function \cs{dumpLaTeXstate} is used to write control sequences
%   and registers, together with their meaning or value in the |.dmp|
%   file. We write informational messages to the terminal during the
%   dump.
%    \begin{macrocode}
\cs_new_protected:Npn \dumpLaTeXstate #1
  {
    \iow_term:x { Dumping~commands~into~(#1.dmp) }
    \iow_open:Nn \c_cs_dump_iow { #1.dmp }
    \tex_message:D { ~ commands }
    \seq_map_inline:Nn \g_cs_trace_seq
      {
        \tex_message:D {.}
        \iow_now:Nx \c_cs_dump_iow
          {
            \token_to_str:N ##1~
            \token_to_meaning:N ##1
          }
      }
    \tex_message:D { ~ registers }
    \seq_map_inline:Nn \g_register_trace_seq
      {
        \tex_message:D {.}
        \iow_now:Nx \c_cs_dump_iow
          {
            \token_to_str:N ##1
            \tex_the:D ##1
          }
      }
    \tex_message:D { ~ finished }
    \iow_close:N \c_cs_dump_iow
  }
%</trace>
%    \end{macrocode}
% \end{macro}
%
%    \begin{macrocode}
%</package>
%    \end{macrocode}
%
% \end{implementation}
%
% \endinput
