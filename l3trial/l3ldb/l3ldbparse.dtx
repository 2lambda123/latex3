% \iffalse meta-comment
%
%% File l3ldbparse.dtx (C) Copyright 2011 Frank Mittelbach, The LaTeX Project
%%                     (C) Copyright 2012,2013,2016-2017,2021 The LaTeX Project
%%
%% It may be distributed and/or modified under the conditions of the
%% LaTeX Project Public License (LPPL), either version 1.3c of this
%% license or (at your option) any later version.  The latest version
%% of this license is in the file
%%
%%    http://www.latex-project.org/lppl.txt
%%
%% This file is part of the ``l3ldb bundle'' (The Work in LPPL)
%% and all files in that bundle must be distributed together.
%%
%% The released version of this bundle is available from CTAN.
%%
%% -----------------------------------------------------------------------
%%
%% The development version of the bundle can be found at
%%
%%    http://www.latex-project.org/svnroot/experimental/trunk/
%%
%% for those people who are interested.
%%
%%%%%%%%%%%
%% NOTE: %%
%%%%%%%%%%%
%%
%%   Snapshots taken from the repository represent work in progress and may
%%   not work or may contain conflicting material!  We therefore ask
%%   people _not_ to put them into distributions, archives, etc. without
%%   prior consultation with The LaTeX Project.
%%
%% -----------------------------------------------------------------------
%%
%<*driver|package>
\RequirePackage{expl3}
%</driver|package>
%<*driver>
\documentclass[full]{l3doc}
\usepackage{textcomp,xparse}
% \usepackage{xparse,xlists,xlists-samples}
% \usepackage{ldcdoc}
\begin{document}
  \DocInput{l3ldbparse.dtx}
\end{document}
%</driver>
% \fi
%
% \title{^^A
%   The \textsf{l3ldbparse} package: parsing LDB entries^^A
% }
%
% \author{^^A
%  The \LaTeX{} Project\thanks
%    {^^A
%      E-mail:
%        \href{mailto:latex-team@latex-project.org}
%          {latex-team@latex-project.org}^^A
%    }^^A
% }
%
% \date{Released 2018-04-30}
%
% \maketitle
%
% \begin{documentation}
%
% \section {Parsing LDB Context Specifications}
%
% A context specification consists of 3 segments.  The 1st segment
% contains begin tags and completed tags only, respectively of the form
% |<foo| and |<foo>|.  Two consecutive tags can be optionally separated
% by a tight or loose binding (\emph{i.e.} a dot or a star), except that
% a completed tag may never be followed by a binding.  The second
% segment consists of end tags (\emph{i.e.} of the form |foo>|)
% occurring in sequence; no bindings are allowed.  If both the 1st and
% 2nd segments are non-empty, then there must be a loose binding between
% them.  The last segment consists of begin tags occurring in sequence.
% There must no binding between the 2nd and 3rd segments.
%
% A context specification is parsed as follows: first, abbreviations
% such as |>...<| are expanded, and various special cases are normalized
% --- for example, |*<-| is turned into the simpler, but semantically
% equivalent, |*|.  Then, the context specification is examined and
% broken into basic constituents, and turned into a list of triples of
% the form (bdg,type,tag), where tag is the identifier naming the tag,
% type is 0 (begin), 1 (end), or 2 (completed), and bdg indicates
% whether this tag was preceded by a tight (1) or a loose binding (2),
% or by no binding at all (0) (\emph{i.e.} it occurred in sequence).
%
% For example, |<A.<B<C><D*E>| is turned into something like:
% \begin{center}
%   |(0,0,A)(1,0,B)(0,2,C)(0,0,D)(2,1,E)|
% \end{center}
% except that we use integer constants instead.
%
% \end{documentation}
%
% \begin{implementation}
%
% \section{\pkg{l3ldbparse} implementation}
%
%^^A todo: more specific error-reporting when parsing.
%
% This is not a real package, but actually just part of the LDB code
% but for the moment we pretend it has its own versioning.
%    \begin{macrocode}
%<*package>
\RequirePackage{expl3}
\ProvidesExplPackage{l3ldbparse}{2018-04-30}{}
  {L3 Experimental resource database parsing}
%    \end{macrocode}
%
% \begin{variable}{\c__ldb_prebdg_mode, \c__ldb_bdg_mode, \c__ldb_less_mode, \c__ldb_begin_mode, \c__ldb_ident_mode}
%   Modes before a binding, after a binding, after |<|, after |<foo|, and after |foo|.
%    \begin{macrocode}
\int_const:Nn \c__ldb_prebdg_mode { 0 }
\int_const:Nn \c__ldb_bdg_mode { 1 }
\int_const:Nn \c__ldb_less_mode { 2 }
\int_const:Nn \c__ldb_begin_mode { 3 }
\int_const:Nn \c__ldb_ident_mode { 4 }
%    \end{macrocode}
% \end{variable}
%
% \begin{variable}{\c__ldb_begin_type, \c__ldb_end_type, \c__ldb_completed_type}
%    \begin{macrocode}
\int_const:Nn \c__ldb_begin_type { 0 }
\int_const:Nn \c__ldb_end_type { 1 }
\int_const:Nn \c__ldb_completed_type { 2 }
%    \end{macrocode}
% \end{variable}
%
% \begin{variable}{\c__ldb_sequenced_bdg, \c__ldb_tight_bdg, \c__ldb_loose_bdg, \c__ldb_other_bdg}
%   The \enquote{other} binding is only used in \pkg{l3ldb}, probably to mark the end of \cs{g_ldb_triples_tl}.
%    \begin{macrocode}
\int_const:Nn \c__ldb_sequenced_bdg { 0 }
\int_const:Nn \c__ldb_tight_bdg { 1 }
\int_const:Nn \c__ldb_loose_bdg { 2 }
\int_const:Nn \c__ldb_other_bdg { 3 }
%    \end{macrocode}
% \end{variable}
%
% \begin{variable}{\c__ldb_initial_segment, \c__ldb_main_segment, \c__ldb_after_segment, \c__ldb_end_segment, \c__ldb_begin_segment}
%    \begin{macrocode}
\int_const:Nn \c__ldb_initial_segment { 0 }
\int_const:Nn \c__ldb_main_segment { 1 }
\int_const:Nn \c__ldb_after_segment { 2 }
\int_const:Nn \c__ldb_end_segment { 3 }
\int_const:Nn \c__ldb_begin_segment { 4 }
%    \end{macrocode}
% \end{variable}
%
% \begin{macro}{\glet_gtmp:N, \gdef_gtmp:n, \gdef_gtmp:x}
%    Very convenient abbreviations for defining a continuation in
%    |\gtmp:w| for immediate use.
%    \begin{macrocode}
\cs_new_protected:Npn \glet_gtmp:N { \cs_gset_eq:NN \gtmp:w }
\cs_new_protected:Npn \gdef_gtmp:n { \cs_gset:Npn   \gtmp:w }
\cs_new_protected:Npn \gdef_gtmp:x { \cs_gset:Npx   \gtmp:w }
%    \end{macrocode}
% \end{macro}
%
% \begin{variable}{\g_ldb_spec_tl, \g_ldb_triples_tl, \g_ldb_tag_tl}
%   We save the original context specification in \cs{g_ldb_spec_tl} and
%   use it when we print an error message.  After parsing this context
%   specification, the corresponding sequence of triples is available in
%   \cs{g_ldb_triples_tl}.  While parsing, whenever we encounter a tag's
%   identifier, we save it temporarily in \cs{g_ldb_tag_tl}.
%    \begin{macrocode}
\tl_new:N \g_ldb_spec_tl
\tl_new:N \g_ldb_triples_tl
\tl_new:N \g_ldb_tag_tl
%    \end{macrocode}
% \end{variable}
%
%^^A todo: give proper names!
% \begin{variable}{\g_ldb_mode, \g_ldb_type, \g_ldb_bdg, \g_ldb_segment}
%   We use these integer variables to keep track of what we have just
%   parsed and of what part of the context we are in.  \cs{g_ldb_mode}
%   encodes the local parsing state, \emph{e.g.} have we just seen a
%   binding, a |<|, an identifier\ldots\ \cs{g_ldb_type} represents the
%   type of the tag spec.  \cs{g_ldb_bdg} is the kind of binding that
%   precedes the tag spec.  \cs{g_ldb_segment} encodes a more global
%   parsing state: are we in the main segment of the context, have we
%   just seen a completed tag, are we now in the sequence of end tags,
%   or in the final sequence of begin tags?
%    \begin{center}
%    \DeleteShortVerb{\|}
%      \begin{tabular}{|l|l|l|l|l|}
%        \cline{2-5}
%          \multicolumn{1}{l}{}          &
%          \multicolumn{1}{|c}{\bf mode} &
%          \multicolumn{1}{|c}{\bf type} &
%          \multicolumn{1}{|c}{\bf bdg}  &
%          \multicolumn{1}{|c|}{\bf segment}\\
%        \hline
%  0 & before bdg          & begin     & sequenced & initial              \\
%  1 & after  bdg          & end       & tight     & main                 \\
%  2 & after \texttt{<}    & completed & loose     & after \texttt{<foo>} \\
%  3 & after \texttt{<foo} &           &           & end sequence         \\
%  4 & after \texttt{foo}  &           &           & begin sequence       \\
%        \hline
%      \end{tabular}
%    \MakeShortVerb{\|}
%    \end{center}
%    \begin{macrocode}
\cs_new_eq:NN \g_ldb_mode    \c__ldb_prebdg_mode
\cs_new_eq:NN \g_ldb_type    \c__ldb_begin_type
\cs_new_eq:NN \g_ldb_bdg     \c__ldb_sequenced_bdg
\cs_new_eq:NN \g_ldb_segment \c__ldb_initial_segment
%    \end{macrocode}
% \end{variable}
%
%^^A todo: turn that to a seq.
% \begin{variable}{\g_ldb_abbrevs_tl}
%   A list of pairs.  Each pair represents an abbreviation and its
%   corresponding replacement text.  We also include pairs such as
%   |{*<-}{*}| for semantic normalization.
%    \begin{macrocode}
\tl_new:N \g_ldb_abbrevs_tl
\tl_gset:Nn \g_ldb_abbrevs_tl
  {
    {>...<}{>-><-<}   {>...}{>->}   {...<}{<-<}
    {<-<-} {<-}       {->->}{->}    {*<-*}{*}
    {*<-.} {*}        {.<-.}{*}     {*<-} {*}
    {<-*}  {*}        {<-.} {*}
  }
%    \end{macrocode}
% \end{variable}
%
% \begin{macro}{\ldb_parse:n}
%   This is the parser's top-level function.  It takes a context
%   specification as an argument, parses it, and places the
%   corresponding triples in \cs{g_ldb_triples_tl}.  It starts off with
%   a few initializations.
%    \begin{macrocode}
\cs_new_protected:Npn \ldb_parse:n #1
  {
    \tl_gset:Nn \g_ldb_spec_tl   {#1}
    \tl_gclear:N  \g_ldb_triples_tl
    \tl_gclear:N  \g_ldb_tag_tl
    \cs_gset_eq:NN \g_ldb_mode   \c__ldb_prebdg_mode
    \cs_gset_eq:NN \g_ldb_type   \c__ldb_begin_type
    \cs_gset_eq:NN \g_ldb_bdg    \c__ldb_sequenced_bdg
    \cs_gset_eq:NN \g_ldb_segment\c__ldb_initial_segment
%    \end{macrocode}
%   Then it proceeds to expand the abbreviations.  The empty pair of
%   abbreviations marks the end of the list of abbreviations.  The
%   reason for \cs{q_mark} is explained later.
%    \begin{macrocode}
    \gdef_gtmp:n { \ldb_expand_next:nnn { #1 \q_mark } }
    \exp_after:wN \gtmp:w \g_ldb_abbrevs_tl { } { }
%    \end{macrocode}
%   When the abbreviation expansion process is done, it then invokes the
%   parser proper.  Finally, we arrive here.  \cs{ldb_parse_finish:}
%   tidies up at the end of the parse, and finally removes \cs{q_error}
%   from the input.
%    \begin{macrocode}
    \ldb_parse_finish: \q_error
  }
%    \end{macrocode}
%   The purpose of \cs{q_error} is to make it possible to abort the
%   current \cs{ldb_parse:n} computation and proceed with the remainder of
%   the job.  The error handler uses \TeX{}'s parameter scanning mechanism
%   to grab and discard everything up to, and including, the \cs{q_error}
%   token.  Naturally, in case the computation completes successfully,
%   it must take care of removing this token.
% \end{macro}
%
% \begin{macro}{\ldb_error:w}
%   Here is the error handler.  It aborts the current \cs{ldb_parse:n}
%   computation by grabbing and discarding everything up to and
%   including the \cs{q_error} token.
%    \begin{macrocode}
\cs_new_protected:Npn \ldb_error:w #1 \q_error
  { \__kernel_msg_error:nn { ldb } { illegal-context } }
\__kernel_msg_new:nnnn { ldb } { illegal-context }
  { Illegal~LDB~context:~\g_ldb_spec_tl.}
  {
    This~context~specification~is~illegal~and~
    will~be~ignored.~See~the~manual.
  }
%    \end{macrocode}
% \end{macro}
%
% \begin{macro}{\ldb_gtmp_error:}
%   Defines \cs{gtmp:w} as a continuation that invokes the error
%   handler.
%    \begin{macrocode}
\cs_new_protected:Npn \ldb_gtmp_error: { \glet_gtmp:N \ldb_error:w }
%    \end{macrocode}
% \end{macro}
%
% \begin{macro}{\ldb_parse_finish:}
%   We must put off outputting a triple until we have seen what comes
%   after a tag identifier, in case it was followed by |>|.  As a
%   result, we may come to the end of the context and still have a
%   pending triple to output.  Here, we make sure to flush it out.
%
%   If mode=0, we are done, no triple is pending: we just make sure to
%   remove the \cs{q_error} token.  If we have just seen a binding or
%   |<|, then the context is incomplete and we signal an error.
%   Otherwise, we flush the pending triple and remove the \cs{q_error}
%   token.
%    \begin{macrocode}
\cs_new_protected:Npn \ldb_parse_finish:
  {
    \if_case:w \g_ldb_mode
      \glet_gtmp:N \ldb_parse_done:w
    \or:   \ldb_gtmp_error:
    \or:   \ldb_gtmp_error:
    \else:
      \gdef_gtmp:n
        {
          \ldb_flush:NNN \c__ldb_begin_type \c__ldb_sequenced_bdg \c__ldb_prebdg_mode
          \ldb_parse_done:w
        }
    \fi:
    \gtmp:w
  }
%    \end{macrocode}
% \end{macro}
%
% \begin{macro}{\ldb_parse_done:w}
%   Removes the \cs{q_error} token when parsing successfully completes.
%    \begin{macrocode}
\cs_new_protected:Npn \ldb_parse_done:w \q_error { }
%    \end{macrocode}
% \end{macro}
%
%^^A todo: use \tl_replace_all, presumably faster.
% \begin{macro}{\ldb_expand_it:w}
%   This macro is redefined during the process of expanding
%   abbreviations.  Its purpose is to scan the context for the current
%   abbreviation.  Let's say the current abbreviation is |AA|.  Then
%   \cs{ldb_expand_it:w} is redefined so that's its parameter text is
%   |#1AA#2\q_stop| and it is essentially invoked on the context to
%   which we append |AA\q_stop| to guarantee a match.  We now come to
%   the justification for \cs{q_mark}.  Suppose the context has the form
%   |...A|, \emph{i.e.} it ends with |A| and contains no occurrence of
%   |AA|.  If \cs{ldb_expand_it:w} was simply invoked on |...AAA\q_stop|
%   we would find a match where the first matching |A| is the one at the
%   end of the context proper and the second one is at the front of the
%   |AA\q_stop| safeguard which we appended to the context.  This is not
%   a proper match.  In order to prevent this problem, we insert the
%   token \cs{q_mark} at the end of the context.  Thus
%   |\ldb_expand_it:w| is invoked on |...A\q_mark AA\q_stop| and the
%   spurious match can no longer happen.  We must use a token that (1)
%   cannot occur in a context, and (2) is distinct from \cs{q_stop}
%   (otherwise, if the context ended with |AA|, we would match the first
%   occurrence of |AA\q_stop| and leave the second occurrence as
%   trailing garbage).
%    \begin{macrocode}
\cs_new_protected:Npn \ldb_expand_it:w { }
%    \end{macrocode}
% \end{macro}
%
% \begin{macro}{\ldb_expand_next:nnn}
%   Grabs the next abbreviation pair (args 2 and~3) and rewrites the
%   context (1st arg) accordingly.  At this point, we have already put
%   \cs{q_mark} at the end of the context.  An empty abbreviation
%   indicates that we have processed all abbreviations.  We now proceed
%   with the actual parsing task.  Otherwise, we redefine
%   \cs{ldb_expand_it:w} to rewrite every occurrence of the current
%   abbreviation, as described earlier: whenever we find a legitimate
%   occurrence of |#2|, we replace it by |#3| and we repeat the process.
%   When we are done, we proceed with the next abbreviation pair.
%    \begin{macrocode}
\cs_new_protected:Npn \ldb_expand_next:nnn #1#2#3
  {
    \tl_if_empty:nTF {#2}
      { \ldb_parse_it:w #1 }
      {
        \cs_gset:Npn \ldb_expand_it:w ##1 #2 ##2 \q_stop
          {
            \tl_if_empty:nTF {##2}
              { \ldb_expand_next:nnn {##1} }
              { \ldb_expand_it:w ##1 #3 ##2 \q_stop }
          }
        \ldb_expand_it:w #1 #2 \q_stop
      }
  }
%    \end{macrocode}
% \end{macro}
%
% \begin{macro}{\ldb_parse_it:w}
%   The entry point into the actual parser.  Note that we need to remove
%   the \cs{q_mark} token.  We then go on to scan for tight bindings.
%    \begin{macrocode}
\cs_new_protected:Npn \ldb_parse_it:w #1 \q_mark
  { \ldb_parse_dot:w #1 . \q_stop }
%    \end{macrocode}
% \end{macro}
%
% \begin{macro}{\ldb_parse_dot:w}
%   Scan for a tight binding.  The 1st argument is guaranteed not to
%   contain any tight binding since \TeX{} scans for the first
%   occurrence; therefore we scan for other things starting with loose
%   bindings.  Then, if the 2nd arg is not empty, we have found a
%   legitimate occurrence of a tight binding and we call
%   |\ldb_push_dot:| to record that fact --- When the 2nd arg is empty,
%   the dot we just found is the one \cs{ldb_parse_it:w} itself appended
%   to catch the default case and guarantee a match.  Finally we proceed
%   to look for more tight bindings in the 2nd arg.
%    \begin{macrocode}
\cs_new_protected:Npn \ldb_parse_dot:w #1 . #2 \q_stop
  {
    \tl_if_empty:nF {#1} { \ldb_parse_star:w #1 * \q_stop }
    \tl_if_empty:nF {#2} { \ldb_push_dot: \ldb_parse_dot:w #2 \q_stop }
  }
%    \end{macrocode}
% \end{macro}
%
% \begin{macro}{\ldb_parse_star:w}
%   Scan for a loose binding.  In the 1st arg we further look for
%   occurrences of |<|.  If the 2nd arg is not empty, we record
%   finding a legitimate loose binding, then proceed to scan the 2nd
%   arg for more loose bindings.
%    \begin{macrocode}
\cs_new_protected:Npn \ldb_parse_star:w #1 * #2 \q_stop
  {
    \tl_if_empty:nF {#1} { \ldb_parse_begin:w #1 < \q_stop }
    \tl_if_empty:nF {#2} { \ldb_push_star: \ldb_parse_star:w #2 \q_stop }
  }
%    \end{macrocode}
% \end{macro}
%
% \begin{macro}{\ldb_parse_begin:w}
%   Scan for |<|.  In the 1st arg we further look for occurrences of
%   |>|.  If the 2nd arg is not empty, we record finding a legitimate
%   |<|, then proceed to scan the 2nd arg for more occurrences of |<|.
%    \begin{macrocode}
\cs_new_protected:Npn \ldb_parse_begin:w #1 < #2 \q_stop
  {
    \tl_if_empty:nF {#1} { \ldb_parse_end:w #1 > \q_stop }
    \tl_if_empty:nF {#2} { \ldb_push_begin: \ldb_parse_begin:w #2 \q_stop }
  }
%    \end{macrocode}
% \end{macro}
%
% \begin{macro}{\ldb_parse_end:w}
%   Scan for |>|.  If the 1st arg is not empty, it must be a tag
%   identifier, and we record it.  If the 2nd arg is not empty, we
%   record finding a legitimate |>|, then proceed to scan the 2nd arg
%   for more occurrences of |>|.
%    \begin{macrocode}
\cs_new_protected:Npn \ldb_parse_end:w #1 > #2 \q_stop
  {
    \tl_if_empty:nF {#1} { \ldb_push_tag:n {#1} }
    \tl_if_empty:nF {#2} { \ldb_push_end: \ldb_parse_end:w #2 \q_stop }
  }
%    \end{macrocode}
% \end{macro}
%
% \begin{macro}{\ldb_push_bdg:N}
%   Takes a constant integer as an argument.  \cs{c__ldb_tight_bdg} for a tight
%   binding, \cs{c__ldb_loose_bdg} for a loose binding.  The only places a binding
%   is allowed is after |<foo| or |foo|, \emph{i.e.} in mode 3 or~4.  In that
%   case, we flush the pending triple and reset \cs{g_ldb_type} to~0,
%   \cs{g_ldb_bdg} to the given constant, and \cs{g_ldb_mode} to~1 to
%   indicate that we have just seen a binding.
%    \begin{macrocode}
\cs_new_protected:Npn \ldb_push_bdg:N #1
  {
    \if_int_compare:w \g_ldb_mode < \c__ldb_begin_mode
      \ldb_gtmp_error:
    \else:
      \gdef_gtmp:n { \ldb_flush:NNN \c__ldb_begin_type #1 \c__ldb_bdg_mode }
    \fi:
    \gtmp:w
  }
%    \end{macrocode}
% \end{macro}
%
% \begin{macro}{\ldb_push_dot:, \ldb_push_star:}
%    Record a tight or a loose binding.
%    \begin{macrocode}
\cs_new_protected:Npn \ldb_push_dot:  { \ldb_push_bdg:N \c__ldb_tight_bdg }
\cs_new_protected:Npn \ldb_push_star: { \ldb_push_bdg:N \c__ldb_loose_bdg }
%    \end{macrocode}
% \end{macro}
%
% \begin{macro}{\ldb_push_begin:}
%   Record finding an occurrence of |<|.  If we are in mode~2, we have
%   just found two consecutive occurrences of |<|.  This is illegal and
%   we signal an error.  If we were in mode~0 or~1 (\emph{i.e.} expecting or
%   having just found a binding), then we just make a note that we found
%   |<| by proceeding to mode~2.  Otherwise, we must flush the pending
%   triple and then reset \cs{g_ldb_type} to~0 (a begin tag),
%   \cs{g_ldb_bdg} to~0 (sequencing: no intervening binding was found),
%   and \cs{g_ldb_mode} to~2 (we just found |<|).
%    \begin{macrocode}
\cs_new_protected:Npn \ldb_push_begin:
  {
    \if_int_compare:w \g_ldb_mode = \c__ldb_less_mode
      \ldb_gtmp_error:
    \else:
      \if_int_compare:w \g_ldb_mode < \c__ldb_less_mode
        \gdef_gtmp:n { \cs_gset_eq:NN \g_ldb_mode \c__ldb_less_mode }
      \else:
        \gdef_gtmp:n { \ldb_flush:NNN \c__ldb_begin_type \c__ldb_sequenced_bdg \c__ldb_less_mode }
      \fi:
    \fi:
    \gtmp:w
  }
%    \end{macrocode}
% \end{macro}
%
% \begin{macro}{\ldb_push_tag:n}
%   Record finding a tag identifier (the argument).  If we are in mode 3
%   or~4, we just found 2 identifiers in sequence.  This should not be
%   possible, but we signal an error anyway.  Otherwise, we record the
%   identifier in \cs{g_ldb_tag_tl}.  If we are in mode~2, then we
%   proceed to mode~3 to indicate that we have found something of the
%   form |<foo|.  If we are in mode~0 or~1, we proceed directly to
%   mode~4 indicating that the identifier was not preceded by |<|.
%    \begin{macrocode}
\cs_new_protected:Npn \ldb_push_tag:n #1
  {
    \if_int_compare:w \g_ldb_mode > \c__ldb_less_mode
      \exp_after:wN \ldb_error:w
    \else:
      \tl_gset:Nn \g_ldb_tag_tl {#1}
      \if_int_compare:w \g_ldb_mode = \c__ldb_less_mode
        \cs_gset_eq:NN \g_ldb_mode \c__ldb_begin_mode
      \else:
        \cs_gset_eq:NN \g_ldb_mode \c__ldb_ident_mode
      \fi:
    \fi:
  }
%    \end{macrocode}
% \end{macro}
%
% \begin{macro}{\ldb_push_end:}
%   We have found an occurrence of |>| and we should flush the pending
%   triple.  If the mode is less than~3, we have not recently seen an
%   identifier.  Therefore, we signal an error.  Otherwise, if the
%   current mode is~4, indicating that we have seen |<foo|, then we now
%   promote \cs{g_ldb_type} to~2 to indicate a completed tag.  If the
%   current mode is~4, then we promote \cs{g_ldb_type} to~1 to indicate an
%   end tag.  Finally we flush the pending triple and reset
%   \cs{g_ldb_type}, \cs{g_ldb_bdg} and \cs{g_ldb_mode} to~0.
%    \begin{macrocode}
\cs_new_protected:Npn \ldb_push_end:
  {
    \if_int_compare:w \g_ldb_mode < \c__ldb_begin_mode
      \ldb_gtmp_error:
    \else:
      \if_int_compare:w \g_ldb_mode = \c__ldb_begin_mode
        \cs_gset_eq:NN \g_ldb_type \c__ldb_completed_type
      \else:
        \cs_gset_eq:NN \g_ldb_type \c__ldb_end_type
      \fi:
      \gdef_gtmp:n { \ldb_flush:NNN \c__ldb_begin_type \c__ldb_sequenced_bdg \c__ldb_prebdg_mode }
    \fi:
    \gtmp:w
  }
%    \end{macrocode}
% \end{macro}
%
% \begin{macro}{\ldb_flush:NNN}
%   Flush the pending triple.  Calls to this functions are always
%   followed by 3 integer constants which \cs{ldb_reset:NNN} uses to
%   update the values of \cs{g_ldb_type}, \cs{g_ldb_bdg} and
%   \cs{g_ldb_mode}.
%    \begin{macrocode}
\cs_new_protected:Npn \ldb_flush:NNN
  {
%    \end{macrocode}
%   First, we select the appropriate handler according to whether we are
%   flushing a begin tag, an end tag, or a completed tag.
%    \begin{macrocode}
    \if_case:w \g_ldb_type
      \glet_gtmp:N \ldb_push_begin_tag:
    \or:
      \glet_gtmp:N \ldb_push_end_tag:
    \else:
      \glet_gtmp:N \ldb_push_compl_tag:
    \fi:
%    \end{macrocode}
%   A binding, \emph{i.e.} \cs{g_ldb_bdg}~$\neq$~0, is not allowed after
%   a completed tag, or in the final end/begin sequences, \emph{i.e.}
%   \cs{g_ldb_segment}~$>$~1.  If we find such a situation, we must
%   signal an error since only sequencing is permitted.
%    \begin{macrocode}
    \if_int_compare:w \g_ldb_bdg > \c__ldb_sequenced_bdg
      \if_int_compare:w \g_ldb_segment > \c__ldb_main_segment
        \ldb_gtmp_error:
      \fi:
    \fi:
%    \end{macrocode}
%   Push the pending triple onto \cs{g_ldb_triples_tl}, execute the
%   appropriate handler, and reset type, bdg and mode as explained
%   earlier.  Note that \cs{gtmp:w} may be the error handler and that we
%   first push the triple anyway; it doesn't hurt and makes the code
%   simpler.
%    \begin{macrocode}
    \ldb_push: \gtmp:w \ldb_reset:NNN
  }
%    \end{macrocode}
% \end{macro}
%
% \begin{macro}{\ldb_push:}
%   Push a new triple (bdg,type,tag) onto \cs{g_ldb_triples_tl}.  We
%   cannot simply use \cs{g_ldb_bdg} as the 1st element of the triple
%   because its value keeps being changed by the parsing algorithm.  We
%   cannot try to expand it, because it is `let' to a non expandable
%   integer constant.  To get around that problem, we use an
%   \cs{if_case:w} to select the appropriate constant.  Another solution
%   would be to use \cs{g_ldb_bdg_tl} instead of \cs{g_ldb_bdg}, and
%   \cs{tl_gset:Nn} instead of \cs{cs_gset_eq:NN}, but that would be
%   less efficient since every numerical test would have to expand
%   \cs{g_ldb_bdg_tl}.
%    \begin{macrocode}
\cs_new_protected:Npn \ldb_push:
  {
    \tl_gput_right:Nx \g_ldb_triples_tl
      {
        \if_case:w \g_ldb_bdg
               \c__ldb_sequenced_bdg
        \or:   \c__ldb_tight_bdg
        \else: \c__ldb_loose_bdg
        \fi:
        \if_case:w \g_ldb_type
               \c__ldb_begin_type
        \or:   \c__ldb_end_type
        \else: \c__ldb_completed_type
        \fi:
        \exp_not:N \ldb_triple:nNN { \exp_not:o\g_ldb_tag_tl }
      }
  }
%    \end{macrocode}
% \end{macro}
%
% \begin{macro}{\ldb_reset:NNN}
%   Update type, bdg and mode with the given values.  \cs{ldb_reset:NNN}
%   is called at the very end of \cs{ldb_flush:NNN}.  This is why every
%   call to \cs{ldb_flush:NNN} must be followed by 3 integer constants.
%    \begin{macrocode}
\cs_new_protected:Npn \ldb_reset:NNN #1#2#3
  {
    \cs_gset_eq:NN \g_ldb_type #1
    \cs_gset_eq:NN \g_ldb_bdg  #2
    \cs_gset_eq:NN \g_ldb_mode #3
  }
%    \end{macrocode}
% \end{macro}
%
% \begin{macro}{\ldb_push_begin_tag:}
%   We have just flushed a triple involving a begin tag.  If we are
%   still in the main part of the context, that's where we remain, in
%   segment~1.  If we are in the end sequence part of the context,
%   \emph{i.e.}  segment~3, we move on to the begin sequence,
%   \emph{i.e.}  segment~4.  If we are already in segment~4, that's
%   where we stay.
%    \begin{macrocode}
\cs_new_protected:Npn \ldb_push_begin_tag:
  {
    \if_int_compare:w \g_ldb_segment<\c__ldb_end_segment
      \cs_gset_eq:NN \g_ldb_segment \c__ldb_main_segment
    \else:
      \cs_gset_eq:NN \g_ldb_segment \c__ldb_begin_segment
    \fi:
  }
%    \end{macrocode}
% \end{macro}
%
% \begin{macro}{\c_ldb_dash_tl}
%   We use this to compare with \cs{g_ldb_tag_tl} and determine whether
%   the tag identifier is |-|.
%    \begin{macrocode}
\tl_const:Nn \c_ldb_dash_tl { - }
%    \end{macrocode}
% \end{macro}
%
% \begin{macro}{\ldb_push_compl_tag:}
%   We have just flushed a triple involving a completed tag.  A
%   completed tag is only allowed in the main part of the context.
%   Therefore, if we are in segment 3 or~4, we signal an error.  Also we
%   check if the identifier is |-|.  If yes, we signal an error because
%   |<->| is illegal.  Otherwise, we simply update \cs{g_ldb_segment}
%   to~2 to indicate that we are now right after a completed tag.
%    \begin{macrocode}
\cs_new_protected:Npn \ldb_push_compl_tag:
  {
    \if_int_compare:w \g_ldb_segment > \c__ldb_after_segment
      \ldb_gtmp_error:
    \else:
      \if_meaning:w \g_ldb_tag_tl \c_ldb_dash_tl
        \ldb_gtmp_error:
      \else:
        \gdef_gtmp:n { \cs_gset_eq:NN \g_ldb_segment \c__ldb_after_segment }
      \fi:
    \fi:
    \gtmp:w
  }
%    \end{macrocode}
% \end{macro}
%
% \begin{macro}{\ldb_push_end_tag:}
%   We have just flushed a triple involving an end tag.  Normally, we
%   would simply promote \cs{g_ldb_segment} to~3 to indicate that we are
%   now in the end sequence of the context.  However, if we are at the
%   beginning of the context, this end tag should not be preceded by a
%   binding.  Also, if we are in the main part of the context, but not
%   at the beginning, the end tag must be preceded by a loose binding.
%   On the other hand, if the previous tag was a completed tag, we must
%   signal an error since a completed tag cannot be followed by a
%   binding.  If we are in the end sequence, then only sequencing is
%   allowed, and if we are in the begin sequence, only begin tags are
%   allowed.
%    \begin{macrocode}
\cs_new_protected:Npn \ldb_push_end_tag:
  {
    \gdef_gtmp:n { \cs_gset_eq:NN \g_ldb_segment \c__ldb_end_segment }
    \if_case:w\g_ldb_segment
           \if_int_compare:w \g_ldb_bdg = \c__ldb_sequenced_bdg \else:
             \ldb_gtmp_error: \fi:
    \or:   \if_int_compare:w \g_ldb_bdg = \c__ldb_loose_bdg  \else:
             \ldb_gtmp_error: \fi:
    \or:   \ldb_gtmp_error:
    \or:   \if_int_compare:w \g_ldb_bdg = \c__ldb_sequenced_bdg \else:
             \ldb_gtmp_error: \fi:
    \else: \ldb_gtmp_error: \fi:
    \gtmp:w
  }
%    \end{macrocode}
% \end{macro}
%
%    \begin{macrocode}
%</package>
%    \end{macrocode}
%
% \end{implementation}
%
% \endinput
