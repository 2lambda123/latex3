% \iffalse
%
%% File l3fp-functions.dtx (C) Copyright 2012-2018,2020,2021 The LaTeX Project
%
% It may be distributed and/or modified under the conditions of the
% LaTeX Project Public License (LPPL), either version 1.3c of this
% license or (at your option) any later version.  The latest version
% of this license is in the file
%
%    http://www.latex-project.org/lppl.txt
%
% This file is part of the "l3trial bundle" (The Work in LPPL)
% and all files in that bundle must be distributed together.
%
% -----------------------------------------------------------------------
%
% The development version of the bundle can be found at
%
%    https://github.com/latex3/latex3
%
% for those people who are interested.
%
%<*driver|package>
\RequirePackage{expl3}
%</driver|package>
%<*driver>
\documentclass[full]{l3doc}
\usepackage{amsmath}
\begin{document}
  \DocInput{\jobname.dtx}
\end{document}
%</driver>
% \fi
%
% \title{^^A
%   The \pkg{l3fp-functions} package\\ Floating point functions^^A
% }
%
% \author{^^A
%  The \LaTeX{} Project\thanks
%    {^^A
%      E-mail:
%        \href{mailto:latex-team@latex-project.org}
%          {latex-team@latex-project.org}^^A
%    }^^A
% }
%
% \date{Released 2018-04-30}
%
% \maketitle
%
% \begin{documentation}
%
% \end{documentation}
%
% \begin{implementation}
%
% \section{\pkg{l3fp-functions} implementation}
%
%    \begin{macrocode}
%<*package>
%    \end{macrocode}
%
%    \begin{macrocode}
%<@@=fp>
%    \end{macrocode}
%
% \subsection{Previously in \pkg{l3fp}: defining new \pkg{l3fp} functions}
%
% \begin{macro}[EXP]{\fp_function:Nw}
%   Parse the argument of the function~|#1| using
%   \cs{@@_parse_operand:Nw} with a precedence of~$16$, and pass the
%   function and argument to \cs{@@_function_apply:nw}.
%    \begin{macrocode}
\cs_new:Npn \fp_function:Nw #1
  {
    \exp_after:wN \@@_function_apply:nw
    \exp_after:wN #1
    \exp:w
      \@@_parse_operand:Nw \c_@@_prec_func_int \@@_parse_expand:w
  }
%    \end{macrocode}
% \end{macro}
%
% \begin{macro}{\fp_new_function:Npn}
% \begin{macro}{\@@_new_function:NNnnn, \@@_new_function:Ncfnn}
% \begin{macro}{\@@_function_args:Nwn}
%   Save the code provided by the user in the control sequence
%   |\__fp_user_#1|.  Define |#1| to call \cs{@@_function_apply:nw}
%   after parsing one operand using \cs{@@_parse_operand:Nw} with
%   precedence~$16$.  The auxiliary \cs{@@_function_args:Nwn} receives
%   the user function and the number of arguments (half of the number of
%   tokens in the parameter text~|#2|), followed by the operand (as a
%   token list of floating points).  It checks the number of arguments,
%   and applies the user function to the arguments (without the outer
%   brace group).
%    \begin{macrocode}
\cs_new_protected:Npn \fp_new_function:Npn #1#2#
  {
    \@@_new_function:Ncfnn #1
      { @@_user_ \cs_to_str:N #1 }
      { \int_eval:n { \tl_count:n {#2} / 2 } }
      {#2}
  }
\cs_new_protected:Npn \@@_new_function:NNnnn #1#2#3#4#5
  {
    \cs_new:Npn #1
      {
        \exp_after:wN \@@_function_apply:nw \exp_after:wN
          {
            \exp_after:wN \@@_function_args:Nwn
            \exp_after:wN #2
            \int_value:w #3 \exp_after:wN ; \exp_after:wN
          }
        \exp:w
          \@@_parse_operand:Nw \c_@@_prec_func_int \@@_parse_expand:w
      }
    \cs_new:Npn #2 #4 {#5}
  }
\cs_generate_variant:Nn \@@_new_function:NNnnn { Ncf }
\cs_new:Npn \@@_function_args:Nwn #1#2; #3
  {
    \int_compare:nNnTF { \tl_count:n {#3} } = {#2}
      { #1 #3 }
      {
        \__kernel_msg_expandable_error:nnnnn
          { fp } { num-args } { #1() } {#2} {#2}
        \c_nan_fp
      }
  }
%    \end{macrocode}
% \end{macro}
% \end{macro}
% \end{macro}
%
% \begin{macro}[EXP]
%   {
%     \@@_function_apply:nw,
%     \@@_function_store:wwNwnn,
%     \@@_function_store_end:wnnn
%   }
%   The auxiliary \cs{@@_function_apply:nw} is called after parsing an
%   operand, so it receives some code~|#1|, then the operand ending
%   with~|@|, then a function such as \cs{@@_parse_infix_+:N} (but not
%   always of this form, see comparisons for instance).  Package the
%   operand (an array) into a token list with floating point items: this
%   is the role of \cs{@@_function_store:wwNwnn} and
%   \cs{@@_function_store_end:wnnn}.  Then apply \cs{@@_parse:n} to the
%   code~|#1| followed by a brace group with this token list.  This
%   results in a floating point result, which is then correctly parsed
%   as the next operand of whatever was looking for one.  The trailing
%   \cs{s_@@_expr_mark} is used as a special infix operator to indicate
%   that the next token has already gone through \cs{@@_parse_infix:NN}.
%    \begin{macrocode}
\cs_new:Npn \@@_function_apply:nw #1#2 @
  {
    \@@_parse:n
      {
        \@@_function_store:wwNwnn #2
          \s_@@_expr_mark \@@_function_store:wwNwnn ;
          \s_@@_expr_mark \@@_function_store_end:wnnn
        \s_@@_expr_stop { } { } {#1}
      }
    \s_@@_expr_mark
  }
\cs_new:Npn \@@_function_store:wwNwnn
    #1; #2 \s_@@_expr_mark #3#4 \s_@@_expr_stop #5#6
  { #3 #2 \s_@@_expr_mark #3#4 \s_@@_expr_stop { #5 #6 } { { #1; } } }
\cs_new:Npn \@@_function_store_end:wnnn
    #1 \s_@@_expr_stop #2#3#4
  { #4 {#2} }
%    \end{macrocode}
% \end{macro}
%
% \subsection{Declaring functions}
%
% \begin{macro}{\fp_new_function:n}
% \begin{macro}[aux]{\@@_new_function:n}
%    \begin{macrocode}
\cs_new_protected:Npn \fp_new_function:n #1
  { \exp_args:No \@@_new_function:n { \tl_to_str:n {#1} } }
\cs_new_protected:Npn \@@_new_function:n #1
  {
    \@@_id_if_invalid:nTF {#1}
      { \msg_error:nnn { fp } { invalid-identifier } {#1} }
      {
        \cs_if_exist:cT { @@_parse_word_#1:N }
          {
            \msg_error:nnn
              { fp } { id-already-defined } {#1}
            \cs_undefine:c { @@_parse_word_#1:N }
            \cs_undefine:c { @@_#1_o:w }
          }
        \@@_function_set_parsing:Nn \cs_gset_eq:NN {#1}
      }
  }
%    \end{macrocode}
% \end{macro}
% \end{macro}
%
% \begin{macro}[aux]
%   {\@@_function_set_parsing:Nn, \@@_function_set_parsing_aux:NNn}
%    \begin{macrocode}
\cs_new:Npn \@@_function_set_parsing:Nn #1#2
  {
    \exp_args:NNc \@@_function_set_parsing_aux:NNn #1
      { @@_parse_word_#2:N } {#2}
  }
\cs_new:Npn \@@_function_set_parsing_aux:NNn #1#2#3
  {
    \cs_set:Npx \@@_tmp:w
      {
        \exp_not:N \@@_parse_function:NNN
        \exp_not:N \@@_function_o:w
        \exp_not:c { @@_#3_o:w }
      }
    \cs_if_eq:NNF #2 \@@_tmp:w
      {
        \cs_if_exist:NTF #2
          {
            \msg_warning:nnnn
              { fp } { id-used-elsewhere } {#3} { function }
            #1 #2 \@@_tmp:w
          }
          {
            \cs_new_eq:NN #2 \scan_stop: % to declare the function
            #1 #2 \@@_tmp:w
          }
      }
  }
%    \end{macrocode}
% \end{macro}
%
% \begin{macro}[aux, EXP]{\@@_function_o:w}
%    \begin{macrocode}
\cs_new:Npn \@@_function_o:w #1#2 @
  {
    \cs_if_exist:NTF #1
      { #1 #2 @ }
      {
        \exp_after:wN \s_@@_symbolic
        \exp_after:wN \@@_symbolic_chk:w
        \exp_after:wN \@@_function_o:w
        \exp_after:wN #1
        \exp_after:wN ,
        \exp_after:wN {
          \exp:w \exp_end_continue_f:w
          \@@_exp_after_array_f:w #2 \s_@@_expr_stop
          \exp_after:wN
        }
        \exp_after:wN ;
      }
  }
%    \end{macrocode}
% \end{macro}
%
% \subsection{Defining functions by their expression}
%
% \begin{variable}{\l_@@_function_arg_int}
%   Labels the arguments of a function being defined.
%    \begin{macrocode}
\int_new:N \l_@@_function_arg_int
%    \end{macrocode}
% \end{variable}
%
% \begin{macro}{\fp_set_function:nnn}
% \begin{macro}[aux]{\@@_set_function:Nnnn}
%   \begin{syntax}
%     \cs{fp_set_function:nnn} \Arg{identifier}
%       \Arg{comma-list of variables} \Arg{expression}
%   \end{syntax}
%   Defines the \meta{identifier} to stand for a function which expects
%   some arguments defined by the \meta{comma-list of variables}, and
%   evaluates to the \meta{expression}.
%    \begin{macrocode}
\cs_new_protected:Npn \fp_set_function:nnn #1
  {
    \exp_args:NNo \@@_set_function:Nnnn \cs_set_eq:cN
      { \tl_to_str:n {#1} }
  }
\cs_new_protected:Npn \@@_set_function:Nnnn #1#2#3#4
  {
    \@@_id_if_invalid:nTF {#2}
      { \msg_error:nnn { fp } { invalid-identifier } {#2} }
      {
        \cs_if_exist:cF { @@_parse_word_#2:N }
          { \@@_function_set_parsing:Nn \cs_set_eq:NN {#2} }
        \group_begin:
          \int_zero:N \l_@@_function_arg_int
          \exp_args:No \clist_map_inline:nn { \tl_to_str:n {#3} }
            {
              \int_incr:N \l_@@_function_arg_int
              \exp_args:Nx \@@_clear_variable:n
                { _ \tex_romannumeral:D \l_@@_function_arg_int }
              \fp_clear_variable:n {##1}
              \cs_set_nopar:cpx { l_@@_variable_##1_fp }
                {
                  \exp_not:N \s__fp_symbolic
                  \exp_not:N \@@_symbolic_chk:w
                  \exp_not:N \@@_function_arg_o:w
                  \int_use:N \l_@@_function_arg_int
                  ########1 , { } ;
                }
            }
          \cs_set:Npn \@@_function_arg_o:w ##1 @
            {
              \exp_after:wN \s_@@_symbolic
              \exp_after:wN \@@_symbolic_chk:w
              \exp_after:wN \@@_function_arg_o:w
              \tex_romannumeral:D
              \@@_exp_after_symbolic_loop:N ##1
                { , \tex_romannumeral:D \use_none:nn }
              \exp_after:wN \c_zero_int
              \exp_after:wN { \exp_after:wN } \exp_after:wN ;
            }
          \fp_set:Nn \l_@@_symbolic_fp {#4}
          \use:x
            {
              \exp_not:n { \cs_gset:Npn \@@_tmp:w ##1 }
                { \exp_not:o { \l_@@_symbolic_fp } }
            }
          \use:x
            {
              \exp_not:n { \cs_gset:Npn \@@_tmp:w ##1 @ }
                {
                  \exp_not:N \@@_exp_after_symbolic_f:nw
                  \exp_not:n { { \exp_after:wN \exp_stop_f: } }
                  \exp_not:o { \@@_tmp:w { . , {##1} } }
                }
            }
        \group_end:
        #1 { @@_#2_o:w } \@@_tmp:w
      }
  }
%    \end{macrocode}
% \end{macro}
%
% \begin{macro}[aux, EXP]
%   {
%     \@@_function_arg_o:w,
%     \@@_function_arg_few:w,
%     \@@_function_arg_get:w
%   }
%    \begin{macrocode}
\cs_new:Npn \@@_function_arg_o:w #1. #2
  {
    \if_meaning:w @ #2
      \exp_after:wN \@@_function_arg_few:w
    \fi:
    \if_int_compare:w #1 = \c_one_int
      \exp_after:wN \@@_function_arg_get:w
    \fi:
    \@@_use_i_until_s:nw
      {
        \exp_after:wN \@@_function_arg_o:w
        \int_value:w \int_eval:n { #1 - 1 } .
      }
      #2
  }
\cs_new:Npn \@@_function_arg_few:w #1 @ { \exp_after:wN \c_nan_fp }
\cs_new:Npn \@@_function_arg_get:w #1#2#3; #4 @
  {
    \@@_exp_after_array_f:w #3; \s_@@_expr_stop
    \exp_after:wN \exp_stop_f:
  }
%    \end{macrocode}
% \end{macro}
% \end{macro}
%
% ^^A todo: add check for number of args
%
%    \begin{macrocode}
%</package>
%    \end{macrocode}
%
% \end{implementation}
%
% \PrintIndex
