% \iffalse
%
%% File l3fp-symbolic.dtx (C) Copyright 2012-2015,2017,2018,2020 The LaTeX3 Project
%
% It may be distributed and/or modified under the conditions of the
% LaTeX Project Public License (LPPL), either version 1.3c of this
% license or (at your option) any later version.  The latest version
% of this license is in the file
%
%    http://www.latex-project.org/lppl.txt
%
% This file is part of the "l3trial bundle" (The Work in LPPL)
% and all files in that bundle must be distributed together.
%
% -----------------------------------------------------------------------
%
% The development version of the bundle can be found at
%
%    https://github.com/latex3/latex3
%
% for those people who are interested.
%
%<*driver|package>
\RequirePackage{expl3}
%</driver|package>
%<*driver>
\documentclass[full]{l3doc}
\usepackage{amsmath}
\begin{document}
  \DocInput{\jobname.dtx}
\end{document}
%</driver>
% \fi
%
% \title{^^A
%   The \pkg{l3fp-symbolic} package\\ Symbolic expressions^^A
% }
%
% \author{^^A
%  The \LaTeX3 Project\thanks
%    {^^A
%      E-mail:
%        \href{mailto:latex-team@latex-project.org}
%          {latex-team@latex-project.org}^^A
%    }^^A
% }
%
% \date{Released 2018-04-30}
%
% \maketitle
%
% \begin{documentation}
%
% \end{documentation}
%
% \begin{implementation}
%
% \section{\pkg{l3fp-symbolic} implementation}
%
%    \begin{macrocode}
%<*package>
%    \end{macrocode}
%
%    \begin{macrocode}
%<@@=fp>
%    \end{macrocode}
%
% \subsection{Misc}
%
% \begin{variable}[int]{\l_@@_symbolic_fp}
%   Scratch floating point.
%    \begin{macrocode}
\fp_new:N \l_@@_symbolic_fp
%    \end{macrocode}
% \end{variable}
%
% \begin{macro}[EXP]{\use:ff}
%   A useful variant.
%    \begin{macrocode}
\cs_generate_variant:Nn \use:nn { ff }
%    \end{macrocode}
% \end{macro}
%
% \subsection{Building blocks for expressions}
%
% Every symbolic expression has the form \cs{s_@@_symbolic}
% \cs{@@_symbolic_chk:w} \meta{operation} |,| \Arg{operands} \meta{junk}
% |;| where the \meta{operation} is a list of \texttt{N}-type tokens,
% the \meta{operands} is an array of floating point objects, and the
% \meta{junk} is to be discarded.  If the outermost operator (last to be
% evaluated) is unary, the expression has the form
% \begin{quote}
%   \cs{s_@@_symbolic} \cs{@@_symbolic_chk:w} \\
%   \cs{@@_types_unary:NNw} |\__fp_|\meta{op}|_o:w| \meta{token} |,| \\
%   |{| \meta{operand} |}| \meta{junk} |;|
% \end{quote}
% where the \meta{op} is a unary operation (|set_sign|, |cos|,
% \ldots{}), and the \meta{token} and \meta{operand} are used as
% arguments for \cs{@@_\meta{op}_o:w} (or the type-specific analog of
% this function).  If the outermost operator is binary, the expression
% has the form
% \begin{quote}
%   \cs{s_@@_symbolic} \cs{@@_symbolic_chk:w} \\
%   \cs{@@_types_binary:Nww} |\__fp_|\meta{op}|_o:ww| |,| \\
%   |{| \meta{operand_1} \meta{operand_2} |}| \meta{junk} |;|
% \end{quote}
% where the \meta{op} is an operation (|+|, |&|, \ldots{}), and
% |\__fp_|\meta{op}|_o:ww| receives the \meta{operands} as arguments.
% If the expression consists of a single variable, it is stored as
% \begin{quote}
%   \cs{s_@@_symbolic} \cs{@@_symbolic_chk:w} \\
%   \cs{@@_variable_o:w} \meta{identifier} |,| \\
%   |{| |}| \meta{junk} |;|
% \end{quote}
%
% Symbolic expressions are stored in a prefix form.  When encountering a
% symbolic expression in a floating point computation, we attempt to
% evaluate the operands as much as possible, and if that yields floating
% point numbers rather than expressions, we apply the operator which
% follows (if the function is known).
%
% For instance, the expression |a + b * sin(c)| is stored as
% \begin{verbatim}
% \s__fp_symbolic \__fp_symbolic_chk:w
%   \__fp_types_binary:Nww \__fp_+_o:ww ,
%   {
%     \s__fp_symbolic \__fp_symbolic_chk:w
%       \__fp_variable_o:w a , { } ;
%     \s__fp_symbolic \__fp_symbolic_chk:w
%       \__fp_types_binary:Nww \__fp_*_o:ww ,
%       {
%         \s__fp_symbolic \__fp_symbolic_chk:w
%           \__fp_variable_o:w b , { } ;
%         \s__fp_symbolic \__fp_symbolic_chk:w
%           \__fp_types_unary:NNw \__fp_sin_o:w \use_i:nn ,
%           {
%             \s__fp_symbolic \__fp_symbolic_chk:w
%               \__fp_variable_o:w c , { } ;
%           } ;
%       } ;
%   } ;
% \end{verbatim}
%
% \begin{variable}[int]{\s_@@_symbolic}
%   Scan mark indicating the start of a symbolic expression.
%    \begin{macrocode}
\scan_new:N \s_@@_symbolic
%    \end{macrocode}
% \end{variable}
%
% \begin{macro}[int]{\@@_symbolic_chk:w}
%   Analog of \cs{@@_chk:w} for symbolic expressions.
%    \begin{macrocode}
\cs_new_protected:Npn \@@_symbolic_chk:w #1,#2#3;
  {
    \__kernel_msg_error:nnx { kernel } { misused-fp }
      {
        \@@_to_tl_dispatch:w
          \s_@@_symbolic \@@_symbolic_chk:w #1,{#2};
      }
  }
%    \end{macrocode}
% \end{macro}
%
% \subsection{Expanding after a symbolic expression}
%
% \begin{macro}[aux, EXP]
%   {\@@_if_has_symbolic:nTF, \@@_if_has_symbolic_aux:w}
%   Tests if |#1| contains \cs{s_@@_symbolic} at top-level.  This test
%   should be precise enough to determine if a given an array contains a
%   symbolic expression or only consists in floating points.  Used in
%   \cs{@@_exp_after_symbolic_f:nw}.
%    \begin{macrocode}
\cs_new:Npn \@@_if_has_symbolic:nTF #1
  {
    \@@_if_has_symbolic_aux:w
      #1             \s_@@_mark \use_i:nn
      \s_@@_symbolic \s_@@_mark \use_ii:nn
    \s_@@_stop
  }
\cs_new:Npn \@@_if_has_symbolic_aux:w
    #1 \s_@@_symbolic #2 \s_@@_mark #3#4 \s_@@_stop { #3 }
%    \end{macrocode}
% \end{macro}
%
% \begin{macro}[int, EXP]{\@@_exp_after_symbolic_f:nw}
% \begin{macro}[aux, EXP]
%   {\@@_exp_after_symbolic_aux:w, \@@_exp_after_symbolic_loop:N}
%   This function does two things: trigger an \texttt{f}-expansion of
%   the argument~|#1| after the following symbolic expression, and
%   evaluate all pieces of the expression which can be evaluated.
%    \begin{macrocode}
\cs_new:Npn \@@_exp_after_symbolic_f:nw
    #1 \s_@@_symbolic \@@_symbolic_chk:w #2, #3#4;
  {
    \exp_after:wN \@@_exp_after_symbolic_aux:w
    \exp:w
    \@@_exp_after_symbolic_loop:N #2
      { , \exp:w \use_none:nn }
    \exp_after:wN \exp_end: \exp_after:wN
      {
        \exp:w \exp_end_continue_f:w
        \@@_exp_after_array_f:w #3 \s_@@_expr_stop
        \exp_after:wN
      }
    \exp_after:wN ;
    \exp:w \exp_end_continue_f:w #1
  }
\cs_new:Npn \@@_exp_after_symbolic_aux:w #1, #2;
  {
    \@@_if_has_symbolic:nTF {#2}
      { \s_@@_symbolic \@@_symbolic_chk:w #1, {#2} ; }
      { #1 #2 @ \prg_do_nothing: }
  }
\cs_new:Npn \@@_exp_after_symbolic_loop:N #1
  {
    \exp_after:wN \exp_end:
    \exp_after:wN #1
    \exp:w
    \@@_exp_after_symbolic_loop:N
  }
%    \end{macrocode}
% \end{macro}
% \end{macro}
%
% \subsection{Applying infix operators to expressions}
%
% \begin{macro}[aux, EXP]{\@@_symbolic_binary_o:Nww}
%   Used when applying infix operators to expressions.
%    \begin{macrocode}
\cs_new:Npn \@@_symbolic_binary_o:Nww #1 #2; #3;
  {
    \@@_exp_after_symbolic_f:nw { \exp_after:wN \exp_stop_f: }
      \s_@@_symbolic \@@_symbolic_chk:w
      \@@_types_binary:Nww #1 , { #2; #3; } ;
  }
%    \end{macrocode}
% \end{macro}
%
% \begin{@makeother}{^} %^^A Hack!
% \begin{@makeother}{|} %^^A Hack!
% \begin{@makeother}{&} %^^A Hack!
% \begin{macro}[EXP, aux]
%   {
%     \@@_symbolic_+_symbolic_o:ww,
%     \@@_symbolic_+_o:ww,
%     \@@_+_symbolic_o:ww,
%     \@@_symbolic_-_symbolic_o:ww,
%     \@@_symbolic_-_o:ww,
%     \@@_-_symbolic_o:ww,
%     \@@_symbolic_*_symbolic_o:ww,
%     \@@_symbolic_*_o:ww,
%     \@@_*_symbolic_o:ww,
%     \@@_symbolic_/_symbolic_o:ww,
%     \@@_symbolic_/_o:ww,
%     \@@_/_symbolic_o:ww,
%     \@@_symbolic_^_symbolic_o:ww,
%     \@@_symbolic_^_o:ww,
%     \@@_^_symbolic_o:ww,
%     \@@_symbolic_|_symbolic_o:ww,
%     \@@_symbolic_|_o:ww,
%     \@@_|_symbolic_o:ww,
%     \@@_symbolic_&_symbolic_o:ww,
%     \@@_symbolic_&_o:ww,
%     \@@_&_symbolic_o:ww,
%   }
%    \begin{macrocode}
\cs_set:Npn \@@_tmp:w #1#2
  {
    \cs_new_nopar:cpn
      { @@_symbolic_#2_symbolic_o:ww }
      { \@@_symbolic_binary_o:Nww #1 }
    \cs_new_eq:cc
      { @@_symbolic_#2         _o:ww }
      { @@_symbolic_#2_symbolic_o:ww }
    \cs_new_eq:cc
      { @@         _#2_symbolic_o:ww }
      { @@_symbolic_#2_symbolic_o:ww }
  }
\tl_map_inline:nn { + - * / ^ & | }
  { \exp_args:Nc \@@_tmp:w { @@_#1_o:ww } {#1} }
%    \end{macrocode}
% \end{macro}
% \end{@makeother}
% \end{@makeother}
% \end{@makeother}
%
% \begin{macro}[EXP, aux]
%   {
%     \@@_symbolic_compare_symbolic_o:NNNNww,
%     \@@_symbolic_compare_o:NNNNww,
%     \@@_compare_symbolic_o:NNNNww,
%   }
%    \begin{macrocode}
 % \cs_new:Npn \@@_symbolic_compare_symbolic_o:NNNNww #1#2#3#4 #5; #6;
 %   {
 %     \str_case:nnF { #1#2#3#4 }
 %       {
 %         { \c_zero_fp \c_zero_fp \c_zero_fp \c_zero_fp }
 %           { \exp_after:wN \c_zero_fp }
 %         { \c_one_fp \c_one_fp \c_one_fp \c_one_fp }
 %           { \exp_after:wN \c_one_fp }
 %       }
 %       {
 %         \@@_exp_after_symbolic_f:nw { \exp_after:wN \exp_stop_f: }
 %           \s_@@_symbolic \@@_symbolic_chk:w \@@_infix_compare_o:w
 %             {
 %               \s_@@_tokens \@@_tokens_exp_not:n { #1#2#3#4 } ;
 %               #5 ;
 %               #6 ;
 %             } ;
 %       }
 %   }
 % \cs_new_eq:NN \@@_symbolic_compare_o:NNNNww
 %   \@@_symbolic_compare_symbolic_o:NNNNww
 % \cs_new_eq:NN \@@_compare_symbolic_o:NNNNww
 %   \@@_symbolic_compare_symbolic_o:NNNNww
%    \end{macrocode}
% \end{macro}
%
% \subsection{Applying prefix functions to expressions}
%
% \begin{macro}[aux, EXP]{\@@_symbolic_unary_o:NNw}
%   Used when applying infix operators to expressions.
%    \begin{macrocode}
\cs_new:Npn \@@_symbolic_unary_o:NNw #1#2#3; @
  {
    \@@_exp_after_symbolic_f:nw { \exp_after:wN \exp_stop_f: }
      \s_@@_symbolic \@@_symbolic_chk:w
      \@@_types_unary:NNw #1#2 , { #3; } ;
  }
%    \end{macrocode}
% \end{macro}
%
% \begin{macro}[EXP, aux]
%   {
%     \@@_symbolic_acos_o:w     ,
%     \@@_symbolic_acsc_o:w     ,
%     \@@_symbolic_asec_o:w     ,
%     \@@_symbolic_asin_o:w     ,
%     \@@_symbolic_cos_o:w      ,
%     \@@_symbolic_cot_o:w      ,
%     \@@_symbolic_csc_o:w      ,
%     \@@_symbolic_exp_o:w      ,
%     \@@_symbolic_ln_o:w       ,
%     \@@_symbolic_not_o:w      ,
%     \@@_symbolic_sec_o:w      ,
%     \@@_symbolic_set_sign_o:w ,
%     \@@_symbolic_sin_o:w      ,
%     \@@_symbolic_tan_o:w      ,
%   }
%    \begin{macrocode}
\tl_map_inline:nn
  {
    {acos} {acsc} {asec} {asin} {cos} {cot} {csc} {exp} {ln}
    {not} {sec} {set_sign} {sin} {sqrt} {tan}
  }
  {
    \cs_new_nopar:cpx { @@_symbolic_#1_o:w }
      {
        \exp_not:N \@@_symbolic_unary_o:NNw
        \exp_not:c { @@_#1_o:w }
      }
  }
%    \end{macrocode}
% \end{macro}
%
% \subsection{Conversions}
%
% \begin{macro}[EXP, int]
%   {
%     \@@_symbolic_to_decimal:w,
%     \@@_symbolic_to_int:w,
%     \@@_symbolic_to_scientific:w
%   }
% \begin{macro}[EXP, aux]{\@@_symbolic_convert:wnnN}
%   Symbolic expressions cannot be converted to decimal, integer, or
%   scientific notation unless they can be reduced to
%    \begin{macrocode}
\cs_set_protected:Npn \@@_tmp:w #1#2#3
  {
    \cs_new_nopar:cpn { @@_symbolic_to_#1:w }
      {
        \exp_after:wN \@@_symbolic_convert:wnnN
        \exp:w \exp_end_continue_f:w
        \@@_exp_after_symbolic_f:nw { { #2 } { fp_to_#1 } #3 }
      }
  }
\@@_tmp:w { decimal    } { 0   } \@@_to_decimal_dispatch:w
\@@_tmp:w { int        } { 0   } \@@_to_int_dispatch:w
\@@_tmp:w { scientific } { nan } \@@_to_scientific_dispatch:w
\cs_new:Npn \@@_symbolic_convert:wnnN #1#2; #3#4#5
  {
    \str_if_eq:nnTF {#1} { \s_@@_symbolic }
      { \@@_invalid_operation:nnw {#3} {#4} #1#2; }
      { #5 #1#2; }
  }
%    \end{macrocode}
% \end{macro}
% \end{macro}
%
% \begin{macro}[EXP, aux]
%   {\@@_symbolic_cs_arg_to_fn:NN, \@@_symbolic_op_arg_to_fn:nN}
%    \begin{macrocode}
\cs_new:Npn \@@_symbolic_cs_arg_to_fn:NN #1
  {
    \exp_args:Nf \@@_symbolic_op_arg_to_fn:nN
      { \@@_types_cs_to_op:N #1 }
  }
\cs_new:Npn \@@_symbolic_op_arg_to_fn:nN #1#2
  {
    \str_case:nnF { #1 #2 }
      {
        { not ? } { ! }
        { set_sign 0 } { abs }
        { set_sign 2 } { - }
      }
      {
        \token_if_eq_meaning:NNTF #2 \use_ii:nn
          { #1 d } {#1}
      }
  }
%    \end{macrocode}
% \end{macro}
%
% \begin{macro}[EXP, int]{\@@_symbolic_to_tl:w}
% \begin{macro}[rEXP, aux]
%   {
%     \@@_symbolic_unary_to_tl:NNw,
%     \@@_symbolic_binary_to_tl:Nww
%   }
%   Converting a symbolic expression to a token list is possible.
%    \begin{macrocode}
\cs_new:Npn \@@_symbolic_to_tl:w
    \s_@@_symbolic \@@_symbolic_chk:w #1#2, #3#4;
  {
    \str_case:nnTF {#1}
      {
        { \@@_types_unary:NNw } { \@@_symbolic_unary_to_tl:NNw }
        { \@@_types_binary:Nww } { \@@_symbolic_binary_to_tl:Nww }
        { \@@_function_o:w } { \@@_symbolic_function_to_tl:Nw }
      }
      { #2, #3 @ }
      { \tl_to_str:n {#2} }
  }
\cs_new:Npn \@@_symbolic_unary_to_tl:NNw #1#2 , #3 @
  {
    \use:ff
      { \@@_symbolic_cs_arg_to_fn:NN #1#2 }
      { \exp_last_unbraced:Nf ( { \@@_to_tl_dispatch:w #3 } ) }
  }
\cs_new:Npn \@@_symbolic_binary_to_tl:Nww #1, #2; #3; @
  {
    \use:ff
      { \exp_last_unbraced:Nf ( { \@@_to_tl_dispatch:w #2; } ) }
      {
        \use:ff
          { \@@_types_cs_to_op:N #1 }
          { \exp_last_unbraced:Nf ( { \@@_to_tl_dispatch:w #3; } ) }
      }
  }
\cs_new:Npn \@@_symbolic_function_to_tl:Nw #1, #2@
  {
    \use:ff { \@@_types_cs_to_op:N #1 }
      { \exp_last_unbraced:Nf ( { \@@_array_to_clist:n {#2} } ) }
  }
%    \end{macrocode}
% \end{macro}
% \end{macro}
%
% \subsection{Identifiers}
%
% Functions defined here are not necessarily tied to symbolic
% expressions.
%
% \begin{macro}[TF]{\@@_id_if_invalid:n}
% \begin{macro}[aux, EXP]{\@@_id_if_invalid_aux:N}
%   If |#1| contains a space, it is not a valid identifier.  Otherwise,
%   loop through letters in |#1|: if it is not a letter, break the loop
%   and return \texttt{true}.  If the end of the loop is reached
%   without finding any non-letter, return \texttt{false}.
%    \begin{macrocode}
\prg_new_protected_conditional:Npnn
    \@@_id_if_invalid:n #1 { T , F , TF }
  {
    \tl_if_empty:nTF {#1}
      { \prg_return_true: }
      {
        \tl_if_in:onTF { \tl_to_str:n {#1} } { ~ }
          { \prg_return_true: }
          {
            \exp_after:wN \@@_id_if_invalid_aux:N \tl_to_str:n {#1}
              { ? \prg_break:n \prg_return_false: }
            \prg_break_point:
          }
      }
  }
\cs_new:Npn \@@_id_if_invalid_aux:N #1
  {
    \use_none:n #1
    \int_compare:nF { `a <= `#1 <= `z }
      {
        \int_compare:nF { `A <= `#1 <= `Z }
          { \prg_break:n \prg_return_true: }
      }
    \@@_id_if_invalid_aux:N
  }
%    \end{macrocode}
% \end{macro}
% \end{macro}
%
% \subsection{Declaring variables and assigning values}
%
% \begin{macro}[aux, EXP]{\@@_variable_o:w}
%   We do not use \cs{exp_last_unbraced:Nv} to extract the value of
%   |\l__fp_variable_#1_fp| because in \cs{fp_set_variable:nn} we define
%   this \texttt{fp} variable to be something which \texttt{f}-expands
%   to an actual floating point, rather than a genuine floating point.
%    \begin{macrocode}
\cs_new:Npn \@@_variable_o:w #1 @ #2
  {
    \fp_if_exist:cTF { l_@@_variable_#1_fp }
      {
        \exp_last_unbraced:Nf \@@_exp_after_array_f:w
          { \use:c { l_@@_variable_#1_fp } } \s_@@_expr_stop
        \exp_after:wN \exp_stop_f: #2
      }
      {
        \token_if_eq_meaning:NNTF #2 \prg_do_nothing:
          {
            \s_@@_symbolic \@@_symbolic_chk:w
              \@@_variable_o:w #1 , { } ;
          }
          {
            \exp_after:wN \s_@@_symbolic
            \exp_after:wN \@@_symbolic_chk:w
            \exp_after:wN \@@_variable_o:w
            \exp:w
            \@@_exp_after_symbolic_loop:N #1
              { , \exp:w \use_none:nn }
            \exp_after:wN \exp_end:
            \exp_after:wN { \exp_after:wN } \exp_after:wN ;
            #2
          }
      }
  }
%    \end{macrocode}
% \end{macro}
%
% \begin{macro}[aux]
%   {\@@_variable_set_parsing:Nn, \@@_variable_set_parsing_aux:NNn}
%    \begin{macrocode}
\cs_new:Npn \@@_variable_set_parsing:Nn #1#2
  {
    \cs_set_nopar:Npn \@@_tmp:w
      {
        \@@_exp_after_symbolic_f:nw { \@@_parse_infix:NN }
        \s_@@_symbolic \@@_symbolic_chk:w
          \@@_variable_o:w #2 , { } ;
      }
    \exp_args:NNc \@@_variable_set_parsing_aux:NNn #1
      { @@_parse_word_#2:N } {#2}
  }
\cs_new:Npn \@@_variable_set_parsing_aux:NNn #1#2#3
  {
    \cs_if_eq:NNF #2 \@@_tmp:w
      {
        \cs_if_exist:NTF #2
          {
            \__kernel_msg_warning:nnnn
              { kernel } { fp-id-used-elsewhere } {#3} { variable }
            #1 #2 \@@_tmp:w
          }
          {
            \cs_new_eq:NN #2 \scan_stop: % to declare the function
            #1 #2 \@@_tmp:w
          }
      }
  }
%    \end{macrocode}
% \end{macro}
%
% \begin{macro}{\fp_clear_variable:n}
% \begin{macro}[aux]{\@@_clear_variable:n}
%    \begin{macrocode}
\cs_new_protected:Npn \fp_clear_variable:n #1
  {
    \@@_id_if_invalid:nTF {#1}
      { \__kernel_msg_error:nnn { kernel } { fp-id-invalid } {#1} }
      { \exp_args:No \@@_clear_variable:n { \tl_to_str:n {#1} } }
  }
\cs_new_protected:Npn \@@_clear_variable:n #1
  {
    \cs_undefine:c { l_@@_variable_#1_fp }
    \@@_variable_set_parsing:Nn \cs_set_eq:NN {#1}
  }
%    \end{macrocode}
% \end{macro}
% \end{macro}
%
% \begin{macro}{\fp_new_variable:n}
% \begin{macro}[aux]{\@@_new_variable:n}
%   Check that |#1| is a valid identifier.  If the identifier is already
%   in use, complain.  Then set |\__fp_parse_word_#1:N| to use
%   |\__fp_variable_o:w|.
%    \begin{macrocode}
\cs_new_protected:Npn \fp_new_variable:n #1
  {
    \@@_id_if_invalid:nTF {#1}
      { \__kernel_msg_error:nnn { kernel } { fp-id-invalid } {#1} }
      { \exp_args:No \@@_new_variable:n { \tl_to_str:n {#1} } }
  }
\cs_new_protected:Npn \@@_new_variable:n #1
  {
    \cs_if_exist:cT { @@_parse_word_#1:N }
      {
        \__kernel_msg_error:nnn
          { kernel } { fp-id-already-defined } {#1}
        \cs_undefine:c { @@_parse_word_#1:N }
        \cs_undefine:c { l_@@_variable_#1_fp }
      }
    \@@_variable_set_parsing:Nn \cs_gset_eq:NN {#1}
  }
%    \end{macrocode}
% \end{macro}
% \end{macro}
%
% \begin{macro}{\fp_set_variable:nn}
% \begin{macro}[aux]{\@@_set_variable:nn}
%   Refuse invalid identifiers.  If the variable does not exist yet,
%   define it just as in \cs{fp_new_variable:n} (but without unnecessary
%   checks).  Then evaluate~|#2|.  If the result contains the
%   identifier~|#1|, we would later get a loop in cases such as
%   \begin{quote}
%     \cs{fp_set_variable:nn} |{A}| |{A}|\\
%     \cs{fp_show:n} |{A}|
%   \end{quote}
%   To detect this, define |\l__fp_variable_#1_fp| to raise an
%   internal flag and evaluate to \texttt{nan}.  Then re-evaluate
%   \cs{l_@@_symbolic_fp}, and store the result in~|#1|.  If the flag is
%   raised, |#1|~was present in \cs{l_@@_symbolic_fp}.  In all cases,
%   the |#1|-free result ends up in |\l__fp_variable_#1_fp|.
%    \begin{macrocode}
\flag_new:n { @@_symbolic }
\cs_new_protected:Npn \fp_set_variable:nn #1
  {
    \@@_id_if_invalid:nTF {#1}
      { \__kernel_msg_error:nnn { kernel } { fp-id-invalid } {#1} }
      { \exp_args:No \@@_set_variable:nn { \tl_to_str:n {#1} } }
  }
\cs_new_protected:Npn \@@_set_variable:nn #1#2
  {
    \@@_variable_set_parsing:Nn \cs_set_eq:NN {#1}
    \fp_set:Nn \l_@@_symbolic_fp {#2}
    \cs_set_nopar:cpn { l_@@_variable_#1_fp }
      { \flag_raise_if_clear:n { @@_symbolic } \c_nan_fp }
    \flag_clear:n { @@_symbolic }
    \fp_set:cn { l_@@_variable_#1_fp } { \l_@@_symbolic_fp }
    \flag_if_raised:nT { @@_symbolic }
      {
        \__kernel_msg_error:nnxxx { kernel } { fp-id-loop }
          { \tl_to_str:n {#1} }
          { \tl_to_str:n {#2} }
          { \fp_to_tl:N \l_@@_symbolic_fp }
      }
  }
%    \end{macrocode}
% \end{macro}
% \end{macro}
%
% \subsection{Turn unknown words to variables}
%
% \begin{macro}[aux, EXP]{\@@_parse_word:Nw}
%    \begin{macrocode}
\cs_undefine:N \@@_parse_word:Nw
\cs_new:Npn \@@_parse_word:Nw #1#2;
  {
    \cs_if_exist_use:cF { @@_parse_word_#2:N }
      {
        \@@_exp_after_symbolic_f:nw { \@@_parse_infix:NN }
        \s_@@_symbolic \@@_symbolic_chk:w
          \@@_variable_o:w #2 , { } ;
      }
      #1
  }
%    \end{macrocode}
% \end{macro}
%
% \subsection{Messages}
%
%    \begin{macrocode}
\__kernel_msg_new:nnnn { kernel } { fp-id-invalid }
  { Floating~point~identifier~'#1'~invalid. }
  {
    \c__msg_coding_error_text_tl
    LaTeX~has~been~asked~to~create~a~new~floating~point~identifier~'#1'~
    but~this~may~only~contain~ASCII~letters.
  }
\__kernel_msg_new:nnnn { kernel } { fp-id-already-defined }
  { Floating~point~identifier~'#1'~already~defined. }
  {
    \c__msg_coding_error_text_tl
    LaTeX~has~been~asked~to~create~a~new~floating~point~identifier~'#1'~
    but~this~name~has~already~been~used~elsewhere.
  }
\__kernel_msg_new:nnnn { kernel } { fp-id-used-elsewhere }
  { Floating~point~identifier~'#1'~already~used~for~something~else. }
  {
    LaTeX~has~been~asked~to~create~a~new~floating~point~identifier~'#1'~
    but~this~name~is~used,~and~is~not~a~user-defined~#2.
  }
\__kernel_msg_new:nnnn { kernel } { fp-id-loop }
  { Variable~'#1'~used~in~the~definition~of~'#1'. }
  {
    \c__msg_coding_error_text_tl
    LaTeX~has~been~asked~to~set~the~floating~point~identifier~'#1'~
    to~the~expression~'#2'.~Evaluating~this~expression~yields~'#3',~
    which~contains~'#1'~itself.
  }
%    \end{macrocode}
%
% \subsection{Road-map}
%
% The following functions are not implemented: |min|, |max|, |?:|,
% comparisons, |round|, |atan|, |acot|.
%
%    \begin{macrocode}
%</package>
%    \end{macrocode}
%
% \end{implementation}
%
% \PrintIndex
