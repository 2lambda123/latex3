\documentclass{article}


\title{Thoughts on Hooks in code}
\author{Frank Mittelbach \and
%                        \and  
%                        \and  
  \LaTeX{} Project Team}

%\date{2018-11-29}
\date{2019-06-24}

\newcounter{hook}
\newcounter{config}

\newenvironment{hook}[3][new]  % new/exists, area, name 
   {\refstepcounter{hook}%
    \begin{itemize}%
    \item[\textbf{Hook (\thehook):}]\textbf{#3} (#2/\textit{#1})}
   {\end{itemize}}

\newenvironment{config}[3][new]  % new/exists, area, name 
   {\refstepcounter{config}%
    \begin{itemize}%
    \item[\textbf{Config (\theconfig):}]\textbf{#3} (#2/\textit{#1})}
   {\end{itemize}}

\newcommand\cs[1]{\texttt{\textbackslash #1}}               
\newcommand\pkg[1]{\texttt{#1}}

\renewcommand\arg[1]{\textit{#1}}      % don't need the math Op in this document I hope :-)

\usepackage[T1]{fontenc}

\begin{document}

\maketitle

\tableofcontents

\section{Introduction}


This document collects some thoughts on places for hooks (i.e., code
interfaces where other code can add material for execution).

I distinguish this from what I call  ``configuration points'', which I think of as
being commands that have a single definition that can be replaced by a
different one but not ``added to'' by different packages. For example,
``add begin document'' is a typical hook, as it is a place where
different packages my want to execute some code. On the other hand
``prepare footins'' is a configuration interface whose sole purpose is
to do something specific to the \cs{footins} box. This may differ
based on the intended outcome, but there can only one definition active at any
time.\footnote{Technically this is of course not quite true, there is
  a certain gray area, but only if some packages very closely
  interrelated. The main point is that ``hooks'' are more about places
  where code gets added without (much) concern about what other
  package want to execute.}


\subsection{Hooks and their interface}

At the moment most existing hooks in \LaTeXe{} are actually named
\cs{@...hook} (with one or two exception). We should keep that
approach for backward compatibility and only change the exceptions.

Currently we do have declarations such as \cs{AtBeginDocument} to add
data to a hook.  The order in which the code gets executed then only
depends on the order of the declarations, i.e., on the order of
package loading.

What is not possible right now is define known dependencies between
different code executed in a certain place, result in the famous ``try
changing the order of package X and Y to resolve the
problem''. Sometimes this works, but sometime a different order would
be needed for different hooks and then of course that advice doesn't
really any longer work.

As an alternative/extension we envision that material added to a hook
will be named (using the package name by default) and that it will be
possible for packages to declare rules that its code for a certain
hook should be placed in relation to code from some other package (if
that package is loaded). This way known dependencies could be
automatically resolved (if declared as a rule by either side---or even
both if they don't conflict) and impossible requirements could be
detected as being incompatible.

Outline of interface:
\begin{verbatim}
 \AddToHook{<hook>}{<name>}{<code>}
 \ClearHook{<hook>}{<name>}              % maybe
 \SetupHook{<hook>}{<name>}{<code>}      % \ClearHook followed by \AddToHook

 \SetupHookRule{<hook>}{<name1>}{<name2>}{<rule>}
\end{verbatim}

\cs{AddToHook} would add additional \arg{code} labeled by \arg{name}
to the \arg{hook}. If there exists already code under that \arg{name}
then it is appended.  \cs{ClearHook} removes all code labeled by
\arg{name}.  \cs{SetupHookRule} defines some relationship between
\arg{name1} and \arg{name2}, for the exact syntax a lot different ways
are possible (should be determined later).

One could also think of \arg{name} to be optional defaulting to the
current package/class name to encourage people to use that as the name
normally.


Existing hook interface commands such as \cs{AtBeginDocument} would
still be supported as follows:
\begin{itemize}
\item
  They add their code under the name ``\texttt{legacy}'' to the hook.
\item
  Other packages could then decide that they should come before or
  after the code labeled \texttt{legacy}.
\end{itemize}
In other words:
\begin{verbatim}
 \def\AtBeginDocument#1{\AddToHook{begindocument}{legacy}{#1}}
 \def\AtEnddocument#1{\AddToHook{enddocument}{legacy}{#1}}
\end{verbatim}

However, there use would be discouraged in favor of explicitly
labeling the hook code with the package name.

There also exist \cs{AtEndOfPackage} and \cs{AtEndOfClass} which uses
\begin{verbatim}
 \@currname.\@currext-h@@k
\end{verbatim}
as the hook name, but I don't think
this needs to be incorporated into the more general mechanism as I
don't think this being usable anywhere outside the actual package
code---that one package writes into that hook of another package is in
theory be possible (as in overwrite some code of that package if it
gets loaded later than me, but I doubt that this makes a
useful/explainable interface).  On the other hand it is surprisingly
often used (in 379 package/classes in TL 2019) so perhaps that
assumption needs some checking.


\subsection{Configuration points and their interface}

As configuration points allow for only a single definition at any one
time they can be simply defined as commands without providing an
explicit interface.

My current suggestion is to all call them \cs{@...config} in the
\LaTeXe{} code and provide them with a default definition. We could
think of offering a configuration interface such as
\begin{verbatim}
 \SetupConfigPoint{@makecol}{preparefootins}{<code>}
\end{verbatim}
but that wouldn't do much other than (globally) replacing the config
command definition with the new one.

I also think that such config point commands should not take arguments
even if they technically have inputs (like the current page box or the
\cs{footins} box etc.).


\section{Hooks and config points in various places}

Collected
are existing \LaTeXe{} hooks (both by the kernel and/or by package) as
well a hooks that do not exist but would be beneficial to have for one
or the other reason. As of now it mainly discusses the \LaTeXe{}
situation.

For each hook/config point we document
\begin{itemize}
\item a ``name'';
\item the area where it applies;
\item whether it is ``new'' or does already exist (if so where);
\item and the major reason why it would be beneficial.
\end{itemize}
If there are several distinct use cases each could be added using
\cs{item}. You can cross reference hooks using
\cs{label}/\cs{ref}.



\subsection{Document structure}



\begin{hook}[kernel]{doc-structure}{documentclass}
\item
  This is really more or less internal (used for handling 2.09
  compatibility) but it shows up in one or two other places in TL.
\end{hook}

\begin{hook}[kernel]{\cs{document}}{begindocument}
\item
\end{hook}

\begin{hook}[kernel]{\cs{enddocument}}{enddocument}
\item
\end{hook}



\subsection{Output routine}

\subsubsection{Making pages}


\begin{hook}[kernel]{\cs{@outputpage}}{begindvibox}
\item
  \cs{AtBeginDvi} is a legacy \LaTeXe{} interface that hooks into the
  first shipout box at the very top. Afterwards the material is
  dropped so it doesn't show up in later boxes. What is special is
  that the hook itself is implemented as a box and each time
  \cs{AtBeginDvi} is called more material is added to that box which
  eventually is unboxed into the first shipout box. in that respect
  that hook is rather ``special'' and right now I'm not sure it really
  has to---use cases please.
\item
  If not then my proposal would be to deprecate it in favor of a
  ``normal'' hook at the same place (i.e., implemented as a token
  list) that would fit into the general model outlined above.
\end{hook}

\begin{hook}{\cs{@outputpage}}{firstshipout}
\item
  Suggested replacement for \texttt{begindvibox} as a normal named
  hook. Executed at the very top inside the first shipoutbox.
\item
  Not being implemented as a box register internally also means that
  it could execute other code than just adding \cs{special}s etc to
  the shipout box.
\end{hook}

\begin{hook}{\cs{@outputpage}}{shipout}
\item
  Executed directly at the top of the shiout box but only for shipouts
  after the first (which uses \texttt{firstshipout}.
\item
  For code that should be executed at all shipouts one need to put it
  into both \texttt{firstshipout} and \texttt{shipout}.
\end{hook}



\subsubsection{Making columns}

\begin{hook}{\cs{@makecol}}{pre}
\item
  \pkg{manyfoot} and similar packages like to take control at the very
  beginning of column generation.
\end{hook}

\begin{hook}{\cs{@makecol}}{post}
\item
  And we may have some post-processing going on after the column is
  made up.
\end{hook}

\begin{config}{\cs{@makecol}}{footins}
\item
  To support manipulation of footnote text (like footnotes as paras,
  in 2 columns etc).

  At this point the assembled footnotes are inside box \cs{footins} and any
  manipulation needs to globally write them back in there.
\end{config}

\begin{config}{\cs{@makecol}}{blocks}
\item
  A column supports the following block elements:
  \begin{itemize}
  \item
    galley text (already inside \cs{@outputbox})
  \item
    footnotes (initially inside \cs{footins})
  \item
    top and bottom floats (initially inside the \LaTeXe{} float lists
    for top and bottom)
  \end{itemize}
  These can appear in different order within the column and in this
  config point the ordering and any special spacing between them is
  specified. The final result is stored in \cs{@outputbox}.

  For specification only the following commands should be used:
  \begin{itemize}
  \item \cs{@makecolappendfootnotes}\arg{separation} The separation
    has to be \cs{vfill} or \cs{@makecol@moveskip} (which is not
    quite the right interface).
  \item \cs{@makecolattachfloats}
  \item \cs{@makecolappendstuff}\arg{vertical material}
  \end{itemize}
  For example to get the footnotes above the bottom floats one would specify:
\begin{verbatim}
 \SetupConfigPoint{@makecol}{blocks}
                  {\@makecol@appendfootnotes {\vfill}%
                   \@makecol@appendfloats}
\end{verbatim}


  Discussion notes:
  \begin{itemize}
  \item
    Maybe \@outputbox{} should start out empty there should be an
    explicit \cs{makecolappendtext}
  \item
    Technically attaching top and bottom floats in one go should be
    enough but perhaps it would be clearer if there are two commands
    \cs{@makecolappendtopfloats} and \cs{@makecolappendbottomfloats}
    even though their order should always be top above bottom.
  \item
    There have to be some clear rules on how \cs{@makecolappendstuff}
    can be used.
  \item
    Should spaces be automatically suppressed in configuration point setups?
  \end{itemize}
\end{config}

\begin{config}{\cs{@makecol}}{splitfootnotemessage}
\item
  Executes if a footnote gets split. By default empty but could be
  used to set up a warning message or similar in that case.
\end{config}


\subsection{Heading commands}

\begin{hook}{heading}{before-break}
\item
  For heading commands it is helpful if one can issue \cs{mark}
  commands, e.g. \cs{PutMark} from \pkg{xmarks} to support running
  header or footer setup. This has to be possible directly after
  heading title (where it is currently supported through commands like
  \cs{sectionmark}) but also directly before the break to determine
  the placement of the heading (e.g., at the very top of a page/column
  or elsewhere). The mark needs to come after the heading has done its
  magic with any preceding space, it can't hide such space from the
  command. In other words it need to happen basically in the middle of
  \cs{addpenalty}.
\item
  For L3 the galley concept might be able to handle this properly, but
  most likely also need more than one point of code injection.
\end{hook}



%\subsection{}

\end{document}
