% \iffalse meta-comment
%
%% File: l3backend-opacity.dtx
%
% Copyright (C) 2021-2023 The LaTeX Project
%
% It may be distributed and/or modified under the conditions of the
% LaTeX Project Public License (LPPL), either version 1.3c of this
% license or (at your option) any later version.  The latest version
% of this license is in the file
%
%    https://www.latex-project.org/lppl.txt
%
% This file is part of the "l3backend bundle" (The Work in LPPL)
% and all files in that bundle must be distributed together.
%
% -----------------------------------------------------------------------
%
% The development version of the bundle can be found at
%
%    https://github.com/latex3/latex3
%
% for those people who are interested.
%
%<*driver>
\documentclass[full,kernel]{l3doc}
\begin{document}
  \DocInput{\jobname.dtx}
\end{document}
%</driver>
% \fi
%
% \title{^^A
%   The \textsf{l3backend-opacity} package\\ Backend opacity support^^A
% }
%
% \author{^^A
%  The \LaTeX{} Project\thanks
%    {^^A
%      E-mail:
%        \href{mailto:latex-team@latex-project.org}
%          {latex-team@latex-project.org}^^A
%    }^^A
% }
%
% \date{Released 2023-03-09}
%
% \maketitle
%
% \begin{documentation}
%
% \end{documentation}
%
% \begin{implementation}
%
% \section{\pkg{l3backend-opacity} implementation}
%
%    \begin{macrocode}
%<*package>
%<@@=opacity>
%    \end{macrocode}
%
% Although opacity is not color, it needs to be managed in a somewhat
% similar way: using a dedicated stack if possible. Depending on the backend,
% that may not be possible. There is also the need to cover fill/stroke setting
% as well as more general running opacity. It is easiest to describe the value
% used in terms of opacity, although commonly this is referred to as
% transparency.
%
%    \begin{macrocode}
%<*dvips>
%    \end{macrocode}
%
% \begin{macro}{\@@_backend_select:n,\@@_backend_select_aux:n}
% \begin{macro}{\@@_backend_fill:n, \@@_backend_stroke:n}
% \begin{macro}{\@@_backend:nnn, \@@_backend:xnn}
%   No stack so set values directly. The need to deal with Distiller and
%   Ghostscript separately means we use a common auxiliary: the two
%   systems require different PostScript for transparency. This is
%   of course not quite as efficient as doing one test for setting all
%   transparency, but it keeps things clearer here. Thanks to Alex Grahn
%   for the detail on testing for GhostScript.
%    \begin{macrocode}
\cs_new_protected:Npn \@@_backend_select:n #1
  {
    \exp_args:Nx \@@_backend_select_aux:n
      { \fp_eval:n { min(max(0,#1),1) } }
  }
\cs_new_protected:Npn \@@_backend_select_aux:n #1
  {
    \@@_backend:nnn {#1} { fill }   { ca }
    \@@_backend:nnn {#1} { stroke } { CA }
  }
\cs_new_protected:Npn \@@_backend_fill:n #1
  {
    \@@_backend:xnn
      { \fp_eval:n { min(max(0,#1),1) } }
      { fill }
      { ca }
  }
\cs_new_protected:Npn \@@_backend_stroke:n #1
  {
    \@@_backend:xnn
      { \fp_eval:n { min(max(0,#1),1) } }
      { stroke }
      { CA }
  }
\cs_new_protected:Npn \@@_backend:nnn #1#2#3
  {
    \__kernel_backend_postscript:n
      {
        product ~ (Ghostscript) ~ search
          {
            pop ~ pop ~ pop ~
            #1 ~ .set #2 constantalpha
          }
          {
            pop ~
            mark ~
            /#3 ~ #1
            /SetTransparency ~
            pdfmark
          }
        ifelse
      }
  }
\cs_generate_variant:Nn \@@_backend:nnn { x }
%    \end{macrocode}
% \end{macro}
% \end{macro}
% \end{macro}
%
%    \begin{macrocode}
%</dvips>
%    \end{macrocode}
%
%    \begin{macrocode}
%<*dvipdfmx|luatex|pdftex|xetex>
%    \end{macrocode}
%
% \begin{variable}{\c_@@_backend_stack_int}
%   Set up a stack, where that is applicable.
%    \begin{macrocode}
\bool_lazy_and:nnT
  { \cs_if_exist_p:N \pdfmanagement_if_active_p: }
  { \pdfmanagement_if_active_p:}
  {
%<*luatex|pdftex>
    \__kernel_color_backend_stack_init:Nnn \c_@@_backend_stack_int
      { page ~ direct } { /opacity 1 ~ gs }
%</luatex|pdftex>
    \pdfmanagement_add:nnn { Page / Resources / ExtGState }
      { opacity 1 } { << /ca ~ 1 /CA ~ 1 >> }
  }
%    \end{macrocode}
% \end{variable}
%
% \begin{variable}{\l_@@_backend_fill_tl, \l_@@_backend_stroke_tl}
%   We use |tl| here for speed: at the backend, this should be reasonable.
%    \begin{macrocode}
\tl_new:N \l_@@_backend_fill_tl
\tl_new:N \l_@@_backend_stroke_tl
%    \end{macrocode}
% \end{variable}
%
% \begin{macro}{\@@_backend_select:n, \@@_backend_select_aux:n}
% \begin{macro}{\@@_backend_reset:}
%   Other than the need to evaluate the opacity as an \texttt{fp}, much the
%   same as color.
%    \begin{macrocode}
\cs_new_protected:Npn \@@_backend_select:n #1
 {
   \exp_args:Nx \@@_backend_select_aux:n
     { \fp_eval:n { min(max(0,#1),1) } }
 }
\cs_new_protected:Npn \@@_backend_select_aux:n #1
  {
    \tl_set:Nn \l_@@_backend_fill_tl {#1}
    \tl_set:Nn \l_@@_backend_stroke_tl {#1}
    \pdfmanagement_add:nnn { Page / Resources / ExtGState }
      { opacity #1 }
      { << /ca ~ #1 /CA ~ #1 >> }
%<*dvipdfmx|xetex>
    \__kernel_backend_literal_pdf:n
%</dvipdfmx|xetex>
%<*luatex|pdftex>
    \__kernel_color_backend_stack_push:nn \c_@@_backend_stack_int
%</luatex|pdftex>
      { /opacity #1 ~ gs }
    \group_insert_after:N \@@_backend_reset:
  }
\bool_lazy_and:nnF
  { \cs_if_exist_p:N \pdfmanagement_if_active_p: }
  { \pdfmanagement_if_active_p:}
  {
    \cs_gset_protected:Npn \@@_backend_select_aux:n #1 { }
  }
\cs_new_protected:Npn \@@_backend_reset:
 {
%<*dvipdfmx|xetex>
   \__kernel_backend_literal_pdf:n
      { /opacity1 ~ gs }
%</dvipdfmx|xetex>
%<*luatex|pdftex>
    \__kernel_color_backend_stack_pop:n \c_@@_backend_stack_int
%</luatex|pdftex>
 }
%    \end{macrocode}
% \end{macro}
% \end{macro}
%
% \begin{macro}{\@@_backend_fill:n, \@@_backend_stroke:n}
% \begin{macro}{\@@_backend_fillstroke:nn, \@@_backend_fillstroke:xx}
%   For separate fill and stroke, we need to work out if we need to do
%   more work or if we can stick to a single setting.
%    \begin{macrocode}
\cs_new_protected:Npn \@@_backend_fill:n #1
  {
    \@@_backend_fill_stroke:xx
      { \fp_eval:n { min(max(0,#1),1) } }
      \l_@@_backend_stroke_tl
  }
\cs_new_protected:Npn \@@_backend_stroke:n #1
  {
    \@@_backend_fill_stroke:xx
      \l_@@_backend_fill_tl
      { \fp_eval:n { min(max(0,#1),1) } }
  }
\cs_new_protected:Npn \@@_backend_fill_stroke:nn #1#2
  {
    \str_if_eq:nnTF {#1} {#2}
      { \@@_backend_select_aux:n {#1} }
      {
        \tl_set:Nn \l_@@_backend_fill_tl {#1}
        \tl_set:Nn \l_@@_backend_stroke_tl {#2}
        \pdfmanagement_add:nnn { Page / Resources / ExtGState }
          { opacity.fill #1 }
          { << /ca ~ #1 >> }
        \pdfmanagement_add:nnn { Page / Resources / ExtGState }
          { opacity.stroke #1 }
          { << /CA ~ #2 >> }
%<*dvipdfmx|xetex>
    \__kernel_backend_literal_pdf:n
%</dvipdfmx|xetex>
%<*luatex|pdftex>
    \__kernel_color_backend_stack_push:nn \c_@@_backend_stack_int
%</luatex|pdftex>
         { /opacity.fill #1 ~ gs /opacity.stroke #2 ~ gs }
        \group_insert_after:N \@@_backend_reset:
      }
  }
\cs_generate_variant:Nn \@@_backend_fill_stroke:nn { xx }
%    \end{macrocode}
% \end{macro}
% \end{macro}
%
%    \begin{macrocode}
%</dvipdfmx|luatex|pdftex|xetex>
%    \end{macrocode}
%
%    \begin{macrocode}
%<*dvisvgm>
%    \end{macrocode}
%
% \begin{macro}{\@@_backend_select:n, \@@_backend_fill:n, \@@_backend_stroke:n}
% \begin{macro}{\@@_backend:nn}
%   Once again, we use a scope here. There is a general opacity function for
%   SVG, but that is of course not set up using the stack.
%    \begin{macrocode}
\cs_new_protected:Npn \@@_backend_select:n #1
  { \@@_backend:nn {#1} { } }
\cs_new_protected:Npn \@@_backend_fill:n #1
  { \@@_backend:nn {#1} { fill- } }
\cs_new_protected:Npn \@@_backend_stroke:n #1
  { \@@_backend:nn { {#1} } { stroke- } }
\cs_new_protected:Npn \@@_backend:nn #1#2
  { \__kernel_backend_scope:x { #2 opacity = " \fp_eval:n { min(max(0,#1),1) } " } }
%    \end{macrocode}
% \end{macro}
% \end{macro}
%
%    \begin{macrocode}
%</dvisvgm>
%    \end{macrocode}
%
%    \begin{macrocode}
%</package>
%    \end{macrocode}
%
% \subsection{Font handling integration}
%
% In \LuaTeX{} we want to use these functions also for transparent fonts
% to avoid interference between both uses of transparency.
%
%    \begin{macrocode}
%<*lua>
%    \end{macrocode}
%
% First we need to check if pdfmanagement is active from Lua.
%    \begin{macrocode}
local pdfmanagement_active do
  local pdfmanagement_if_active_p = token.create'pdfmanagement_if_active_p:'
  local cmd = pdfmanagement_if_active_p.cmdname
  if cmd == 'undefined_cs' then
    pdfmanagement_active = false
  else
    token.put_next(pdfmanagement_if_active_p)
    pdfmanagement_active = token.scan_int() ~= 0
  end
end

if pdfmanagement_active and luaotfload and luaotfload.set_transparent_colorstack then
  luaotfload.set_transparent_colorstack(token.create'c__opacity_backend_stack_int'.index)

  local transparent_register = {
    token.create'pdfmanagement_add:nnn',
    token.new(0, 1),
      'Page/Resources/ExtGState',
    token.new(0, 2),
    token.new(0, 1),
      '',
    token.new(0, 2),
    token.new(0, 1),
      '<</ca ',
      '',
      '/CA ',
      '',
      '>>',
    token.new(0, 2),
  }
  luatexbase.add_to_callback('luaotfload.parse_transparent', function(value)
    value = (octet * -1):match(value)
    if not value then
      tex.error'Invalid transparency value'
      return
    end
    value = value:sub(1, -2)
    local result = 'opacity' .. value
    tex.runtoks(function()
      transparent_register[6], transparent_register[10], transparent_register[12] = result, value, value
      tex.sprint(-2, transparent_register)
    end)
    return '/' .. result .. ' gs'
  end, 'l3opacity')
end
%    \end{macrocode}
%
%    \begin{macrocode}
%</lua>
%    \end{macrocode}
%
%
% \end{implementation}
%
% \PrintIndex
